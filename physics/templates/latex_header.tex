% XeLaTeXで \( と \) が未定義になる場合の対策
% まず amsmath を読み込んで \( と \) を確実に定義する
\RequirePackage{amsmath}

% 日本語の改行と折り返しを改善するための設定
\usepackage{fixltx2e}  % LaTeXの古いコマンドを修正(\( と \) を含む)
\usepackage{geometry}
\usepackage{amssymb}
\usepackage{breqn}
\usepackage{textcomp}

% \textbfを確実に定義(unicode-mathが読み込まれた後)
\makeatletter
\@ifundefined{textbf}{%
  \newcommand{\textbf}[1]{{\bfseries #1}}%
}{}
\makeatother

% amsmathパッケージ読み込み後に確実に \begin{math} と \end{math} に定義する
% \let を使って既存の定義を保存してから再定義(未定義の場合はエラーになるので、try-catch的な処理)
\makeatletter
% \( の定義を試みる(既に定義されている場合は保存してから再定義)
\expandafter\let\expandafter\temp@lparen\csname (\endcsname
\ifx\temp@lparen\relax
  % 未定義の場合
  \def\({\begin{math}}%
\else
  % 既に定義されている場合(念のため再定義)
  \def\({\begin{math}}%
\fi
% \) の定義を試みる
\expandafter\let\expandafter\temp@rparen\csname )\endcsname
\ifx\temp@rparen\relax
  % 未定義の場合
  \def\){\end{math}}%
\else
  % 既に定義されている場合(念のため再定義)
  \def\){\end{math}}%
\fi
\makeatother

% 文書開始時にも確実に定義されるようにする(二重の安全策)
\AtBeginDocument{%
  \makeatletter
  \def\({\begin{math}}%
  \def\){\end{math}}%
  \makeatother
}

% ページレイアウトの調整
\geometry{
  a4paper,
  margin=2cm
}

% \textbfが未定義の場合の対策(念のため再定義)
\makeatletter
\@ifundefined{textbf}{%
  \newcommand{\textbf}[1]{{\bfseries #1}}%
}{}
\makeatother

% 日本語の改行設定(XeLaTeXの標準機能)- 最重要設定
\XeTeXlinebreaklocale "ja"
\XeTeXlinebreakskip = 0pt plus 1pt minus 0.1pt

% 長いテキストの折り返しを許可(強力な設定)
\sloppy

% 日本語の改行を大幅に改善
\setlength{\emergencystretch}{15em}
\tolerance=10000
\pretolerance=10000
\setlength{\hfuzz}{2pt}
\setlength{\vfuzz}{2pt}
\setlength{\overfullrule}{0pt}
\hbadness=10000

% ハイフネーションを無効化(日本語では不要)
\hyphenpenalty=10000
\exhyphenpenalty=10000
\doublehyphendemerits=10000
\finalhyphendemerits=10000
\adjdemerits=10000
\brokenpenalty=10000

% 長い数式の折り返し
\allowdisplaybreaks

% 段落の設定
\setlength{\parindent}{1em}
\setlength{\parskip}{0.3em plus 0.1em minus 0.1em}

% 行間の調整
\linespread{1.15}

% ============================================
% 目次とアウトラインの設定
% ============================================

% hyperrefパッケージ(PDFのブックマークとリンク)
\usepackage{hyperref}
\hypersetup{
    colorlinks=true,
    linkcolor=blue,
    filecolor=magenta,      
    urlcolor=cyan,
    citecolor=green,
    pdftitle={力学特論 演習問題 解答},
    pdfauthor={},
    pdfsubject={力学特論},
    pdfkeywords={力学, 相対論, ラグランジアン, ハミルトニアン},
    bookmarks=true,
    bookmarksopen=true,
    bookmarksnumbered=true,
    pdfstartview={FitH},
    pdfpagemode=UseOutlines
}

% 目次の深さを設定(セクション、サブセクション、サブサブセクションまで)
\setcounter{tocdepth}{3}
\setcounter{secnumdepth}{3}

% 図のキャプションの設定
\usepackage{caption}
\captionsetup{font=small, labelfont=bf}

% 図の参照を改善
\usepackage{graphicx}
\graphicspath{{physics/adPhysics/}}

% ページ番号の設定
\usepackage{fancyhdr}
\pagestyle{plain}
