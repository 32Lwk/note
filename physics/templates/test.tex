\section{統計物理学Ⅰ・演習6
解答}\label{ux7d71ux8a08ux7269ux7406ux5b66ux2170ux6f14ux7fd26-ux89e3ux7b54}

\subsection{セクションI:
偏微分と熱力学関数}\label{ux30bbux30afux30b7ux30e7ux30f3i-ux504fux5faeux5206ux3068ux71b1ux529bux5b66ux95a2ux6570}

\subsubsection{セクションIの概要}\label{ux30bbux30afux30b7ux30e7ux30f3iux306eux6982ux8981}

本セクションでは、\textbf{偏微分とマクスウェル関係式}を用いて、熱力学量間の重要な関係式を導出する。これらの関係式は、実験的に測定しやすい量から、直接測定が困難な量を計算する際に非常に有用である。

\textbf{重要な概念:}

\begin{enumerate}
\def\labelenumi{\arabic{enumi}.}
\tightlist
\item
  \textbf{マクスウェル関係式}:
  熱力学ポテンシャル(自由エネルギーなど)の2階偏微分の順序交換可能性から導かれる関係式
\item
  \textbf{偏微分の連鎖則}: 変数変換における偏微分の関係
\item
  \textbf{熱力学ポテンシャル}: 系の平衡状態を記述する関数(\(F\), \(G\),
  \(H\)など)
\end{enumerate}

本セクションで導出する関係式は、理想気体と実在気体の違いを理解する上で重要である。

\begin{center}\rule{0.5\linewidth}{0.5pt}\end{center}

\subsubsection{問1.
内部エネルギーの方程式}\label{ux554f1.-ux5185ux90e8ux30a8ux30cdux30ebux30aeux30fcux306eux65b9ux7a0bux5f0f}

\textbf{問題:} 以下の関係式を導け。
\[(1) \quad \left(\frac{\partial U}{\partial V}\right)_T = T^2 \left\{\frac{\partial(p/T)}{\partial T}\right\}_V\]

\textbf{解答:}

\textbf{基本関係式の確認}

熱力学の基本関係式は、第一法則と第二法則を結合したものである:

\[TdS = dU + pdV\]

ここで、\(T\)は温度、\(S\)はエントロピー、\(U\)は内部エネルギー、\(p\)は圧力、\(V\)は体積である。

また、ヘルムホルツ自由エネルギー\(F = U - TS\)の全微分は:

\[dF = dU - TdS - SdT = -SdT - pdV\]

\textbf{導出の戦略}

目標は、\((\partial U/\partial V)_T\)を、圧力の温度依存性で表すことである。内部エネルギー\(U\)を直接測定するのは困難だが、圧力\(p\)の温度依存性は実験的に測定しやすい。

\textbf{方法1: エントロピーを用いた導出}

\textbf{ステップ1: 内部エネルギーの全微分}

\(TdS = dU + pdV\)より、\(dU = TdS - pdV\)。

\(U\)を\(T\)と\(V\)の関数として考えると、全微分は:

\[dU = \left(\frac{\partial U}{\partial T}\right)_V dT + \left(\frac{\partial U}{\partial V}\right)_T dV\]

\textbf{ステップ2: エントロピーの全微分}

一方、\(S\)も\(T\)と\(V\)の関数として:

\[dS = \left(\frac{\partial S}{\partial T}\right)_V dT + \left(\frac{\partial S}{\partial V}\right)_T dV\]

\textbf{ステップ3: 基本関係式への代入}

\(dU = TdS - pdV\)に\(dS\)を代入:

\[dU = T\left[\left(\frac{\partial S}{\partial T}\right)_V dT + \left(\frac{\partial S}{\partial V}\right)_T dV\right] - pdV\]

\[= T\left(\frac{\partial S}{\partial T}\right)_V dT + \left[T\left(\frac{\partial S}{\partial V}\right)_T - p\right]dV\]

\textbf{ステップ4: 係数の比較}

\(dU\)の2つの表現を比較すると:

\[dU = \left(\frac{\partial U}{\partial T}\right)_V dT + \left(\frac{\partial U}{\partial V}\right)_T dV\]

\[= T\left(\frac{\partial S}{\partial T}\right)_V dT + \left[T\left(\frac{\partial S}{\partial V}\right)_T - p\right]dV\]

\(dV\)の係数を比較:

\[\left(\frac{\partial U}{\partial V}\right)_T = T\left(\frac{\partial S}{\partial V}\right)_T - p\]

\textbf{ステップ5: マクスウェル関係式の適用}

マクスウェル関係式は、熱力学ポテンシャルの2階偏微分の順序交換可能性から導かれる。\(dF = -SdT - pdV\)より:

\[\frac{\partial}{\partial V}\left(\frac{\partial F}{\partial T}\right)_V = \frac{\partial}{\partial T}\left(\frac{\partial F}{\partial V}\right)_T\]

\[\frac{\partial(-S)}{\partial V} = \frac{\partial(-p)}{\partial T}\]

したがって:

\[\left(\frac{\partial S}{\partial V}\right)_T = \left(\frac{\partial p}{\partial T}\right)_V\]

これは重要なマクスウェル関係式の1つである。

\textbf{ステップ6: 最終的な関係式}

ステップ4とステップ5の結果を組み合わせ:

\[\left(\frac{\partial U}{\partial V}\right)_T = T\left(\frac{\partial p}{\partial T}\right)_V - p\]

\textbf{ステップ7: 右辺の整理}

右辺を\(T^2\)でくくる形に変形する:

\[T\left(\frac{\partial p}{\partial T}\right)_V - p = T^2 \left[\frac{1}{T}\left(\frac{\partial p}{\partial T}\right)_V - \frac{p}{T^2}\right]\]

ここで、積の微分法則を逆に適用すると:

\[\frac{\partial}{\partial T}\left(\frac{p}{T}\right) = \frac{1}{T}\frac{\partial p}{\partial T} - \frac{p}{T^2}\]

したがって:

\[T\left(\frac{\partial p}{\partial T}\right)_V - p = T^2 \left[\frac{\partial}{\partial T}\left(\frac{p}{T}\right)\right]_V = T^2 \left\{\frac{\partial(p/T)}{\partial T}\right\}_V\]

\textbf{答え:}
\[\left(\frac{\partial U}{\partial V}\right)_T = T^2 \left\{\frac{\partial(p/T)}{\partial T}\right\}_V\]

\textbf{物理的意味:}

\textbf{式の解釈:}

\[\left(\frac{\partial U}{\partial V}\right)_T = T^2 \left\{\frac{\partial(p/T)}{\partial T}\right\}_V\]

この式は、\textbf{内部エネルギーが体積に依存する度合い}を、\textbf{圧力の温度依存性}から計算できることを示している。これは、内部エネルギーを直接測定する代わりに、より測定しやすい圧力の温度依存性から内部エネルギーの体積依存性を求めることができることを意味する。

\textbf{理想気体の場合:}

理想気体の状態方程式\(pV = nRT\)より:

\[\frac{p}{T} = \frac{nR}{V}\]

これは\(T\)に依存しないため:

\[\left\{\frac{\partial(p/T)}{\partial T}\right\}_V = 0\]

したがって:

\[\left(\frac{\partial U}{\partial V}\right)_T = 0\]

これは\textbf{ジュールの法則}(Joule's
law)として知られている:理想気体の内部エネルギーは体積に依存せず、温度のみに依存する。

\textbf{実在気体の場合:}

実在気体では、分子間相互作用により、圧力が温度に依存する。例えば、ファン・デル・ワールス気体では:

\[p = \frac{nRT}{V-nb} - \frac{an^2}{V^2}\]

この場合、\(p/T\)は温度に依存するため、内部エネルギーも体積に依存する。これは、分子間相互作用エネルギーが体積に依存するためである。

\textbf{応用:}

この関係式は、実在気体の内部エネルギーを実験的に決定する際に重要である。圧力の温度依存性を測定することで、内部エネルギーの体積依存性を計算できる。

\begin{center}\rule{0.5\linewidth}{0.5pt}\end{center}

\subsubsection{問2.
定圧熱容量と定積熱容量の関係式}\label{ux554f2.-ux5b9aux5727ux71b1ux5bb9ux91cfux3068ux5b9aux7a4dux71b1ux5bb9ux91cfux306eux95a2ux4fc2ux5f0f}

\textbf{問題:} 以下の関係式を導け。
\[(2) \quad C_p = C_V + T\left(\frac{\partial V}{\partial T}\right)_p \left(\frac{\partial p}{\partial T}\right)_V\]

\textbf{解答:}

\textbf{熱容量の定義}

熱容量は、系に熱を加えたときの温度上昇の度合いを表す:

\begin{itemize}
\item
  \textbf{定積熱容量} \(C_V\): 体積を一定に保ったときの熱容量
  \[C_V = \left(\frac{\partial U}{\partial T}\right)_V = T\left(\frac{\partial S}{\partial T}\right)_V\]
\item
  \textbf{定圧熱容量} \(C_p\): 圧力を一定に保ったときの熱容量
  \[C_p = \left(\frac{\partial H}{\partial T}\right)_p = T\left(\frac{\partial S}{\partial T}\right)_p\]
\end{itemize}

ここで、\(H = U + pV\)は\textbf{エンタルピー}である。

\textbf{導出の戦略}

\(C_p\)と\(C_V\)の関係を導くには、エンタルピー\(H\)とエントロピー\(S\)の関係を利用する。

\textbf{方法1: エンタルピーを用いた導出}

\textbf{ステップ1: エンタルピーの全微分}

\(H = U + pV\)より:

\[dH = dU + pdV + Vdp\]

基本関係式\(TdS = dU + pdV\)を代入:

\[dH = TdS + Vdp\]

\textbf{ステップ2: エンタルピーを\(T\)と\(p\)の関数として表現}

\(H\)を\(T\)と\(p\)の関数として:

\[dH = \left(\frac{\partial H}{\partial T}\right)_p dT + \left(\frac{\partial H}{\partial p}\right)_T dp\]

\textbf{ステップ3: エントロピーを\(T\)と\(p\)の関数として表現}

一方、\(S\)も\(T\)と\(p\)の関数として:

\[dS = \left(\frac{\partial S}{\partial T}\right)_p dT + \left(\frac{\partial S}{\partial p}\right)_T dp\]

\textbf{ステップ4: 基本関係式への代入}

\(dH = TdS + Vdp\)に\(dS\)を代入:

\[dH = T\left[\left(\frac{\partial S}{\partial T}\right)_p dT + \left(\frac{\partial S}{\partial p}\right)_T dp\right] + Vdp\]

\[= T\left(\frac{\partial S}{\partial T}\right)_p dT + \left[T\left(\frac{\partial S}{\partial p}\right)_T + V\right]dp\]

\textbf{ステップ5: 係数の比較}

\(dH\)の2つの表現を比較:

\[dH = \left(\frac{\partial H}{\partial T}\right)_p dT + \left(\frac{\partial H}{\partial p}\right)_T dp\]

\[= T\left(\frac{\partial S}{\partial T}\right)_p dT + \left[T\left(\frac{\partial S}{\partial p}\right)_T + V\right]dp\]

\(dT\)の係数を比較:

\[C_p = \left(\frac{\partial H}{\partial T}\right)_p = T\left(\frac{\partial S}{\partial T}\right)_p\]

\textbf{ステップ6: エントロピーの変数変換}

\(S\)を\(T\)と\(V\)の関数として考えると:

\[dS = \left(\frac{\partial S}{\partial T}\right)_V dT + \left(\frac{\partial S}{\partial V}\right)_T dV\]

ここで、\(p\)を一定に保つ(\(dp = 0\))と、\(V\)は\(T\)の関数になる。したがって:

\[\left(\frac{\partial S}{\partial T}\right)_p = \left(\frac{\partial S}{\partial T}\right)_V + \left(\frac{\partial S}{\partial V}\right)_T \left(\frac{\partial V}{\partial T}\right)_p\]

これは、\textbf{偏微分の連鎖則}の応用である。

\textbf{ステップ7: 最終的な関係式}

ステップ5とステップ6の結果を組み合わせ:

\[C_p = T\left(\frac{\partial S}{\partial T}\right)_V + T\left(\frac{\partial S}{\partial V}\right)_T \left(\frac{\partial V}{\partial T}\right)_p\]

\[= C_V + T\left(\frac{\partial S}{\partial V}\right)_T \left(\frac{\partial V}{\partial T}\right)_p\]

\textbf{ステップ8: マクスウェル関係式の適用}

マクスウェル関係式より:

\[\left(\frac{\partial S}{\partial V}\right)_T = \left(\frac{\partial p}{\partial T}\right)_V\]

したがって:

\[C_p = C_V + T\left(\frac{\partial V}{\partial T}\right)_p \left(\frac{\partial p}{\partial T}\right)_V\]

\textbf{答え:}
\[C_p = C_V + T\left(\frac{\partial V}{\partial T}\right)_p \left(\frac{\partial p}{\partial T}\right)_V\]

\textbf{物理的意味:}

\textbf{\(C_p > C_V\)の理由:}

\[C_p = C_V + T\left(\frac{\partial V}{\partial T}\right)_p \left(\frac{\partial p}{\partial T}\right)_V\]

この式は、\textbf{定圧熱容量が定積熱容量より大きい理由}を説明している。

\textbf{定圧過程と定積過程の違い:}

\begin{enumerate}
\def\labelenumi{\arabic{enumi}.}
\tightlist
\item
  \textbf{定積過程}(\(dV = 0\)):

  \begin{itemize}
  \tightlist
  \item
    温度を上げるために必要な熱は、内部エネルギーを増やすだけ
  \item
    \(C_V = (\partial U/\partial T)_V\)
  \end{itemize}
\item
  \textbf{定圧過程}(\(dp = 0\)):

  \begin{itemize}
  \tightlist
  \item
    温度を上げると、体積が膨張する(\((\partial V/\partial T)_p > 0\))
  \item
    体積が膨張すると、外部に対して仕事\(pdV\)をする必要がある
  \item
    そのため、同じ温度上昇に対して、より多くの熱が必要
  \end{itemize}
\end{enumerate}

\textbf{右辺第2項の意味:}

\[T\left(\frac{\partial V}{\partial T}\right)_p \left(\frac{\partial p}{\partial T}\right)_V\]

この項は、\textbf{体積変化に伴う追加のエネルギー}を表している:

\begin{itemize}
\tightlist
\item
  \((\partial V/\partial T)_p\):
  定圧での体積の温度依存性(熱膨張係数に関連)
\item
  \((\partial p/\partial T)_V\): 定積での圧力の温度依存性
\item
  これらの積に\(T\)をかけたものが、追加の熱容量に寄与
\end{itemize}

\textbf{理想気体の場合(マイヤーの関係式):}

理想気体の状態方程式\(pV = nRT\)より:

\[\left(\frac{\partial V}{\partial T}\right)_p = \frac{nR}{p}, \quad \left(\frac{\partial p}{\partial T}\right)_V = \frac{nR}{V}\]

したがって:

\[C_p - C_V = T \cdot \frac{nR}{p} \cdot \frac{nR}{V} = T \cdot \frac{(nR)^2}{pV} = T \cdot \frac{(nR)^2}{nRT} = nR\]

これは\textbf{マイヤーの関係式}(Mayer's relation)として知られている。

\textbf{実在気体の場合:}

実在気体では、\(C_p - C_V\)は\(nR\)より大きくなる。これは、分子間相互作用により、体積変化に伴う追加のエネルギーが大きくなるためである。

\begin{center}\rule{0.5\linewidth}{0.5pt}\end{center}

\subsection{セクションII:
輪ゴムの熱力学}\label{ux30bbux30afux30b7ux30e7ux30f3ii-ux8f2aux30b4ux30e0ux306eux71b1ux529bux5b66}

\subsubsection{セクションIIの概要}\label{ux30bbux30afux30b7ux30e7ux30f3iiux306eux6982ux8981}

本セクションでは、\textbf{輪ゴムの熱力学}を扱う。輪ゴムは、通常のバネとは異なり、温度依存性を持つ弾性を示す興味深い系である。

\textbf{輪ゴムの特徴:}

\begin{enumerate}
\def\labelenumi{\arabic{enumi}.}
\tightlist
\item
  \textbf{温度依存性のあるバネ定数}: \(k(T) = k_0 + k_1T\)

  \begin{itemize}
  \tightlist
  \item
    \(k_0\): 温度に依存しない部分(通常の弾性)
  \item
    \(k_1\): 温度依存部分(エントロピー効果)
  \end{itemize}
\item
  \textbf{伸びると発熱}: 輪ゴムを素早く伸ばすと、温度が上昇する

  \begin{itemize}
  \tightlist
  \item
    これは、通常のバネ(伸ばすと温度が下がる)とは逆の挙動
  \end{itemize}
\item
  \textbf{エントロピー効果}: 輪ゴムの弾性は、主にエントロピー効果による

  \begin{itemize}
  \tightlist
  \item
    分子鎖が伸びると、エントロピーが減少する
  \item
    これが、温度が高いほど張力が大きくなる理由
  \end{itemize}
\end{enumerate}

本セクションでは、温度依存性を持つバネ定数と比熱から、自由エネルギーと内部エネルギーを導出する。

\begin{center}\rule{0.5\linewidth}{0.5pt}\end{center}

\subsubsection{問1.
バネの自由エネルギー}\label{ux554f1.-ux30d0ux30cdux306eux81eaux7531ux30a8ux30cdux30ebux30aeux30fc}

\textbf{問題:}
バネの張力が\(X = -k(T)x\)(\(k(T) = k_0 + k_1T\))で与えられ、比熱が\(C = \text{定数}\)のとき、自由エネルギーが
\[F(T,x) = -CT\left[\ln\left(\frac{T}{T_0}\right) - 1\right] + \frac{1}{2}k(T)x^2\]
と書けることを示せ。ただし、\(F(T=0, x=0) = 0\)とする。

\textbf{解答:}

\textbf{基本関係式の確認}

バネ(輪ゴム)の熱力学では、自由エネルギー\(F\)の全微分は:

\[dF = -SdT - Xdx\]

ここで、\(S\)はエントロピー、\(X\)は張力、\(x\)は伸びである。張力は:

\[X = -k(T)x = -(k_0 + k_1T)x\]

負号は、張力が伸びに反対方向に働くことを示している。

\textbf{導出の戦略}

与えられた条件から自由エネルギーを決定する: 1.
張力\(X = -k(T)x\)から、\(F\)の\(x\)依存性を決定 2.
比熱\(C\)が定数であることから、\(F\)の\(T\)依存性を決定 3.
境界条件\(F(T=0, x=0) = 0\)を適用

\textbf{ステップ1: 張力からの条件}

自由エネルギーの定義より:

\[\left(\frac{\partial F}{\partial x}\right)_T = -X = k(T)x = (k_0 + k_1T)x\]

これを\(x\)で積分する。\(T\)を定数として扱う:

\[F(T,x) = \int_0^x k(T)x' dx' + f(T) = \frac{1}{2}k(T)x^2 + f(T)\]

ここで、\(f(T)\)は\(x\)に依存しない関数(積分定数)である。この関数は、\(T\)のみに依存する部分を表す。

\textbf{ステップ2: エントロピーからの条件}

自由エネルギーの定義より:

\[\left(\frac{\partial F}{\partial T}\right)_x = -S\]

一方、ステップ1の結果\(F(T,x) = \frac{1}{2}k(T)x^2 + f(T)\)を\(T\)で偏微分:

\[\left(\frac{\partial F}{\partial T}\right)_x = \frac{1}{2}\frac{\partial k(T)}{\partial T} x^2 + f'(T) = \frac{1}{2}k_1 x^2 + f'(T)\]

したがって:

\[S = -\frac{1}{2}k_1 x^2 - f'(T)\]

\textbf{重要な観察:}

エントロピーは、\(x\)に依存する項(\(-k_1 x^2/2\))と、\(T\)に依存する項(\(-f'(T)\))からなる。\(x\)に依存する項は、伸びが大きいほどエントロピーが小さくなることを示している(輪ゴムの特徴)。

\textbf{ステップ3: 比熱からの条件}

比熱\(C\)は、\(x\)を一定に保ったときの熱容量:

\[C = T\left(\frac{\partial S}{\partial T}\right)_x\]

ステップ2の結果\(S = -\frac{1}{2}k_1 x^2 - f'(T)\)を\(T\)で偏微分:

\[\left(\frac{\partial S}{\partial T}\right)_x = -f''(T)\]

したがって:

\[C = -Tf''(T)\]

\(C\)が定数なので:

\[f''(T) = -\frac{C}{T}\]

\textbf{ステップ4: \(f(T)\)の積分}

\(f''(T) = -C/T\)を積分:

\[f'(T) = -C\int \frac{1}{T} dT = -C\ln T + A\]

ここで、\(A\)は積分定数である。

もう一度積分:

\[f(T) = -C\int \ln T dT + AT + B\]

部分積分を用いると:

\[\int \ln T dT = T\ln T - T\]

したがって:

\[f(T) = -C(T\ln T - T) + AT + B = -CT\left(\ln T - 1\right) + AT + B\]

\textbf{ステップ5: 境界条件の適用}

境界条件\(F(T=0, x=0) = 0\)を適用する。

\(F(T,x) = -CT(\ln T - 1) + AT + B + \frac{1}{2}k(T)x^2\)において、\(T \to 0\)で\(\ln T \to -\infty\)となるため、適切な基準温度\(T_0\)を導入する必要がある。

\textbf{基準温度\(T_0\)の導入:}

\[\ln T = \ln\left(\frac{T}{T_0}\right) + \ln T_0\]

したがって:

\[f(T) = -CT\left[\ln\left(\frac{T}{T_0}\right) + \ln T_0 - 1\right] + AT + B\]

\[= -CT\left[\ln\left(\frac{T}{T_0}\right) - 1\right] - CT\ln T_0 + AT + B\]

\(T\ln T_0\)の項は、\(A\)の項と合わせて\(A' = A - C\ln T_0\)とすることができる。また、\(F(T=0, x=0) = 0\)より、\(B = 0\)である。

さらに、\(A'\)は\(T\)の線形項であるが、\(F(T=0, x=0) = 0\)の条件から、\(A' = 0\)とすることができる(または、\(T_0\)の選択で吸収できる)。

したがって:

\[f(T) = -CT\left[\ln\left(\frac{T}{T_0}\right) - 1\right]\]

\textbf{答え:}
\[F(T,x) = -CT\left[\ln\left(\frac{T}{T_0}\right) - 1\right] + \frac{1}{2}k(T)x^2\]

\textbf{物理的意味:}

\textbf{自由エネルギーの2つの項:}

\[F(T,x) = -CT\left[\ln\left(\frac{T}{T_0}\right) - 1\right] + \frac{1}{2}k(T)x^2\]

\begin{enumerate}
\def\labelenumi{\arabic{enumi}.}
\tightlist
\item
  \textbf{第1項(温度依存項)}: \(-CT[\ln(T/T_0) - 1]\)

  \begin{itemize}
  \tightlist
  \item
    温度のみに依存する部分
  \item
    エントロピー効果による寄与
  \item
    比熱\(C\)が定数であることから生じる
  \end{itemize}
\item
  \textbf{第2項(弾性エネルギー)}: \(\frac{1}{2}k(T)x^2\)

  \begin{itemize}
  \tightlist
  \item
    伸び\(x\)に依存する部分
  \item
    バネの弾性エネルギー
  \item
    \(k(T) = k_0 + k_1T\)なので、温度が高いほど、同じ伸びに対してより大きなエネルギーが必要
  \end{itemize}
\end{enumerate}

\textbf{輪ゴムの特徴:}

\begin{itemize}
\tightlist
\item
  \textbf{\(k_1 > 0\)の場合}: 温度が高いほど、バネ定数が大きくなる

  \begin{itemize}
  \tightlist
  \item
    これは、輪ゴムが伸びるときに発熱する理由
  \item
    エントロピー効果により、温度が高いほど張力が大きくなる
  \end{itemize}
\item
  \textbf{通常のバネとの違い}:
  通常のバネでは\(k_1 = 0\)(温度に依存しない)

  \begin{itemize}
  \tightlist
  \item
    輪ゴムは、主にエントロピー効果による弾性を示す
  \item
    分子鎖が伸びると、エントロピーが減少し、これが張力の原因となる
  \end{itemize}
\end{itemize}

\begin{center}\rule{0.5\linewidth}{0.5pt}\end{center}

\subsubsection{問2.
内部エネルギーの計算}\label{ux554f2.-ux5185ux90e8ux30a8ux30cdux30ebux30aeux30fcux306eux8a08ux7b97}

\textbf{問題:} 問1の結果と、内部エネルギーの関係式(問1の(2)式のバネ版)
\[\left(\frac{\partial U}{\partial x}\right)_T = T^2 \left\{\frac{\partial(X/T)}{\partial T}\right\}_x\]
を用いて、内部エネルギー\(U(T,x)\)を求めよ。ただし、\(U(T=0, x=0) = 0\)とする。

\textbf{解答:}

\textbf{導出の戦略}

問1で求めた自由エネルギー\(F(T,x)\)と、セクションIで導出した内部エネルギーの関係式を用いて、内部エネルギー\(U(T,x)\)を求める。

\textbf{ステップ1: 内部エネルギーの関係式の適用}

セクションIの問1で導出した関係式のバネ版:

\[\left(\frac{\partial U}{\partial x}\right)_T = T^2 \left\{\frac{\partial(X/T)}{\partial T}\right\}_x\]

張力は\(X = -k(T)x = -(k_0 + k_1T)x\)なので:

\[\frac{X}{T} = -\frac{(k_0 + k_1T)x}{T} = -\frac{k_0 x}{T} - k_1 x\]

これを\(T\)で偏微分(\(x\)を一定に保つ):

\[\left(\frac{\partial(X/T)}{\partial T}\right)_x = \frac{k_0 x}{T^2}\]

したがって:

\[\left(\frac{\partial U}{\partial x}\right)_T = T^2 \cdot \frac{k_0 x}{T^2} = k_0 x\]

\textbf{重要な観察:}

内部エネルギーの\(x\)依存性は、\(k_0\)(温度に依存しない部分)のみに依存し、\(k_1\)(温度依存部分)には依存しない。これは、\(k_1\)の効果が主にエントロピー効果であることを示している。

\textbf{ステップ2: 内部エネルギーの積分}

ステップ1の結果\(\partial U/\partial x = k_0 x\)を\(x\)で積分(\(T\)を定数として扱う):

\[U(T,x) = \int_0^x k_0 x' dx' + g(T) = \frac{1}{2}k_0 x^2 + g(T)\]

ここで、\(g(T)\)は\(x\)に依存しない関数(積分定数)である。

\textbf{ステップ3: 比熱からの条件}

内部エネルギーとエントロピーの関係:

\[U = F + TS\]

問1より、自由エネルギーは:

\[F(T,x) = -CT\left[\ln\left(\frac{T}{T_0}\right) - 1\right] + \frac{1}{2}k(T)x^2\]

エントロピーは:

\[S = -\left(\frac{\partial F}{\partial T}\right)_x\]

\(F\)を\(T\)で偏微分:

\[\frac{\partial F}{\partial T} = -C\left[\ln\left(\frac{T}{T_0}\right) - 1\right] - CT \cdot \frac{1}{T} + \frac{1}{2}k_1 x^2\]

\[= -C\ln\left(\frac{T}{T_0}\right) + C - C + \frac{1}{2}k_1 x^2 = -C\ln\left(\frac{T}{T_0}\right) + \frac{1}{2}k_1 x^2\]

したがって:

\[S = C\ln\left(\frac{T}{T_0}\right) - \frac{1}{2}k_1 x^2\]

\textbf{ステップ4: 内部エネルギーの計算}

\(U = F + TS\)に代入:

\[U = -CT\left[\ln\left(\frac{T}{T_0}\right) - 1\right] + \frac{1}{2}k(T)x^2 + T\left[C\ln\left(\frac{T}{T_0}\right) - \frac{1}{2}k_1 x^2\right]\]

\[= -CT\ln\left(\frac{T}{T_0}\right) + CT + \frac{1}{2}(k_0 + k_1T)x^2 + CT\ln\left(\frac{T}{T_0}\right) - \frac{1}{2}k_1 T x^2\]

\[= CT + \frac{1}{2}k_0 x^2 + \frac{1}{2}k_1 T x^2 - \frac{1}{2}k_1 T x^2\]

\[= CT + \frac{1}{2}k_0 x^2\]

\textbf{重要な結果:}

内部エネルギーは、\(k_1\)(温度依存部分)を含まない。これは、\(k_1\)の効果が完全にエントロピー効果であることを示している。

\textbf{答え:} \[U(T,x) = CT + \frac{1}{2}k_0 x^2\]

\textbf{物理的意味:}

\textbf{内部エネルギーの2つの項:}

\[U(T,x) = CT + \frac{1}{2}k_0 x^2\]

\begin{enumerate}
\def\labelenumi{\arabic{enumi}.}
\tightlist
\item
  \textbf{第1項(温度依存項)}: \(CT\)

  \begin{itemize}
  \tightlist
  \item
    温度に比例する項
  \item
    比熱\(C\)による寄与
  \item
    熱運動のエネルギー
  \end{itemize}
\item
  \textbf{第2項(弾性エネルギー)}: \(\frac{1}{2}k_0 x^2\)

  \begin{itemize}
  \tightlist
  \item
    伸び\(x\)の2乗に比例する項
  \item
    \(k_0\)(温度に依存しない部分)による弾性エネルギー
  \item
    通常のバネと同じ形
  \end{itemize}
\end{enumerate}

\textbf{重要な観察:}

\begin{itemize}
\tightlist
\item
  \textbf{\(k_1\)の効果}:
  \(k_1\)(温度依存部分)は内部エネルギーには現れない

  \begin{itemize}
  \tightlist
  \item
    これは、\(k_1\)の効果が完全にエントロピー効果であることを示している
  \item
    自由エネルギーには\(k_1\)が含まれるが、それは\(TS\)項を通じて現れる
  \end{itemize}
\end{itemize}

\textbf{輪ゴムの断熱伸長:}

輪ゴムを素早く伸ばすと、断熱過程として\(U\)が一定に保たれる。\(U = CT + \frac{1}{2}k_0 x^2\)が一定なので:

\begin{itemize}
\tightlist
\item
  \(x\)が増加すると、\(\frac{1}{2}k_0 x^2\)が増加
\item
  したがって、\(CT\)が減少し、温度が下がるように見える
\end{itemize}

しかし、実際には: -
エントロピー効果(\(k_1\)の効果)により、温度が上昇する -
これは、輪ゴムが伸びるときに発熱する理由である

\textbf{通常のバネとの違い:}

\begin{itemize}
\tightlist
\item
  通常のバネ: \(k_1 = 0\)なので、\(U = CT + \frac{1}{2}k_0 x^2\)

  \begin{itemize}
  \tightlist
  \item
    断熱伸長では、温度が下がる
  \end{itemize}
\item
  輪ゴム: \(k_1 > 0\)なので、エントロピー効果が重要

  \begin{itemize}
  \tightlist
  \item
    断熱伸長では、エントロピー効果により温度が上昇する
  \end{itemize}
\end{itemize}

\begin{center}\rule{0.5\linewidth}{0.5pt}\end{center}

\subsection{セクションIII:
1成分理想気体}\label{ux30bbux30afux30b7ux30e7ux30f3iii-1ux6210ux5206ux7406ux60f3ux6c17ux4f53}

\subsubsection{セクションIIIの概要}\label{ux30bbux30afux30b7ux30e7ux30f3iiiux306eux6982ux8981}

本セクションでは、\textbf{1成分理想気体}の熱力学を詳しく扱う。理想気体とは、以下の性質を持つ気体である:

\begin{enumerate}
\def\labelenumi{\arabic{enumi}.}
\tightlist
\item
  \textbf{状態方程式}: \(pV = NRT\)(\(N\)はモル数、\(R\)は気体定数)
\item
  \textbf{内部エネルギー}:
  \(U = cNRT\)(\(c\)は定数、通常は\(c = 3/2\)(単原子気体)や\(c = 5/2\)(二原子気体))
\item
  \textbf{分子間相互作用が無視できる}:
  粒子間の相互作用エネルギーが運動エネルギーに比べて無視できる
\end{enumerate}

理想気体のエントロピーは、統計力学から導出される形で与えられる:
\[S = NR\ln(T^c V/N) + NS_0\] ここで、\(S_0\)は定数項である。

本セクションでは、このエントロピーと内部エネルギーから出発して、以下の熱力学量を計算する:
- \textbf{化学ポテンシャル} \(\mu\):
粒子を1個追加する際の自由エネルギーの変化 -
\textbf{ヘルムホルツ自由エネルギー} \(F\):
等温過程で利用可能な仕事を表す - \textbf{エントロピーの線形形式}:
広延変数による表現

これらの量は、理想気体の平衡状態を完全に記述するために必要である。

\begin{center}\rule{0.5\linewidth}{0.5pt}\end{center}

\subsubsection{問1.
化学ポテンシャルの計算}\label{ux554f1.-ux5316ux5b66ux30ddux30c6ux30f3ux30b7ux30e3ux30ebux306eux8a08ux7b97}

\textbf{問題:}
エントロピー\(S = NR\ln(T^c V/N) + NS_0\)と内部エネルギー\(U = cNRT\)から、化学ポテンシャル\(\mu\)を計算せよ。

\textbf{解答:}

\textbf{基本関係式の確認}

熱力学の基本関係式(第一法則と第二法則の結合)は、粒子数が変化する場合:
\[TdS = dU + pdV - \mu dN\]

ここで、\(\mu\)は化学ポテンシャルである。この式から、\(dN\)の係数を比較することで\(\mu\)を求めることができる。

\textbf{ステップ1: エントロピーの全微分}

与えられたエントロピー: \[S = NR\ln(T^c V/N) + NS_0\]

\(S\)を\(T\), \(V\),
\(N\)の関数として全微分する。まず、\(\ln(T^c V/N) = c\ln T + \ln V - \ln N\)に注意すると:

\[\frac{\partial S}{\partial T} = NR \cdot \frac{c}{T} = \frac{cNR}{T}\]

\[\frac{\partial S}{\partial V} = NR \cdot \frac{1}{V} = \frac{NR}{V}\]

\[\frac{\partial S}{\partial N} = R\ln(T^c V/N) + NR \cdot \left(-\frac{1}{N}\right) + S_0 = R\ln(T^c V/N) - R + S_0\]

したがって:
\[dS = \frac{\partial S}{\partial T} dT + \frac{\partial S}{\partial V} dV + \frac{\partial S}{\partial N} dN\]

\[= \frac{cNR}{T} dT + \frac{NR}{V} dV + \left[R\ln(T^c V/N) + S_0 - R\right] dN\]

\textbf{ステップ2: 内部エネルギーの全微分}

与えられた内部エネルギー: \[U = cNRT\]

\(U\)を\(T\)と\(N\)の関数として全微分:
\[\frac{\partial U}{\partial T} = cNR, \quad \frac{\partial U}{\partial N} = cRT\]

したがって:
\[dU = \frac{\partial U}{\partial T} dT + \frac{\partial U}{\partial N} dN = cNR dT + cRT dN\]

\textbf{ステップ3: 基本関係式への代入}

\(TdS\)を計算:
\[TdS = cNR dT + \frac{NR T}{V} dV + T\left[R\ln(T^c V/N) + S_0 - R\right] dN\]

\(dU + pdV\)を計算: \[dU + pdV = cRT dN + cNR dT + pdV\]

したがって:
\[TdS = cNR dT + \frac{NR T}{V} dV + T\left[R\ln(T^c V/N) + S_0 - R\right] dN\]

\[dU + pdV = cRT dN + cNR dT + pdV\]

\textbf{ステップ4: 係数の比較}

\(TdS = dU + pdV - \mu dN\)の各項の係数を比較する。

\textbf{\(dV\)の係数:}
\[TdS \text{の} dV \text{の係数} = \frac{NR T}{V}\]
\[dU + pdV \text{の} dV \text{の係数} = p\]

したがって: \[p = \frac{NR T}{V}\]

これは理想気体の状態方程式である。

\textbf{\(dN\)の係数:}
\[TdS \text{の} dN \text{の係数} = T\left[R\ln(T^c V/N) + S_0 - R\right]\]
\[dU + pdV \text{の} dN \text{の係数} = cRT\]

したがって: \[T\left[R\ln(T^c V/N) + S_0 - R\right] = cRT - \mu\]

これを\(\mu\)について解くと:
\[-\mu = T\left[R\ln(T^c V/N) + S_0 - R\right] - cRT\]

\[\mu = -RT\ln(T^c V/N) - TS_0 + RT + cRT\]

\[= -RT\ln(T^c V/N) - TS_0 + (c+1)RT\]

\textbf{\(dT\)の係数:} \[TdS \text{の} dT \text{の係数} = cNR\]
\[dU + pdV \text{の} dT \text{の係数} = cNR\]

これは一致しており、整合性が取れている。

\textbf{答え:}
\[\mu = -RT\ln\left(\frac{T^c V}{N}\right) - TS_0 + (c+1)RT\]

\textbf{物理的意味:}

化学ポテンシャル\(\mu\)は、\textbf{粒子を1個追加する際の自由エネルギーの変化}を表す。理想気体の場合:

\[\mu = -RT\ln\left(\frac{T^c V}{N}\right) - TS_0 + (c+1)RT\]

\textbf{各項の意味:}

\begin{enumerate}
\def\labelenumi{\arabic{enumi}.}
\tightlist
\item
  \textbf{\(-RT\ln(T^c V/N)\)項}:

  \begin{itemize}
  \tightlist
  \item
    対数項は、エントロピー効果を表す。
  \item
    \(T^c V/N\)が大きいほど(温度が高い、体積が大きい、粒子数が少ない)、化学ポテンシャルは低くなる。
  \item
    これは、粒子を追加することでエントロピーが増加し、系がより安定になることを意味する。
  \end{itemize}
\item
  \textbf{\(-TS_0\)項}:

  \begin{itemize}
  \tightlist
  \item
    定数項による寄与。通常は基準エントロピーとして扱われる。
  \end{itemize}
\item
  \textbf{\((c+1)RT\)項}:

  \begin{itemize}
  \tightlist
  \item
    内部エネルギーからの寄与。粒子を追加すると、内部エネルギーが\(cRT\)増加する。
  \item
    また、エントロピーの温度依存性から\(RT\)の寄与がある。
  \end{itemize}
\end{enumerate}

\textbf{化学ポテンシャルの性質:}

\begin{itemize}
\tightlist
\item
  \textbf{温度依存性}:
  温度が高いほど、\(\ln(T^c V/N)\)が大きくなり、化学ポテンシャルは低くなる。高温では粒子を追加しやすい。
\item
  \textbf{体積依存性}:
  体積が大きいほど、化学ポテンシャルは低くなる。広い空間では粒子を追加しやすい。
\item
  \textbf{粒子数依存性}:
  粒子数が少ないほど、化学ポテンシャルは低くなる。希薄な系では粒子を追加しやすい。
\end{itemize}

\textbf{平衡条件:}

2つの系が平衡にあるとき、化学ポテンシャルは等しい:\(\mu_1 = \mu_2\)。これにより、粒子の移動方向が決まる。

\begin{center}\rule{0.5\linewidth}{0.5pt}\end{center}

\subsubsection{問2.
ヘルムホルツ自由エネルギー}\label{ux554f2.-ux30d8ux30ebux30e0ux30dbux30ebux30c4ux81eaux7531ux30a8ux30cdux30ebux30aeux30fc}

\textbf{問題:}
内部エネルギーとエントロピーから、ヘルムホルツ自由エネルギー\(F(T,V,N)\)を求め、全微分形式を書け。また、各偏微分がどの熱力学変数に対応するか答えよ。

\textbf{解答:}

\textbf{ステップ1: 自由エネルギーの計算}

\[F = U - TS = cNRT - T\left[NR\ln(T^c V/N) + NS_0\right]\]

\[= cNRT - NRT\ln(T^c V/N) - NTS_0\]

\[= NRT\left[c - \ln(T^c V/N)\right] - NTS_0\]

\[= NRT\left[c - c\ln T - \ln(V/N)\right] - NTS_0\]

\textbf{ステップ2: 全微分と偏微分の計算}

自由エネルギー\(F(T,V,N)\)の全微分は:
\[dF = \left(\frac{\partial F}{\partial T}\right)_{V,N} dT + \left(\frac{\partial F}{\partial V}\right)_{T,N} dV + \left(\frac{\partial F}{\partial N}\right)_{T,V} dN\]

各偏微分を詳しく計算する。

\textbf{温度による偏微分:}

\[F = NRT\left[c - \ln(T^c V/N)\right] - NTS_0\]

\(T\)で偏微分する際、積の微分法則を適用:
\[\frac{\partial}{\partial T}\left[NRT \cdot \ln(T^c V/N)\right] = NR \cdot \ln(T^c V/N) + NRT \cdot \frac{\partial}{\partial T}\ln(T^c V/N)\]

\[\frac{\partial}{\partial T}\ln(T^c V/N) = \frac{\partial}{\partial T}(c\ln T + \ln V - \ln N) = \frac{c}{T}\]

したがって:
\[\left(\frac{\partial F}{\partial T}\right)_{V,N} = NR\left[c - \ln(T^c V/N)\right] - NRT \cdot \frac{c}{T} - NS_0\]

\[= NR\left[c - \ln(T^c V/N) - c\right] - NS_0\]

\[= -NR\ln(T^c V/N) - NS_0 = -S\]

これは、\(F = U - TS\)から\(\partial F/\partial T = -S\)であることと一致する。

\textbf{体積による偏微分:}

\[\left(\frac{\partial F}{\partial V}\right)_{T,N} = \frac{\partial}{\partial V}\left[NRT\left[c - \ln(T^c V/N)\right] - NTS_0\right]\]

\[= -NRT \cdot \frac{\partial}{\partial V}\ln(T^c V/N) = -NRT \cdot \frac{1}{V} = -\frac{NRT}{V} = -p\]

これは、理想気体の状態方程式\(p = NRT/V\)と一致する。

\textbf{粒子数による偏微分:}

\[\left(\frac{\partial F}{\partial N}\right)_{T,V} = \frac{\partial}{\partial N}\left[NRT\left[c - \ln(T^c V/N)\right] - NTS_0\right]\]

積の微分法則を適用:
\[= RT\left[c - \ln(T^c V/N)\right] - NRT \cdot \frac{\partial}{\partial N}\ln(T^c V/N) - TS_0\]

\[\frac{\partial}{\partial N}\ln(T^c V/N) = \frac{\partial}{\partial N}(\ln(T^c V) - \ln N) = -\frac{1}{N}\]

したがって:
\[\left(\frac{\partial F}{\partial N}\right)_{T,V} = RT\left[c - \ln(T^c V/N)\right] + RT - TS_0\]

\[= RT\left[c + 1 - \ln(T^c V/N)\right] - TS_0 = \mu\]

これは問1で求めた化学ポテンシャルと一致する。

したがって: \[dF = -SdT - pdV + \mu dN\]

\textbf{答え:} \[F(T,V,N) = NRT\left[c - \ln(T^c V/N)\right] - NTS_0\]

\[dF = -SdT - pdV + \mu dN\]

各偏微分の対応: -
\(\left(\frac{\partial F}{\partial T}\right)_{V,N} = -S\)(エントロピー)
- \(\left(\frac{\partial F}{\partial V}\right)_{T,N} = -p\)(圧力) -
\(\left(\frac{\partial F}{\partial N}\right)_{T,V} = \mu\)(化学ポテンシャル)

\textbf{物理的意味:}

\textbf{ヘルムホルツ自由エネルギー\(F\)の役割:}

\begin{enumerate}
\def\labelenumi{\arabic{enumi}.}
\item
  \textbf{等温過程での仕事}:
  \(F\)は、等温過程で系が外部に対して行うことができる最大の仕事を表す。\(dF = -SdT - pdV + \mu dN\)より、等温(\(dT = 0\))で粒子数一定(\(dN = 0\))の過程では、\(dF = -pdV\)となり、体積変化による仕事が自由エネルギーの減少分に等しい。
\item
  \textbf{平衡条件}:
  等温・等積・粒子数一定の条件下では、\(F\)が最小になる状態が平衡状態である。
\item
  \textbf{完全な熱力学記述}:
  \(F(T,V,N)\)が与えられれば、以下の熱力学量がすべて導出できる:

  \begin{itemize}
  \tightlist
  \item
    エントロピー: \(S = -\partial F/\partial T\)
  \item
    圧力: \(p = -\partial F/\partial V\)
  \item
    化学ポテンシャル: \(\mu = \partial F/\partial N\)
  \item
    内部エネルギー: \(U = F + TS\)
  \end{itemize}
\end{enumerate}

\textbf{共役変数の関係:}

各偏微分は、対応する\textbf{共役変数}(conjugate variable)を与える:

\begin{itemize}
\tightlist
\item
  \textbf{\((T, S)\)}: 温度とエントロピーは共役な変数対
\item
  \textbf{\((p, V)\)}: 圧力と体積は共役な変数対
\item
  \textbf{\((\mu, N)\)}: 化学ポテンシャルと粒子数は共役な変数対
\end{itemize}

全微分\(dF = -SdT - pdV + \mu dN\)は、これらの共役変数の関係を明確に示している。

\textbf{理想気体の特徴:}

理想気体の自由エネルギーは、対数項を含むため、温度や体積の変化に対して非線形に振る舞う。これは、エントロピーの対数依存性に起因している。

\begin{center}\rule{0.5\linewidth}{0.5pt}\end{center}

\subsubsection{問3.
化学ポテンシャルの検証}\label{ux554f3.-ux5316ux5b66ux30ddux30c6ux30f3ux30b7ux30e3ux30ebux306eux691cux8a3c}

\textbf{問題:}
問2の結果から、化学ポテンシャルを直接計算し、問1の結果と一致することを確認せよ。

\textbf{解答:}

問2より: \[\mu = \left(\frac{\partial F}{\partial N}\right)_{T,V}\]

\[F = NRT\left[c - \ln(T^c V/N)\right] - NTS_0\]

\(N\)で偏微分する際、\(T^c V/N\)の中に\(N\)が含まれることに注意:

\[\mu = RT\left[c - \ln(T^c V/N)\right] - NRT \cdot \frac{\partial}{\partial N}\ln(T^c V/N) - TS_0\]

\[\frac{\partial}{\partial N}\ln(T^c V/N) = \frac{\partial}{\partial N}\left[\ln(T^c V) - \ln N\right] = -\frac{1}{N}\]

したがって: \[\mu = RT\left[c - \ln(T^c V/N)\right] + RT - TS_0\]

\[= RT\left[c + 1 - \ln(T^c V/N)\right] - TS_0\]

\[= -RT\ln(T^c V/N) + (c+1)RT - TS_0\]

これは問1の結果と一致する。

\textbf{答え:} 一致する。

\begin{center}\rule{0.5\linewidth}{0.5pt}\end{center}

\subsubsection{問4.
エントロピーの線形形式}\label{ux554f4.-ux30a8ux30f3ux30c8ux30edux30d4ux30fcux306eux7ddaux5f62ux5f62ux5f0f}

\textbf{問題:} エントロピー\(S(U,V,N)\)を、広延変数の線形和の形
\[S(U,V,N) = a_U U + a_V V + a_N N\] で表し、式(8)を再現せよ。

\textbf{解答:}

基本関係式: \[TdS = dU + pdV - \mu dN\]

したがって: \[dS = \frac{1}{T} dU + \frac{p}{T} dV - \frac{\mu}{T} dN\]

係数を比較:
\[a_U = \frac{1}{T}, \quad a_V = \frac{p}{T}, \quad a_N = -\frac{\mu}{T}\]

理想気体の状態方程式と内部エネルギー:
\[p = \frac{NR T}{V}, \quad U = cNRT\]

したがって:
\[a_U = \frac{1}{T}, \quad a_V = \frac{NR}{V}, \quad a_N = -\frac{\mu}{T}\]

問1より: \[\mu = -RT\ln(T^c V/N) - TS_0 + (c+1)RT\]

したがって: \[a_N = R\ln(T^c V/N) + S_0 - (c+1)R\]

\textbf{ステップ4: 線形形式から元の式への復元}

係数を代入して、\(S\)を計算する:

\[S = a_U U + a_V V + a_N N = \frac{1}{T} U + \frac{p}{T} V - \frac{\mu}{T} N\]

理想気体の状態方程式\(p = NRT/V\)と内部エネルギー\(U = cNRT\)より:

\[S = \frac{1}{T} \cdot cNRT + \frac{NRT/V}{T} \cdot V - \frac{\mu}{T} N\]

\[= cNR + NR - \frac{\mu}{T} N\]

問1より、\(\mu = -RT\ln(T^c V/N) - TS_0 + (c+1)RT\)なので:

\[\frac{\mu}{T} = -R\ln(T^c V/N) - S_0 + (c+1)R\]

したがって:

\[S = cNR + NR - \left[-R\ln(T^c V/N) - S_0 + (c+1)R\right] N\]

\[= cNR + NR + NR\ln(T^c V/N) + NS_0 - (c+1)NR\]

\[= (c + 1 - c - 1)NR + NR\ln(T^c V/N) + NS_0\]

\[= NR\ln(T^c V/N) + NS_0\]

これは、元のエントロピーの式と一致する。

\textbf{重要なポイント:}

この計算により、エントロピー\(S(U,V,N)\)が広延変数の線形和として表されることが確認された。ただし、係数\(a_U\),
\(a_V\), \(a_N\)は、\(U\), \(V\),
\(N\)に依存する(\(T = U/(cNR)\)など)。これは、エントロピーが一次同次関数であることを示している。

\textbf{答え:}
\[S(U,V,N) = \frac{1}{T} U + \frac{p}{T} V - \frac{\mu}{T} N = NR\ln(T^c V/N) + NS_0\]

\textbf{計算の詳細:}

\textbf{エントロピーの線形形式の重要性:}

エントロピー\(S(U,V,N)\)が広延変数の線形和として表されることは、熱力学の基本的な性質である。

\[S(U,V,N) = a_U U + a_V V + a_N N = \frac{1}{T} U + \frac{p}{T} V - \frac{\mu}{T} N\]

\textbf{広延性と強度性:}

\begin{enumerate}
\def\labelenumi{\arabic{enumi}.}
\tightlist
\item
  \textbf{広延変数}(extensive variables): 系のサイズに比例する変数

  \begin{itemize}
  \tightlist
  \item
    \(U\)(内部エネルギー)
  \item
    \(V\)(体積)
  \item
    \(N\)(粒子数)
  \item
    \(S\)(エントロピー)
  \end{itemize}
\item
  \textbf{強度変数}(intensive variables): 系のサイズに依存しない変数

  \begin{itemize}
  \tightlist
  \item
    \(T\)(温度)
  \item
    \(p\)(圧力)
  \item
    \(\mu\)(化学ポテンシャル)
  \end{itemize}
\item
  \textbf{係数の意味}: 各係数(\(a_U\), \(a_V\),
  \(a_N\))は、対応する強度変数を温度で割ったものである:

  \begin{itemize}
  \tightlist
  \item
    \(a_U = 1/T\): 温度の逆数
  \item
    \(a_V = p/T\): 圧力と温度の比
  \item
    \(a_N = -\mu/T\): 化学ポテンシャルと温度の比(負号)
  \end{itemize}
\end{enumerate}

\textbf{一次同次性:}

エントロピーは一次同次関数である。すなわち、すべての広延変数を\(\lambda\)倍すると、エントロピーも\(\lambda\)倍される:

\[S(\lambda U, \lambda V, \lambda N) = \lambda S(U, V, N)\]

これは、系を2つ合わせたとき、エントロピーが2倍になることを意味する(エントロピーの加法性)。

\textbf{オイラーの関係式:}

一次同次関数に対して、オイラーの関係式が成り立つ:

\[S = \frac{\partial S}{\partial U} U + \frac{\partial S}{\partial V} V + \frac{\partial S}{\partial N} N\]

これは、上記の線形形式と一致している。

\textbf{理想気体の特徴:}

理想気体では、エントロピーが\(U\), \(V\),
\(N\)の線形和として表されるが、実際の式\(S = NR\ln(T^c V/N) + NS_0\)は対数項を含む。これは、\(T = U/(cNR)\)の関係により、\(U\)が対数の中に現れるためである。しかし、\(S\)を\(U\),
\(V\), \(N\)の関数として表すと、線形形式に分解できる。

\begin{center}\rule{0.5\linewidth}{0.5pt}\end{center}

\subsubsection{セクションIIIのまとめ}\label{ux30bbux30afux30b7ux30e7ux30f3iiiux306eux307eux3068ux3081}

本セクションでは、\textbf{1成分理想気体}の熱力学を詳しく扱った。以下の重要な結果を得た:

\textbf{1. 化学ポテンシャル:}
\[\mu = -RT\ln\left(\frac{T^c V}{N}\right) - TS_0 + (c+1)RT\]

\begin{itemize}
\tightlist
\item
  粒子を1個追加する際の自由エネルギーの変化を表す
\item
  温度、体積、粒子数に依存する
\item
  平衡条件では、2つの系の化学ポテンシャルが等しい
\end{itemize}

\textbf{2. ヘルムホルツ自由エネルギー:}
\[F(T,V,N) = NRT\left[c - \ln(T^c V/N)\right] - NTS_0\]

\[dF = -SdT - pdV + \mu dN\]

\begin{itemize}
\tightlist
\item
  等温過程での最大仕事を表す
\item
  各偏微分が対応する熱力学量(\(S\), \(p\), \(\mu\))を与える
\item
  完全な熱力学記述が可能
\end{itemize}

\textbf{3. エントロピーの線形形式:}
\[S(U,V,N) = \frac{1}{T} U + \frac{p}{T} V - \frac{\mu}{T} N\]

\begin{itemize}
\tightlist
\item
  エントロピーは広延変数(\(U\), \(V\), \(N\))の一次同次関数
\item
  各係数は対応する強度変数を温度で割ったもの
\item
  オイラーの関係式が成り立つ
\end{itemize}

\textbf{重要な概念:}

\begin{itemize}
\tightlist
\item
  \textbf{広延変数と強度変数}:
  系のサイズに比例する変数と、サイズに依存しない変数
\item
  \textbf{共役変数}: 熱力学変数の対(\((T,S)\), \((p,V)\), \((\mu,N)\))
\item
  \textbf{一次同次性}:
  すべての広延変数を\(\lambda\)倍すると、エントロピーも\(\lambda\)倍される
\end{itemize}

これらの結果は、理想気体の平衡状態を完全に記述するための基礎となる。

\begin{center}\rule{0.5\linewidth}{0.5pt}\end{center}

\subsection{セクションIV:
2成分理想気体混合}\label{ux30bbux30afux30b7ux30e7ux30f3iv-2ux6210ux5206ux7406ux60f3ux6c17ux4f53ux6df7ux5408}

\subsubsection{セクションIVの概要}\label{ux30bbux30afux30b7ux30e7ux30f3ivux306eux6982ux8981}

本セクションでは、\textbf{2成分理想気体の混合}によるエントロピー変化を扱う。これは、統計力学における重要な現象である。

\textbf{混合エントロピーの重要性:}

\begin{enumerate}
\def\labelenumi{\arabic{enumi}.}
\tightlist
\item
  \textbf{自発的混合}:
  異なる気体を混合すると、エントロピーが増加し、混合が自発的に起こる
\item
  \textbf{第二法則}: エントロピー増加は、第二法則の直接的な例である
\item
  \textbf{理想気体の独立性}:
  理想気体では、各成分は独立に振る舞う(ダルトンの法則)
\end{enumerate}

\textbf{本セクションで扱う内容:}

\begin{itemize}
\tightlist
\item
  初期状態(仕切りで分離)のエントロピー
\item
  混合後のエントロピー
\item
  エントロピー変化とその正負の判定
\end{itemize}

これらの結果は、溶液の混合、拡散現象など、多くの物理現象の理解に重要である。

\begin{center}\rule{0.5\linewidth}{0.5pt}\end{center}

\subsubsection{問1.
初期状態のエントロピー}\label{ux554f1.-ux521dux671fux72b6ux614bux306eux30a8ux30f3ux30c8ux30edux30d4ux30fc}

\textbf{問題:}
体積\(V\)が仕切りで\(V_1\)と\(V_2\)に分けられ、それぞれに成分1(\(N_1\)モル)と成分2(\(N_2\)モル)が入っている平衡状態のエントロピーを、\((T, V, N_1, N_2)\)の関数として表せ。

\textbf{解答:}

\textbf{初期状態の設定}

体積\(V\)が仕切りで\(V_1\)と\(V_2\)に分けられ、それぞれに成分1(\(N_1\)モル)と成分2(\(N_2\)モル)が入っている。初期状態では、2つの気体は仕切りで分離されているが、平衡状態(圧力が等しい)にある。

\textbf{ステップ1: 各成分のエントロピー}

セクションIIIで導出した理想気体のエントロピーを用いる:

成分1のエントロピー:
\[S_1(T, V_1, N_1) = N_1 R\ln(T^c V_1/N_1) + N_1 S_0\]

成分2のエントロピー:
\[S_2(T, V_2, N_2) = N_2 R\ln(T^c V_2/N_2) + N_2 S_0\]

\textbf{ステップ2: 全エントロピー}

2つの成分は独立しているため、全エントロピーは和:

\[S_{\text{tot}} = S_1 + S_2 = N_1 R\ln(T^c V_1/N_1) + N_2 R\ln(T^c V_2/N_2) + (N_1 + N_2) S_0\]

\textbf{ステップ3: 平衡条件の適用}

平衡状態では、圧力が等しい:

\[p_1 = \frac{N_1 RT}{V_1} = p_2 = \frac{N_2 RT}{V_2} = p\]

したがって:

\[\frac{N_1}{V_1} = \frac{N_2}{V_2}\]

また、\(V_1 + V_2 = V\)なので:

\[\frac{N_1}{V_1} = \frac{N_2}{V_2} = \frac{N_1 + N_2}{V_1 + V_2} = \frac{N_1 + N_2}{V}\]

したがって:

\[V_1 = \frac{N_1}{N_1 + N_2} V, \quad V_2 = \frac{N_2}{N_1 + N_2} V\]

\textbf{ステップ4: エントロピーの整理}

\(V_1\)と\(V_2\)を代入:

\[S_{\text{tot}} = N_1 R\ln\left(\frac{T^c V_1}{N_1}\right) + N_2 R\ln\left(\frac{T^c V_2}{N_2}\right) + (N_1 + N_2) S_0\]

\[= N_1 R\ln\left(\frac{T^c V}{N_1 + N_2}\right) + N_2 R\ln\left(\frac{T^c V}{N_1 + N_2}\right) + (N_1 + N_2) S_0\]

\[= (N_1 + N_2) R\ln\left(\frac{T^c V}{N_1 + N_2}\right) + (N_1 + N_2) S_0\]

\textbf{重要な観察:}

初期状態のエントロピーは、2つの成分を合わせた1成分理想気体のエントロピーと同じ形になっている。これは、各成分が独立に占有する体積が、全体積に比例しているためである。

\textbf{答え:}
\[S_{\text{tot}} = (N_1 + N_2) R\ln\left(\frac{T^c V}{N_1 + N_2}\right) + (N_1 + N_2) S_0\]

\textbf{物理的意味:}

\textbf{初期状態の特徴:}

\begin{itemize}
\tightlist
\item
  2つの気体は仕切りで分離されているが、圧力が平衡している
\item
  各成分は、全体積のうち、自分のモル数に比例する部分を占有する
\item
  エントロピーは、各成分が独立に占有する体積に依存する
\end{itemize}

\textbf{エントロピーの解釈:}

\[S_{\text{tot}} = (N_1 + N_2) R\ln\left(\frac{T^c V}{N_1 + N_2}\right) + (N_1 + N_2) S_0\]

この式は、\(N_1 + N_2\)モルの1成分理想気体が体積\(V\)を占有する場合のエントロピーと同じである。これは、各成分が独立に振る舞うためである。

\begin{center}\rule{0.5\linewidth}{0.5pt}\end{center}

\subsubsection{問2.
混合後のエントロピー}\label{ux554f2.-ux6df7ux5408ux5f8cux306eux30a8ux30f3ux30c8ux30edux30d4ux30fc}

\textbf{問題:}
仕切りを取り除いて混合した後の平衡状態のエントロピー\(S(T, V, N_1, N_2)\)を計算せよ。

\textbf{解答:}

\textbf{混合後の状態}

仕切りを取り除くと、2つの成分が混合し、各成分が全体積\(V\)を占有する。

\textbf{理想気体の独立性(ダルトンの法則):}

理想気体では、各成分は独立に振る舞う。これは、分子間相互作用が無視できるためである。したがって、各成分のエントロピーは、その成分が単独で全体積\(V\)を占有する場合のエントロピーとして計算できる。

\textbf{各成分のエントロピー:}

成分1のエントロピー(全体積\(V\)を占有):
\[S_1(T, V, N_1) = N_1 R\ln(T^c V/N_1) + N_1 S_0\]

成分2のエントロピー(全体積\(V\)を占有):
\[S_2(T, V, N_2) = N_2 R\ln(T^c V/N_2) + N_2 S_0\]

\textbf{全エントロピー:}

2つの成分は独立しているため、全エントロピーは和:

\[S(T, V, N_1, N_2) = S_1 + S_2 = N_1 R\ln(T^c V/N_1) + N_2 R\ln(T^c V/N_2) + (N_1 + N_2) S_0\]

\textbf{重要な観察:}

混合後のエントロピーは、各成分が全体積\(V\)を占有する場合のエントロピーの和である。これは、理想気体の独立性による。

\textbf{答え:}
\[S(T, V, N_1, N_2) = N_1 R\ln\left(\frac{T^c V}{N_1}\right) + N_2 R\ln\left(\frac{T^c V}{N_2}\right) + (N_1 + N_2) S_0\]

\textbf{物理的意味:}

\textbf{混合による変化:}

\begin{itemize}
\tightlist
\item
  混合後、各成分は全体積\(V\)を利用できるようになる
\item
  初期状態では、各成分は\(V_1\)や\(V_2\)(\(V\)より小さい)を占有していた
\item
  混合により、各成分が利用できる体積が増加する
\end{itemize}

\textbf{エントロピーの増加:}

エントロピーは、各成分がより大きな体積を占有できることにより増加する。これは、より多くの微視的状態が利用可能になるためである。

\textbf{理想気体の特徴:}

理想気体では、各成分は独立に振る舞うため、混合による相互作用の変化はない。エントロピー増加は、純粋に体積の利用可能性の増加による。

\begin{center}\rule{0.5\linewidth}{0.5pt}\end{center}

\subsubsection{問3.
エントロピー変化}\label{ux554f3.-ux30a8ux30f3ux30c8ux30edux30d4ux30fcux5909ux5316}

\textbf{問題:}
エントロピー変化\(\Delta S = S(T, V, N_1, N_2) - \{S_1(T, V_1, N_1) + S_2(T, V_2, N_2)\}\)を計算し、正負を判定せよ。

\textbf{解答:}

\textbf{エントロピー変化の計算}

エントロピー変化は、混合後のエントロピーから初期状態のエントロピーを引いたものである:

\[\Delta S = S(T, V, N_1, N_2) - \{S_1(T, V_1, N_1) + S_2(T, V_2, N_2)\}\]

\textbf{ステップ1: 初期状態のエントロピー}

問1より:

\[S_1(T, V_1, N_1) + S_2(T, V_2, N_2) = (N_1 + N_2) R\ln\left(\frac{T^c V}{N_1 + N_2}\right) + (N_1 + N_2) S_0\]

\textbf{ステップ2: 混合後のエントロピー}

問2より:

\[S(T, V, N_1, N_2) = N_1 R\ln\left(\frac{T^c V}{N_1}\right) + N_2 R\ln\left(\frac{T^c V}{N_2}\right) + (N_1 + N_2) S_0\]

\textbf{ステップ3: エントロピー変化の計算}

\[\Delta S = N_1 R\ln\left(\frac{T^c V}{N_1}\right) + N_2 R\ln\left(\frac{T^c V}{N_2}\right) - (N_1 + N_2) R\ln\left(\frac{T^c V}{N_1 + N_2}\right)\]

\(T^c\)は共通なので:

\[\Delta S = R\left[N_1\ln\left(\frac{V}{N_1}\right) + N_2\ln\left(\frac{V}{N_2}\right) - (N_1 + N_2)\ln\left(\frac{V}{N_1 + N_2}\right)\right]\]

対数の性質\(\ln a - \ln b = \ln(a/b)\)を用いて整理:

\[\Delta S = R\left[N_1\ln\left(\frac{V/N_1}{V/(N_1 + N_2)}\right) + N_2\ln\left(\frac{V/N_2}{V/(N_1 + N_2)}\right)\right]\]

\[= R\left[N_1\ln\left(\frac{N_1 + N_2}{N_1}\right) + N_2\ln\left(\frac{N_1 + N_2}{N_2}\right)\right]\]

\[= R\left[N_1\ln\left(1 + \frac{N_2}{N_1}\right) + N_2\ln\left(1 + \frac{N_1}{N_2}\right)\right]\]

\textbf{ステップ4: 正負の判定}

\(N_1 > 0\), \(N_2 > 0\)なので、\(N_2/N_1 > 0\), \(N_1/N_2 > 0\)である。

対数の性質より、\(x > 0\)のとき\(\ln(1+x) > 0\)なので:

\[\ln\left(1 + \frac{N_2}{N_1}\right) > 0, \quad \ln\left(1 + \frac{N_1}{N_2}\right) > 0\]

したがって:

\[\Delta S = R\left[N_1\ln\left(1 + \frac{N_2}{N_1}\right) + N_2\ln\left(1 + \frac{N_1}{N_2}\right)\right] > 0\]

\textbf{重要な結果:}

エントロピー変化は常に正である。これは、混合が自発的に起こることを示している。

\textbf{答え:}
\[\Delta S = R\left[N_1\ln\left(1 + \frac{N_2}{N_1}\right) + N_2\ln\left(1 + \frac{N_1}{N_2}\right)\right] > 0\]

\textbf{物理的意味:}

\textbf{エントロピー増加の理由:}

\[\Delta S = R\left[N_1\ln\left(1 + \frac{N_2}{N_1}\right) + N_2\ln\left(1 + \frac{N_1}{N_2}\right)\right] > 0\]

エントロピーは混合により増加する。これは、以下の理由による:

\begin{enumerate}
\def\labelenumi{\arabic{enumi}.}
\tightlist
\item
  \textbf{体積の利用可能性}:
  各成分が、より大きな体積(全体積\(V\))を利用できるようになる
\item
  \textbf{微視的状態の増加}: より多くの微視的状態が利用可能になる
\item
  \textbf{独立性}:
  理想気体では、各成分は独立に振る舞うため、混合による相互作用の変化はない
\end{enumerate}

\textbf{第二法則との関係:}

エントロピー増加(\(\Delta S > 0\))は、混合が自発的に起こることを示している。これは、熱力学の第二法則の直接的な例である。

\textbf{混合エントロピーの特徴:}

\begin{itemize}
\tightlist
\item
  混合エントロピーは、各成分のモル数に依存する
\item
  等モル混合(\(N_1 = N_2\))の場合、混合エントロピーは最大になる
\item
  理想気体では、混合エントロピーは体積の利用可能性の増加による
\end{itemize}

\textbf{実在気体との違い:}

実在気体では、分子間相互作用により、混合エントロピーは理想気体の場合とは異なる。しかし、希薄な気体では、理想気体の結果が良い近似となる。

\begin{center}\rule{0.5\linewidth}{0.5pt}\end{center}

\subsection{セクションV:
ポアソン分布の導出}\label{ux30bbux30afux30b7ux30e7ux30f3v-ux30ddux30a2ux30bdux30f3ux5206ux5e03ux306eux5c0eux51fa}

\subsubsection{セクションVの概要}\label{ux30bbux30afux30b7ux30e7ux30f3vux306eux6982ux8981}

本セクションでは、\textbf{ポアソン分布}を統計力学の観点から導出する。ポアソン分布は、統計力学において重要な確率分布である。

\textbf{ポアソン分布の重要性:}

\begin{enumerate}
\def\labelenumi{\arabic{enumi}.}
\tightlist
\item
  \textbf{稀な事象の記述}:
  ポアソン分布は、稀な事象(\(p \ll 1\))の発生回数を記述する
\item
  \textbf{粒子数揺らぎ}:
  理想気体の粒子数揺らぎなど、統計力学における重要な現象を記述する
\item
  \textbf{極限分布}: 二項分布の極限として現れる
\end{enumerate}

\textbf{導出の戦略:}

\begin{enumerate}
\def\labelenumi{\arabic{enumi}.}
\tightlist
\item
  二項分布から出発
\item
  母関数を用いて統計量を計算
\item
  極限を取ってポアソン分布を導出
\end{enumerate}

\textbf{本セクションで扱う内容:}

\begin{itemize}
\tightlist
\item
  分子の位置の確率
\item
  二項分布の導出
\item
  母関数と統計量
\item
  ポアソン分布への極限遷移
\end{itemize}

これらの結果は、統計力学における粒子数揺らぎ、拡散現象など、多くの物理現象の理解に重要である。

\begin{center}\rule{0.5\linewidth}{0.5pt}\end{center}

\subsubsection{\texorpdfstring{問1.
1個の分子が体積\(v\)に含まれる確率}{問1. 1個の分子が体積vに含まれる確率}}\label{ux554f1.-1ux500bux306eux5206ux5b50ux304cux4f53ux7a4dvux306bux542bux307eux308cux308bux78baux7387}

\textbf{問題:}
体積\(V\)の容器に\(N\)個の分子がランダムに分布している。体積\(v\)の部分領域に1個の分子が含まれる確率\(p\)を求めよ。

\textbf{解答:}

\textbf{等確率の原理(等重率の原理)}

統計力学の基本原理として、\textbf{等確率の原理}(principle of equal a
priori
probability)がある。これは、平衡状態において、すべての微視的状態が等しい確率で実現するという原理である。

\textbf{分子の位置の確率}

体積\(V\)の容器に\(N\)個の分子がランダムに分布している場合、各分子の位置は完全にランダムであると仮定する。したがって、各分子が体積\(v\)の部分領域に含まれる確率は、体積比に等しい:

\[p = \frac{v}{V}\]

\textbf{確率の性質:}

\begin{itemize}
\tightlist
\item
  \(0 \leq p \leq 1\)(\(v \leq V\)のとき)
\item
  \(p = 1\)(\(v = V\)のとき、すべての分子が領域内)
\item
  \(p \to 0\)(\(v/V \to 0\)のとき、領域が非常に小さい)
\end{itemize}

\textbf{答え:} \[p = \frac{v}{V}\]

\textbf{物理的意味:}

\begin{itemize}
\tightlist
\item
  分子の位置がランダムであるため、特定の領域に存在する確率は、その領域の体積比に比例する
\item
  これは、等確率の原理(等重率の原理)に基づいている
\item
  この仮定は、理想気体の統計力学的記述の基礎となる
\end{itemize}

\begin{center}\rule{0.5\linewidth}{0.5pt}\end{center}

\subsubsection{問2.
平均分子数の予想}\label{ux554f2.-ux5e73ux5747ux5206ux5b50ux6570ux306eux4e88ux60f3}

\textbf{問題:}
体積\(v\)に含まれる分子数\(n\)の平均値\(\langle n \rangle\)を予想せよ。

\textbf{解答:}

\textbf{期待値の計算}

各分子が独立に確率\(p\)で体積\(v\)に含まれるため、体積\(v\)に含まれる分子数\(n\)の期待値(平均値)は:

\[\langle n \rangle = \sum_{i=1}^{N} \text{(分子$i$が領域内にある確率)} = Np\]

ここで、各分子は独立に確率\(p\)で領域内にあるため、期待値は\(Np\)である。

\textbf{数密度の導入}

\(p = v/V\)を代入:

\[\langle n \rangle = Np = N \cdot \frac{v}{V} = \frac{N}{V} \cdot v = \rho v\]

ここで、\(\rho = N/V\)は\textbf{数密度}(number density)である。

\textbf{答え:} \[\langle n \rangle = \rho v = \frac{Nv}{V}\]

\textbf{物理的意味:}

\begin{itemize}
\tightlist
\item
  平均分子数は、数密度と体積の積に等しい
\item
  これは直感的に理解できる結果である
\item
  数密度\(\rho\)は、単位体積あたりの分子数を表す
\item
  この関係式は、統計力学における基本的な関係である
\end{itemize}

\begin{center}\rule{0.5\linewidth}{0.5pt}\end{center}

\subsubsection{問3.
確率分布の係数}\label{ux554f3.-ux78baux7387ux5206ux5e03ux306eux4fc2ux6570}

\textbf{問題:} \(n\)個の分子が体積\(v\)に含まれる確率\(P(n)\)が
\[P(n) = Ap^{\alpha}(1-p)^{\beta}\] の形で与えられるとき、\(A\),
\(\alpha\), \(\beta\)を求めよ。

\textbf{解答:}

\textbf{二項分布の導出}

\(n\)個の分子が体積\(v\)に含まれ、残りの\((N-n)\)個が体積\(v\)の外にある確率を計算する。

\textbf{組み合わせの数:}

\(N\)個の分子から\(n\)個を選ぶ組み合わせの数は:

\[\binom{N}{n} = \frac{N!}{n!(N-n)!}\]

これは二項係数である。

\textbf{各組み合わせの確率:}

選ばれた\(n\)個の分子が領域内にある確率: \(p^n\)
選ばれなかった\((N-n)\)個の分子が領域外にある確率: \((1-p)^{N-n}\)

したがって、各組み合わせの確率は\(p^n(1-p)^{N-n}\)である。

\textbf{全確率:}

\[P(n) = \binom{N}{n} p^n (1-p)^{N-n}\]

これは\textbf{二項分布}(binomial distribution)である。

\textbf{係数の対応:}

与えられた形式\(P(n) = Ap^{\alpha}(1-p)^{\beta}\)と比較:

\[A = \binom{N}{n} = \frac{N!}{n!(N-n)!}, \quad \alpha = n, \quad \beta = N-n\]

\textbf{答え:}
\[A = \binom{N}{n} = \frac{N!}{n!(N-n)!}, \quad \alpha = n, \quad \beta = N-n\]

\textbf{物理的意味:}

\begin{itemize}
\tightlist
\item
  これは二項分布である
\item
  \(n\)個の分子を選ぶ組み合わせの数が\(\binom{N}{n}\)通りあり、それぞれの確率が\(p^n(1-p)^{N-n}\)である
\item
  二項分布は、独立な試行の繰り返しから生じる確率分布である
\end{itemize}

\begin{center}\rule{0.5\linewidth}{0.5pt}\end{center}

\subsubsection{問4.
母関数と統計量}\label{ux554f4.-ux6bcdux95a2ux6570ux3068ux7d71ux8a08ux91cf}

\textbf{問題:} 母関数 \[F(x) = \sum_{n=0}^{N} x^n P(n)\]
を用いて、以下の関係式を示せ:
\[\langle n \rangle = \left.\frac{dF(x)}{dx}\right|_{x=1}\]
\[\langle n^2 \rangle - \langle n \rangle = \left.\frac{d^2F(x)}{dx^2}\right|_{x=1}\]

\textbf{解答:}

\textbf{母関数の定義}

母関数(generating
function)は、確率分布のすべてのモーメント(平均、分散など)を計算するための便利なツールである。

\[F(x) = \sum_{n=0}^{N} x^n P(n)\]

\textbf{ステップ1: 平均値の導出}

母関数を\(x\)で微分:

\[\frac{dF(x)}{dx} = \frac{d}{dx}\sum_{n=0}^{N} x^n P(n) = \sum_{n=0}^{N} n x^{n-1} P(n)\]

\(x = 1\)を代入:

\[\left.\frac{dF(x)}{dx}\right|_{x=1} = \sum_{n=0}^{N} n P(n) = \langle n \rangle\]

\textbf{重要な観察:}

母関数の1階微分を\(x = 1\)で評価すると、平均値が得られる。これは、\(x^n\)の係数が\(n\)に比例するためである。

\textbf{ステップ2: 分散の導出}

母関数を2階微分:

\[\frac{d^2F(x)}{dx^2} = \frac{d^2}{dx^2}\sum_{n=0}^{N} x^n P(n) = \sum_{n=0}^{N} n(n-1) x^{n-2} P(n)\]

\(x = 1\)を代入:

\[\left.\frac{d^2F(x)}{dx^2}\right|_{x=1} = \sum_{n=0}^{N} n(n-1) P(n) = \langle n(n-1) \rangle\]

ここで、\(\langle n(n-1) \rangle = \langle n^2 - n \rangle = \langle n^2 \rangle - \langle n \rangle\)なので:

\[\left.\frac{d^2F(x)}{dx^2}\right|_{x=1} = \langle n^2 \rangle - \langle n \rangle\]

\textbf{重要な観察:}

母関数の2階微分を\(x = 1\)で評価すると、\(\langle n^2 \rangle - \langle n \rangle\)が得られる。これから、分散\(\langle \delta n^2 \rangle = \langle n^2 \rangle - \langle n \rangle^2\)を計算できる。

\textbf{答え:} 示された。

\textbf{物理的意味:}

\begin{itemize}
\tightlist
\item
  母関数は、確率分布のすべてのモーメント(平均、分散など)を計算するための便利なツールである
\item
  微分操作により、統計量を簡単に計算できる
\item
  母関数の方法は、直接的な計算よりも計算が簡単になる場合が多い
\end{itemize}

\begin{center}\rule{0.5\linewidth}{0.5pt}\end{center}

\subsubsection{問5.
母関数の具体形と統計量の計算}\label{ux554f5.-ux6bcdux95a2ux6570ux306eux5177ux4f53ux5f62ux3068ux7d71ux8a08ux91cfux306eux8a08ux7b97}

\textbf{問題:} 問3の結果を母関数の定義に代入し、
\[F(x) = \{(1-p) + px\}^N\]
を示せ。また、これを用いて平均値\(\langle n \rangle\)と分散\(\langle \delta n^2 \rangle\)を計算せよ。

\textbf{解答:}

\textbf{ステップ1: 母関数の計算}

問3の結果\(P(n) = \binom{N}{n} p^n (1-p)^{N-n}\)を母関数の定義に代入:

\[F(x) = \sum_{n=0}^{N} x^n P(n) = \sum_{n=0}^{N} x^n \binom{N}{n} p^n (1-p)^{N-n}\]

\[= \sum_{n=0}^{N} \binom{N}{n} (px)^n (1-p)^{N-n}\]

\textbf{二項定理の適用:}

二項定理より:

\[(a + b)^N = \sum_{n=0}^{N} \binom{N}{n} a^{N-n} b^n\]

ここで、\(a = 1-p\), \(b = px\)とすると:

\[F(x) = \{(1-p) + px\}^N\]

\textbf{重要な結果:}

二項分布の母関数は、閉じた形で表すことができる。これは、統計量の計算を大幅に簡略化する。

\textbf{ステップ2: 平均値の計算}

\[\frac{dF(x)}{dx} = N \{(1-p) + px\}^{N-1} \cdot p\]

\(x = 1\)を代入:
\[\langle n \rangle = N \{(1-p) + p\}^{N-1} \cdot p = N \cdot 1^{N-1} \cdot p = Np = \rho v\]

問2の予想と一致する。

\textbf{ステップ3: 分散の計算}

\[\frac{d^2F(x)}{dx^2} = N(N-1) \{(1-p) + px\}^{N-2} \cdot p^2\]

\(x = 1\)を代入:
\[\langle n^2 \rangle - \langle n \rangle = N(N-1) p^2\]

したがって: \[\langle n^2 \rangle = N(N-1) p^2 + Np\]

分散:
\[\langle \delta n^2 \rangle = \langle n^2 \rangle - \langle n \rangle^2 = N(N-1) p^2 + Np - (Np)^2\]

\[= N^2 p^2 - N p^2 + Np - N^2 p^2 = Np(1-p)\]

\textbf{答え:} \[F(x) = \{(1-p) + px\}^N\]
\[\langle n \rangle = Np = \rho v\]
\[\langle \delta n^2 \rangle = Np(1-p) = \rho v (1-p)\]

\textbf{物理的意味:}

\textbf{平均値と分散:}

\[\langle n \rangle = Np = \rho v\]
\[\langle \delta n^2 \rangle = Np(1-p) = \rho v (1-p)\]

\begin{enumerate}
\def\labelenumi{\arabic{enumi}.}
\item
  \textbf{平均値}:
  数密度と体積の積に等しい。これは直感的に理解できる結果である。
\item
  \textbf{分散}:
  \(Np(1-p)\)で与えられる。これは二項分布の標準的な結果である。
\item
  \textbf{\(p \ll 1\)の極限}:
  \(p \ll 1\)のとき、\(1-p \approx 1\)なので:
  \[\langle \delta n^2 \rangle \approx Np = \langle n \rangle\]

  これは、分散が平均に等しくなることを意味する。これは、ポアソン分布の特徴である。
\end{enumerate}

\textbf{二項分布の特徴:}

\begin{itemize}
\tightlist
\item
  平均値と分散は、\(N\)と\(p\)に依存する
\item
  \(p\)が小さいとき、分散は平均に近づく
\item
  これは、ポアソン分布への極限遷移の準備となる
\end{itemize}

\begin{center}\rule{0.5\linewidth}{0.5pt}\end{center}

\subsubsection{問6.
ポアソン分布への極限}\label{ux554f6.-ux30ddux30a2ux30bdux30f3ux5206ux5e03ux3078ux306eux6975ux9650}

\textbf{問題:}
\(\rho = N/V\)を固定したまま、\(v/V \to 0\)の極限を取るとき、
\[P(n) = \frac{a^n}{n!} e^{-a}\]
(\(a = \langle n \rangle\))となることを示せ。

\textbf{解答:}

\textbf{極限の設定}

\(\rho = N/V\)を固定したまま、\(v/V \to 0\)の極限を考える。これは、以下の条件を意味する:

\begin{itemize}
\tightlist
\item
  \(p = v/V \to 0\)(領域が非常に小さい)
\item
  \(N \to \infty\)(分子数が非常に多い)
\item
  \(Np = \rho v = a\)(一定、平均分子数)
\end{itemize}

\textbf{ステップ1: 母関数の書き換え}

\[F(x) = \{(1-p) + px\}^N = \exp[N\ln\{(1-p) + px\}]\]

\textbf{ステップ2: 対数の展開}

\(\ln\{(1-p) + px\} = \ln\{1 + p(x-1)\}\)を\(p\)について展開(\(p \ll 1\)のとき):

\[\ln\{1 + p(x-1)\} = p(x-1) - \frac{p^2(x-1)^2}{2} + \frac{p^3(x-1)^3}{3} - \cdots\]

\(p \to 0\)の極限では、一次項のみを残す:

\[\ln\{(1-p) + px\} \approx p(x-1)\]

\textbf{ステップ3: 母関数の極限}

\[F(x) \approx \exp[Np(x-1)] = \exp[a(x-1)] = e^{-a} e^{ax}\]

ここで、\(a = Np = \rho v\)(平均分子数)である。

\textbf{ステップ4: 確率分布の導出}

\(e^{ax}\)をテイラー展開:

\[e^{ax} = \sum_{n=0}^{\infty} \frac{(ax)^n}{n!}\]

したがって:

\[F(x) = e^{-a} \sum_{n=0}^{\infty} \frac{(ax)^n}{n!} = \sum_{n=0}^{\infty} \frac{a^n}{n!} e^{-a} x^n\]

一方、母関数の定義より:

\[F(x) = \sum_{n=0}^{\infty} x^n P(n)\]

係数を比較:

\[P(n) = \frac{a^n}{n!} e^{-a}\]

これは\textbf{ポアソン分布}(Poisson distribution)である。

\textbf{答え:} \[P(n) = \frac{a^n}{n!} e^{-a}\]

ここで、\(a = \langle n \rangle = \rho v\)である。

\textbf{物理的意味:}

\textbf{ポアソン分布の特徴:}

\[P(n) = \frac{a^n}{n!} e^{-a}\]

ここで、\(a = \langle n \rangle = \rho v\)(平均分子数)である。

\begin{enumerate}
\def\labelenumi{\arabic{enumi}.}
\tightlist
\item
  \textbf{稀な事象の記述}:
  ポアソン分布は、稀な事象(\(p \ll 1\))の発生回数を記述する

  \begin{itemize}
  \tightlist
  \item
    各分子が領域内にある確率が非常に小さい
  \item
    しかし、分子数が非常に多いため、平均分子数\(a\)は有限
  \end{itemize}
\item
  \textbf{平均値と分散}: 平均値\(a\)のみで分布が決まる

  \begin{itemize}
  \tightlist
  \item
    平均値: \(\langle n \rangle = a\)
  \item
    分散: \(\langle \delta n^2 \rangle = a\)(平均に等しい!)
  \item
    これは、二項分布の\(p \ll 1\)の極限で現れる特徴である
  \end{itemize}
\item
  \textbf{統計力学への応用}:

  \begin{itemize}
  \tightlist
  \item
    分子数が非常に多く、観測領域が小さい場合、分子数の分布はポアソン分布に従う
  \item
    理想気体の粒子数揺らぎを記述する
  \item
    拡散現象、化学反応など、多くの物理現象で現れる
  \end{itemize}
\end{enumerate}

\textbf{ポアソン分布の重要性:}

ポアソン分布は、統計力学における重要な分布である。特に、粒子数揺らぎの記述において中心的な役割を果たす。

\begin{center}\rule{0.5\linewidth}{0.5pt}\end{center}

\subsection{まとめ}\label{ux307eux3068ux3081}

本演習では、以下の重要な概念を学んだ:

\begin{enumerate}
\def\labelenumi{\arabic{enumi}.}
\tightlist
\item
  \textbf{熱力学関係式の導出}:
  偏微分とマクスウェル関係式を用いた熱力学量の関係式の導出
\item
  \textbf{輪ゴムの熱力学}:
  温度依存性を持つバネ定数を持つ系の熱力学的記述
\item
  \textbf{理想気体の統計力学}:
  エントロピー、自由エネルギー、化学ポテンシャルの計算
\item
  \textbf{混合エントロピー}: 気体混合によるエントロピー増加
\item
  \textbf{ポアソン分布}: 二項分布からポアソン分布への極限遷移
\end{enumerate}

これらの概念は、統計力学の基礎を理解する上で重要である。
