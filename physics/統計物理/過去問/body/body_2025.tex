%======================================================================
% 2025年度 統計物理1 本試験(2025年1月29日 10:30--12:00)
%======================================================================
\part{2025年度 本試験}
\setcounter{section}{0}
\renewcommand{\theHsection}{2025.\arabic{section}}
\renewcommand{\theHsubsection}{2025.\arabic{section}.\arabic{subsection}}

\section{問題I:架空の気体(類題:演習5-I, 演習6-III)}

\subsection{問題}

ある架空の気体を考える。この気体の $N$ モルのエントロピーは、内部エネルギー $U$ と体積 $V$ の関数として以下で与えられるとする($R$ は気体定数、$S_0$ は定数):
\begin{equation}
S(U,V) = N R \ln\left[\left(\frac{U}{N}\right)^{3} \left(\frac{V}{N}\right)^2\right] + N S_0.
\end{equation}
以下の問に答えよ。

\begin{enumerate}
\item 系の温度 $T$ を内部エネルギー $U$ を用いて表せ。
\item 断熱自由膨張により、系が $(T,V)$ から $(T',V')$ へ変化したとする($V < V'$)。$T'$ を $T$, $V$, $V'$, $R$, $N$ のいずれかの記号を用いて表せ。
\item この後、系を準静的断熱過程により $(T',V')$ から $(T'',V)$ ともとの体積 $V$ まで圧縮する。一連の操作における、初期状態 $(T,V)$ と終状態 $(T'',V)$ のエントロピー変化を、$T$, $V$, $V'$, $R$, $N$ のいずれかの記号を用いて表せ。
\end{enumerate}

\subsection{解答}

\paragraph{この問題のポイント}

この問題では、エントロピーが $S(U,V)$ の形で与えられている。温度は $1/T = (\partial S/\partial U)_V$ で定義されるので、$S$ を $U$ で微分すれば $T$ が求まる。問2・問3は断熱自由膨張と準静的断熱圧縮の組み合わせで、2024年度問題Iと同様に、$T'=T$(断熱自由膨張で温度不変)が成り立つ。本問の $S(U,V)$ では $\Delta S = 2 N R \ln(V'/V)$ となる。

\paragraph{解き方の流れ}

\begin{enumerate}
\item 問1:$1/T = (\partial S/\partial U)_V$ から $S(U,V)$ を $U$ で微分して $T$ を求める。
\item 問2:断熱自由膨張では $\Delta U = 0$ なので $T' = T$。
\item 問3:準静的断熱では $S$ 一定。$T''$ を断熱関係式で求め、$\Delta S = S_{\mathrm{f}} - S_{\mathrm{i}}$ を計算する。
\end{enumerate}

\paragraph{用語の説明}
\begin{itemize}
\item \textbf{断熱自由膨張}:外界と熱・仕事のやりとりがない条件下で気体が膨張する過程。非準静的で不可逆。$\Delta U = 0$ なので $T' = T$。
\item \textbf{準静的断熱過程}:断熱しながら無限にゆっくり変化させる過程。可逆とみなせ、この間エントロピー $S$ は一定。
\end{itemize}

\paragraph{問1:温度 $T$ を $U$ で表す}

熱力学の関係式 $1/T = (\partial S/\partial U)_V$ を用いる。$S$ は
$S = NR \ln[(U/N)^{3}(V/N)^2] + N S_0$ であり、
\[
\ln[(U/N)^{3}(V/N)^2] = 3\ln(U/N) + 2\ln(V/N)
\]
なので、$U$ で偏微分すると $\frac{\partial}{\partial U}\ln(U/N) = \frac{1}{U}$。よって
\begin{equation}
\frac{\partial S}{\partial U} = N R \cdot 3 \cdot \frac{1}{U}
= \frac{3NR}{U}.
\end{equation}
($N$ モル全体の $S$ を $U$ で微分しているので、$N$ が残る。$U$ は全内部エネルギーである。)したがって
\begin{equation}
\frac{1}{T} = \left(\frac{\partial S}{\partial U}\right)_V = \frac{3NR}{U}
\quad \Rightarrow \quad
\boxed{T = \frac{U}{3NR}}.
\end{equation}
(この気体では $U \propto T$ であり、$U = 3 N R T$ なので $c=3$ の理想気体に相当する。)

\paragraph{なぜ $1/T = \partial S/\partial U$ か(原理的な説明)}

熱力学では、\textbf{温度 $T$ はエントロピー $S$ を内部エネルギー $U$ で偏微分したときの逆数}として定義される:$1/T = (\partial S/\partial U)_V$。体積一定のとき、$dU = T\,dS$ なので、$U$ を増やすには熱を加える必要があり、その「熱の入りやすさ」の逆が $T$ である。$S = NR\ln[(U/N)^{3}(V/N)^2] + NS_0$ を $U$ で微分すると $\partial S/\partial U = 3NR/U$ となり、$1/T = 3NR/U$、すなわち $T = U/(3NR)$ が得られる。$U = 3NRT$ なので、この架空の気体は $c=3$ の理想気体に相当する $U$-$T$ 関係を持つ。

\begin{figure}[H]
\centering
\includegraphics[width=0.8\textwidth]{figures/past2025_ex1_why_T.png}
\caption{問題I:$1/T = (\partial S/\partial U)_V$ の意味。$S(U,V)$ の $U$ による傾きの逆数が温度である。}
\label{fig:past2025_ex1_why_T}
\end{figure}

\paragraph{問2:断熱自由膨張後の温度 $T'$}

断熱自由膨張では $\Delta U = 0$ なので $U' = U$。$T = U/(3NR)$ より $T$ は $U$ のみに依存するので、$U' = U$ なら $T' = T$。したがって
\begin{equation}
\boxed{T' = T}.
\end{equation}

\paragraph{問3:一連の操作のエントロピー変化}

準静的断熱過程 $(T',V') \to (T'',V)$ ではエントロピーは一定なので $dS = 0$。$S(U,V) = NR\ln[(U/N)^{3}(V/N)^2] + NS_0$ において、$S$ が一定なら対数の中身 $(U/N)^{3}(V/N)^2$ が一定でなければならない。$N$ は定数なので、これは $U^3 V^2 = \mathrm{const}$ と同値である(断熱可逆過程の関係式)。

次に、この関係式を温度 $T$ で表す。問1で求めた $U = 3NRT$ を $U^3 V^2 = \mathrm{const}$ に代入すると
\begin{equation}
(3NRT)^3 V^2 = \mathrm{const}.
\end{equation}
$(3NR)^3$ は定数なので、両辺を $(3NR)^3$ で割ると $T^3 V^2 = \mathrm{const}$ が得られる。この関係式を状態 $(T',V')$ と $(T'',V)$ に適用すると $(T')^3 (V')^2 = (T'')^3 V^2$。問2より $T'=T$ なので
\begin{equation}
T'' = T \left(\frac{V'}{V}\right)^{2/3}.
\end{equation}
エントロピー変化は、初期 $S_{\mathrm{i}} = NR\ln[(U/N)^{3}(V/N)^2] + NS_0$、\\
終状態 $S_{\mathrm{f}} = NR\ln[(U''/N)^{3}(V/N)^2] + NS_0$ である。
$U'' = 3 N R T''$ なので
\begin{equation}
\Delta S = S_{\mathrm{f}} - S_{\mathrm{i}}
= N R \ln\frac{(U''/N)^{3}}{(U/N)^{3}}
= N R \cdot 3 \ln\frac{T''}{T}
= 3 N R \ln\frac{T''}{T}.
\end{equation}
$T''/T = (V'/V)^{2/3}$ を代入して
\begin{equation}
\Delta S = 3 N R \cdot \frac{2}{3} \ln\frac{V'}{V}
= 2 N R \ln\frac{V'}{V}.
\end{equation}
よって
\begin{equation}
\boxed{\Delta S = 2 N R \ln\frac{V'}{V}}.
\end{equation}
($V' > V$ なので $\Delta S > 0$。断熱自由膨張は不可逆過程である。)

\paragraph{なぜ一連の操作で $\Delta S = 2 N R \ln(V'/V)$ か(物理的考察)}

問2より断熱自由膨張では $T' = T$ なので内部エネルギーは不変。準静的断熱過程 $(T',V') \to (T'',V)$ ではエントロピーは一定なので、\textbf{全体のエントロピー増加は断熱自由膨張の段階だけで生じる}。$S(U,V) = NR\ln[(U/N)^3(V/N)^2] + NS_0$ の形から、体積 $V$ から $V'$ に不可逆に膨張したときのエントロピー増加は $NR \cdot 2\ln(V'/V) = 2NR\ln(V'/V)$ である($U$ 一定で $V$ のみ変化)。2024年度問題Iと同様に、気体がより広い体積に広がったことによる「配置の無秩序さ」の増加がエントロピー増大の原因である。

\begin{figure}[H]
\centering
\includegraphics[width=0.85\textwidth]{figures/past2025_ex1_path.png}
\caption{問題Iの過程:$(T,V) \to (T',V')$(断熱自由膨張、$T'=T$)、$(T',V') \to (T'',V)$(準静的断熱圧縮)。本問では $T'' = T(V'/V)^{2/3}$、$\Delta S = 2 N R \ln(V'/V)$。}
\label{fig:past2025_ex1_path}
\end{figure}

\begin{figure}[H]
\centering
\includegraphics[width=0.8\textwidth]{figures/past2025_ex1_entropy.png}
\caption{問題I:架空の気体の $S(U,V) = NR\ln[(U/N)^3(V/N)^2] + NS_0$ と状態変化の経路。断熱自由膨張で $\Delta S = 2NR\ln(V'/V)$。}
\label{fig:past2025_ex1}
\end{figure}

%----------------------------------------------------------------------
\section{問題II:2種類の理想気体(類題:2023年度 問題I;演習3-II, 演習5-III--IV)}
%----------------------------------------------------------------------

\subsection{問題}

右図のように、容器が壁で左右に仕切られており、左には温度 $T_L$ の理想気体Aが 1 モル、右には温度 $T_R$ の理想気体Bが 2 モル入っている。左右の部屋の体積はそれぞれ $V_L$, $V_R$ とする。仕切り壁は断熱壁であり、$T_L \neq T_R$ とする。容器全体は外界から孤立している。

体積 $V$ の容器に入った $N$ モルの理想気体のエントロピーは、内部エネルギー $U$ と体積 $V$ の関数として
\begin{equation}
S(U,V) = N R \ln\left[\left(\frac{U}{N}\right)^c \left(\frac{V}{N}\right)\right] + N S_0
\end{equation}
で与えられる($R$ は気体定数、$c$ は物質に依存する定数、$S_0$ は定数)。理想気体Aの $c$ は $c_A = 3$、理想気体Bの $c$ は $c_B = 3/2$ とする。以下の問に答えよ。

\begin{enumerate}
\item 左右を隔てる断熱壁を(気体の移動がないように)透熱壁に置き換えた。十分に時間が経過したあと、左右の気体の温度は等しくなった。この終状態の温度 $T$ を求めよ。
\item 問1の終状態における、容器全体のエントロピーは、初期状態におけるそれと比べてどれだけ変化したか。$R$, $T_L$, $T_R$, $V_L$, $V_R$, $S_0$ のいずれかの記号を用いて示せ。
\item 問2で計算した全エントロピー変化が正であることを証明せよ。
\item 同じ初期状態から出発して、最終的に温度が等しくなるような操作を考える。問1の操作だけではない。操作の途中で、外部から仕切り壁を左右に動かすことができるようにしたり、仕切り壁を断熱壁から透熱壁に交換する(あるいはその逆)ことを自在にできるようにすれば、操作次第で、終状態の温度を問1の $T$ より低くもできるし、高くもできる。ただし、終状態の左右の部屋の体積 $V_L$ と $V_R$ は初期状態と同じであり、また操作を通して左右の部屋の間で気体の移動もできないものとする。この操作で達成することができる最低温度 $T_{\mathrm{min}}$ を求めよ。
\end{enumerate}

\subsection{解答}

\paragraph{この問題のポイント}

左右に\textbf{異なる種類}の理想気体(AとB)が入っており、仕切りは断熱壁なので初期は $T_L \neq T_R$ でもよい。透熱壁に替えると熱が流れ、両室の温度が等しくなる。孤立系なので全内部エネルギーは保存し、$c_A N_A = 3$、$c_B N_B = 3$ で両室の熱容量が等しいため、終温度は $T = (T_L + T_R)/2$ となる。問2ではそのときのエントロピー変化を求める。

\paragraph{解き方の流れ}

\begin{enumerate}
\item 問1:$U_{\mathrm{initial}} = c_A N_A R T_L + c_B N_B R T_R$、$U_{\mathrm{final}} = (c_A N_A + c_B N_B) R T$。$U_{\mathrm{initial}} = U_{\mathrm{final}}$ から $T$ を求める。
\item 問2:初期と終状態の各室のエントロピーを $S(U,V)$ の式で書き、合計の差をとる。
\item 問3:問2の $\Delta S = 3R \ln[(T_L+T_R)^2/(4 T_L T_R)]$ について、相加・相乗平均の不等式 $(T_L+T_R)^2 \ge 4 T_L T_R$(等号は $T_L=T_R$ のときのみ)を用いて $\Delta S > 0$ を示す。
\item 問4:可逆操作では $S_{\mathrm{f}}(T) \ge S_{\mathrm{i}}$ が成り立つ。等号を満たす最小の $T$ が $T_{\mathrm{min}}$ であり、$S_{\mathrm{f}} - S_{\mathrm{i}} = 3R \ln(T^2/(T_L T_R)) = 0$ から $T_{\mathrm{min}} = \sqrt{T_L T_R}$ を導く。
\end{enumerate}

\paragraph{記号の整理}

左室:気体A、$N_A = 1$ モル、$c_A = 3$、初期温度 $T_L$、体積 $V_L$。右室:気体B、$N_B = 2$ モル、$c_B = 3/2$、初期温度 $T_R$、体積 $V_R$。内部エネルギーは $U = c N R T$ で与えられる(理想気体)。$c_A N_A = 3$、$c_B N_B = 3$ なので、両室の熱容量($c N R$)は等しい。

\paragraph{問1:終状態の温度 $T$}

孤立系なので全内部エネルギーは保存する。初期の全内部エネルギーは
\begin{equation}
U_{\mathrm{initial}} = c_A N_A R T_L + c_B N_B R T_R
= 3 \cdot 1 \cdot R \cdot T_L + \frac{3}{2} \cdot 2 \cdot R \cdot T_R
= 3 R T_L + 3 R T_R = 3R(T_L + T_R).
\end{equation}
終状態では左室の温度は $T$、右室の温度も $T$ なので
\begin{equation}
U_{\mathrm{final}} = c_A N_A R T + c_B N_B R T
= 3 R T + 3 R T = 6 R T.
\end{equation}
$U_{\mathrm{initial}} = U_{\mathrm{final}}$ より $3R(T_L + T_R) = 6 R T$、したがって
\begin{equation}
\boxed{T = \frac{T_L + T_R}{2}}.
\end{equation}

\paragraph{なぜこの温度になるか(原理的な説明)}

透熱壁に替えると、高温の左室($T_L$)から低温の右室($T_R$)へ熱が流れる。孤立系なので全内部エネルギーは保存する。左室の内部エネルギーは $U_A = c_A N_A R T = 3 R T$($N_A=1$, $c_A=3$)、右室は $U_B = c_B N_B R T = 3 R T$($N_B=2$, $c_B=3/2$)なので、終状態では両室とも $3RT$ ずつ持つ。初期の全内部エネルギーは $3 R T_L + 3 R T_R = 3R(T_L+T_R)$ なので、保存則 $6 R T = 3R(T_L+T_R)$ より $T = (T_L+T_R)/2$ である。\textbf{両室の熱容量が等しい}($c_A N_A = c_B N_B = 3$)ため、単純な算術平均になる。熱容量が異なる場合は加重平均 $T = (c_A N_A T_L + c_B N_B T_R)/(c_A N_A + c_B N_B)$ となる(2023再問題Iと同様)。

\begin{figure}[H]
\centering
\includegraphics[width=0.8\textwidth]{figures/past2025_ex2_energy_balance.png}
\caption{問題II:透熱壁を通して熱が流れ、全内部エネルギー保存から終状態の温度 $T$ が決まる。}
\label{fig:past2025_ex2_energy}
\end{figure}

\paragraph{問2:エントロピー変化}

与えられた $S(U,V) = N R \ln[(U/N)^c (V/N)] + N S_0$ に、各室の $U = c N R T$ を代入してエントロピーを求める。初期状態の左室は $U_L = c_A N_A R T_L = 3 R T_L$($N_A=1$, $c_A=3$)なので
\begin{equation}
S_L^{\mathrm{i}} = R \ln[(3RT_L)^3 V_L] + S_0.
\end{equation}
右室は $U_R = c_B N_B R T_R = 3 R T_R$、$N_B = 2$ なので $U_R/N_B = 3RT_R/2$。$S(U,V) = N R \ln[(U/N)^c (V/N)] + N S_0$ に代入して $(U_R/N_B)^{c_B} = (3RT_R/2)^{3/2}$、$V_R/N_B = V_R/2$ より
\begin{equation}
S_R^{\mathrm{i}} = 2 R \ln\left[\left(\frac{3RT_R}{2}\right)^{3/2} \frac{V_R}{2}\right] + 2 S_0.
\end{equation}
終状態では左室:温度 $T$、体積 $V_L$、$U_L' = 3 R T$。右室:温度 $T$、体積 $V_R$、$U_R' = 3 R T$。したがって
\begin{equation}
S_L^{\mathrm{f}} = R \ln[(3RT)^{3} V_L] + S_0, \quad
S_R^{\mathrm{f}} = 2 R \ln\left[\left(\frac{3RT}{2}\right)^{3/2} \frac{V_R}{2}\right] + 2 S_0.
\end{equation}
エントロピー変化は
\begin{align}
\Delta S &= (S_L^{\mathrm{f}} - S_L^{\mathrm{i}}) + (S_R^{\mathrm{f}} - S_R^{\mathrm{i}}) \notag \\
&= R \ln\frac{(3RT)^3}{(3RT_L)^3}
+ 2R \ln\frac{(3RT/2)^{3/2}}{(3RT_R/2)^{3/2}}
= 3R \ln\frac{T}{T_L} + 3R \ln\frac{T}{T_R}
= 3R \ln\frac{T^2}{T_L T_R}.
\end{align}
$T = (T_L + T_R)/2$ を代入して
\begin{equation}
\boxed{\Delta S = 3R \ln\frac{T^2}{T_L T_R}
= 3R \ln\frac{(T_L+T_R)^2}{4 T_L T_R}}.
\end{equation}
($T_L \neq T_R$ のとき $(T_L+T_R)^2 > 4 T_L T_R$ なので $\Delta S > 0$。熱が高温から低温に流れる不可逆過程である。)

\paragraph{なぜ $\Delta S > 0$ か(物理的考察)}

$T = (T_L+T_R)/2$ のとき、左室では $T_L \to T$、右室では $T_R \to T$ と変化する。$T_L > T_R$ とすると、左室は熱を失いエントロピーが減り、右室は熱を得てエントロピーが増える。熱力学第二法則により、不可逆過程(熱が高温から低温へ流れる)では全エントロピーは増大するので、右室の増加が左室の減少を上回り $\Delta S > 0$ となる。式 $(T_L+T_R)^2 \ge 4 T_L T_R$(等号は $T_L=T_R$ のとき)から、$T_L \neq T_R$ なら $\Delta S = 3R \ln[(T_L+T_R)^2/(4 T_L T_R)] > 0$ である。

\paragraph{問3:$\Delta S > 0$ の証明}

問2で得た全エントロピー変化は
\begin{equation}
\Delta S = 3R \ln\frac{(T_L+T_R)^2}{4 T_L T_R}.
\end{equation}
$\Delta S > 0$ であるためには、対数の中身が $1$ より大きければよい。すなわち $(T_L+T_R)^2/(4 T_L T_R) > 1$、つまり $(T_L+T_R)^2 > 4 T_L T_R$ を示せばよい。

\textbf{ステップ1:}左辺と右辺の差を計算する。
\begin{equation}
(T_L + T_R)^2 - 4 T_L T_R
= T_L^2 + 2 T_L T_R + T_R^2 - 4 T_L T_R
= T_L^2 - 2 T_L T_R + T_R^2
= (T_L - T_R)^2.
\end{equation}
$(T_L - T_R)^2$ は $T_L$, $T_R$ が実数である限り常に $0$ 以上である。

\textbf{ステップ2:}等号が成り立つ条件を確認する。$(T_L - T_R)^2 = 0$ のとき、かつそのときに限り $T_L = T_R$ である。問題の仮定で $T_L \neq T_R$ としているので、$(T_L - T_R)^2 > 0$ である。

\textbf{ステップ3:}したがって $(T_L+T_R)^2 - 4 T_L T_R > 0$、すなわち $(T_L+T_R)^2 > 4 T_L T_R$ が成り立つ。$T_L$, $T_R > 0$ なので $4 T_L T_R > 0$ であり、
\begin{equation}
\frac{(T_L+T_R)^2}{4 T_L T_R} > 1
\quad \Rightarrow \quad
\ln\frac{(T_L+T_R)^2}{4 T_L T_R} > 0.
\end{equation}
$R > 0$ なので
\begin{equation}
\boxed{\Delta S = 3R \ln\frac{(T_L+T_R)^2}{4 T_L T_R} > 0}.
\end{equation}
以上で、問2で計算した全エントロピー変化が正であることが証明された。

\paragraph{問3の補足(相加・相乗平均との関係)}

不等式 $(T_L+T_R)^2 \ge 4 T_L T_R$ は、正の数 $T_L$, $T_R$ についての\textbf{相加・相乗平均の不等式} $\frac{T_L+T_R}{2} \ge \sqrt{T_L T_R}$(等号は $T_L=T_R$ のときのみ)の両辺を $2$ 乗したものである。問1の終温度 $T = (T_L+T_R)/2$ は算術平均であり、$\sqrt{T_L T_R}$ は幾何平均である。$T_L \neq T_R$ のとき算術平均は幾何平均より大きいので、$(T_L+T_R)/2 > \sqrt{T_L T_R}$ となり、$(T_L+T_R)^2 > 4 T_L T_R$ が従う。

\paragraph{問4:達成できる最低温度 $T_{\mathrm{min}}$}

操作の制約は次の通りである:終状態の左右の体積は $V_L$, $V_R$ のまま(初期と同じ)、左右の部屋の間で気体の移動はない。仕切り壁を動かしたり断熱・透熱を切り替えたりできるので、\textbf{可逆操作}を組み合わせて、終状態で両室の温度を同じ $T$ に揃えることができる。

\textbf{熱力学第二法則}より、可逆操作を含む過程では、系のエントロピーは減少しない。すなわち、終状態のエントロピー $S_{\mathrm{f}}(T)$ は初期状態のエントロピー $S_{\mathrm{i}}$ 以上でなければならない:
\begin{equation}
S_{\mathrm{f}}(T) \ge S_{\mathrm{i}}.
\end{equation}
終状態では左室が $(T, V_L)$、右室が $(T, V_R)$ なので、問2と同様の $S(U,V)$ の式を用いると、両室のエントロピー合計 $S_{\mathrm{f}}(T)$ は $T$ の増加関数である($T$ が大きいほど $S_{\mathrm{f}}$ は大きい)。したがって、$S_{\mathrm{f}}(T) \ge S_{\mathrm{i}}$ を満たす最小の $T$ が、達成可能な\textbf{最低温度} $T_{\mathrm{min}}$ である。そのときは等号 $S_{\mathrm{f}}(T_{\mathrm{min}}) = S_{\mathrm{i}}$ が成り立ち、可逆過程で実現できる。

問2で求めたエントロピー変化の式を、終状態の温度を $T$ として書くと
\begin{equation}
S_{\mathrm{f}}(T) - S_{\mathrm{i}}
= 3R \ln\frac{T}{T_L} + 3R \ln\frac{T}{T_R}
= 3R \ln\frac{T^2}{T_L T_R}.
\end{equation}
(左室:$N_A=1$, $c_A=3$ より $R \ln(T/T_L)^3 = 3R \ln(T/T_L)$。右室:$N_B=2$, $c_B=3/2$ より $2R \cdot (3/2)\ln(T/T_R) = 3R \ln(T/T_R)$。合計で $3R \ln(T^2/(T_L T_R))$ となる。)$S_{\mathrm{f}}(T) = S_{\mathrm{i}}$ のとき
\begin{equation}
3R \ln\frac{T^2}{T_L T_R} = 0
\quad \Rightarrow \quad
\ln\frac{T^2}{T_L T_R} = 0
\quad \Rightarrow \quad
\frac{T^2}{T_L T_R} = 1.
\end{equation}
$T > 0$ なので $T^2 = T_L T_R$、したがって $T = \sqrt{T_L T_R}$。これが $S_{\mathrm{f}} = S_{\mathrm{i}}$ を満たす唯一の正の温度であり、可逆過程で達成可能な最低温度である。よって
\begin{equation}
\boxed{T_{\mathrm{min}} = \sqrt{T_L T_R}}.
\end{equation}
($T_L$ と $T_R$ の\textbf{幾何平均}である。問1の $T = (T_L+T_R)/2$(算術平均)とは異なる。)

\paragraph{なぜ幾何平均が最低温度か(原理的な説明)}

問1の操作(透熱壁に替えて熱平衡に達するだけ)では、孤立系なので全内部エネルギーが保存し、終温度は $T = (T_L+T_R)/2$ に一意に決まる。一方、仕切り壁を動かしたり断熱・透熱を切り替えたりする\textbf{可逆操作}を許すと、外界と仕事のやりとりが生じ得るが、ここでは「終状態の体積は $V_L$, $V_R$ のまま」という制約だけが課されている。可逆過程ではエントロピーを増やさないので、$S_{\mathrm{f}} \ge S_{\mathrm{i}}$ の等号が成り立つような最小の $T$ まで下げられる。その $T$ が $T_{\mathrm{min}} = \sqrt{T_L T_R}$ である。相加・相乗平均の不等式より $\sqrt{T_L T_R} \le (T_L+T_R)/2$(等号は $T_L=T_R$ のときのみ)なので、$T_{\mathrm{min}} \le T$(問1の温度)であり、$T_L \neq T_R$ のときは $T_{\mathrm{min}} < T$ となる。つまり、可逆操作を駆使すれば、問1では到達しない「より低い終温度」を実現できる。

$T_{\mathrm{min}}$ を達成する操作の例:左室を断熱壁で囲い準静的断熱膨張させて冷却し、右室を準静的断熱圧縮して加熱するなど、両室のエントロピーを変えずに可逆的に温度を揃える。一方を断熱膨張・他方を断熱圧縮する組み合わせで、最終的に体積を $V_L$, $V_R$ に戻したときの共通温度が $T_{\mathrm{min}}$ になるようにできる。

\begin{figure}[H]
\centering
\includegraphics[width=0.75\textwidth]{figures/past2025_ex2_setup.png}
\caption{問題IIの設定:左に気体A 1モル($T_L,V_L$)、右に気体B 2モル($T_R,V_R$)。仕切りは断熱壁。}
\label{fig:past2025_ex2}
\end{figure}

\begin{figure}[H]
\centering
\includegraphics[width=0.8\textwidth]{figures/past2025_ex2_DeltaS_Tmin.png}
\caption{問題II 問3・問4:左は $\Delta S > 0$ の証明に用いる不等式 $(T_L+T_R)^2 > 4 T_L T_R$($T_L \neq T_R$ のとき)。右は達成できる終温度の範囲。$T_{\mathrm{min}} = \sqrt{T_L T_R}$(幾何平均)$\le T \le$ 問1の $T = (T_L+T_R)/2$(算術平均)。}
\label{fig:past2025_ex2_DeltaS_Tmin}
\end{figure}

%----------------------------------------------------------------------
\section{問題III:変な気体(類題:2023年度 問題II、2024年度 問題III;演習6-I, 演習4-III)}
%----------------------------------------------------------------------

\subsection{問題}

体積 $V$ の中に、ある気体が入っている。この系のエントロピーが $S(T,V) = \sigma T^3 V$ で与えられるとする($\sigma$ は正の定数)。以下の問に答えよ。

\begin{enumerate}
\item この系の内部エネルギー $E(T,V)$ を、$\sigma$, $T$, $V$ を用いて表わせ。ただし $E(T=0,V)=0$ とする。
\item この系の圧力 $p(T,V)$ を、$\sigma$, $T$, $V$ を用いて表わせ。
\item この気体が温度 $T$ の熱源と接して熱平衡にあり、体積 $V$ は一定とする。このときの揺らぎの分散 $\langle \delta E^2 \rangle$ を、ボルツマン定数 $k_B$、$\sigma$、$T$、$V$ を用いて表わせ。ここで $\delta E = E - \langle E \rangle$ である。
\end{enumerate}

ヒント:定積比熱は $C_V = (\partial E/\partial T)_V = T(\partial S/\partial T)_V$ とも書ける。$S$ を $T,V$ の関数とみなすと $T\,dS = C_V\,dT + \bigl((\partial E/\partial V)_T + p\bigr)\,dV$ と書ける。

\subsection{解答}

【類題】2023年度 問題II、2024年度 問題III と同一内容です。演習問題解説では演習6-I(熱力学の関係式)、演習4-III(内部エネルギーの方程式)が関連します。

\paragraph{この問題のポイント}

エントロピーが $S(T,V) = \sigma T^3 V$ という「変な」形で与えられた気体を扱う。問1はヒントの $C_V = T(\partial S/\partial T)_V$ から定積比熱を求め、$E(T=0,V)=0$ として $E$ を積分で求める。問2は $T\,dS = dE + p\,dV$ から圧力 $p$ を求める。問3は「熱源と接触した系」なので\textbf{カノニカル分布}が成り立ち、統計力学の公式 $\langle \delta E^2 \rangle = k_B T^2 C_V$ を用いる(用語集・2023年度 問題II 問3を参照)。詳細な導出・図は2023年度 問題IIを参照。

\paragraph{なぜこのように解くか}
問1:熱力学では体積一定のとき $dE = T\,dS$ なので $(\partial E/\partial T)_V = T(\partial S/\partial T)_V = C_V$。$S(T,V)$ を $T$ で微分して $C_V$ を得て、$E(T=0,V)=0$ で積分する。問2:$T\,dS = dE + p\,dV$ で $T$ 一定のとき $T(\partial S/\partial V)_T = (\partial E/\partial V)_T + p$ なので、$S$ と $E$ の $V$ 微分から $p$ を求める。問3:熱源と接触しているのでカノニカル分布が適用され、エネルギーの分散は $\langle \delta E^2 \rangle = k_B T^2 C_V$ で与えられる。

\paragraph{問1:$E(T,V)$}

ヒントの「定積比熱は $C_V = T(\partial S/\partial T)_V$ とも書ける」を用いる。熱力学の $T\,dS = dE + p\,dV$ で体積一定とすると $dE = T\,dS$ なので $(\partial E/\partial T)_V = T(\partial S/\partial T)_V = C_V$ となる。$S = \sigma T^3 V$ より $(\partial S/\partial T)_V = 3\sigma T^2 V$ なので $C_V = T \cdot 3\sigma T^2 V = 3\sigma T^3 V$。$E(T=0,V)=0$ として $E = \int_0^T C_V\,dT'$ より
\begin{equation}
\boxed{E(T,V) = \frac{3}{4}\sigma T^4 V}.
\end{equation}

\paragraph{問2:$p(T,V)$}

ヒントの $T\,dS = C_V\,dT + \bigl((\partial E/\partial V)_T + p\bigr)\,dV$ で、$T$ 一定のとき $dT=0$ なので $T(\partial S/\partial V)_T = (\partial E/\partial V)_T + p$ が成り立つ。$(\partial S/\partial V)_T = \sigma T^3$、$(\partial E/\partial V)_T = (3/4)\sigma T^4$ を代入して $p = \sigma T^4 - (3/4)\sigma T^4 = (1/4)\sigma T^4$。よって
\begin{equation}
\boxed{p(T,V) = \frac{1}{4}\sigma T^4}.
\end{equation}
($p$ は $V$ に依存しない。)

\paragraph{問3:$\langle \delta E^2 \rangle$}

「温度 $T$ の熱源と接して熱平衡にあり」とあるので\textbf{カノニカル分布}が適用される(用語集「カノニカル分布」参照)。このとき系のエネルギーは熱源とのやりとりで揺らぎ、その分散は $\langle \delta E^2 \rangle = k_B T^2 C_V$ で与えられる(なぜこの公式でよいかは2023年度 問題II 問3・本試験問題III 問6を参照)。問1で $C_V = 3\sigma T^3 V$ を求めているので、これを代入して
\begin{equation}
\boxed{\langle \delta E^2 \rangle = 3\sigma k_B T^5 V}.
\end{equation}

詳細な導出・物理的考察は2023年度 問題II(本冊)を参照してください。$E(T,V)$ と $p(T,V)$ の概念図は図\ref{fig:past2023_ex2_E_p}(2023年度 問題II)を参照。

%----------------------------------------------------------------------
\section{問題IV:2準位系の統計力学(カノニカルとミクロカノニカル)(類題:演習7-I, 演習7-III, 演習7-IV)}
%----------------------------------------------------------------------

\subsection{問題}

一般に、$N$ 個の独立な調和振動子を量子的に扱うと、各振動子のエネルギーは $E_\ell = \varepsilon m_\ell$($m_\ell = 0, 1, 2, \ldots$)と離散的になる。この問題では話を簡単にするため、\textbf{各振動子がとり得るエネルギーを最初の2つ、すなわち $E_0 = 0$ と $E_1 = \varepsilon$ の2つの値だけに限定した特別な場合(2準位系)}を考える。

\textbf{理論的な背景:}
\begin{itemize}
\item \textbf{系が孤立している場合}:系が量子状態 $i$ にある確率は、全エネルギー $E_{\mathrm{tot}}$ で決まる状態数 $W_N(E_{\mathrm{tot}})$ を用いて $P(i) = 1/W_N(E_{\mathrm{tot}})$ で与えられる。系のエントロピーは $S = k_B \ln W_N(E)$ と表される。
\item \textbf{系が温度 $T$ の大きな熱源と接している場合}:系が量子状態 $i$ にある確率は、分配関数 $Z_N(T)$ を用いて $P(i) = e^{-\beta E_i}/Z_N(T)$ で与えられる。ここで $E_i$ は状態 $i$ のエネルギー、$\beta = 1/(k_B T)$ とする。
\end{itemize}

\textbf{第1部:系が温度 $T$ の大きな熱源と接している場合(カノニカル分布)}
\begin{enumerate}
\item $N=1$ のとき、1粒子分配関数 $Z_1(T)$ を計算せよ。
\item $N=1$ のとき、エネルギーの平均値 $\langle E \rangle$ を温度の関数として求めよ。
\item $N$ 個のときの分配関数 $Z_N$ を計算せよ。
\item $Z_N$ を用いて、$N$ 個の系のエネルギーの平均値 $\langle E \rangle$ を求めよ。
\end{enumerate}

\textbf{第2部:系が孤立している場合(ミクロカノニカル分布)}

$N_0$ 個の振動子がエネルギー $E_0=0$、$N_1$ 個がエネルギー $E_1=\varepsilon$ にあるとする。全振動子数は $N_{\mathrm{tot}} = N_0 + N_1$、全エネルギーは $E_{\mathrm{tot}} = N_1 \varepsilon$ である。
\begin{enumerate}
\setcounter{enumi}{4}
\item このときの微視的状態の数(状態数)$W_N(E_{\mathrm{tot}})$ を $N_0$ と $N_1$ を用いて表せ。
\item 問5の結果から、系のエントロピー $S$ を $N_{\mathrm{tot}}$ と $E_{\mathrm{tot}}$ を用いて表せ。$N_{\mathrm{tot}}$, $N_0$, $N_1$ は十分大きいとし、必要ならスターリングの公式 $\ln N! \approx N\ln N - N$ を用いてよい。
\item 問6の結果とエントロピーの性質を用いて、全エネルギー $E_{\mathrm{tot}}$ を温度 $T$ の関数として表せ。
\end{enumerate}

\subsection{解答}

\paragraph{この問題のポイント}

各粒子が\textbf{2つのエネルギー準位} $E_0=0$ と $E_1=\varepsilon$ のみを取る系を扱う。第1部は熱源と接触した系(カノニカル分布)で分配関数 $Z$ と平均エネルギー $\langle E \rangle$ を求める。第2部は孤立系(ミクロカノニカル分布)で状態数 $W_N$、エントロピー $S = k_B \ln W_N$、そして $1/T = (\partial S/\partial E)_{N}$ から $E_{\mathrm{tot}}(T)$ を導く。問7で、ミクロカノニカルから得た $E_{\mathrm{tot}}(T)$ がカノニカル分布の $\langle E \rangle$ と一致することを確認できる。

\paragraph{解き方の流れ}

\begin{enumerate}
\item 問1:1粒子が2状態なので $Z_1 = e^{-\beta\cdot 0} + e^{-\beta\varepsilon} = 1 + e^{-\beta\varepsilon}$。
\item 問2:$\langle E \rangle = -\partial\ln Z_1/\partial\beta$ または $\langle E \rangle = (0\cdot 1 + \varepsilon\cdot e^{-\beta\varepsilon})/Z_1$。
\item 問3:$N$ 個が独立なので $Z_N = (Z_1)^N$。
\item 問4:$\langle E \rangle = N\times$(1粒子の平均)。
\item 問5:全 $N_0+N_1$ 個のうち $N_1$ 個を「エネルギー $\varepsilon$ を取る粒子」に割り当てる組み合わせ。解答は $N_0$, $N_1$ のみで $W_N = (N_0+N_1)!/(N_0!\,N_1!)$ と書く。
\item 問6:$S = k_B \ln W_N$。スターリングの公式で $\ln W_N$ を $N_0$, $N_1$(または $N_{\mathrm{tot}}$, $E_{\mathrm{tot}}$)で表す。
\item 問7:熱力学の関係 $1/T = (\partial S/\partial E_{\mathrm{tot}})_{N_{\mathrm{tot}}}$ を用い、$E_{\mathrm{tot}}$ について解く。
\end{enumerate}

\paragraph{問1:$Z_1(T)$}

\textbf{分配関数の定義}:カノニカル分布では、分配関数 $Z$ は「すべての状態 $i$ についてボルツマン因子 $e^{-\beta E_i}$ を足し合わせたもの」である:$Z = \sum_i e^{-\beta E_i}$。これにより確率 $P(i) = e^{-\beta E_i}/Z$ が規格化される($\sum_i P(i) = 1$)。

1個の振動子は $E_0=0$ または $E_1=\varepsilon$ の2状態のみを取るので、状態の和は2項のみである:
\begin{equation}
Z_1 = \sum_{i \in \{0,1\}} e^{-\beta E_i}
= e^{-\beta E_0} + e^{-\beta E_1}
= e^{-\beta \cdot 0} + e^{-\beta \varepsilon}
= e^0 + e^{-\beta\varepsilon}
= \boxed{1 + e^{-\beta\varepsilon}}.
\end{equation}
確率は $P(E_0) = 1/Z_1$、$P(E_1) = e^{-\beta\varepsilon}/Z_1$ であり、$P(E_0) + P(E_1) = 1$ が成り立つ。

\paragraph{問2:$N=1$ のときの $\langle E \rangle$}

\textbf{方法1(定義から直接)}:平均エネルギーは $\langle E \rangle = \sum_i E_i P(i) = (1/Z_1)\sum_i E_i e^{-\beta E_i}$ である。2状態では $E_0=0$、$E_1=\varepsilon$ なので
\begin{equation}
\langle E \rangle = \frac{0 \cdot e^{-\beta \cdot 0} + \varepsilon \cdot e^{-\beta\varepsilon}}{Z_1}
= \frac{\varepsilon\,e^{-\beta\varepsilon}}{1 + e^{-\beta\varepsilon}}.
\end{equation}

\textbf{方法2(分配関数から)}:一般的に $\langle E \rangle = -\partial\ln Z/\partial\beta$ が成り立つ(導出:$\ln Z = \ln\sum_i e^{-\beta E_i}$ を $\beta$ で微分すると $\frac{1}{Z}\sum_i (-E_i)e^{-\beta E_i} = -\langle E \rangle$)。$Z_1 = 1 + e^{-\beta\varepsilon}$ より
\begin{equation}
\ln Z_1 = \ln(1 + e^{-\beta\varepsilon}), \quad
\frac{\partial \ln Z_1}{\partial \beta} = \frac{- \varepsilon\,e^{-\beta\varepsilon}}{1 + e^{-\beta\varepsilon}}.
\end{equation}
したがって $\langle E \rangle = -\partial\ln Z_1/\partial\beta = \varepsilon\,e^{-\beta\varepsilon}/(1+e^{-\beta\varepsilon})$。分子・分母に $e^{\beta\varepsilon}$ をかけると $\varepsilon/(e^{\beta\varepsilon}+1)$ となる。よって
\begin{equation}
\boxed{\langle E \rangle = \frac{\varepsilon}{e^{\beta\varepsilon}+1} = \frac{\varepsilon\,e^{-\beta\varepsilon}}{1+e^{-\beta\varepsilon}}}.
\end{equation}

\paragraph{問3:$Z_N$}

$N$ 個の振動子が\textbf{互いに独立}であるとき、全系の状態は各粒子の状態の組み合わせで表される。粒子 $j$ が状態 $i_j \in \{0,1\}$ にあるとき、全エネルギーは $E_{i_1} + E_{i_2} + \cdots + E_{i_N}$ である。分配関数は全状態について $e^{-\beta E}$ を足し合わせるので
\begin{equation}
Z_N = \sum_{i_1,i_2,\ldots,i_N} e^{-\beta(E_{i_1}+E_{i_2}+\cdots+E_{i_N})}
= \sum_{i_1} e^{-\beta E_{i_1}} \sum_{i_2} e^{-\beta E_{i_2}} \cdots \sum_{i_N} e^{-\beta E_{i_N}}
= (Z_1)^N.
\end{equation}
(各 $i_j$ について和は独立に行えるので、$N$ 個の1粒子分配関数の積になる。)したがって
\begin{equation}
\boxed{Z_N = (Z_1)^N = (1 + e^{-\beta\varepsilon})^N}.
\end{equation}

\paragraph{問4:$Z_N$ を用いた $\langle E \rangle$}

$\langle E \rangle = -\partial\ln Z_N/\partial\beta$ を用いる。$Z_N = (Z_1)^N$ より
\begin{equation}
\ln Z_N = N \ln Z_1, \quad
\frac{\partial \ln Z_N}{\partial \beta} = N \,\frac{\partial \ln Z_1}{\partial \beta}.
\end{equation}
したがって
\begin{equation}
\langle E \rangle = -\frac{\partial \ln Z_N}{\partial \beta}
= N\left(-\frac{\partial \ln Z_1}{\partial \beta}\right)
= N \langle E \rangle_1
= N\,\frac{\varepsilon}{e^{\beta\varepsilon}+1}.
\end{equation}
($\langle E \rangle_1$ は問2で求めた1粒子の平均エネルギーである。)よって
\begin{equation}
\boxed{\langle E \rangle = N\,\frac{\varepsilon}{e^{\beta\varepsilon}+1}}.
\end{equation}

\paragraph{問5:状態数 $W_N(E_{\mathrm{tot}})$}

問題文の指示どおり、$N_0$ と $N_1$ のみを用いて表す。全振動子数は $N_0 + N_1$ であり、このうち $N_1$ 個がエネルギー $\varepsilon$ を、$N_0$ 個が $E_0=0$ を取る。

\textbf{微視的状態の数}:各粒子が $E_0=0$ か $E_1=\varepsilon$ のどちらかを取る。全 $N_0+N_1$ 個の粒子のうち、\textbf{どの $N_1$ 個}がエネルギー $\varepsilon$ を取るかを決めると、残り $N_0$ 個は自動的に $E_0=0$ を取る。したがって、状態数は「$N_0+N_1$ 個のうち $N_1$ 個を選ぶ組み合わせの数」に等しい:
\begin{equation}
W_N = \binom{N_0+N_1}{N_1}
= \frac{(N_0+N_1)!}{N_1!\,(N_0+N_1-N_1)!}
= \frac{(N_0+N_1)!}{N_0!\,N_1!}.
\end{equation}
(具体例:$N_0=2$、$N_1=1$ のとき、3個のうち1個を選ぶので $W_N = \binom{3}{1} = 3$。)したがって、$N_0$ と $N_1$ のみを用いた答えは
\begin{equation}
\boxed{W_N(E_{\mathrm{tot}}) = \frac{(N_0+N_1)!}{N_0!\,N_1!}}.
\end{equation}

\begin{figure}[H]
\centering
\includegraphics[width=0.75\textwidth]{figures/past2025_ex4_combination.png}
\caption{問題IV 問5:$N_{\mathrm{tot}}$ 個のうち $N_1$ 個がエネルギー $\varepsilon$ を取る組み合わせ。図は $N_{\mathrm{tot}}=5$、$N_1=2$ の例($W = \binom{5}{2}=10$ 通り)。}
\label{fig:past2025_ex4_combination}
\end{figure}

\paragraph{問6:エントロピー $S(N_{\mathrm{tot}}, E_{\mathrm{tot}})$}

ボルツマンの関係式 $S = k_B \ln W_N(E_{\mathrm{tot}})$ を用いる。$N_1 = E_{\mathrm{tot}}/\varepsilon$、$N_0 = N_{\mathrm{tot}} - N_1$ とする。

スターリングの公式 $\ln N! \approx N\ln N - N$($N$ が大きいとき)を各階乗に適用する:
\begin{align}
\ln(N_{\mathrm{tot}}!) &\approx N_{\mathrm{tot}}\ln N_{\mathrm{tot}} - N_{\mathrm{tot}}, \\
\ln(N_0!) &\approx N_0\ln N_0 - N_0, \\
\ln(N_1!) &\approx N_1\ln N_1 - N_1.
\end{align}
$W_N = N_{\mathrm{tot}}!/(N_0!\,N_1!)$ より $\ln W_N = \ln(N_{\mathrm{tot}}!) - \ln(N_0!) - \ln(N_1!)$ なので
\begin{align}
\ln W_N &\approx \bigl(N_{\mathrm{tot}}\ln N_{\mathrm{tot}} - N_{\mathrm{tot}}\bigr)
- \bigl(N_0\ln N_0 - N_0\bigr) - \bigl(N_1\ln N_1 - N_1\bigr) \notag \\
&= N_{\mathrm{tot}}\ln N_{\mathrm{tot}} - N_0\ln N_0 - N_1\ln N_1
- (N_{\mathrm{tot}} - N_0 - N_1).
\end{align}
ここで $N_0 + N_1 = N_{\mathrm{tot}}$ なので $N_{\mathrm{tot}} - N_0 - N_1 = 0$ となり、線形項は打ち消し合う。したがって
\begin{equation}
\ln W_N = N_{\mathrm{tot}}\ln N_{\mathrm{tot}} - N_0\ln N_0 - N_1\ln N_1.
\end{equation}
$S = k_B \ln W_N$ より
\begin{equation}
\boxed{S = k_B \bigl( N_{\mathrm{tot}}\ln N_{\mathrm{tot}} - N_0\ln N_0 - N_1\ln N_1 \bigr)},
\end{equation}
ただし $N_0 = N_{\mathrm{tot}} - E_{\mathrm{tot}}/\varepsilon$、$N_1 = E_{\mathrm{tot}}/\varepsilon$ である。問題の指示どおり $N_{\mathrm{tot}}$ と $E_{\mathrm{tot}}$ のみで書けば、$n_1 = E_{\mathrm{tot}}/\varepsilon$ として
\begin{equation}
S = k_B \left[ N_{\mathrm{tot}}\ln N_{\mathrm{tot}}
- (N_{\mathrm{tot}}-n_1)\ln(N_{\mathrm{tot}}-n_1) - n_1\ln n_1 \right].
\end{equation}

\paragraph{問7:$E_{\mathrm{tot}}$ を温度 $T$ の関数として表す}

\textbf{なぜ $1/T = \partial S/\partial E_{\mathrm{tot}}$ を使うか}:ミクロカノニカル分布では、エントロピー $S(E_{\mathrm{tot}})$ が与えられたとき、熱力学の定義で温度は $1/T = (\partial S/\partial E_{\mathrm{tot}})_{N_{\mathrm{tot}}}$ である。逆に、温度 $T$ が指定されたとき、この関係式を $E_{\mathrm{tot}}$ について解けば $E_{\mathrm{tot}}(T)$ が得られる。

\textbf{ステップ1:}問6の $S = k_B(N_{\mathrm{tot}}\ln N_{\mathrm{tot}} - N_0\ln N_0 - N_1\ln N_1)$ を $E_{\mathrm{tot}}$ で微分する。$N_{\mathrm{tot}}$ は定数であり、$N_0 = N_{\mathrm{tot}} - E_{\mathrm{tot}}/\varepsilon$、$N_1 = E_{\mathrm{tot}}/\varepsilon$ より
\begin{equation}
\frac{\partial N_0}{\partial E_{\mathrm{tot}}} = -\frac{1}{\varepsilon}, \quad
\frac{\partial N_1}{\partial E_{\mathrm{tot}}} = \frac{1}{\varepsilon}.
\end{equation}
$N_{\mathrm{tot}}\ln N_{\mathrm{tot}}$ は $E_{\mathrm{tot}}$ に依存しないので微分は $0$。$-N_0\ln N_0$ の微分は、合成関数の公式 $\frac{\partial}{\partial E}(-N_0\ln N_0) = -(\ln N_0 + 1)\frac{\partial N_0}{\partial E}$ より $-(\ln N_0 + 1)(-1/\varepsilon) = (\ln N_0 + 1)/\varepsilon$。同様に $-N_1\ln N_1$ の微分は $-(\ln N_1 + 1)(1/\varepsilon)$。したがって
\begin{equation}
\frac{\partial S}{\partial E_{\mathrm{tot}}} = k_B\left( \frac{\ln N_0 + 1}{\varepsilon} - \frac{\ln N_1 + 1}{\varepsilon} \right)
= \frac{k_B}{\varepsilon}(\ln N_0 - \ln N_1)
= \frac{k_B}{\varepsilon} \ln\frac{N_0}{N_1}.
\end{equation}
$1/T = \partial S/\partial E_{\mathrm{tot}}$ より
\begin{equation}
\frac{1}{T} = \frac{k_B}{\varepsilon} \ln\frac{N_0}{N_1}
\quad \Rightarrow \quad
\ln\frac{N_0}{N_1} = \frac{\varepsilon}{k_B T} = \beta\varepsilon
\quad \Rightarrow \quad
\frac{N_0}{N_1} = e^{\beta\varepsilon}.
\end{equation}

\textbf{ステップ2:}$N_0/N_1 = e^{\beta\varepsilon}$ を $N_1$ について解く。$N_0 = N_{\mathrm{tot}} - N_1$ なので
\begin{equation}
\frac{N_{\mathrm{tot}} - N_1}{N_1} = e^{\beta\varepsilon}
\quad \Rightarrow \quad
\frac{N_{\mathrm{tot}}}{N_1} - 1 = e^{\beta\varepsilon}
\quad \Rightarrow \quad
\frac{N_{\mathrm{tot}}}{N_1} = 1 + e^{\beta\varepsilon}.
\end{equation}
よって $N_1 = N_{\mathrm{tot}}/(1 + e^{\beta\varepsilon})$。$E_{\mathrm{tot}} = N_1 \varepsilon$ なので
\begin{equation}
E_{\mathrm{tot}} = N_{\mathrm{tot}}\,\frac{\varepsilon}{1+e^{\beta\varepsilon}}.
\end{equation}
$\beta = 1/(k_B T)$ より、$e^{\beta\varepsilon} = e^{\varepsilon/(k_B T)}$ である。問題の指示どおり温度 $T$ の関数として表すと
\begin{equation}
\boxed{E_{\mathrm{tot}} = N_{\mathrm{tot}}\,\frac{\varepsilon}{1 + e^{\varepsilon/(k_B T)}}}.
\end{equation}
これは問4のカノニカル分布による平均エネルギー $\langle E \rangle$ と一致する。すなわち、孤立系でエントロピーを最大化するエネルギーと、熱源と接触した系の平均エネルギーが一致する(熱力学極限での等価性)。

\paragraph{なぜ $1/T = \partial S/\partial E$ か(問7の補足)}

熱力学では、断熱壁で囲まれた系のエントロピー $S(E,V)$ に対して $1/T = (\partial S/\partial E)_V$ が成り立つ。問6で求めた $S$ を $E_{\mathrm{tot}}$ で微分した逆数が温度 $T$ であり、その $T$ と $E_{\mathrm{tot}}$ の関係を式で表したものが問7の答えである。

\begin{figure}[H]
\centering
\includegraphics[width=0.6\textwidth]{figures/past2025_ex4_two_level.png}
\caption{問題IV:2準位系。各粒子は $E_0=0$ または $E_1=\varepsilon$ のどちらかの状態のみを取る。}
\label{fig:past2025_ex4_two_level}
\end{figure}
