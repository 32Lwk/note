%======================================================================
% 2024年度 統計物理1 本試験(2024年1月24日 10:30--12:00)
%======================================================================
\part{2024年度 本試験}
\setcounter{section}{0}

\section{問題I:断熱自由膨張とエントロピー変化(類題:演習5-II)}

\subsection{問題}

$N$ モルの理想気体のエントロピー $S(T,V)$ と内部エネルギー $U(T,V)$ は以下で与えられる($T$:温度、$V$:体積、$R$:気体定数、$c$, $S_0$:定数):
\begin{align}
S(T,V) &= NR \ln\left(\frac{T^c V}{N}\right) + N S_0, \\
U(T,V) &= c N R T.
\end{align}

\begin{enumerate}
\item 系が断熱自由膨張により状態 $(T,V)$ から $(T',V')$ へ変化したとする($V < V'$)。$T'$ を $T$, $V$, $V'$, $c$, $R$, $N$ のいずれかの記号を用いて表せ。
\item 問1の後、系を準静的断熱過程により $(T',V')$ から $(T'',V)$ ともとの体積 $V$ まで圧縮する。一連の操作
\[
(T,V) \xrightarrow{\text{断熱自由膨張}} (T',V')
\xrightarrow{\text{準静的断熱過程}} (T'',V)
\]
における、初期状態 $(T,V)$ と終状態 $(T'',V)$ のエントロピー変化を、$T$, $V$, $V'$, $c$, $R$, $N$ のいずれかの記号を用いて表せ。
\end{enumerate}

\subsection{解答}

\paragraph{この問題のポイント(初学者向け)}

断熱自由膨張では\textbf{外界と熱のやりとりも仕事のやりとりもない}ため、内部エネルギー $U$ は変わらない。理想気体では $U$ は $T$ のみに依存するので、$T' = T$ となる。一方、エントロピーは不可逆過程で増大するので、一連の操作の後には $\Delta S > 0$ となる。問2では「準静的断熱過程では $S$ 一定」を使い、$\Delta S$ を断熱自由膨張の部分だけで表す。

\paragraph{解き方の流れ}

\begin{enumerate}
\item 問1:断熱自由膨張では $Q=0$, $W=0$ なので $\Delta U = 0$。理想気体では $U = c N R T$ で $U$ は $V$ に依存しないので、$T' = T$。
\item 問2:準静的断熱過程では $S$ 一定なので、$\Delta S$ は初期 $(T,V)$ と終状態 $(T'',V)$ のエントロピー差。断熱過程の関係式で $T''$ を $T,V,V'$ で表し、$\Delta S$ に代入する。
\end{enumerate}

\paragraph{用語の説明}
\begin{itemize}
\item \textbf{断熱自由膨張}:外界と熱のやりとりがなく(断熱)、外から仕事もされない条件下で気体が膨張する過程。非準静的で不可逆。熱力学第一法則 $dU = \delta Q - \delta W$ より $Q=0$, $W=0$ なら $\Delta U=0$。
\item \textbf{準静的断熱過程}:断熱しながら無限にゆっくり変化させる過程。可逆とみなせる。可逆断熱過程ではエントロピーは一定($dS = \delta Q_{\mathrm{rev}}/T = 0$)。
\item \textbf{理想気体の断熱過程}:$T^c V = \mathrm{const}$ が成り立つ($c$ は $U = c N R T$ の係数)。これは $dU = -p\,dV$ と $p = NRT/V$ から導かれる。
\end{itemize}

\paragraph{問1:断熱自由膨張後の温度 $T'$}

断熱自由膨張では $Q=0$, $W=0$ なので $\Delta U = 0$。理想気体では $U = c N R T$ であり、$U$ は $T$ のみに依存するので、$U$ が変わらなければ $T$ も変わらない。したがって
\begin{equation}
\boxed{T' = T}.
\end{equation}

\paragraph{なぜ温度が変わらないか(原理的な説明)}

断熱自由膨張では、外界と熱のやりとりがなく($Q=0$)、また気体が外に仕事をしない($W=0$)。熱力学第一法則 $dU = \delta Q - \delta W$ より $\Delta U = 0$ である。理想気体では内部エネルギー $U$ は温度 $T$ のみの関数($U = c N R T$)であり、体積 $V$ には依存しない。したがって、体積が $V$ から $V'$ に増えても $U$ が一定なら $T$ も不変である。\textbf{理想気体の断熱自由膨張では温度は変化しない}。これは理想気体の仮定(分子間力なし)に基づく結果であり、実在気体ではジュールの実験でわかるようにわずかに温度が変化する場合がある。

\paragraph{問2:一連の操作のエントロピー変化}

初期状態のエントロピーは $S_{\mathrm{i}} = NR\ln(T^c V/N) + N S_0$。終状態 $(T'',V)$ のエントロピーは $S_{\mathrm{f}} = NR\ln((T'')^c V/N) + N S_0$。したがって
\begin{equation}
\Delta S = S_{\mathrm{f}} - S_{\mathrm{i}}
= NR \ln\frac{(T'')^c}{T^c}
= N R c \ln\frac{T''}{T}.
\end{equation}
問1より $T' = T$。準静的断熱過程 $(T',V') \to (T'',V)$ ではエントロピーが一定なので、\textbf{理想気体の断熱過程の関係式} $T^c V = \mathrm{const}$ を用いる。導出:断熱では $dU = -p\,dV$。$U = c N R T$ より $dU = c N R\,dT$、$p = NRT/V$ より $c N R\,dT = -(NRT/V)\,dV$。$T$, $V$ で分離して積分すると $c \ln T = -\ln V + \mathrm{const}$、すなわち $T^c V = \mathrm{const}$。したがって $(T')^c V' = (T'')^c V$。$T'=T$ なので $T^c V' = (T'')^c V$、よって
\begin{equation}
T'' = T \left(\frac{V'}{V}\right)^{1/c}.
\end{equation}
($c$ は $U = c N R T$ の係数であり、断熱指数 $\gamma = (c+1)/c$ を使うと $T V^{\gamma-1} = T V^{1/c}$ なので $T^c V = \mathrm{const}$ と $T V^{\gamma-1} = \mathrm{const}$ は同じ関係である。)

これを $\Delta S$ に代入する:
\begin{equation}
\Delta S = N R c \ln\frac{T''}{T}
= N R c \ln\left(\frac{V'}{V}\right)^{1/c}
= N R \ln\frac{V'}{V}.
\end{equation}
よって
\begin{equation}
\boxed{\Delta S = N R \ln\frac{V'}{V}}.
\end{equation}
($V' > V$ なので $\Delta S > 0$。断熱自由膨張は不可逆過程のため、エントロピーが増大する。)

\paragraph{なぜエントロピーが増大するか(物理的考察)}

一連の操作のうち、\textbf{断熱自由膨張} $(T,V) \to (T',V')$ は不可逆過程である。気体が急に膨張するため、途中の状態は平衡ではなく、同じ体積変化を準静的に行う経路と比べて「無秩序に」広がる。熱力学第二法則により、孤立系の不可逆過程ではエントロピーは増大する。準静的断熱過程 $(T',V') \to (T'',V)$ ではエントロピーは一定なので、\textbf{全体の $\Delta S$ は断熱自由膨張の段階で生じたエントロピー増加} $N R \ln(V'/V)$ に等しい。$V' > V$ なので $\ln(V'/V) > 0$ であり、気体がより広い体積に広がった分だけ「配置の無秩序さ」が増し、エントロピーが増えたと解釈できる。

\begin{figure}[H]
\centering
\includegraphics[width=0.85\textwidth]{figures/past2024_ex1_why_entropy.png}
\caption{問題I:断熱自由膨張で気体が $V \to V'$ に広がるとエントロピーが増大する。準静的断熱圧縮では $S$ 一定。}
\label{fig:past2024_ex1_why_entropy}
\end{figure}

\begin{figure}[H]
\centering
\includegraphics[width=0.85\textwidth]{figures/past2024_ex1_path.png}
\caption{問題Iの過程:$(T,V) \to (T',V')$(断熱自由膨張)、$(T',V') \to (T'',V)$(準静的断熱圧縮)。}
\label{fig:past2024_ex1_path}
\end{figure}

%----------------------------------------------------------------------
\section{問題II:左右に仕切られた容器内の理想気体(類題:演習6-IV--V)}
%----------------------------------------------------------------------

\subsection{問題}

体積 $V$ の断熱壁の容器が、中央で左右に仕切られた透熱壁で分かれている。透熱壁は左右に自由に動ける。系の温度を $T$ とする。左に $N_A$ モル、右に $N_B$ モルの単原子分子理想気体が入っており、左右の圧力は釣り合い、熱平衡にある。

$N_A$ モルと $N_B$ モルの2種類の(単原子)理想気体からなる混合系のエントロピーは
\begin{equation}
S(T,V,N_A,N_B) = N_A R \ln\frac{T^c V}{N_A} + N_A S_{0,A}
+ N_B R \ln\frac{T^c V}{N_B} + N_B S_{0,B}
\end{equation}
で与えられる($c$, $S_{0,A}$, $S_{0,B}$ は定数)。

\begin{enumerate}
\item 左右が同一種の気体の場合、左右を隔てる壁を静かに取り外した。十分時間が経った後の全エントロピー変化 $\Delta S$ を、$T$, $R$, $c$, $N_A$, $N_B$ で表せ。
\item 左右が異なる種類の気体の場合、壁にかかる圧力 $p$ を、$T$, $R$, $c$, $V$, $N_A$, $N_B$ で表せ。
\item 問2の状況で壁を静かに取り外した。十分時間が経った後のエントロピー変化 $\Delta S$ を、$V$ を用いずに表せ。
\item 問3で求めた全エントロピー変化が非負であることを示せ。
\item 問3の終状態における気体Aの化学ポテンシャル $\mu_A$ を導け。
\end{enumerate}

\subsection{解答}

\paragraph{この問題のポイント(初学者向け)}

問1は\textbf{同一種}の気体で壁を取り外す場合。左右とも同じ気体で、もともと温度・圧力が釣り合っているので、壁を取ってもマクロには同じ状態であり $\Delta S = 0$。問2〜5は\textbf{異なる種類}の気体(例:ネオンとアルゴン)。問2は壁にかかる圧力、問3は壁を取り外したときのエントロピー増加(混合のエントロピー)、問4はその非負性の証明、問5は化学ポテンシャルである。

\paragraph{解き方の流れ}

\begin{enumerate}
\item 問1:初期と終状態のエントロピーを $S(T,V,N)$ の式で書き、$V_A/V = N_A/(N_A+N_B)$ などを使って $\Delta S$ を計算すると 0 になる。
\item 問2:$p V = (N_A+N_B) R T$ より $p$ を求める。
\item 問3:各気体が体積 $V_A$, $V_B$ から $V$ に広がるので、$\Delta S = N_A R \ln(V/V_A) + N_B R \ln(V/V_B)$。$V/V_A$, $V/V_B$ を $N_A$, $N_B$ で表す。
\item 問4:$x_A \ln x_A + x_B \ln x_B \le 0$($x_A+x_B=1$)を用いる。
\item 問5:理想気体の化学ポテンシャル $\mu_A = \mu_A^0(T) + R T \ln(p_A/p^0)$ で、$p_A = x_A p$ とする。
\end{enumerate}

\paragraph{記号と設定}

左室:体積 $V_A$、モル数 $N_A$、圧力 $p$。右室:体積 $V_B$、モル数 $N_B$、圧力 $p$。透熱壁で温度 $T$ が共通。圧力釣り合い:$p V_A = N_A R T$、$p V_B = N_B R T$。全体の体積は $V = V_A + V_B$。

\paragraph{問1:同一種で壁を取り外したときの $\Delta S$}

初期:左室 $S_A^{\mathrm{i}} = N_A R \ln(T^c V_A/N_A) + N_A S_0$、右室 $S_B^{\mathrm{i}} = N_B R \ln(T^c V_B/N_B) + N_B S_0$(同一種なので $S_{0,A}=S_{0,B}=S_0$)。終状態:全体で体積 $V$、モル数 $N_A+N_B$、温度 $T$。混合後のエントロピーは、単一の理想気体として
\begin{equation}
S_{\mathrm{f}} = (N_A+N_B) R \ln\frac{T^c V}{N_A+N_B} + (N_A+N_B) S_0.
\end{equation}
初期の全エントロピーは
\begin{equation}
S_{\mathrm{i}} = N_A R \ln\frac{T^c V_A}{N_A} + N_A S_0
+ N_B R \ln\frac{T^c V_B}{N_B} + N_B S_0.
\end{equation}
変化は
\begin{align}
\Delta S &= (N_A+N_B) R \ln\frac{T^c V}{N_A+N_B}
- N_A R \ln\frac{T^c V_A}{N_A}
- N_B R \ln\frac{T^c V_B}{N_B} \notag \\
&= N_A R \ln\frac{V}{V_A}\frac{N_A}{N_A+N_B}
+ N_B R \ln\frac{V}{V_B}\frac{N_B}{N_A+N_B}.
\end{align}
同一種で温度 $T$・圧力 $p$ が共通なので、状態方程式 $p V_A = N_A R T$、$p V_B = N_B R T$、$V_A + V_B = V$ が成り立つ。$p V = (N_A+N_B) R T$ なので $V_A/V = (N_A R T/p)/((N_A+N_B) R T/p) = N_A/(N_A+N_B)$、すなわち $V_A = V N_A/(N_A+N_B)$。同様に $V_B = V N_B/(N_A+N_B)$。したがって
\begin{equation}
\frac{V}{V_A} = \frac{N_A+N_B}{N_A}, \quad
\frac{V}{V_B} = \frac{N_A+N_B}{N_B}.
\end{equation}
これらを $\Delta S$ の式に代入すると、
$\frac{V}{V_A}\frac{N_A}{N_A+N_B} = \frac{N_A+N_B}{N_A}\cdot\frac{N_A}{N_A+N_B} = 1$、同様に $\frac{V}{V_B}\frac{N_B}{N_A+N_B} = 1$。したがって
\begin{equation}
\Delta S = N_A R \ln 1 + N_B R \ln 1 = 0.
\end{equation}
同一種で圧力・温度が釣り合っているとき、壁を取り外してもマクロな状態は変わらず可逆である。よって
\begin{equation}
\boxed{\Delta S = 0}.
\end{equation}

\paragraph{なぜ同一種では $\Delta S = 0$ か(物理的考察)}

左右が同一種の気体で、透熱壁で温度 $T$ が共通、圧力も釣り合っているとき、左室の体積 $V_A$ と右室の体積 $V_B$ の比は $V_A : V_B = N_A : N_B$ である。壁を取り外すと、両方の気体が全体積 $V = V_A + V_B$ に広がるが、\textbf{同一種なので区別がつかず}、マクロには「$N_A+N_B$ モルの同一気体が体積 $V$ にある」という1つの平衡状態になる。初期状態も、圧力・温度が同じなので、実質同じマクロ状態を別の仕切り方で表現しているに過ぎない。したがって可逆的に壁を元に戻せ、$\Delta S = 0$ である。

\paragraph{問2:異なる種類のとき壁にかかる圧力 $p$}

左室の状態方程式:$p V_A = N_A R T$(単原子理想気体)。右室:$p V_B = N_B R T$。壁にかかる圧力は左右で等しく、釣り合いのとき左から $p$、右から $p$ なので、壁に働く正味の力は 0。壁にかかる圧力の「大きさ」は $p$ である。左室の体積を $V_A$ とすると $p V_A = N_A R T$、$V_A + V_B = V$、$p V_B = N_B R T$ より $p(V_A+V_B) = (N_A+N_B) R T$、よって $p = (N_A+N_B) R T / V$。したがって
\begin{equation}
\boxed{p = \frac{(N_A+N_B) R T}{V}}.
\end{equation}

\paragraph{問3:異なる種類で壁を取り外したときの $\Delta S$($V$ を用いずに)}

初期:左室 $N_A$ モル・体積 $V_A$・温度 $T$、右室 $N_B$ モル・体積 $V_B$・温度 $T$。$V_A + V_B = V$。終状態:混合気体、体積 $V$、温度 $T$、$N_A$ モルの気体Aと $N_B$ モルの気体Bが同じ体積 $V$ を占める。$\Delta S$ は $N_A R \ln(V/V_A) + N_B R \ln(V/V_B)$ となるが、問題では $V$ を用いずに表すので、$V_A$, $V_B$, $V$ を $p V_A = N_A R T$、$p V = (N_A+N_B) R T$ などで $N_A$, $N_B$, $T$, $R$ で表して $V$ を消去する。

初期エントロピー:
\begin{equation}
S_{\mathrm{i}} = N_A R \ln\frac{T^c V_A}{N_A} + N_A S_{0,A}
+ N_B R \ln\frac{T^c V_B}{N_B} + N_B S_{0,B}.
\end{equation}
終状態:気体Aは体積 $V_A$ から $V$ に、気体Bは $V_B$ から $V$ に広がる。混合系のエントロピーは(各成分が体積 $V$ を持つとして)
\begin{equation}
S_{\mathrm{f}} = N_A R \ln\frac{T^c V}{N_A} + N_A S_{0,A}
+ N_B R \ln\frac{T^c V}{N_B} + N_B S_{0,B}.
\end{equation}
よって
\begin{equation}
\Delta S = N_A R \ln\frac{V}{V_A} + N_B R \ln\frac{V}{V_B}.
\end{equation}
$p V_A = N_A R T$、$p V = (N_A+N_B) R T$ より $V/V_A = p V/(p V_A) = (N_A+N_B) R T / (N_A R T) = (N_A+N_B)/N_A$。同様に $V/V_B = (N_A+N_B)/N_B$。よって
\begin{equation}
\boxed{\Delta S = N_A R \ln\frac{N_A+N_B}{N_A} + N_B R \ln\frac{N_A+N_B}{N_B}}.
\end{equation}

\paragraph{問4:$\Delta S \ge 0$ の証明}

問3の結果 $\Delta S = N_A R \ln\frac{N_A+N_B}{N_A} + N_B R \ln\frac{N_A+N_B}{N_B}$ を、モル分率 $x_A = N_A/(N_A+N_B)$、$x_B = N_B/(N_A+N_B)$($x_A+x_B=1$)で書き直す。$\ln\frac{N_A+N_B}{N_A} = \ln(1/x_A) = -\ln x_A$、$\ln\frac{N_A+N_B}{N_B} = -\ln x_B$ なので
\begin{equation}
\Delta S = N_A R (-\ln x_A) + N_B R (-\ln x_B)
= -R (N_A \ln x_A + N_B \ln x_B).
\end{equation}
$N_A = (N_A+N_B) x_A$、$N_B = (N_A+N_B) x_B$ を代入すると $N_A \ln x_A + N_B \ln x_B = (N_A+N_B)(x_A \ln x_A + x_B \ln x_B)$ である。したがって
\begin{equation}
\frac{\Delta S}{R} = -(N_A+N_B)(x_A \ln x_A + x_B \ln x_B).
\end{equation}
$0 < x < 1$ のとき $\ln x < 0$ なので $x \ln x < 0$($x=1$ のときは $\ln 1 = 0$ なので $x\ln x = 0$)。したがって $x_A \ln x_A \le 0$、$x_B \ln x_B \le 0$(等号は $x_A=1$ または $x_B=1$ のときのみ)であり、$\Delta S \ge 0$ となる。等号は $x_A=1$ または $x_B=1$、すなわち一方の気体だけが存在するときである。

\paragraph{なぜ異種混合で $\Delta S > 0$ か(物理的考察)}

異なる種類の気体(例:ネオンとアルゴン)が壁を取り外して混合すると、\textbf{各気体が相手の領域にも広がる}。気体Aは体積 $V_A$ から $V$ に、気体Bは $V_B$ から $V$ に広がり、両方が同じ空間を占める。異種なので「どちらがどちらか」は区別でき、この「混合」は不可逆である(自然には分離しない)。配置の取り方が増えた分だけエントロピーが増大し、$\Delta S = N_A R \ln[(N_A+N_B)/N_A] + N_B R \ln[(N_A+N_B)/N_B] > 0$ となる。熱力学第二法則(孤立系のエントロピーは減少しない)の反映である。

\begin{figure}[H]
\centering
\includegraphics[width=0.8\textwidth]{figures/past2024_ex2_entropy_same_diff.png}
\caption{問題II:同一種で壁を取り外すと $\Delta S = 0$(可逆)。異種で混合すると $\Delta S > 0$(不可逆)。}
\label{fig:past2024_ex2_entropy}
\end{figure}

\paragraph{問5:終状態の気体Aの化学ポテンシャル $\mu_A$}

化学ポテンシャルは $\mu_A = (\partial G/\partial N_A)_{T,p,N_B}$ で定義される。理想気体では、ギブス自由エネルギー $G$ から $\mu_A = \mu_A^0(T) + R T \ln(p_A/p^0)$ が導かれる($\mu_A^0(T)$ は標準圧力 $p^0$ における化学ポテンシャル)。終状態では気体Aの分圧は $p_A = p \cdot N_A/(N_A+N_B) = x_A p$(ダルトンの分圧の法則)。$p = (N_A+N_B) R T / V$ なので $p_A = N_A R T / V$ でもある。標準状態を $p^0$ とすると
\begin{equation}
\mu_A = \mu_A^0(T) + R T \ln\frac{p_A}{p^0}
= \mu_A^0(T) + R T \ln\frac{N_A R T}{V p^0}.
\end{equation}
あるいは、モル分率 $x_A = N_A/(N_A+N_B)$ と全圧 $p$ を用いて $p_A = x_A p$ なので
\begin{equation}
\boxed{\mu_A = \mu_A^0(T) + R T \ln\frac{x_A p}{p^0}}.
\end{equation}

\paragraph{化学ポテンシャルの物理的意味}

化学ポテンシャル $\mu_A$ は、成分Aの粒子を1モル追加したときのギブス自由エネルギー $G$ の増分である。理想気体では $\mu_A = \mu_A^0(T) + R T \ln(p_A/p^0)$ となり、\textbf{分圧 $p_A$ が高いほど $\mu_A$ は大きい}。混合後の終状態では、気体Aは全圧 $p$ のうち分圧 $p_A = x_A p$ で存在する。$x_A < 1$ なので $p_A < p$ であり、純粋なAだけのときより $\mu_A$ は低くなる。これは「混合によってAの化学ポテンシャルが下がり、拡散が進む」という直感と一致する。

\begin{figure}[H]
\centering
\includegraphics[width=0.75\textwidth]{figures/past2024_ex2_setup.png}
\caption{問題IIの設定:透熱壁で仕切られた容器。左 $N_A$ モル、右 $N_B$ モル。}
\label{fig:past2024_ex2_setup}
\end{figure}

%----------------------------------------------------------------------
\section{問題III:変な気体(類題:2023年度 問題II;演習6-I, 演習4-III)}
%----------------------------------------------------------------------

\subsection{問題}

体積 $V$ の中に、ある気体が入っている。この系のエントロピーが $S(T,V) = \sigma T^3 V$ で与えられるとする($\sigma$ は正の定数)。以下の問に答えよ。

\begin{enumerate}
\item この系の内部エネルギー $E(T,V)$ を、$\sigma$, $T$, $V$ を用いて表わせ。ただし $E(T=0,V)=0$ とする。
\item この系の圧力 $p(T,V)$ を、$\sigma$, $T$, $V$ を用いて表わせ。
\item この気体が温度 $T$ の熱源と接して熱平衡にあり、体積 $V$ は一定とする。このときの揺らぎの分散 $\langle \delta E^2 \rangle$ を、ボルツマン定数 $k_B$、$\sigma$、$T$、$V$ を用いて表わせ。ここで $\delta E = E - \langle E \rangle$ である。
\end{enumerate}

ヒント:定積比熱は $C_V = (\partial E/\partial T)_V = T(\partial S/\partial T)_V$ とも書ける。$S$ を $T,V$ の関数とみなすと $T\,dS = C_V\,dT + \bigl((\partial E/\partial V)_T + p\bigr)\,dV$ と書ける。

\subsection{解答}

【類題】2023年度 問題II と同一内容です。演習問題解説では演習6-I(熱力学の関係式)、演習4-III(内部エネルギーの方程式)が関連します。

\paragraph{なぜこのように解くか(問1・問2)}
問1ではエントロピー $S(T,V)$ が与えられているので、熱力学の関係 $C_V = T(\partial S/\partial T)_V$(体積一定での定積比熱)から $C_V$ を求め、$E(T=0,V)=0$ のもとで $E = \int_0^T C_V\,dT'$ で内部エネルギーを求める。問2では $T\,dS = dE + p\,dV$ を $T$ 一定で $dV$ の係数比較し、$(\partial S/\partial V)_T$ と $(\partial E/\partial V)_T$ から $p$ を導く。

\paragraph{問1:$E(T,V)$}

ヒントの $C_V = T(\partial S/\partial T)_V$ を用いる。$S = \sigma T^3 V$ より $(\partial S/\partial T)_V = 3\sigma T^2 V$ なので $C_V = T \cdot 3\sigma T^2 V = 3\sigma T^3 V$。体積一定で $E(T=0,V)=0$ として $E(T,V) = \int_0^T C_V(T',V)\,dT'$ を計算すると
\begin{equation}
\boxed{E(T,V) = \frac{3}{4}\sigma T^4 V}.
\end{equation}

\paragraph{問2:$p(T,V)$}

熱力学の基本関係式 $T\,dS = dE + p\,dV$ で $T$ 一定とすると $T(\partial S/\partial V)_T = (\partial E/\partial V)_T + p$。$(\partial S/\partial V)_T = \sigma T^3$、$(\partial E/\partial V)_T = (3/4)\sigma T^4$ を代入して $T\cdot\sigma T^3 = (3/4)\sigma T^4 + p$、したがって
\begin{equation}
\boxed{p(T,V) = \frac{1}{4}\sigma T^4}.
\end{equation}
($p$ は $V$ に依存しない。)

\paragraph{問3:$\langle \delta E^2 \rangle$}

問題文の「温度 $T$ の熱源と接して熱平衡にあり」から\textbf{カノニカル分布}が適用される(用語集参照)。カノニカル分布では、系のエネルギーの分散は $\langle \delta E^2 \rangle = k_B T^2 C_V$ で与えられる($C_V$ は定積比熱。なぜこの公式でよいかは2023年度 問題II 問3・本試験問題III 問6を参照)。問1で $C_V = 3\sigma T^3 V$ を求めているので、これを代入して
\begin{equation}
\boxed{\langle \delta E^2 \rangle = 3\sigma k_B T^5 V}.
\end{equation}

詳細な導出・物理的考察は2023年度 問題II(本冊)を参照してください。図は図\ref{fig:past2023_ex2_E_p}(2023年度 問題II)を参照。

%----------------------------------------------------------------------
\section{問題IV:N個の独立な調和振動子(類題:2023年度 問題III;演習7-III)}
%----------------------------------------------------------------------

\subsection{問題}

$N$ 個の独立な調和振動子を量子的に扱う。$i$ 番目の振動子のエネルギーは $E_i = \hbar\omega m_i$($m_i = 0,1,2,\ldots$)とし、$\omega$ は振動子の角振動数である。この系が温度 $T$ の大きな熱源と接しているとき、系が量子状態 $i$ にある確率は分配関数 $Z_N$ を用いて $P(i) = e^{-\beta E_i}/Z_N$ で与えられる。ここで $\beta = 1/(k_B T)$ である。状態 $i$ は量子数の組 $(m_1,\ldots,m_N)$ で指定され、全エネルギーは $E_i = \hbar\omega(m_1+\cdots+m_N) \equiv \hbar\omega M$、$M = m_1+\cdots+m_N$ とする。

\begin{enumerate}
\item $N=1$ のとき $Z_1$ を計算せよ。
\item $N=1$ のときエネルギーの平均値 $\langle E \rangle$ を計算せよ。
\item $Z_N$ を計算せよ。
\item $Z_N$ を用いてエネルギーの平均値 $\langle E \rangle$ を計算せよ。
\item $Z_N = \sum_{M=0}^\infty W_N(M)\,e^{-\beta\hbar\omega M}$ とする。$W_N(M)$ を $N$ と $M$ で表せ。
\item 系がエネルギー $E$ を取る確率は $P(E) \propto W_N(M)\,e^{-\beta E}$ であり、$E^*$ で鋭いピークを持つ。$M \gg 1$ かつ $N \gg 1$ として、$E^*$ を $\beta$, $\hbar\omega$, $N$ などで表せ。必要ならスターリングの公式 $\ln N! \approx N\ln N - N$ を用いてよい。
\end{enumerate}

\subsection{解答}

【類題】2023年度 問題III と同一内容です。演習問題解説では演習7-III(調和振動子の分配関数)が類題です。

\paragraph{問1:$Z_1$}

1個の振動子のエネルギーは $E_m = \hbar\omega m$($m = 0,1,2,\ldots$)。分配関数は $Z_1 = \sum_{m=0}^\infty e^{-\beta\hbar\omega m}$ で、$|e^{-\beta\hbar\omega}|<1$ のとき等比級数の和として
\begin{equation}
\boxed{Z_1 = \frac{1}{1 - e^{-\beta\hbar\omega}}}.
\end{equation}

\paragraph{問2:$N=1$ のときの $\langle E \rangle$}

カノニカル分布では平均エネルギーは $\langle E \rangle = -\partial\ln Z/\partial\beta$ で与えられる($\ln Z$ を $\beta$ で微分すると $-\langle E \rangle$ になるため。2023年度 問題III 問2の「なぜ」を参照)。$\ln Z_1 = -\ln(1-e^{-\beta\hbar\omega})$ を $\beta$ で微分して
\begin{equation}
\boxed{\langle E \rangle = \frac{\hbar\omega}{e^{\beta\hbar\omega}-1}}.
\end{equation}

\paragraph{問3:$Z_N$}

各振動子が独立なので、全状態の和は各振動子の分配関数の積になる:$Z_N = (Z_1)^N$。よって
\begin{equation}
\boxed{Z_N = \left(\frac{1}{1-e^{-\beta\hbar\omega}}\right)^N}.
\end{equation}

\paragraph{問4:$Z_N$ を用いた $\langle E \rangle$}

$\langle E \rangle = -\partial\ln Z_N/\partial\beta = -N\,\partial\ln Z_1/\partial\beta$ より、問2の結果を $N$ 倍して
\begin{equation}
\boxed{\langle E \rangle = N\,\frac{\hbar\omega}{e^{\beta\hbar\omega}-1}}.
\end{equation}

\paragraph{問5:$W_N(M)$}

$W_N(M)$ は「全エネルギーが $E = \hbar\omega M$ であるような状態の数」である。$M = m_1+\cdots+m_N$ を満たす非負整数の組 $(m_1,\ldots,m_N)$ の個数は、$M$ 個のボールを $N$ 個の箱に分ける重複組合せに等しく、
\begin{equation}
\boxed{W_N(M) = \frac{(M+N-1)!}{(N-1)!\,M!}}.
\end{equation}

\paragraph{問6:最確エネルギー $E^*$}

系がエネルギー $E = \hbar\omega M$ を取る確率は $P(E) \propto W_N(M)\,e^{-\beta E}$(状態数 $W_N(M)$ とボルツマン因子の積)なので、これが最大になる $M$ を求める。$P$ の最大は $\ln P$ の最大と同じだから、$\ln P$ を $M$ で微分して 0 とおく:$\partial[\ln W_N(M) - \beta\hbar\omega M]/\partial M = 0$。$M \gg 1$, $N \gg 1$ でスターリングの公式 $\ln n! \approx n\ln n - n$ を用いると $\partial\ln W_N(M)/\partial M \approx \ln((M+N)/M)$ となる(2023年度 問題III 問6の導出参照)。したがって $\ln\frac{M+N}{M} = \beta\hbar\omega$ より $1+N/M = e^{\beta\hbar\omega}$、$M = N/(e^{\beta\hbar\omega}-1)$。よって
\begin{equation}
\boxed{E^* = \frac{N\hbar\omega}{e^{\beta\hbar\omega}-1}}.
\end{equation}
(問4の $\langle E \rangle$ と一致する。$N \gg 1$ のとき $P(E)$ は $E^*$ で鋭いピークを持つ。)

詳細な導出・物理的考察は2023年度 問題III(本冊)を参照してください。図は図\ref{fig:past2023_oscillator}、図\ref{fig:past2023_ex3_W_P}(2023年度 問題III)を参照。
