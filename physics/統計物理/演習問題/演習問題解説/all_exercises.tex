\documentclass[11pt,a4paper]{ltjsarticle}
\usepackage[no-math]{luatexja-fontspec}
\setmainjfont{Hiragino Mincho ProN}[
  UprightFont=*,
  BoldFont=*,
  ItalicFont=*,
  BoldItalicFont=*
]
\setsansjfont{Hiragino Kaku Gothic ProN}[
  UprightFont=*,
  BoldFont=*,
  ItalicFont=*,
  BoldItalicFont=*
]
\usepackage{amsmath,amssymb}
\usepackage{graphicx}
\usepackage{geometry}
\geometry{margin=2.5cm}
\usepackage{float}
\usepackage[draft=false]{hyperref}
\hypersetup{
    colorlinks=true,
    linkcolor=blue,
    citecolor=blue,
    urlcolor=blue,
    pdfusetitle=true
}

\title{統計物理学Ⅰ 演習問題 解答・解説}
\author{名古屋大学 理学部}
\date{2025年度}

\begin{document}

% 表紙:1ページ目に直接描画(\maketitle は改ページで空白になることがあるため、カスタム表紙を使用)
\pagestyle{empty}
\thispagestyle{empty}
\begin{center}
  \vspace*{2cm}
  {\LARGE 統計物理学Ⅰ 演習問題 解答・解説}\\[3cm]
  \vfill
  {\large 名古屋大学 理学部}\\[1em]
  {\large 2025年度}
  \vfill
\end{center}
\newpage
\pagestyle{plain}

\tableofcontents
\newpage

% 演習1 (2025/10/10実施)
\section{演習1 (2025年10月10日実施)}

\subsection{I. 部屋を温めるための熱}

\subsubsection{問題}

体積 $V = 27\,\mathrm{m}^3$、1気圧、天井の高さ2mの部屋を考える。空気の定積比熱 $C_v = 0.7\,\mathrm{J/(g{\cdot}deg)}$(定数と仮定)、空気の密度 $\rho = 1.3 \times 10^{-3}\,\mathrm{g/cm}^3$ とする。

\begin{enumerate}
    \item 0℃の部屋を20℃まで温めるのに必要な熱量(J)を求めよ。定積条件下とする。
    \item 問1の熱量を供給するのにかかる時間を求めよ。エアコンの暖房能力を $P = 2000\,\mathrm{W}$ と仮定する。
\end{enumerate}

\subsubsection{解答}

\paragraph{この問題で学ぶこと}

熱量計算の基本(質量、比熱、温度変化から熱量を求める)と、単位の換算・整合性の確認。定積条件のもとで、供給した熱がすべて内部エネルギーの増加に使われる。

\paragraph{解き方の流れ}

\begin{enumerate}
    \item 密度と体積から空気の質量を求める(単位換算に注意)
    \item 定積比熱の式 $Q = m C_v \Delta T$ に代入して熱量を計算
    \item 問2:仕事率 $P$ を用いて、時間 $t = Q/P$ を求める
\end{enumerate}

\paragraph{用語の説明(初学者向け)}

\begin{itemize}
    \item \textbf{定積比熱 $C_v$}:体積を一定に保ったまま、物質の単位質量(1 g)の温度を1 K(または1 ℃)上げるのに必要な熱量である。単位は $\mathrm{J/(g{\cdot}K)}$ や $\mathrm{J/(g{\cdot}deg)}$。
    \item \textbf{なぜ定積条件か}:部屋の空気を温めるとき、窓や壁がしっかりしていれば体積はほぼ一定とみなせる。現実には多少の膨張はあるが、近似として定積を仮定する。
    \item \textbf{定積過程}:体積 $V$ を変えずに行う状態変化。このとき気体は仕事をしない($W = 0$)ので、供給した熱量はすべて内部エネルギーの増加に使われる。
    \item \textbf{セルシウスとケルビン}:温度の差を考えるとき、0℃と20℃の差は20 Kである。熱力学では絶対温度(ケルビン)を用いる。$T(\mathrm{K}) = T(℃) + 273.15$ であり、$\Delta T = 20\,\mathrm{K}$ である。
\end{itemize}

\paragraph{問題の理解と設定}

部屋の空気を 0℃(273 K)から 20℃(293 K)まで温めるのに必要な熱量を求める。部屋の体積と空気の密度から空気の質量を求め、定積比熱の定義を用いて熱量を計算する。

\paragraph{使用する物理法則}

定積比熱の定義:
\begin{equation}
C_v = \frac{1}{m}\left(\frac{\partial Q}{\partial T}\right)_V
\end{equation}
添字 $_V$ は「体積 $V$ を固定して」という意味である。つまり、体積を変えずに温度だけを微小に $dT$ 変化させたとき、必要な熱量は $dQ = m C_v \, dT$ である。

\textbf{$Q = m C_v \Delta T$ が成り立つ条件}:$C_v$ が定数と仮定されているとき、$dQ = m C_v \, dT$ を温度 $T_1$ から $T_2$ まで積分すると、この式が得られる。$C_v$ が温度に依存する場合は、積分 $\int m C_v(T)\,dT$ が必要になる。

$C_v$ が定数と仮定されているので、温度 $T_1$ から $T_2$ まで積分すると
\begin{equation}
Q = \int_{T_1}^{T_2} m C_v \, dT = m C_v (T_2 - T_1) = m C_v \Delta T
\end{equation}
が成り立つ。

\paragraph{問1: 必要な熱量の計算}

\textbf{ステップ1:単位の換算}

密度の単位を SI 系に揃える。$1\,\mathrm{g} = 10^{-3}\,\mathrm{kg}$、$1\,\mathrm{cm} = 10^{-2}\,\mathrm{m}$ より $1\,\mathrm{cm}^3 = 10^{-6}\,\mathrm{m}^3$ である。したがって
\begin{equation}
\rho = 1.3 \times 10^{-3}\,\frac{\mathrm{g}}{\mathrm{cm}^3}
= 1.3 \times 10^{-3} \times \frac{10^{-3}\,\mathrm{kg}}{10^{-6}\,\mathrm{m}^3}
= 1.3 \times 10^{-3} \times 10^{3}\,\frac{\mathrm{kg}}{\mathrm{m}^3}
= 1.3\,\frac{\mathrm{kg}}{\mathrm{m}^3}
\end{equation}
($1\,\mathrm{g/cm}^3 = 1000\,\mathrm{kg/m}^3$ を用いた。)すなわち、$1\,\mathrm{m}^3$ の空気の質量は 1.3 kg である。

\textbf{ステップ2:空気の質量}

\begin{equation}
m = \rho V = 1.3\,\frac{\mathrm{kg}}{\mathrm{m}^3} \times 27\,\mathrm{m}^3 = 35.1\,\mathrm{kg} = 3.51 \times 10^4\,\mathrm{g}
\end{equation}

\textbf{ステップ3:熱量}

$\Delta T = 293 - 273 = 20\,\mathrm{K}$ として
\begin{align}
Q &= m C_v \Delta T = 3.51 \times 10^4\,\mathrm{g} \times 0.7\,\frac{\mathrm{J}}{\mathrm{g{\cdot}K}} \times 20\,\mathrm{K} \\
&= 4.914 \times 10^5\,\mathrm{J} \approx 4.9 \times 10^5\,\mathrm{J} = 490\,\mathrm{kJ}
\end{align}

\textbf{答}:約 $4.9 \times 10^5\,\mathrm{J}$(約 490 kJ)

\paragraph{物理的意味}

490 kJ は、1 kWh $= 3.6 \times 10^6\,\mathrm{J}$ の約 0.14 倍である。したがって、約 0.14 kWh の電気エネルギーに相当する。家庭用エアコンで数分程度の消費量である。

\textbf{単位のチェック}:$m$ は g または kg、$C_v$ は J/(g$\cdot$K)、$\Delta T$ は K。$m C_v \Delta T$ の単位は $\mathrm{g} \cdot \mathrm{J/(g{\cdot}K)} \cdot \mathrm{K} = \mathrm{J}$ となり、熱量の単位と一致する。

\paragraph{問2: 暖房にかかる時間}

仕事率 $P$ は単位時間あたりのエネルギーで、$P = dQ/dt$ と定義される。次元は $\mathrm{J/s} = \mathrm{W}$(ワット)である。熱量 $Q$ を仕事率 $P$ で供給するのに要する時間は
\begin{equation}
t = \frac{Q}{P} = \frac{4.914 \times 10^5\,\mathrm{J}}{2000\,\mathrm{W}} = \frac{4.914 \times 10^5}{2000}\,\mathrm{s} = 245.7\,\mathrm{s} \approx 4.1\,\mathrm{分}
\end{equation}
($\mathrm{J/W} = \mathrm{J/(J/s)} = \mathrm{s}$ である。)

\textbf{答}:約 4.1 分

\textbf{補足}:実際の暖房では、壁や窓からの熱損失があるため、部屋を一定温度に保つには継続的に熱を供給する必要があり、温めるだけでも通常はもう少し時間がかかる場合がある。

\textbf{なぜ $Q = m C_v \Delta T$ で求まるのか}:定積比熱 $C_v$ の定義は「単位質量を1 K温めるのに必要な熱量」である。質量 $m$ を $\Delta T$ K 温めるには、その積 $m \times C_v \times \Delta T$ が必要になる。これは線形関係(比例)であり、図\ref{fig:ex1_heat_capacity}のように $Q$-$T$ グラフでは傾き $m C_v$ の直線になる。

\begin{figure}[H]
    \centering
    \includegraphics[width=0.7\textwidth]{figures/ex1_heat_capacity.png}
    \caption{定積条件での熱量と温度変化の関係。$Q = m C_v \Delta T$ は比例関係であり、直線の傾きが $m C_v$ である。}
    \label{fig:ex1_heat_capacity}
\end{figure}

%--------------------------------------
\subsection{II. 理想気体の状態方程式の絵}

\subsubsection{この問題で学ぶこと}

状態方程式 $pV = NRT$ が $(p, V, T)$ 空間に曲面を定めること。等温線・等圧線・等積線は、その曲面を異なる方向で切った切り口である。後の熱力学(エントロピー、経路)の議論の基礎となる可視化である。

\subsubsection{問題}

理想気体の状態方程式 $pV = NRT$ において、$V$ を $x$ 軸、$p$ を $y$ 軸、$T$ を $z$ 軸とする3次元図をスケッチせよ。等温線、等圧線、等積線を書き入れよ。

\subsubsection{解答}

\paragraph{用語の説明}

\begin{itemize}
    \item \textbf{状態方程式}:気体の圧力 $p$、体積 $V$、温度 $T$ の間に成り立つ関係式。理想気体では $pV = NRT$($N$ はモル数、$R$ は気体定数)。$N$ と $R$ を一定とすると、この式は「$p,V,T$ のうち2つを決めれば残り1つが決まる」ことを意味し、平衡状態を一意に定める。
    \item \textbf{等温線}:温度 $T$ を一定にしたときの $p$ と $V$ の関係。$pV = NRT = \mathrm{const}$ なので、$p = \mathrm{const}/V$ となり、$p$-$V$ 平面では直角双曲線(反比例のグラフ)になる。
    \item \textbf{等圧線}:圧力 $p$ を一定にしたときの $V$ と $T$ の関係。$V = (NR/p)T$ なので $V \propto T$ の直線。
    \item \textbf{等積線}:体積 $V$ を一定にしたときの $p$ と $T$ の関係。$p = (NR/V)T$ なので $p \propto T$ の直線。
\end{itemize}

\textbf{各変数の物理的意味}:$p$ は圧力(単位面積あたりの力)、$V$ は体積、$T$ は絶対温度(K)、$N$ はモル数、$R$ は気体定数。$N$ と $R$ が一定のとき、$p,V,T$ のうち2つを決めると残り1つが決まる。

\textbf{2次元への投影}:$p$-$V$ 図は $T$ を固定した断面(等温線が双曲線)、$T$-$V$ 図は $p$ を固定した断面(等圧線が直線)である。

\paragraph{軸の対応}

問題文の指定によると、$V$ を $x$ 軸、$p$ を $y$ 軸、$T$ を $z$ 軸とする。3次元空間 $(V, p, T)$ において、状態方程式 $pV = NRT$ を満たす点の集合が曲面をなす。

\paragraph{図の説明}

図\ref{fig:ex1_ideal_gas_3d}に、この曲面と、その上に描かれた等温線・等圧線の例を示す。

\begin{itemize}
    \item \textbf{等温線}(赤):$T = 300\,\mathrm{K}$ などの「水平面」($T$ 一定の面)で曲面を切った切り口。$p$ と $V$ は反比例する双曲線になる。
    \item \textbf{等圧線}(青):$p = 1\,\mathrm{気圧}$ などの「垂直面」($p$ 一定の面)で曲面を切った切り口。$V$ が増えると $T$ も増える直線になる。
    \item \textbf{等積線}:$V$ を固定した「縦方向」の線($V$ 一定の直線)。この線上では $p$ と $T$ が比例する。
\end{itemize}

\textbf{見やすくする工夫}:異なる温度(例:200 K、300 K、400 K)で複数の等温線を、異なる圧力で複数の等圧線を描き入れると、曲面の形が把握しやすくなる。

\textbf{なぜ等温線は双曲線になるのか}:高校数学で学んだ反比例 $y = k/x$ を思い出してください。$T$ を固定すると $pV = NRT = \mathrm{const}$ より $p = \mathrm{const}/V$。つまり $p$ は $V$ に反比例するので、$p$-$V$ 平面では直角双曲線になる。同様に、等圧線では $V \propto T$(比例)、等積線では $p \propto T$(比例)である。

\textbf{物理的考察}:この曲面の上の任意の1点が、気体の1つの平衡状態を表す。状態変化はこの曲面上を動く経路として表される。等温膨張は曲面に沿った水平な曲線、断熱膨張は別の曲線になる(演習3、5で詳しく扱う)。

\begin{figure}[H]
    \centering
    \includegraphics[width=0.8\textwidth]{figures/ex1_ideal_gas_3d.png}
    \caption{理想気体の状態方程式 $pV = NRT$ の3次元表示。曲面上の曲線が等温線・等圧線の例である。}
    \label{fig:ex1_ideal_gas_3d}
\end{figure}

%--------------------------------------
\subsection{III. 気体分子の速さ}

\subsubsection{この問題で学ぶこと}

等分配則 $\langle \frac{1}{2}m v_x^2 \rangle = \frac{1}{2}k_B T$ から分子の速度の目安を求める。音速が圧縮率と密度で決まる理由を理解する。分子運動論とマクロな量(温度、音速)のつながりを体得する。

\subsubsection{問題}

分子の運動エネルギーについて $\langle \frac{1}{2}m v_x^2 \rangle = \frac{1}{2}k_B T$ が成り立つ。空気分子を平均質量 $M$ の1成分気体と近似し、$T = 290\,\mathrm{K}$ における $\sqrt{\langle v_x^2 \rangle}$(m/s)を見積もれ。また、音速 $c_s = 1/\sqrt{\rho \chi}$($\chi$ は等温圧縮率)を見積もれ。$R = 8.3\,\mathrm{J/(mol{\cdot}deg)}$、$N_A = 6.0 \times 10^{23}$、空気の分子量 $M = 29$、密度 $\rho = 1.3\,\mathrm{kg/m}^3$ を用いる。

\subsubsection{解答}

\paragraph{用語の説明}

\begin{itemize}
    \item \textbf{$\langle \cdots \rangle$}:アンサンブル平均または熱平衡における期待値。多くの分子についての平均を表す。
    \item \textbf{ボルツマン定数 $k_B$}:$k_B = R/N_A \approx 1.38 \times 10^{-23}\,\mathrm{J/K}$。気体定数 $R$ をアボガドロ数 $N_A$ で割ったもので、1分子あたりの気体定数といえる。
    \item \textbf{等分配則}:熱平衡にある系では、各自由度($x,y,z$ 各方向の運動など)に平均して $\frac{1}{2} k_B T$ の運動エネルギーが配分される。ここでは $x$ 方向の1自由度に注目し、$\langle \frac{1}{2}m v_x^2 \rangle = \frac{1}{2}k_B T$ が成り立つ。
    \item \textbf{注意}:$v_x$ は速度の $x$ 成分であり、速さ $v = \sqrt{v_x^2 + v_y^2 + v_z^2}$ とは異なる。$\sqrt{\langle v_x^2 \rangle}$ は $x$ 方向の速さの「二乗平均平方根(RMS)」であり、速さの目安として用いる。
    \item \textbf{圧縮率 $\chi$}:物質の「押し縮めやすさ」。$\chi = -\frac{1}{V}\left(\frac{\partial V}{\partial p}\right)_T$ で定義される。バネの「柔らかさ」の逆数のような量。音波は媒質の密度の疎密波なので、圧縮率が小さい(かたい)ほど、疎密が速く伝わり音速は大きい。
\end{itemize}

\paragraph{問1: 分子の速さ}

与えられた関係式 $\langle \frac{1}{2}m v_x^2 \rangle = \frac{1}{2}k_B T$ の両辺を $m/2$ で割ると
\begin{equation}
\langle v_x^2 \rangle = \frac{k_B T}{m}
\end{equation}
1分子の質量 $m$:モル質量 $M$(kg/mol)をアボガドロ数 $N_A$ で割ると、1分子あたりの質量になる。$m = M/N_A$。ボルツマン定数 $k_B$:気体定数 $R$ は1 mol あたりの定数なので、$R$ を $N_A$ で割ると1分子あたりの定数 $k_B = R/N_A$ となる。したがって
\begin{equation}
\langle v_x^2 \rangle = \frac{k_B T}{m} = \frac{(R/N_A) T}{M/N_A} = \frac{RT}{M}
\end{equation}
$M = 29\,\mathrm{g/mol} = 29 \times 10^{-3}\,\mathrm{kg/mol}$、$T = 290\,\mathrm{K}$、$R = 8.3\,\mathrm{J/(mol{\cdot}K)}$ を代入。単位の約分:$\mathrm{J/(mol{\cdot}K)} \times \mathrm{K} / (\mathrm{kg/mol}) = \mathrm{J/kg} = \mathrm{m}^2/\mathrm{s}^2$($\mathrm{J} = \mathrm{kg{\cdot}m^2/s^2}$ より)。
\begin{align}
\langle v_x^2 \rangle &= \frac{8.3 \times 290}{29 \times 10^{-3}} = \frac{2407}{0.029} \approx 8.30 \times 10^4\,\mathrm{m}^2/\mathrm{s}^2
\end{align}
したがって
\begin{equation}
\sqrt{\langle v_x^2 \rangle} = \sqrt{8.30 \times 10^4} \approx 288\,\mathrm{m/s}
\end{equation}
これは $x$ 方向の速さの「二乗平均平方根」(RMS)であり、速さの目安として用いる。

\textbf{答}:約 288 m/s

\textbf{感覚}:288 m/s は時速約 1000 km であり、音速(約 340 m/s)に近い。空気分子は非常に速く運動している。

\paragraph{問2: 音速}

\textbf{ステップ1:等温圧縮率}

等温圧縮率は $\chi = -\frac{1}{V}\left(\frac{\partial V}{\partial p}\right)_T$ で定義される。理想気体の状態方程式 $pV = NRT$ より、$T$ 一定のとき $V = NRT/p$ なので
\begin{equation}
\left(\frac{\partial V}{\partial p}\right)_T = -\frac{NRT}{p^2} = -\frac{V}{p}
\end{equation}
よって $\chi = -\frac{1}{V} \cdot \left(-\frac{V}{p}\right) = \frac{1}{p}$。

\textbf{ステップ2:音速の公式}

音波は媒質の密度の疎密(粗密)の波である。連続体力学の波動方程式から、音速は
\begin{equation}
c_s = \frac{1}{\sqrt{\rho \chi}}
\end{equation}
で与えられる。導出の概略:密度 $\rho$ と圧縮率 $\chi$ が、波動の伝播速度に $\sqrt{1/(\rho\chi)}$ の形で入る。密度が大きく押し縮めやすい($\chi$ が大きい)ほど、疎密が伝わりにくく、音速は遅くなる。

\textbf{ステップ3:数値計算}

$p = 1.013 \times 10^5\,\mathrm{Pa}$(1気圧)、$\rho = 1.3\,\mathrm{kg/m}^3$ として
\begin{equation}
c_s = \sqrt{\frac{1}{\rho \chi}} = \sqrt{\frac{p}{\rho}} = \sqrt{\frac{1.013 \times 10^5}{1.3}} \approx \sqrt{7.79 \times 10^4} \approx 279\,\mathrm{m/s}
\end{equation}

\textbf{答}:約 280 m/s

\paragraph{物理的意味・補足}

\textbf{等温と断熱の違い}:実際の音波は、圧縮・膨張が非常に速いため、周囲との熱交換がなく断熱過程として伝わる。そのため音速の計算には断熱圧縮率を用いる。理想気体では $c_s = \sqrt{\gamma p/\rho} \approx 343\,\mathrm{m/s}$($\gamma = 1.4$)となり、実測値に近い。本問で用いた等温圧縮率による 280 m/s は、音速のオーダー(数百 m/s)を理解するための見積もりである。断熱圧縮率の方が小さいため、断熱で計算した音速は等温より大きくなる。

\textbf{なぜ $\sqrt{\langle v_x^2 \rangle}$ と音速が同程度なのか}:音速は媒質の分子の運動と密接に関係する。粗いイメージでは、音波は分子の衝突によって隣へ運動が伝わる「連鎖」であり、分子の熱運動の速さと同じオーダーで伝播する。$\sqrt{\langle v_x^2 \rangle} \approx 288\,\mathrm{m/s}$ と音速 $\approx 340\,\mathrm{m/s}$ が同程度なのは偶然ではなく、同じ物理(分子の運動)に根ざしている。

%--------------------------------------
\subsection{IV. ガウス積分}

\subsubsection{この問題で学ぶこと}

ガウス分布の正規化定数、平均、分散の計算。変数変換とガウス積分 $I = \sqrt{\pi}$ の導出技法。奇関数の積分がゼロになることを利用する。統計力学ではゆらぎの分布として頻出する。

\subsubsection{問題}

ガウス分布 $P(x) = C e^{-(x-\mu)^2/(2\sigma^2)}$($-\infty < x < \infty$)について、
\begin{enumerate}
    \item 正規化条件 $\int_{-\infty}^{\infty} P(x)\,dx = 1$ から $C$ を求めよ。
    \item 平均 $\langle x \rangle = \int_{-\infty}^{\infty} x P(x)\,dx$ を計算せよ。
    \item 分散 $\langle \delta x^2 \rangle = \langle (x - \langle x \rangle)^2 \rangle$ を計算せよ。
\end{enumerate}

\subsubsection{解答}

\paragraph{用語の説明}

\begin{itemize}
    \item \textbf{正規化条件}:確率分布では、すべての事象の確率の合計が 1 でなければならない。そうでないと確率として意味を持たない。連続変数の場合、$\int_{-\infty}^{\infty} P(x)\,dx = 1$ を満たすように定数 $C$ を決める。この操作を正規化という。
    \item \textbf{$\mu$(平均)}:分布の「中心」の位置。問2で求めるように、この分布では $\langle x \rangle = \mu$ である。
    \item \textbf{$\sigma$(標準偏差)}:分布の「広がり」の目安。問3で求めるように、$\langle (x-\mu)^2 \rangle = \sigma^2$ であり、$\sigma$ が大きいほど分布は幅広くなる。
\end{itemize}

\textbf{統計力学での意義}:ガウス分布(正規分布)は、中心極限定理により多くの独立な変数の和が従う分布であり、平衡状態におけるゆらぎの分布として統計力学で繰り返し現れる。

\paragraph{問1: 正規化定数 $C$}

正規化条件より
\begin{equation}
\int_{-\infty}^{\infty} C e^{-(x-\mu)^2/(2\sigma^2)}\,dx = 1
\end{equation}
変数変換 $y = \frac{x - \mu}{\sqrt{2}\sigma}$ を行う。この変換を選ぶ理由は、指数の肩 $(x-\mu)^2/(2\sigma^2)$ を $-y^2$ の形にし、標準的なガウス積分に帰着させるためである。このとき $x = \mu + \sqrt{2}\sigma y$、$dx = \sqrt{2}\sigma\,dy$ であり、$x \to \pm\infty$ のとき $y \to \pm\infty$ である。したがって
\begin{align}
1 &= C \int_{-\infty}^{\infty} e^{-y^2} \cdot \sqrt{2}\sigma\,dy = C\sqrt{2}\sigma \int_{-\infty}^{\infty} e^{-y^2}\,dy
\end{align}
ここで、ガウス積分 $I = \int_{-\infty}^{\infty} e^{-y^2}\,dy = \sqrt{\pi}$ を用いる。

\textbf{ガウス積分の導出}:積分 $I = \int_{-\infty}^{\infty} e^{-y^2}\,dy$ を直接求めるのは困難なので、$I^2$ を計算する「2乗するテクニック」を用いる。$I^2 = \left(\int_{-\infty}^{\infty} e^{-y^2}\,dy\right)^2 = \int_{-\infty}^{\infty} \int_{-\infty}^{\infty} e^{-(y^2 + z^2)}\,dy\,dz$ とおく。$y$-$z$ 平面で極座標 $y = r\cos\theta$、$z = r\sin\theta$ に変換する。ヤコビアンは $r$ であり、$dy\,dz = r\,dr\,d\theta$、$y^2 + z^2 = r^2$ である。したがって
\begin{equation}
I^2 = \int_0^{2\pi} d\theta \int_0^{\infty} e^{-r^2} r\,dr = 2\pi \left[-\frac{1}{2}e^{-r^2}\right]_0^{\infty} = 2\pi \cdot \frac{1}{2} = \pi
\end{equation}
ゆえに $I = \sqrt{\pi}$($I > 0$ より)。

したがって
\begin{equation}
1 = C\sqrt{2}\sigma \cdot \sqrt{\pi} = C\sqrt{2\pi}\sigma \quad \Rightarrow \quad C = \frac{1}{\sqrt{2\pi}\sigma}
\end{equation}

\textbf{答}:$C = \frac{1}{\sqrt{2\pi}\sigma}$

\paragraph{問2: 平均 $\langle x \rangle$}

$x = (x - \mu) + \mu$ と分解する。被積分関数は
\begin{equation}
x P(x) = [(x-\mu) + \mu] \cdot C e^{-(x-\mu)^2/(2\sigma^2)}
\end{equation}
\begin{itemize}
    \item $(x-\mu)$ の項:$(x-\mu) e^{-(x-\mu)^2/(2\sigma^2)}$ は $x = \mu$ に関して奇関数($x-\mu$ を $- (x-\mu)$ に置き換えると符号が反転する)。$-\infty$ から $\infty$ までの積分は 0 である。
    \item $\mu$ の項:$\mu \cdot C e^{-(x-\mu)^2/(2\sigma^2)}$ の積分は、正規化条件より $\mu \cdot 1 = \mu$ である。
\end{itemize}
よって
\begin{equation}
\langle x \rangle = \int_{-\infty}^{\infty} x P(x)\,dx = 0 + \mu = \mu
\end{equation}

\textbf{答}:$\langle x \rangle = \mu$

\paragraph{問3: 分散}

分散は $\langle \delta x^2 \rangle = \langle (x - \langle x \rangle)^2 \rangle$ で定義される。$(x - \langle x \rangle)^2 = x^2 - 2\langle x \rangle x + \langle x \rangle^2$ を展開し、平均をとると $\langle x^2 \rangle - 2\langle x \rangle^2 + \langle x \rangle^2 = \langle x^2 \rangle - \langle x \rangle^2$ となる。$\langle x \rangle = \mu$ なので $\langle \delta x^2 \rangle = \langle x^2 \rangle - \mu^2$ である。

$\langle x^2 \rangle$ を計算する。変数変換 $y = (x-\mu)/(\sqrt{2}\sigma)$、すなわち $x = \mu + \sqrt{2}\sigma y$、$dx = \sqrt{2}\sigma\,dy$ を用いると
\begin{align}
\langle x^2 \rangle &= C \int_{-\infty}^{\infty} x^2 e^{-(x-\mu)^2/(2\sigma^2)}\,dx \\
&= \frac{1}{\sqrt{\pi}} \int_{-\infty}^{\infty} (\mu + \sqrt{2}\sigma y)^2 e^{-y^2}\,dy
\end{align}
$(a+b)^2 = a^2 + 2ab + b^2$ と展開する。
\begin{itemize}
    \item $\mu^2$ の項:$\frac{\mu^2}{\sqrt{\pi}} \int_{-\infty}^{\infty} e^{-y^2}\,dy = \mu^2 \cdot 1 = \mu^2$
    \item $2\mu\sqrt{2}\sigma y$ の項:$y e^{-y^2}$ は奇関数なので積分は 0
    \item $2\sigma^2 y^2$ の項:$\frac{2\sigma^2}{\sqrt{\pi}} \int_{-\infty}^{\infty} y^2 e^{-y^2}\,dy$ を計算する
\end{itemize}
$y^2 e^{-y^2}$ の積分は、部分積分を用いる。$u = y$、$dv = y e^{-y^2} dy$ とおく($y e^{-y^2}$ は $-\frac{1}{2}e^{-y^2}$ の微分なので積分しやすい。$u=y$ を微分すると $du=dy$ で次数が下がる)。$du = dy$、$v = -\frac{1}{2} e^{-y^2}$ であり、
\begin{align}
\int_{-\infty}^{\infty} y^2 e^{-y^2}\,dy &= \left[-\frac{y}{2}e^{-y^2}\right]_{-\infty}^{\infty} + \frac{1}{2}\int_{-\infty}^{\infty} e^{-y^2}\,dy = 0 + \frac{\sqrt{\pi}}{2} = \frac{\sqrt{\pi}}{2}
\end{align}
したがって
\begin{equation}
\langle x^2 \rangle = \mu^2 + \frac{2\sigma^2}{\sqrt{\pi}} \cdot \frac{\sqrt{\pi}}{2} = \mu^2 + \sigma^2
\end{equation}
よって
\begin{equation}
\langle \delta x^2 \rangle = \langle x^2 \rangle - \mu^2 = \sigma^2
\end{equation}

\textbf{答}:$\langle \delta x^2 \rangle = \sigma^2$

\begin{figure}[H]
    \centering
    \includegraphics[width=0.7\textwidth]{figures/ex1_gaussian.png}
    \caption{ガウス分布 $P(x) = \frac{1}{\sqrt{2\pi}\sigma} e^{-(x-\mu)^2/(2\sigma^2)}$ の概形。$\mu$ が中心、$\sigma$ が広がりを表す。}
    \label{fig:ex1_gaussian}
\end{figure}

\textbf{なぜ奇関数の積分は0なのか}:$f(-y) = -f(y)$ を満たす奇関数を $-\infty$ から $\infty$ まで積分すると、正の部分と負の部分が打ち消し合う。$(x-\mu) e^{-(x-\mu)^2/(2\sigma^2)}$ は $x=\mu$ に関して対称に符号が反転するため、全体の積分は0である。

\textbf{物理的考察}:$\mu$ は分布の中心(最頻値かつ平均)、$\sigma$ はゆらぎの典型的な幅である。$|\Delta x| \sim \sigma$ の範囲に約68\%の確率が含まれる。統計力学では、熱平衡でのエネルギーや粒子数のゆらぎがガウス型になることが多い。

%--------------------------------------
\subsection{V. ガンマ関数}

\subsubsection{この問題で学ぶこと}

階乗 $n!$ を積分で表す。部分積分により $\Gamma(n+1) = n \Gamma(n)$ という漸化式が得られ、$\Gamma(1)=1$ から帰納的に $n!$ が求まる。統計力学の状態数計算で必須の道具である。

\subsubsection{問題}

$n! = \int_0^{\infty} t^n e^{-t}\,dt$ が成り立つことを示せ。

\subsubsection{解答}

\paragraph{用語の説明}

\begin{itemize}
    \item \textbf{ガンマ関数 $\Gamma(z)$}:$\Gamma(z) = \int_0^{\infty} t^{z-1} e^{-t}\,dt$ で定義される。$z$ が正の整数 $n+1$ のとき $\Gamma(n+1) = n!$ となり、階乗 $n!$ の実数(さらには複素数)への一般化を与える。$n$ が自然数でない場合にも「階乗らしい」量を定義できる。
\end{itemize}

\textbf{統計力学での利用}:状態数の計算やスターリング近似(演習7)で $\Gamma$ 関数が現れる。

\paragraph{証明の方針}

$\Gamma(n+1) = \int_0^{\infty} t^n e^{-t}\,dt$ とおき、部分積分を繰り返して $\Gamma(n+1) = n!$ を示す。

\paragraph{部分積分の実行}

$u = t^n$、$dv = e^{-t} dt$ とおくと、$du = n t^{n-1} dt$、$v = -e^{-t}$ である。部分積分の公式 $\int u\,dv = uv - \int v\,du$ より
\begin{align}
\Gamma(n+1) &= \int_0^{\infty} t^n e^{-t}\,dt = \left[-t^n e^{-t}\right]_0^{\infty} - \int_0^{\infty} (-e^{-t}) \cdot n t^{n-1}\,dt \\
&= \left[-t^n e^{-t}\right]_0^{\infty} + n \int_0^{\infty} t^{n-1} e^{-t}\,dt = n \Gamma(n)
\end{align}

\textbf{境界項について}:$t \to 0$ のとき $t^n e^{-t} \to 0$($n \geq 1$ のとき)。$t \to \infty$ のとき、指数関数 $e^{-t}$ の減衰が $t^n$ の増大より速いため、$t^n e^{-t} \to 0$ である。よって $\left[-t^n e^{-t}\right]_0^{\infty} = 0 - 0 = 0$。

\paragraph{帰納的な計算}

$\Gamma(1) = \int_0^{\infty} e^{-t}\,dt = \left[-e^{-t}\right]_0^{\infty} = 0 - (-1) = 1$ である。

$n=1$ のとき:$\Gamma(2) = 1 \cdot \Gamma(1) = 1 = 1!$

$n=2$ のとき:$\Gamma(3) = 2 \cdot \Gamma(2) = 2 \cdot 1 = 2 = 2!$

$n=3$ のとき:$\Gamma(4) = 3 \cdot \Gamma(3) = 3 \cdot 2 = 6 = 3!$

一般の $n$ に対して、漸化式 $\Gamma(n+1) = n \Gamma(n)$ を繰り返し用いると
\begin{equation}
\Gamma(n+1) = n \cdot \Gamma(n) = n \cdot (n-1) \cdot \Gamma(n-1) = \cdots = n \cdot (n-1) \cdots 2 \cdot 1 \cdot \Gamma(1) = n!
\end{equation}
となる(帰納法:$\Gamma(k+1)=k!$ なら $\Gamma(k+2)=(k+1)\Gamma(k+1)=(k+1)!$)。

\textbf{$n=0$ の場合}:$0!$ は通常 1 と定義される。$\Gamma(1)=1$ なので、$\Gamma(0+1)=0!$ は 1 と整合する。

\textbf{なぜ部分積分で $u=t^n$, $dv=e^{-t}dt$ を選ぶのか}:$e^{-t}$ は積分しても $e^{-t}$ のまま(符号変化のみ)で扱いやすい。$t^n$ は微分すると次数が1つ減る。これを繰り返すと $t^{n-1}$, $t^{n-2}$ ... と下がり、最終的に $\Gamma(1) = \int e^{-t}dt = 1$ に帰着する。この「次数を下げる」戦略が漸化式 $\Gamma(n+1) = n\Gamma(n)$ を生む。

\textbf{物理的考察}:$\Gamma$ 関数は、$n$ が自然数でない場合(例:$n=1/2$ のとき $\Gamma(3/2)=\sqrt{\pi}/2$)にも意味を持つ。統計力学の分配関数や状態数の計算で、非整数の $n$ に対する「階乗らしい量」が必要になる場面がある。

% 演習2 (2025/10/24実施)
\section{演習2 (2025年10月24日実施)}

\subsection{I. ファンデルワールスの状態方程式}

\subsubsection{この問題で学ぶこと}

実在気体を理想気体より精密に記述するファンデルワールス方程式の性質。希薄極限で理想気体に帰着すること、臨界点の求め方、無次元化により物質に依存しない普遍形が得られること。

\subsubsection{問題}

ファンデルワールスの状態方程式
\begin{equation}
p = \frac{NRT}{V - bN} - \frac{aN^2}{V^2}
\end{equation}
について、$a$, $b$ は正の定数とする。
\begin{enumerate}
    \item 密度 $N/V \ll 1$ のとき、理想気体の状態方程式になることを示せ。
    \item 臨界温度 $T_c$、臨界体積 $V_c$、臨界圧力 $p_c$ を $R, N, a, b$ で表せ。
    \item 無次元化 $\tilde{p} = p/p_c$、$\tilde{V} = V/V_c$、$\tilde{T} = T/T_c$ で $\tilde{p} = \frac{8\tilde{T}}{3\tilde{V}-1} - \frac{3}{\tilde{V}^2}$ となることを示せ。
\end{enumerate}

\subsubsection{解答}

\paragraph{前提知識:理想気体と実在気体}

高校物理で学んだ理想気体の状態方程式 $pV = nRT$ を思い出してください。理想気体では、分子の大きさを無視し、分子間の引力も無視します。しかし実在の気体(窒素、酸素、二酸化炭素など)では、分子には有限の大きさがあり、分子同士には引き合う力(ファンデルワールス力)があります。

\textbf{ファンデルワールス方程式の直感的な理解:}

\begin{itemize}
    \item \textbf{第1項の $V - bN$}:分子自身の体積を考慮した補正。分子は「場所を取る」ので、実際に気体が動き回れる体積は $V$ より小さい。$b$ は1モルあたりの「排除体積」で、$bN$ だけ有効体積が減る。箱の中にボールを詰めると、ボールの体積分だけ空きスペースが減る、というイメージ。
    \item \textbf{第2項の $-aN^2/V^2$}:分子間引力の補正。分子同士が引き合うため、壁に衝突する際の勢いが弱まる(内側に引っ張られる)。その結果、理想気体より圧力は小さくなる。引力は互いに近い分子のペアに働くので、密度の2乗 $(N/V)^2$ に比例する。
\end{itemize}

\paragraph{導出の戦略}

問1では、密度 $N/V \ll 1$(希薄な気体)のとき、$bN/V$ と $aN^2/V^2$ が無視でき、理想気体に戻ることを示す。テイラー展開 $\frac{1}{1-x} \approx 1+x$ を用いる。

問2では、臨界点を $p$-$V$ 曲線の変曲点として求める。計算の都合で $x=1/V$ と置くと見通しが良くなる。

問3では、臨界量で無次元化することで、物質によらない普遍形が得られることを示す。

\paragraph{用語の説明}

\begin{itemize}
    \item \textbf{臨界点}:気体と液体の区別がなくなる点。$p$-$V$ 図で、$T$ が高いときは $p(V)$ は単調減少だが、$T$ を下げていくとある温度で「変曲点」が現れ、それ以下では極大・極小を持つ「S字形」になる。臨界点は、変曲点がちょうど現れる温度・圧力・体積($T_c$, $p_c$, $V_c$)である。
    \item \textbf{変曲点の条件}:曲線の傾きの変化がゼロになる点。数学的には $\frac{dp}{dV}=0$ かつ $\frac{d^2p}{dV^2}=0$(または $V$ の代わりに $x=1/V$ を使った形)。
\end{itemize}

\paragraph{問1: 理想気体への帰着}

密度 $n = N/V \ll 1$ のとき、気体は希薄で分子間の相互作用や分子の体積の影響が無視できる。この極限で理想気体の式 $pV = NRT$ に戻ることを示す。

\textbf{ステップ1}:$bN/V \ll 1$ なので、$(1 - bN/V)^{-1}$ を $bN/V$ の1次までテイラー展開する。$|x| \ll 1$ のとき $\frac{1}{1-x} = 1 + x + x^2 + \cdots \approx 1 + x$ である。
\begin{equation}
\frac{1}{V - bN} = \frac{1}{V(1 - bN/V)} = \frac{1}{V} \cdot \frac{1}{1 - bN/V} \approx \frac{1}{V}\left(1 + \frac{bN}{V}\right)
\end{equation}

\textbf{ステップ2}:第1項を展開し、第2項と合わせる。
\begin{align}
p &\approx \frac{NRT}{V}\left(1 + \frac{bN}{V}\right) - \frac{aN^2}{V^2} \\
&= \frac{NRT}{V} + \frac{NRT \cdot bN}{V^2} - \frac{aN^2}{V^2} = \frac{NRT}{V} + \frac{N^2}{V^2}(RTb - a)
\end{align}
密度 $N/V \ll 1$ のとき、$N^2/V^2 = (N/V)^2$ は $O((N/V)^2)$ であり、第1項の $NRT/V$ は $O(N/V)$ である。$N/V \ll 1$ なら $(N/V)^2 \ll N/V$ なので、第2項以降を無視し、
\begin{equation}
p \approx \frac{NRT}{V}
\end{equation}
となり、理想気体の状態方程式 $pV = NRT$ に帰着する。

\paragraph{問2: 臨界点}

\textbf{幾何学的意味}:$p$ を $V$ の関数としてみると、$T$ が高いときは $p(V)$ は単調減少である。$T$ を下げていくと、ある温度で $p(V)$ に変曲点が現れ、それ以上下げると極大・極小が生じて「S字形」になる。臨界点は、この変曲点がちょうど現れる温度・圧力・体積である。

計算の都合上 $x = 1/V$ とおく。$V = 1/x$ を代入して
\begin{equation}
p = \frac{NRT x}{1 - bNx} - aN^2 x^2
\end{equation}
臨界点では、$x$ を変数とするとき
\begin{equation}
\left(\frac{\partial p}{\partial x}\right)_T = 0, \quad \left(\frac{\partial^2 p}{\partial x^2}\right)_T = 0
\end{equation}
($p$-$V$ 図の変曲点条件を $x = 1/V$ で書き直したものである。)

\textbf{第1式}:$\frac{\partial}{\partial x}\left(\frac{x}{1-bNx}\right) = \frac{1 \cdot (1-bNx) - x \cdot (-bN)}{(1-bNx)^2} = \frac{1}{(1-bNx)^2}$ より
\begin{equation}
\frac{\partial p}{\partial x} = \frac{NRT}{(1-bNx)^2} - 2aN^2 x = 0 \quad \Rightarrow \quad NRT = 2aN^2 x (1-bNx)^2
\end{equation}

\textbf{第2式}:$\frac{\partial^2 p}{\partial x^2} = \frac{2NRT \cdot bN}{(1-bNx)^3} - 2aN^2 = 0$ より
\begin{equation}
NRT \cdot bN = aN^2 (1-bNx)^3
\end{equation}

第1式を第2式に代入:$2aN^2 x (1-bNx)^2 \cdot bN = aN^2 (1-bNx)^3$。$aN^2 \neq 0$、$1-bNx \neq 0$ で割って
\begin{equation}
2bN x (1-bNx) = (1-bNx)^2 \quad \Rightarrow \quad 2bN x = 1 - bNx \quad \Rightarrow \quad 3bN x_c = 1
\end{equation}
よって $x_c = 1/(3bN)$、すなわち $V_c = 1/x_c = 3bN$。

また $1 - bNx_c = 1 - 1/3 = 2/3$。第1式から
\begin{equation}
T_c = \frac{2aN^2 x_c (1-bNx_c)^2}{NR} = \frac{2aN \cdot (1/3) \cdot (2/3)^2}{R} = \frac{8a}{27Rb}
\end{equation}
$p_c$ は $x_c$、$T_c$ を元の式に代入して
\begin{equation}
p_c = \frac{NRT_c x_c}{1-bNx_c} - aN^2 x_c^2 = \frac{NRT_c/(3bN)}{2/3} - \frac{aN^2}{9b^2 N^2} = \frac{RT_c}{2b} - \frac{a}{9b^2}
\end{equation}
$T_c = 8a/(27Rb)$ を代入して
\begin{equation}
p_c = \frac{R}{2b} \cdot \frac{8a}{27Rb} - \frac{a}{9b^2} = \frac{4a}{27b^2} - \frac{3a}{27b^2} = \frac{a}{27b^2}
\end{equation}

\textbf{答}:$T_c = \frac{8a}{27Rb}$、$V_c = 3bN$、$p_c = \frac{a}{27b^2}$

\paragraph{問3: 無次元化}

$V = \tilde{V} V_c = \tilde{V} \cdot 3bN$、$T = \tilde{T} T_c = \tilde{T} \cdot 8a/(27Rb)$、$p = \tilde{p} p_c = \tilde{p} \cdot a/(27b^2)$ を代入する。
\begin{align}
\tilde{p} \frac{a}{27b^2} &= \frac{NR \cdot \tilde{T} \cdot 8a/(27Rb)}{\tilde{V} \cdot 3bN - bN} - \frac{aN^2}{(\tilde{V} \cdot 3bN)^2} \\
&= \frac{8a\tilde{T} N/(27Rb)}{bN(3\tilde{V}-1)} - \frac{a}{9\tilde{V}^2 b^2} = \frac{8a\tilde{T}}{27Rb^2(3\tilde{V}-1)} - \frac{a}{9\tilde{V}^2 b^2}
\end{align}
両辺に $27b^2/a$ をかけると
\begin{equation}
\tilde{p} = \frac{8\tilde{T}}{3\tilde{V}-1} - \frac{3}{\tilde{V}^2}
\end{equation}
物質によらない普遍的な形になる。

\textbf{直感的な理解}:無次元化により、水素でも二酸化炭素でも、$\tilde{p}$, $\tilde{V}$, $\tilde{T}$ で表せば同じ方程式に従う。この「対応状態の原理」は、異なる物質の振る舞いを比較するときに便利である。

\textbf{物理的意味と考察}:ファンデルワールス方程式は、理想気体の2つの欠点(分子の大きさ・分子間引力)をそれぞれ $b$, $a$ で補正したものである。臨界点以上の温度では気体と液体の区別がつかず、臨界点以下で凝縮が起こる。$a$, $b$ の典型的な値は、水($H_2O$)で $a \sim 0.55\,\mathrm{Pa{\cdot}m^6/mol^2}$、$b \sim 30 \times 10^{-6}\,\mathrm{m^3/mol}$ のオーダーである。

\begin{figure}[H]
    \centering
    \includegraphics[width=\textwidth]{figures/ex2_van_der_waals.png}
    \caption{ファンデルワールス状態方程式の無次元化。$\tilde{T} > 1$ で単調、$\tilde{T} \approx 1$ で変曲点、$\tilde{T} < 1$ で極値が現れる。}
    \label{fig:ex2_vdw}
\end{figure}

%--------------------------------------
\subsection{II. 気体のする仕事}

\subsubsection{この問題で学ぶこと}

気体が膨張するときにする仕事の計算方法。$p$-$V$ 図上で、経路に沿った積分 $W = \int p\,dV$ が仕事を表すこと。準静的過程ではこの積分が可能であること。ファンデルワールス気体では、条件によって「変な答え」が出る理由を理解する。

\subsubsection{問題}

$N$ mol の理想気体が準静的等温過程で $V_1$ から $V_2$($V_1 < V_2$)まで膨張するときの仕事 $W$ を求めよ。次にファンデルワールス気体で同様の計算をし、「変な答え」になる場合の意味を説明せよ。

\subsubsection{解答}

\paragraph{前提知識:仕事と $p$-$V$ 図}

力学で学んだ「仕事 $=$ 力 $\times$ 変位」を思い出してください。気体がピストンを押して膨張するとき、気体がする仕事は $W = \int F\,dx = \int p S\,dx = \int p\,dV$ である($S$ は断面積、$dV = S\,dx$)。

\textbf{準静的過程とは?}

変化が無限にゆっくりで、途中のどの瞬間も平衡状態とみなせる過程。このとき、圧力 $p$ が各瞬間で一意的に定まり、$p$-$V$ 図上の曲線に沿って積分できる。急速な膨張では、途中の状態が非平衡になり、単純な積分はできない。

\textbf{符号の確認}:膨張($V_2 > V_1$)のとき $dV > 0$、$W = \int p\,dV > 0$。気体は外界に仕事をする(ピストンを押す)。

\paragraph{導出の戦略}

理想気体では $p = NRT/V$ を代入して積分する。$\int dV/V = \ln V$ を用いる。ファンデルワールス気体では、$p$ の式をそのまま積分する。「変な答え」の2つのケース($V < bN$、相転移領域)を、物理的に説明する。

\paragraph{理想気体}

$p = NRT/V$ より。初学者向けに $\int dV/V = \ln V$ である($\frac{d}{dV}\ln V = 1/V$ の逆)。
\begin{equation}
W = \int_{V_1}^{V_2} p\,dV = NRT \int_{V_1}^{V_2} \frac{dV}{V} = NRT \left[\ln V\right]_{V_1}^{V_2} = NRT \ln\frac{V_2}{V_1}
\end{equation}

\paragraph{ファンデルワールス気体}

$p = \frac{NRT}{V-bN} - \frac{aN^2}{V^2}$ を積分する。
\begin{align}
W &= \int_{V_1}^{V_2} \frac{NRT}{V-bN}\,dV - \int_{V_1}^{V_2} \frac{aN^2}{V^2}\,dV \\
&= NRT \left[\ln(V-bN)\right]_{V_1}^{V_2} - aN^2 \left[-\frac{1}{V}\right]_{V_1}^{V_2} \\
&= NRT \ln\frac{V_2-bN}{V_1-bN} + aN^2\left(\frac{1}{V_2} - \frac{1}{V_1}\right)
\end{align}
$V_2 > V_1$ のとき $1/V_2 - 1/V_1 < 0$ なので、第2項は負。分子間引力の補正により、理想気体より取り出せる仕事は少なくなる。

\paragraph{「変な答え」の意味}

\begin{enumerate}
    \item \textbf{$V < bN$ の範囲}:$V - bN < 0$ だと式が意味を持たない($\ln(V-bN)$ が定義されない)。体積が分子の排除体積より小さい状態は、この模型では考えない。実際には液体や固体の領域である。
    \item \textbf{相転移領域}:$T < T_c$ のとき、$p(V)$ は単調でなく極大・極小を持つ「S字形」になる。しかし実在の気体では、気体と液体が共存する区間では圧力は一定(蒸気圧)であり、$p$-$V$ 図には水平線が現れる。ファンデルワールス式のS字形部分をそのまま積分すると、$V$ が増えるのに $p$ が減る区間があり、負の寄与が出て「変な答え」になる。正しくはマクスウェルの構成で水平線に置き換える必要がある。
\end{enumerate}

\textbf{直感的な理解}:$p$-$V$ 図で、曲線の下の面積が仕事である。理想気体の等温線は単調減少なので、面積は正。ファンデルワールスのS字形では、$V$ を増やす区間で $p$ が一時的に増える部分があり、その区間の「面積」が負になり得る。

\textbf{物理的意味と考察}:準静的過程では、気体は平衡状態の連続を経由する。相転移が起こる場合、気液共存状態では圧力が一定であり、ファンデルワールス式はその区間を正しく記述しない。実際の等温圧縮では、ある体積で凝縮が始まり、液体が増えていく間は圧力が変わらない。

%--------------------------------------
\subsection{III. 圧力の高さ依存性}

\subsubsection{この問題で学ぶこと}

大気圧が高度とともに減少する理由を、力のつり合いから理解する。理想気体の状態方程式と組み合わせて、指数関数的な圧力分布( barometric formula)を導く。変数分離による微分方程式の解法の典型例である。

\subsubsection{問題}

細長い容器に空気を入れ、地上に垂直に立てる。温度一定として、高さ $z$ における圧力 $p(z)$ の表式を導け。$p(z) = p_0 e^{-Mgz/(RT)}$ となることを示し、$T=290\,\mathrm{K}$ で圧力が半分になる高さを求めよ。

\subsubsection{解答}

\paragraph{前提知識:圧力と力のつり合い}

圧力 $p$ は「単位面積あたりに働く力」である。断面積 $S$ の面には、$pS$ の力が垂直に働く。静止している流体では、任意の部分に働く力の合力はゼロでなければならない(力のつり合い)。

\textbf{日常的な例}:プールに潜ると、深いほど耳が痛くなる。これは水深が増すと、その上の水の重さがのしかかり、圧力が増すためである。空気の柱も同様に、高い位置ほど「上の空気の重さ」が少ないので、圧力は低くなる。

\paragraph{導出の戦略}

\begin{enumerate}
    \item 高さ $z$ から $z+dz$ の薄い空気の層を考え、力のつり合いから $dp/dz = -\rho g$ を得る。
    \item 理想気体の状態方程式から $\rho = Mp/(RT)$ を導き、$dp/dz$ を $p$ の式で表す。
    \item 変数分離して積分し、$p(z) = p_0 e^{-Mgz/(RT)}$ を得る。
    \item $p(z)/p_0 = 1/2$ の条件から、圧力が半分になる高さを求める。
\end{enumerate}

\paragraph{問題の理解}

\textbf{なぜ高さで圧力が変わるか}:容器内の空気の柱を考える。高さ $z$ の位置にある空気には、その上の空気の重さがのしかかる。下にあるほど「上にのしかかる空気」が多いので、圧力は下の方が高くなる。微小な厚さ $dz$ の空気の層について、上向きの力(下面の圧力)と下向きの力(上面の圧力+層の重さ)のつり合いから微分方程式を立て、積分して $p(z)$ を求める。

\paragraph{力のつり合い}

図\ref{fig:ex2_pressure_concept}のように、高さ $z$ から $z+dz$ の薄い円筒(断面積 $S$)を考える。
\begin{itemize}
    \item 下面には上向きに $p(z) S$ の力
    \item 上面には下向きに $p(z+dz) S$ の力
    \item 円筒内の空気の質量 $m = \rho(z) S\,dz$ に重力 $mg$ が下向きに働く
\end{itemize}
つり合いの式:$p(z)S = p(z+dz)S + \rho(z) S\,dz \cdot g$。$p(z+dz) - p(z) \approx \frac{dp}{dz} dz$ より
\begin{equation}
-\frac{dp}{dz} = \rho(z) g
\end{equation}

\paragraph{密度の表式}

理想気体の状態方程式 $pV = NRT$ より、単位体積あたりのモル数は $n = N/V = p/(RT)$。密度 $\rho$ は質量/体積であり、$\rho = n M = (p/RT) \cdot M = M p/(RT)$($M$ はモル質量)。したがって
\begin{equation}
-\frac{dp}{dz} = \frac{Mg}{RT} p
\end{equation}
変数分離して $\frac{dp}{p} = -\frac{Mg}{RT} dz$。両辺を積分し、$z=0$ で $p = p_0$ とすると
\begin{equation}
\ln\frac{p(z)}{p_0} = -\frac{Mg}{RT} z \quad \Rightarrow \quad p(z) = p_0 \exp\left(-\frac{Mgz}{RT}\right)
\end{equation}

\paragraph{圧力が半分になる高さ}

$p(z)/p_0 = 1/2$ のとき $\exp(-Mgz/(RT)) = 1/2$。両辺の自然対数をとり
\begin{equation}
-\frac{Mgz}{RT} = \ln\frac{1}{2} = -\ln 2 \quad \Rightarrow \quad z = \frac{RT}{Mg} \ln 2
\end{equation}
$R = 8.3\,\mathrm{J/(mol{\cdot}K)}$、$T = 290\,\mathrm{K}$、$M = 29 \times 10^{-3}\,\mathrm{kg/mol}$、$g = 9.8\,\mathrm{m/s}^2$、$\ln 2 \approx 0.693$ を代入:
\begin{equation}
z = \frac{8.3 \times 290}{0.029 \times 9.8} \times 0.693 \approx 5840\,\mathrm{m} \approx 5.8\,\mathrm{km}
\end{equation}

\textbf{答}:約 5.8 km

\textbf{感覚}:5.8 km は富士山(約 3.8 km)より高く、成層圏(約 11 km)より低い。中程度の高山で気圧が半分程度になる目安である。

\textbf{補足}:実際の大気では温度が高度とともに変化するため、この結果とはずれる。高山の気圧が低いのは、上空の空気の重さが小さいためである。富士山(約 3.8 km)では気圧は約 0.6 気圧程度になる。

\textbf{変数分離の操作}:$\frac{dp}{p} = -\frac{Mg}{RT} dz$ で、左辺は $p$ のみ、右辺は $z$ のみの関数なので、両辺を積分できる。$\int \frac{dp}{p} = \ln p$、$\int dz = z$ である。積分定数は、境界条件 $z=0$ で $p=p_0$ から決める。

\textbf{直感的な理解}:指数関数 $e^{-Mgz/(RT)}$ は、$z$ が増えると減少する。$Mg/(RT)$ が大きい(分子が重い、温度が低い)ほど、圧力の減衰は速い。高山では気圧が低く、登山時に息が切れる一因である。

\textbf{物理的意味と考察}:約 5.8 km で圧力が半分になる。富士山(3776 m)では約 0.6 気圧、エベレスト(8849 m)では約 0.3 気圧程度である。実際の大気では温度が高度で変化するため、この単純な式とはずれる。対流圏では約 6.5 K/km の割合で温度が下がる(演習3-IV 参照)。

\begin{figure}[H]
    \centering
    \includegraphics[width=0.6\textwidth]{figures/ex2_pressure_height.png}
    \caption{圧力の高さ依存性 $p(z) = p_0 e^{-Mgz/(RT)}$。}
    \label{fig:ex2_pressure}
\end{figure}

\begin{figure}[H]
    \centering
    \includegraphics[width=0.5\textwidth]{figures/ex2_pressure_concept.png}
    \caption{微小円筒に作用する力。上下の圧力差が、円筒内の空気の重さとつり合う。}
    \label{fig:ex2_pressure_concept}
\end{figure}

%--------------------------------------
\subsection{IV. 示量変数と示強変数}

\subsubsection{この問題で学ぶこと}

熱力学で重要な「示量変数」と「示強変数」の違い。示量変数が「1次同次」であることから、オイラーの関係式 $F = \sum_i f_i X_i$ が導かれる。偏微分で得られる $f_i$ が示強変数であることを示す。

\subsubsection{問題}

示量変数 $F$ が $\lambda F(X_1,\ldots,X_n,x_1,\ldots) = F(\lambda X_1,\ldots,\lambda X_n,x_1,\ldots)$ を満たすとき、オイラーの関係式 $F = \sum_i f_i X_i$($f_i = \partial F/\partial X_i$)を示せ。また $f_i$ が示強変数であることを示せ。

\subsubsection{解答}

\paragraph{前提知識:示量変数と示強変数の違い}

熱力学では、系の性質を表す変数を2種類に分ける。

\textbf{示量変数(extensive)}:系の「大きさ」に比例する量。例:体積 $V$、内部エネルギー $U$、エントロピー $S$、粒子数 $N$。同じ物質の系を2つ合わせると、示量変数は2倍になる。コップの水を2つ合わせると、体積も質量も2倍になる。

\textbf{示強変数(intensive)}:系の「大きさ」によらない量。例:圧力 $p$、温度 $T$、化学ポテンシャル $\mu$、密度。平衡にある同じ物質どうしを接触させても、これらの値は変わらない。2つのコップの水を接触させても、温度は等しいまま(熱平衡)。

\textbf{数学的な定義}:示量変数 $F$ が $X_1,\ldots,X_n$ に依存するとき、$X_i$ をすべて $\lambda$ 倍にすると $F$ も $\lambda$ 倍になる:$F(\lambda X_1,\ldots,\lambda X_n, x_1,\ldots) = \lambda F(X_1,\ldots,X_n, x_1,\ldots)$。これを「1次同次」という。

\paragraph{導出の戦略}

オイラー関係式:1次同次関数 $F$ を $\lambda$ で微分し、$\lambda=1$ とおく。合成関数の微分と連鎖律を用いる。

$f_i$ が示強であること:$F(\lambda X) = \lambda F(X)$ を $X_j$ で偏微分し、$f_j(\lambda X) = f_j(X)$ を示す。

\paragraph{オイラーの関係式の導出}

$F(\lambda X_1,\ldots,\lambda X_n, x_1,\ldots) = \lambda F(X_1,\ldots,X_n, x_1,\ldots)$ の両辺を $\lambda$ で微分する。左辺は合成関数の微分により
\begin{equation}
\sum_{i=1}^n X_i \frac{\partial F}{\partial (\lambda X_i)}\bigg|_{(\lambda X_1,\ldots)} = \frac{\partial}{\partial \lambda}\left[\lambda F(X_1,\ldots,X_n)\right] = F(X_1,\ldots,X_n)
\end{equation}
$\lambda = 1$ とおくと $\frac{\partial F}{\partial (\lambda X_i)}\big|_{\lambda=1} = \frac{\partial F}{\partial X_i}$ なので
\begin{equation}
\sum_{i=1}^n X_i \frac{\partial F}{\partial X_i} = F
\end{equation}
$f_i = \frac{\partial F}{\partial X_i}$ とおけば $F = \sum_i f_i X_i$(オイラーの関係式)。

\paragraph{$f_i$ が示強変数であること}

示強変数は、$X_j$ を $\lambda$ 倍にしたとき値が変わらないものである。$f_i = \frac{\partial F}{\partial X_i}$ について、$F$ の示量性 $F(\lambda X) = \lambda F(X)$ を $X_j$ で偏微分すると
\begin{equation}
\lambda \frac{\partial F}{\partial (\lambda X_j)}\bigg|_{(\lambda X)} = \lambda \frac{\partial F}{\partial X_j}\bigg|_{(X)}
\end{equation}
左辺は $\frac{\partial F}{\partial X_j}\big|_{(\lambda X)}$ に等しい。よって $\frac{\partial F}{\partial X_j}(\lambda X) = \frac{\partial F}{\partial X_j}(X)$。すなわち $f_j$ は $X$ を $\lambda$ 倍しても変わらないから、示強変数である。

\paragraph{問3の具体例}

(i) $F = \sqrt{XY}$:$f_X = \frac{\sqrt{Y}}{2\sqrt{X}}$、$f_Y = \frac{\sqrt{X}}{2\sqrt{Y}}$。$f_X X + f_Y Y = \frac{1}{2}\sqrt{XY} + \frac{1}{2}\sqrt{XY} = \sqrt{XY} = F$。$F(\lambda X, \lambda Y) = \sqrt{\lambda X \cdot \lambda Y} = \lambda \sqrt{XY} = \lambda F$ なので示量変数。

(ii) $F = \ln(X/Y)$:$F(\lambda X, \lambda Y) = \ln(\lambda X/(\lambda Y)) = \ln(X/Y) = F(X,Y)$ より、$F$ は $\lambda$ 倍しても変わらない。これは示強変数である(示量変数の定義 $F(\lambda X) = \lambda F(X)$ は満たさない)。

(iii) $F = X^2/\sqrt{X^2+Y^2}$:$f_X = \frac{2X(X^2+Y^2)^{1/2} - X^2 \cdot X(X^2+Y^2)^{-1/2}}{X^2+Y^2} = \frac{X(X^2+2Y^2)}{(X^2+Y^2)^{3/2}}$ など。$f_X X + f_Y Y = F$ の確認は計算により行う。$F(\lambda X, \lambda Y) = \lambda^2 F(X,Y)/\lambda = \lambda F(X,Y)$ なので示量変数。

\textbf{物理的意味と考察}:熱力学では $U = TS - pV + \mu N$ がオイラー関係の例である($S$, $V$, $N$ が示量、$T$, $-p$, $\mu$ が示強)。示量変数は「強度×示量」の和で表せる、という構造が熱力学の基礎にある。

% 演習3 (2025/11/07実施)
\section{演習3 (2025年11月7日実施)}

\subsection{I. 理想気体の2状態を結ぶ3つの経路}

\subsubsection{この問題で学ぶこと}

状態量(内部エネルギー $U$)と過程量(仕事 $W$、熱 $Q$)の違い。$\Delta U$ は経路に依らないが、$W$ と $Q$ は経路で異なる。熱力学第1法則 $\Delta U = Q - W$ の適用。

\subsubsection{問題}

状態1 $(T_1, p_1, V_1)$ から状態2 $(T_2, p_2, V_2)$ へ至る3つの経路について、$\Delta U$、$W$、$Q$ を求めよ。
\begin{enumerate}
    \item 経路1:定積過程 $1 \to A$、定圧過程 $A \to 2$
    \item 経路2:等温過程 $1 \to B$、定積過程 $B \to 2$
    \item 経路3:断熱過程 $1 \to C$、定積過程 $C \to 2$
\end{enumerate}

\subsubsection{解答}

\paragraph{前提知識:熱力学第1法則と状態量・過程量}

エネルギー保存則の熱力学版が第1法則である。系の内部エネルギー $U$ の変化 $\Delta U$ は、系が吸収した熱 $Q$ から系が外界にした仕事 $W$ を引いたものに等しい:
\begin{equation}
\Delta U = Q - W
\end{equation}
\textbf{符号の慣習}:$Q > 0$ は系が熱を吸収、$W > 0$ は系が外界に仕事をする(膨張でピストンを押す場合)。

\textbf{状態量と過程量の違い(重要)}:

\begin{itemize}
    \item \textbf{状態量}:現在の状態($T$, $p$, $V$ など)だけで決まり、そこへ至る経路に依存しない。内部エネルギー $U$、エントロピー $S$ は状態量。
    \item \textbf{過程量}:どの経路で変化したかに依存する。熱 $Q$ と仕事 $W$ は過程量。同じ状態1から状態2へ行くのに、経路が違えば $Q$ と $W$ は異なる。しかし $Q - W = \Delta U$ は常に同じ($\Delta U$ が状態量だから)。
\end{itemize}

\paragraph{用語の説明}

\begin{itemize}
    \item \textbf{定積過程}:体積一定。$dV = 0$ より $W = \int p\,dV = 0$。ピストンを固定したまま加熱するイメージ。
    \item \textbf{定圧過程}:圧力一定。$W = p \Delta V$。大気圧にさらして膨張・圧縮させる。
    \item \textbf{等温過程}:温度一定。理想気体では $U$ が $T$ のみに依存するので $\Delta U = 0$、よって $Q = W$。
    \item \textbf{断熱過程}:熱の出入りなし。$Q = 0$、したがって $\Delta U = -W$。断熱材で囲んだ変化。
\end{itemize}

\paragraph{導出の戦略}

理想気体では $U = cNRT$ なので、$\Delta U = cNR(T_2 - T_1)$ は経路に依らない。各経路で、中間点の状態を状態方程式から求め、各過程の $W$, $Q$ を第1法則と過程の定義から計算する。

\paragraph{基本事項}

理想気体では $U = c N R T$($c$ は $C_V/(NR)$ に相当する無次元定数、単原子なら $c = 3/2$)であり、$U$ は温度 $T$ のみの関数である。したがって、どの経路を通っても状態1から状態2への内部エネルギー変化は
\begin{equation}
\Delta U = cNR(T_2 - T_1)
\end{equation}
で、経路に依存しない。一方、$Q$ と $W$ は過程により異なる。

\paragraph{経路1:定積 $\to$ 定圧}

中間点Aは $V_A = V_1$(状態1と同じ体積)、$T_A = T_2$(状態2と同じ温度)。状態方程式 $pV = NRT$ より $p_A = NR T_2 / V_1$。

\textbf{過程 $1 \to A$(定積)}:$W_{1A} = 0$。第1法則より $Q_{1A} = \Delta U_{1A} = cNR(T_2 - T_1)$。

\textbf{過程 $A \to 2$(定圧)}:$W_{A2} = p_2(V_2 - V_1)$。$\Delta U_{A2} = cNR(T_2 - T_2) = 0$ なので $Q_{A2} = W_{A2} = p_2(V_2 - V_1)$。

\textbf{合計}:$W_{(1)} = p_2(V_2 - V_1)$、$Q_{(1)} = cNR(T_2 - T_1) + p_2(V_2 - V_1)$。

\textbf{なぜ $V_A=V_1$、$T_A=T_2$ か}:経路1の定義は「定積過程 $1\to A$、定圧過程 $A\to 2$」。定積では体積を変えないので $V_A = V_1$。次に定圧で $A\to 2$ とするので、$A$ では圧力 $p_A$ で、状態2に至る。状態2の温度は $T_2$ なので、定積 $1\to A$ の終点 $A$ では $T_A = T_2$ とするのが自然(定積で $T_1$ から $T_2$ まで温める)。

\paragraph{経路2:等温 $\to$ 定積}

中間点Bは $T_B = T_1$、$p_B = p_2$。$V_B = NR T_1 / p_2$。

\textbf{過程 $1 \to B$(等温)}:$\Delta U_{1B} = 0$。$W_{1B} = \int_{V_1}^{V_B} p\,dV = NRT_1 \int_{V_1}^{V_B} \frac{dV}{V} = NRT_1 \ln(V_B/V_1)$。$pV = NRT$ より $V_B/V_1 = p_1/p_2$ なので $W_{1B} = NRT_1 \ln(p_1/p_2)$。$Q_{1B} = W_{1B}$。

\textbf{過程 $B \to 2$(定積)}:$W_{B2} = 0$。$Q_{B2} = \Delta U_{B2} = cNR(T_2 - T_1)$。

\textbf{合計}:$W_{(2)} = NRT_1 \ln(p_1/p_2)$、$Q_{(2)} = NRT_1 \ln(p_1/p_2) + cNR(T_2 - T_1)$。

\textbf{なぜ $T_B=T_1$、$p_B=p_2$ か}:経路2は「等温過程 $1\to B$、定積過程 $B\to 2$」。等温では温度一定なので $T_B = T_1$。次に定積で $B\to 2$ なので、$B$ の体積は状態2の体積 $V_2$ に等しい。また、定積過程で状態2($p_2,V_2,T_2$)に至るため、$B$ では $p_B = p_2$、$V_B = V_2$ である。状態方程式から $V_B = NRT_B/p_B = NRT_1/p_2$。

\paragraph{経路3:断熱 $\to$ 定積}

中間点Cは $V_C = V_2$、$T_C = T_2$(断熱過程の終点が状態2の体積・温度である)。断熱線 $T V^{\gamma-1} = \mathrm{const}$ より $T_1 V_1^{\gamma-1} = T_2 V_2^{\gamma-1}$ が成り立つ(状態1と2の関係によっては、経路が存在しない場合もあるが、ここでは状態2が断熱線上にあると仮定)。

\textbf{過程 $1 \to C$(断熱)}:$Q_{1C} = 0$。$\Delta U_{1C} = cNR(T_2 - T_1)$ より $W_{1C} = -\Delta U_{1C} = -cNR(T_2 - T_1)$(系は膨張して仕事をしている)。

\textbf{過程 $C \to 2$(定積)}:$W_{C2} = 0$。$Q_{C2} = \Delta U_{C2} = cNR(T_2 - T_2) = 0$(Cと2は同じ状態)。

\textbf{合計}:$W_{(3)} = -cNR(T_2 - T_1)$、$Q_{(3)} = cNR(T_2 - T_1)$。

\textbf{確認}:いずれの経路でも $\Delta U = cNR(T_2 - T_1)$ であり、$Q - W = \Delta U$ が成り立つ。$W$ と $Q$ は経路により異なるが、その差 $Q - W$ は常に $\Delta U$ に等しい。

\textbf{まとめ}:内部エネルギー $U$ は状態量(経路に依存しない)であり、仕事 $W$ と熱 $Q$ は過程量(経路に依存する)である。この対比が本問題の教育的意図である。

\textbf{比較表}:各経路の $W$、$Q$、$\Delta U$ をまとめると次のようになる。

\begin{center}
\begin{tabular}{lccc}
\hline
経路 & $W$ & $Q$ & $\Delta U$ \\
\hline
経路1(定積→定圧) & $p_2(V_2-V_1)$ & $cNR(T_2-T_1)+p_2(V_2-V_1)$ & $cNR(T_2-T_1)$ \\
経路2(等温→定積) & $NRT_1\ln(p_1/p_2)$ & $NRT_1\ln(p_1/p_2)+cNR(T_2-T_1)$ & $cNR(T_2-T_1)$ \\
経路3(断熱→定積) & $-cNR(T_2-T_1)$ & $cNR(T_2-T_1)$ & $cNR(T_2-T_1)$ \\
\hline
\end{tabular}
\end{center}
いずれも $\Delta U$ は同じで、$Q - W = \Delta U$ が成り立つ。図\ref{fig:ex3_paths}では経路1が緑、経路2が青、経路3が赤で示されている。

\begin{figure}[H]
    \centering
    \includegraphics[width=0.8\textwidth]{figures/ex3_pv_paths.png}
    \caption{3つの経路のT-V図。経路1は緑、経路2は青、経路3は赤。}
    \label{fig:ex3_paths}
\end{figure}

%--------------------------------------
\subsection{II. 2つの理想気体の熱接触}

\subsubsection{この問題で学ぶこと}

温度の異なる2つの系が熱接触すると、熱が高温から低温へ流れ、最終的に両者の温度が等しくなる(熱平衡)。断熱条件では全エネルギーが保存するため、平衡温度が決まる。

\subsubsection{問題}

2つの理想気体(系1: $(T_1, V)$、系2: $(T_2, V)$)が断熱壁で隔てられている。透熱壁に置き換えた後の平衡温度 $T$ を求めよ。

\subsubsection{解答}

\paragraph{前提知識:透熱壁と断熱壁}

\textbf{断熱壁}:熱を通さない壁。2つの系を隔てると、熱のやりとりは起こらない。

\textbf{透熱壁}:熱を通す壁。2つの系を接触させると、温度の高い方から低い方へ熱が流れる(第2法則)。やがて両方の温度が等しくなり、熱平衡に達する。

\textbf{日常的な例}:熱いお茶と冷たい水を混ぜると、ぬるい温度になる。これは熱が高温側から低温側へ流れ、全体のエネルギーが保存しながら平衡に達するためである。

\paragraph{導出の戦略}

全体は断熱されているので、系1と系2の内部エネルギーの和は変化しない。平衡では両者の温度が $T$ に等しくなる。$U_1 + U_2 =$ 一定、かつ $U = cNRT$ の形から、$T$ を求める。

\paragraph{エネルギー保存}

系1のモル数を $N_1$、系2のモル数を $N_2$ とする。各系とも $U = cNR T$ の形だから、平衡温度を $T$ とすると
\begin{equation}
U_1 + U_2 = cN_1 R T_1 + cN_2 R T_2 = c(N_1 + N_2) R T
\end{equation}
両方の系が同じモル数 $N$、同じ体積 $V$ の場合、$N_1 = N_2 = N$ として
\begin{equation}
2cNR T = cNR(T_1 + T_2) \quad \Rightarrow \quad T = \frac{T_1 + T_2}{2}
\end{equation}

\textbf{答}:$T = \frac{T_1 + T_2}{2}$

\textbf{設定の確認}:$N_1 = N_2 = N$、両系とも体積 $V$ が同じ、$c$ も同じと仮定している。比熱が異なる場合は $T = (c_1 N_1 T_1 + c_2 N_2 T_2)/(c_1 N_1 + c_2 N_2)$ となる。

\textbf{平衡に達するまで}:熱接触直後は非平衡であるが、本問では最終的な平衡状態のみを扱う。熱は高温から低温へ流れる(第2法則)。この過程でエントロピーは増大する(演習5で詳述)。

\textbf{直感的な理解}:$T_1 > T_2$ なら、系1から系2へ熱が流れ、系1は冷え系2は温まる。平衡温度 $T$ は $T_2 < T < T_1$ の範囲にあり、同じモル数・同じ $c$ のときは算術平均になる。

\textbf{物理的意味と考察}:モル数や比熱が異なる場合は $T = (c_1 N_1 T_1 + c_2 N_2 T_2)/(c_1 N_1 + c_2 N_2)$ となる。熱容量が大きい系ほど、平衡温度への寄与が大きい。

%--------------------------------------
\subsection{III. Mayerサイクル}

\subsubsection{この問題で学ぶこと}

Mayerサイクルは、自由断熱膨張・等圧圧縮・等積加熱の3過程からなる。1サイクルで $\Delta U = 0$ であることと、各過程のエネルギー収支から、$C_p - C_V = R$(Mayerの関係式)を導く。

\subsubsection{問題}

自由断熱膨張 $1 \to 2$、等圧過程 $2 \to 3$、等積過程 $3 \to 1$ からなるMayerサイクルについて、各過程の $W$、$Q$、$\Delta U$ を求め、$C_p = C_V + R$ を確認せよ。

\subsubsection{解答}

\paragraph{前提知識:自由断熱膨張}

\textbf{自由膨張}:壁を瞬間的に取り除き、気体が真空に膨張する。準静的ではなく、途中は非平衡。気体は外界に仕事をしない(真空を押しても仕事はゼロ)ので $W = 0$。断熱なので $Q = 0$。第1法則より $\Delta U = 0$。理想気体では $U$ が $T$ のみに依存するので、$T$ は不変。

\textbf{サイクル}:状態1に戻る閉じた経路。1サイクルで $\Delta U_{\mathrm{tot}} = 0$(出発点と到着点が同じだから)。

\paragraph{導出の戦略}

各過程の $W$, $Q$, $\Delta U$ を個別に求める。1サイクルで $Q_{23} + Q_{31} = W_{\mathrm{cyc}}$ が成り立つ($\Delta U = 0$ より)。$Q_{23}$, $Q_{31}$ を $C_p$, $C_V$ で表し、$W_{\mathrm{cyc}}$ を状態方程式から求めて、$C_p - C_V = R$ を得る。

\paragraph{$C_p$ と $C_V$ の定義}

定圧熱容量 $C_p = (\partial Q/\partial T)_p$:圧力一定で温度を $dT$ 上げるのに必要な熱 $dQ$。定積熱容量 $C_V = (\partial Q/\partial T)_V$:体積一定で温度を $dT$ 上げるのに必要な熱。理想気体では $C_V = cNR$($c=3/2$ で単原子、$c=5/2$ で二原子分子常温)。

\paragraph{過程の整理}

状態1: $(p_1, V_1, T_1)$、状態2: $(p_2, V_2, T_1)$($V_2 > V_1$、自由膨張で温度不変)、状態3: $(p_2, V_1, T_3)$。状態3は等圧圧縮 $2\to 3$ で体積を $V_2$ から $V_1$ に戻した点であり、$p_2$、$V_1$ を満たす。$1 \to 2$ は自由膨張(瞬間的、非準静的)、$2 \to 3$ は等圧圧縮、$3 \to 1$ は等積加熱。

\paragraph{各過程の $W$、$Q$、$\Delta U$}

\textbf{$1 \to 2$(自由断熱膨張)}:ピストンを一気に引くので、気体は仕事をしない($W = 0$)。断熱なので $Q = 0$。第1法則より $\Delta U = 0$。理想気体では $U$ が $T$ のみに依存するので、$T_2 = T_1$ である。

\textbf{$2 \to 3$(等圧過程)}:$W_{23} = p_2(V_1 - V_2) < 0$(外界が気体に仕事)。$\Delta U_{23} = cNR(T_3 - T_2) = cNR(T_3 - T_1)$。$Q_{23} = \Delta U_{23} + W_{23}$。定圧過程では $Q_{23} = C_p(T_3 - T_2) = C_p(T_3 - T_1)$。

\textbf{$3 \to 1$(等積過程)}:$W_{31} = 0$。$\Delta U_{31} = cNR(T_1 - T_3)$。$Q_{31} = C_V(T_1 - T_3)$。

\paragraph{$C_p = C_V + R$ の導出}

1サイクルで $\Delta U_{\mathrm{tot}} = 0$ なので $Q_{23} + Q_{31} = W_{\mathrm{cyc}}$(外界がした正味の仕事)。また、系が吸収する熱の正味は $Q_{23} + Q_{31}$ である。

$Q_{23} = C_p(T_3 - T_1)$、$Q_{31} = C_V(T_1 - T_3) = -C_V(T_3 - T_1)$。よって $Q_{23} + Q_{31} = (C_p - C_V)(T_3 - T_1)$。

\textbf{$T_3 < T_1$ の理由}:状態2と3で $p_2$ が同じ。$p_2 V_1 = NR T_3$、$p_2 V_2 = NR T_1$。$V_1 < V_2$ だから $T_3 < T_1$ である。

$T_3$ と $T_1$ の関係を求める。状態2と3で $p_2 V_1 = NR T_3$、$p_2 V_2 = NR T_2 = NR T_1$。したがって $T_3/T_1 = V_1/V_2 < 1$、つまり $T_3 < T_1$。ゆえに $T_3 - T_1 < 0$。

1サイクルの仕事:$W_{\mathrm{cyc}} = W_{23} + W_{31} = p_2(V_1 - V_2)$。$Q_{23} + Q_{31} = W_{\mathrm{cyc}}$ より
\begin{equation}
(C_p - C_V)(T_3 - T_1) = p_2(V_1 - V_2)
\end{equation}
$p_2(V_1 - V_2) = NR(T_1 - T_3)$(状態2,3の $pV = NRT$ から)なので
\begin{equation}
(C_p - C_V)(T_3 - T_1) = NR(T_1 - T_3) = -NR(T_3 - T_1)
\end{equation}
$T_3 \neq T_1$ で割って $C_p - C_V = NR$。1モルあたりでは $C_p - C_V = R$。すなわち $C_p = C_V + R$(Mayerの関係式)。

\textbf{直感的な理解}:定圧で温めると、体積も膨張する。その膨張に伴う仕事分、定積で温めるより多くの熱が必要。これが $C_p > C_V$ の理由である。

\textbf{物理的意味と考察}:Mayerの関係式 $C_p = C_V + R$ は、1モルあたりの定圧熱容量が定積熱容量より気体定数 $R$ だけ大きいことを示す。単原子気体では $C_V = \frac{3}{2}R$、$C_p = \frac{5}{2}R$。二原子分子(常温)では $C_V = \frac{5}{2}R$、$C_p = \frac{7}{2}R$。

\begin{figure}[H]
    \centering
    \includegraphics[width=0.7\textwidth]{figures/ex3_mayer_cycle.png}
    \caption{Mayerサイクル。$1 \to 2$ は自由断熱膨張、$2 \to 3$ は等圧、$3 \to 1$ は等積。}
    \label{fig:ex3_mayer}
\end{figure}

%--------------------------------------
\subsection{IV. 大気の断熱減率}

\subsubsection{この問題で学ぶこと}

空気の塊が上昇して膨張するとき、周囲と熱交換がなければ断熱過程となる。断熱膨張で温度が下がる(断熱減率)。ポアソンの式 $pV^\gamma = \mathrm{const}$ と気圧の高度依存を組み合わせて、温度の高度勾配を導く。

\subsubsection{問題}

断熱過程で $pV^\gamma = \mathrm{const}$、$T p^{1/\gamma - 1} = \mathrm{const}$ が成り立つことを示し、気圧の式と組み合わせて $dT/dz = -(\gamma-1)Mg/(\gamma R)$ を導け。$\gamma = 1.41$ のとき、高度1kmあたりの温度低下を見積もれ。

\subsubsection{解答}

\paragraph{前提知識:断熱減率とは}

\textbf{なぜ山頂は寒いか}:地表で温められた空気が上昇すると、気圧が下がり膨張する。膨張するとき気体は仕事をするが、周囲との熱交換がなければ(断熱)、内部エネルギーが減って温度が下がる。これが「断熱減率」の直感的な理由である。

\textbf{比熱比 $\gamma$}:$\gamma = C_p/C_V$。単原子理想気体では $\gamma = 5/3$、二原子分子(常温)では $\gamma \approx 1.4$。

\paragraph{導出の戦略}

\begin{enumerate}
    \item 断熱過程で $dQ = 0$、第1法則 $dU = -p\,dV$ と $dU = C_V dT$ から、$TV^{\gamma-1} = \mathrm{const}$ および $pV^\gamma = \mathrm{const}$(ポアソン式)を導く。
    \item 状態方程式で $V$ を消し、$T p^{1/\gamma - 1} = \mathrm{const}$ を得る。
    \item この式を $z$ で微分し、$dp/dz = -Mgp/(RT)$ を代入して $dT/dz$ を求める。
\end{enumerate}

\paragraph{ポアソンの式 $pV^\gamma = \mathrm{const}$ の導出}

断熱過程では $dQ = 0$ なので、第1法則 $dU = dQ - p\,dV$ より $dU = -p\,dV$。理想気体で $dU = C_V dT$、$p = NRT/V$ だから
\begin{equation}
C_V dT = -\frac{NRT}{V} dV \quad \Rightarrow \quad \frac{dT}{T} + \frac{NR}{C_V} \frac{dV}{V} = 0
\end{equation}
$\gamma - 1 = C_p/C_V - 1 = (C_p - C_V)/C_V = R/C_V$(1モルあたり。Mayerの関係 $C_p - C_V = R$ を用いた)。$N$ モルでは $\frac{NR}{C_V} = \gamma - 1$。積分して
\begin{equation}
\ln T + (\gamma - 1) \ln V = \mathrm{const} \quad \Rightarrow \quad T V^{\gamma-1} = \mathrm{const}
\end{equation}
$pV = NRT$ より $T = pV/(NR)$ を代入すると $p V^{\gamma} = \mathrm{const}$(ポアソンの式)。

\paragraph{$T p^{1/\gamma - 1} = \mathrm{const}$}

$pV^\gamma = \mathrm{const}$ と $pV = NRT$ から $V$ を消す。$V = NRT/p$ なので $(NRT/p) \cdot p^\gamma = \mathrm{const}$、すなわち $T p^{\gamma-1}/p = T p^{\gamma-2} = \mathrm{const}$? 正しくは $p V^\gamma = p (NRT/p)^\gamma = p^{1-\gamma} (NRT)^\gamma = \mathrm{const}$。$T^\gamma p^{1-\gamma} = \mathrm{const}$。両辺を $1/\gamma$ 乗して $T p^{(1-\gamma)/\gamma} = \mathrm{const}$。$(1-\gamma)/\gamma = 1/\gamma - 1$ なので $T p^{1/\gamma - 1} = \mathrm{const}$。

\paragraph{微分形}

$\ln T + (1/\gamma - 1) \ln p = \mathrm{const}$ を $z$ で微分して
\begin{equation}
\frac{1}{T} \frac{dT}{dz} + \left(\frac{1}{\gamma} - 1\right) \frac{1}{p} \frac{dp}{dz} = 0
\end{equation}
気圧の式 $dp/dz = -Mgp/(RT)$ を代入して
\begin{equation}
\frac{dT}{dz} = -T \left(\frac{1}{\gamma} - 1\right) \frac{1}{p} \cdot \left(-\frac{Mgp}{RT}\right) = \frac{(\gamma-1)Mg}{\gamma R} \cdot \frac{T}{p} \cdot \frac{p}{RT} \cdot \ldots
\end{equation}
計算:$T p^{(1-\gamma)/\gamma} = \mathrm{const}$ の両辺の対数をとり $z$ で微分する。
\begin{equation}
\frac{1}{T} \frac{dT}{dz} + \frac{1-\gamma}{\gamma} \cdot \frac{1}{p} \frac{dp}{dz} = 0
\end{equation}
気圧の式 $\frac{dp}{dz} = -\frac{Mgp}{RT}$ より $\frac{1}{p}\frac{dp}{dz} = -\frac{Mg}{RT}$。代入して
\begin{equation}
\frac{1}{T} \frac{dT}{dz} = -\frac{1-\gamma}{\gamma} \cdot \left(-\frac{Mg}{RT}\right) = \frac{1-\gamma}{\gamma} \cdot \frac{Mg}{RT}
\end{equation}
$1-\gamma < 0$ なので右辺は負。よって $\frac{dT}{dz} = T \cdot \frac{1-\gamma}{\gamma} \cdot \frac{Mg}{RT} = \frac{(1-\gamma)Mg}{\gamma R} = -\frac{(\gamma-1)Mg}{\gamma R}$
\begin{equation}
\frac{dT}{dz} = -\frac{(\gamma-1)Mg}{\gamma R}
\end{equation}

\paragraph{数値計算}

$\gamma = 1.41$、$M = 28.9 \times 10^{-3}\,\mathrm{kg/mol}$、$g = 9.8\,\mathrm{m/s}^2$、$R = 8.32\,\mathrm{J/(mol{\cdot}K)}$ として
\begin{equation}
\left|\frac{dT}{dz}\right| = \frac{0.41 \times 28.9 \times 10^{-3} \times 9.8}{1.41 \times 8.32} \approx 9.8 \times 10^{-3}\,\mathrm{K/m} \approx 9.8\,\mathrm{K/km}
\end{equation}
高度が1 km 上がるごとに、約 10 K 温度が下がる。

\paragraph{現実との対応}

\textbf{環境減率と断熱減率の違い}:環境減率(約 6.5 K/km)は実際の大気の観測値。断熱減率(約 10 K/km)は、空気の塊が断熱的に上昇したときの理論値。水蒸気の凝結による潜熱放出で、実際の大気は断熱減率よりゆるやかに温度が下がる。実際の大気の「環境 lapse rate」は約 6.5 K/km で、断熱減率(約 10 K/km)より小さい。これは、水蒸気の凝結による潜熱放出や、地表からの放射などが影響するためである。それでも、乾いた空気の塊が上昇するときの冷却の目安として断熱減率は重要である。

\begin{figure}[H]
    \centering
    \includegraphics[width=0.6\textwidth]{figures/ex3_adiabatic_lapse.png}
    \caption{大気の断熱減率に従う温度の高度分布。}
    \label{fig:ex3_adiabatic}
\end{figure}

% 演習4 (2025/11/21実施)
\section{演習4 (2025年11月21日実施)}

\subsection{I. 最大仕事の原理}

\subsubsection{この問題で学ぶこと}

同じ初期状態から同じ最終状態へ変化させるとき、準静的過程で得られる仕事が最大である。自由膨張では仕事はゼロ、断熱+等温の2段階でも準静的等温過程より小さい仕事しか得られない。

\subsubsection{問題}

$N$ mol の理想気体を $(T, V_i)$ から $(T, V_f)$($V_i < V_f$)へ変化させる。
\begin{enumerate}
    \item 準静的等温過程でする仕事 $W_{\mathrm{IQS}}$ を求めよ。
    \item 自由膨張では $W = 0$ で $W \leq W_{\mathrm{IQS}}$ を示せ。
    \item 準静的断熱過程で $(T', V_f)$ にしたときの仕事 $W_{\mathrm{AQS}}$ を求めよ。
    \item 断熱壁を除き温度 $T$ の熱源と接触させ $(T, V_f)$ にするときの仕事を求めよ。
    \item $W_{\mathrm{AQS}} + W' \leq W_{\mathrm{IQS}}$ を示せ。
\end{enumerate}

\subsubsection{解答}

\paragraph{前提知識:最大仕事の原理}

熱力学第2法則の一つの帰結として、同じ初期状態から同じ最終状態へ変化させるとき、準静的(可逆)過程で得られる仕事が最大である。非準静的過程では、摩擦や乱流などでエネルギーが散逸し、得られる仕事は小さいかゼロになる。

\textbf{直感的な理解}:ゆっくり慎重に膨張させれば、ピストンの動きに伴う仕事を最大限取り出せる。一気に壁を取れば(自由膨張)、気体は仕事をせずに膨張し、$W = 0$。

\paragraph{導出の戦略}

問1:準静的等温では $p = NRT/V$ を積分。問2:自由膨張では $W = 0$、$W_{\mathrm{IQS}} > 0$ だから $W \leq W_{\mathrm{IQS}}$。問3:断熱過程で $W_{\mathrm{AQS}} = -\Delta U$。断熱線 $T V^{\gamma-1} = \mathrm{const}$(準静的断熱で成り立つ)から $T'$ を求める。問4:「断熱壁を除く」とは、熱源(温度 $T$)と接触させて熱平衡にさせる操作。体積 $V_f$ のまま温度が $T'$ から $T$ へ変化する定積過程となり $W' = 0$。問5:$W_{\mathrm{tot}} = W_{\mathrm{AQS}}$ と $W_{\mathrm{IQS}}$ を比較し、不等式を示す。$x = V_i/V_f$ とおき、$f(x)$ の増減を $f'(x)$ の符号で調べる。

\textbf{全体のストーリー}:問1で準静的等温の最大仕事 $W_{\mathrm{IQS}}$ を求める。問2で自由膨張では $W=0$ であることを示す。問3--5で、断熱+等温の2段階では $W_{\mathrm{AQS}} < W_{\mathrm{IQS}}$ となり、準静的等温より少ない仕事しか得られない。

\paragraph{問1}

準静的等温過程では $p = NRT/V$ より
\begin{equation}
W_{\mathrm{IQS}} = \int_{V_i}^{V_f} p\,dV = NRT \int_{V_i}^{V_f} \frac{dV}{V} = NRT \ln\frac{V_f}{V_i}
\end{equation}
$V_f > V_i$ なので $W_{\mathrm{IQS}} > 0$。

\paragraph{問2}

自由膨張では気体が仕事をしないので $W = 0$。$W_{\mathrm{IQS}} > 0$ であるから $W \leq W_{\mathrm{IQS}}$ が成り立つ。

\paragraph{問3}

断熱過程では $Q = 0$ なので $\Delta U = -W_{\mathrm{AQS}}$。理想気体で $U = cNRT$ より $W_{\mathrm{AQS}} = cNR(T - T')$。断熱線 $T V^{\gamma-1} = \mathrm{const}$ から $T V_i^{\gamma-1} = T' V_f^{\gamma-1}$、したがって $T' = T (V_i/V_f)^{\gamma-1}$。ゆえに
\begin{equation}
W_{\mathrm{AQS}} = cNRT\left[1 - \left(\frac{V_i}{V_f}\right)^{\gamma-1}\right]
\end{equation}
$T' < T$ なので $W_{\mathrm{AQS}} > 0$(系は膨張して仕事をする)。

\paragraph{問4}

断熱壁を除いて熱源(温度 $T$)と接触させると、体積 $V_f$ のまま温度が $T'$ から $T$ へ変化する定積過程になる。定積過程では $W' = 0$。

\paragraph{問5}

$W_{\mathrm{tot}} = W_{\mathrm{AQS}} + W' = W_{\mathrm{AQS}}$。$W_{\mathrm{tot}} \leq W_{\mathrm{IQS}}$ を示すには、$cNRT[1 - (V_i/V_f)^{\gamma-1}] \leq NRT \ln(V_f/V_i)$ を示せばよい。$x = V_i/V_f < 1$ とおくと、$1 - x^{\gamma-1} \leq \ln(1/x) = -\ln x$ を示す。$f(x) = -\ln x - (1 - x^{\gamma-1})$ とおくと $f(1) = 0$、$f'(x) = -1/x + (\gamma-1)x^{\gamma-2}$。$x < 1$ で $f'(x) > 0$ のとき $f(x) < 0$、すなわち $1 - x^{\gamma-1} < -\ln x$。単原子気体では $\gamma = 5/3$、$c = 3/2$ で、この不等式が成り立つ。よって $W_{\mathrm{tot}} \leq W_{\mathrm{IQS}}$。

\textbf{なぜ断熱+等温の2段階では等温1段階より仕事が少ないのか}:断熱膨張では温度が下がり($T' < T$)、その後の定積加熱では仕事をしない。一方、等温過程では常に熱源から熱を供給しながら膨張するので、各瞬間の圧力が高く保たれ、$p$-$V$ 図上の面積(=仕事)が最大になる。図\ref{fig:ex4_max_work}のように、等温線の下の面積が得られる仕事を表す。断熱+等温では、途中で温度が下がる区間があり、その分仕事が取り出しにくくなる。

\begin{figure}[H]
    \centering
    \includegraphics[width=0.7\textwidth]{figures/ex4_max_work.png}
    \caption{最大仕事の原理。準静的等温過程では曲線下の面積が仕事になる。自由膨張では経路が定義できず $W=0$。}
    \label{fig:ex4_max_work}
\end{figure}

%--------------------------------------
\subsection{II. 熱機関の効率}

\subsubsection{問題}

カルノー機関の効率 $\eta_C$ を $T_H$、$T_L$ で表せ。オットー機関の効率を $T_A, T_B, T_C, T_D$ および $V_1, V_2$ で表し、$\eta < \eta_C$ を示せ。

\subsubsection{解答}

\paragraph{前提知識:熱機関と効率}

\textbf{熱機関}:高温熱源から熱 $Q_H$ を吸収し、一部を仕事 $W$ に変換し、残り $Q_L$ を低温熱源に捨てるサイクル運転。自動車のエンジン、発電所のタービンなどがその例。

\textbf{効率}:$\eta = W/Q_H = 1 - Q_L/Q_H$。吸収した熱のうち、どれだけを仕事に変えられるか。捨てる熱 $Q_L$ が少ないほど効率が高い。

\textbf{カルノーの定理}:同じ2つの熱源の間で動作するあらゆる熱機関のうち、可逆機関(カルノー機関)が最高効率を達成する。その効率は $\eta_C = 1 - T_L/T_H$ であり、熱源の温度だけで決まる。

\textbf{オットー機関}:ガソリンエンジンの理想化。4ストロークサイクルを、A→B(断熱圧縮)、B→C(定積吸熱)、C→D(断熱膨張)、D→A(定積放熱)の4過程で近似する。各過程で何が起こるか:断熱では $Q=0$、定積では $W=0$ で $Q=C_V\Delta T$。高温熱源・低温熱源の温度差が大きいほど効率が上がる(カルノー効率 $\eta_C = 1 - T_L/T_H$ 参照)。

\paragraph{導出の戦略}

カルノー機関:可逆過程では $Q_H/T_H = Q_L/T_L$ が成り立つ。$W = Q_H - Q_L$ から $\eta$ を求める。オットー機関:各過程の $Q$ を $C_V$ と温度で表し、$\eta = 1 - Q_L/Q_H$ を計算。断熱関係 $TV^{\gamma-1} = \mathrm{const}$ で温度比を圧縮比で表す。

\paragraph{カルノー機関}

$\eta_C = 1 - T_L/T_H$。高温熱源温度 $T_H$ と低温熱源温度 $T_L$ だけで決まる。

\paragraph{オットー機関}

図\ref{fig:ex4_cycles}(b) のように、A $\to$ B(断熱圧縮)、B $\to$ C(定積吸熱)、C $\to$ D(断熱膨張)、D $\to$ A(定積放熱)である。吸熱は B $\to$ C で $Q_H = C_V(T_C - T_B)$、放熱は D $\to$ A で $Q_L = C_V(T_D - T_A)$。効率は
\begin{equation}
\eta = 1 - \frac{Q_L}{Q_H} = 1 - \frac{T_D - T_A}{T_C - T_B}
\end{equation}
断熱過程 A $\to$ B、C $\to$ D で $T V^{\gamma-1} = \mathrm{const}$ より $T_B/T_A = (V_1/V_2)^{\gamma-1}$、$T_C/T_D = (V_1/V_2)^{\gamma-1}$。よって $T_B/T_A = T_C/T_D = r^{\gamma-1}$($r = V_1/V_2 > 1$)。これより $T_D = T_C/r^{\gamma-1}$、$T_A = T_B/r^{\gamma-1}$。代入して $\eta = 1 - r^{1-\gamma}$。

カルノー機関は $T_H$ と $T_L$ の間で動作するが、オットー機関では吸熱時の温度は $T_B \sim T_C$、放熱時は $T_A \sim T_D$ であり、実質的な動作温度範囲は $T_A$ から $T_C$ まで。カルノー機関を同じ $T_A$、$T_C$ で動作させた場合の効率は $1 - T_A/T_C$。$r > 1$ のとき $r^{1-\gamma} > T_A/T_C$ となるため $\eta < \eta_C$ である。

\begin{figure}[H]
    \centering
    \includegraphics[width=0.9\textwidth]{figures/ex4_carnot_otto.png}
    \caption{カルノーサイクルとオットーサイクルのT-V線図。}
    \label{fig:ex4_cycles}
\end{figure}

%--------------------------------------
\subsection{III. 内部エネルギーの方程式}

\subsubsection{問題}

カルノーの定理と熱力学第1法則から $\left(\frac{\partial U}{\partial V}\right)_T = -p + T\left(\frac{\partial p}{\partial T}\right)_V$ を導け。

\subsubsection{解答}

\paragraph{なぜこの式が成り立つのか(導出の原理)}

理想気体では $(\partial U/\partial V)_T = 0$ であるが、ファンデルワールス気体など一般の系では $U$ が $V$ に依存する。この依存性を、熱力学第1法則とカルノーの定理(可逆機関での $Q_H/T_H = Q_L/T_L$)から導く。エントロピー $S$ が状態量であることと、可逆過程で $dS = dQ/T$ が成り立つことから、カルノーの定理が導かれる。

\paragraph{微小カルノーサイクル}

図\ref{fig:ex4_mini_carnot}に示すように、$p$-$V$ 図上で温度 $T$ と $T+dT$ の2本の等温線、および2本の断熱線で囲まれる微小な四角形(実際は曲線で囲まれる)を考える。

$p$-$V$ 図上で、温度 $T$ と $T+dT$ の2本の等温線、および2本の断熱線で囲まれる微小サイクルを考える。頂点を A $\to$ B $\to$ C $\to$ D $\to$ A とする。B $\to$ C が等温($T+dT$)、D $\to$ A が等温($T$)とする。

カルノーの定理から $Q_H/(T+dT) = Q_L/T$。$dT$ が小さいとして $Q_H \approx Q_L(1 + dT/T)$、したがって $Q_H - Q_L \approx Q_L \cdot dT/T$。

熱力学第1法則:1サイクルで $\Delta U = 0$ なので、$Q_H - Q_L$ はこのサイクルで系がする正味の仕事 $W_{\mathrm{cyc}}$ に等しい。仕事は $p$-$V$ 図上の面積で、$dT$ の1次で
\begin{equation}
W_{\mathrm{cyc}} = Q_H - Q_L \approx dT \int_{V_D}^{V_B} \left(\frac{\partial p}{\partial T}\right)_V dV
\end{equation}
(等温線間の圧力差 $\approx (\partial p/\partial T)_V dT$ を体積で積分したもの)。

一方、等温過程 D $\to$ A で系が吸収する熱 $Q_L$ は、$dQ = dU + p\,dV$ より $Q_L = \int_{V_D}^{V_A} (dU + p\,dV)$。等温では $dU = (\partial U/\partial V)_T dV$ だから
\begin{equation}
Q_L = \int_{V_D}^{V_A} \left[\left(\frac{\partial U}{\partial V}\right)_T + p\right] dV
\end{equation}
$Q_H - Q_L = Q_L \cdot dT/T$ と $W_{\mathrm{cyc}}$ の式を比較し、$dT$ の1次の係数を等置すると
\begin{equation}
\left(\frac{\partial U}{\partial V}\right)_T + p = T \left(\frac{\partial p}{\partial T}\right)_V
\end{equation}
ゆえに $\displaystyle \left(\frac{\partial U}{\partial V}\right)_T = -p + T\left(\frac{\partial p}{\partial T}\right)_V$。

%--------------------------------------
\subsection{IV--V. ファンデルワールス気体と断熱線}

\subsubsection{問題}

ファンデルワールス気体の内部エネルギー、断熱線、カルノー効率を求め、2つの断熱線が交わらないことを示せ。

\subsubsection{解答}

\paragraph{内部エネルギー}

$(\partial U/\partial V)_T = aN^2/V^2$ は、分子間引力のポテンシャルエネルギー $-\frac{aN^2}{V}$ を $V$ で微分したものに対応する。$V$ が大きいと引力ポテンシャルは小さく、$U$ の $V$ 依存部分は $-\frac{aN^2}{V}$ となる。

$\left(\frac{\partial U}{\partial V}\right)_T = -p + T\left(\frac{\partial p}{\partial T}\right)_V$ に $p = \frac{NRT}{V-bN} - \frac{aN^2}{V^2}$ を代入する。$\left(\frac{\partial p}{\partial T}\right)_V = \frac{NR}{V-bN}$ だから
\begin{equation}
\left(\frac{\partial U}{\partial V}\right)_T = -\left(\frac{NRT}{V-bN} - \frac{aN^2}{V^2}\right) + T \cdot \frac{NR}{V-bN} = \frac{aN^2}{V^2}
\end{equation}
$V$ で積分して $U = -\frac{aN^2}{V} + f(T)$。$U$ の $T$ 依存性は理想気体と同様 $cNRT$ なので $U = cNRT - \frac{aN^2}{V} + \mathrm{const}$。

\paragraph{断熱線}

断熱過程では $dU + p\,dV = 0$。$dU = C_V dT + \frac{aN^2}{V^2}dV$、$p\,dV = \left(\frac{NRT}{V-bN} - \frac{aN^2}{V^2}\right)dV$。加えると
\begin{equation}
C_V dT + \frac{NRT}{V-bN} dV = 0 \quad \Rightarrow \quad \frac{C_V}{NR} \frac{dT}{T} + \frac{dV}{V-bN} = 0
\end{equation}
$c = C_V/(NR)$ とおくと $\frac{dT}{T} + \frac{1}{c}\frac{dV}{V-bN} = 0$。積分して $c \ln T + \ln(V-bN) = \mathrm{const}$、すなわち
\begin{equation}
T^c (V - bN) = \mathrm{const}
\end{equation}
($c = C_V/(NR)$ は無次元定数。単原子なら $C_V = \frac{3}{2}NR$ で $c = 3/2$。)

\paragraph{2つの断熱線が交わらないこと}

\textbf{なぜ交わると矛盾するのか}:2つの断熱線が交わると仮定する。交点と等温線で三角形のサイクルを作ることができる(図\ref{fig:ex4_adiabatic}参照)。このサイクルでは、等温過程で熱を吸収し、2つの断熱過程では熱の出入りがない。正味で熱を吸収して仕事をするが、捨てる熱がなく、低温熱源が存在しない。つまり「一つの熱源から熱を取り出して正味の仕事に変える」ことになり、ケルビンの原理(熱力学第2法則の一表現)に反する。よって2つの断熱線は交わらない。

\textbf{物理的考察}:断熱線は $S = \mathrm{const}$ の曲線である。異なる断熱線は異なる $S$ の値に対応する。$S$ は状態 $(T,V)$ の一価関数なので、同じ $(T,V)$ で異なる $S$ は存在せず、2つの断熱線が交わることはない。

\begin{figure}[H]
    \centering
    \includegraphics[width=0.7\textwidth]{figures/ex4_adiabatic_intersection.png}
    \caption{2つの断熱線が交わると仮定した場合の矛盾(概念図)。}
    \label{fig:ex4_adiabatic}
\end{figure}

% 演習5 (2025/12/05実施)
\section{演習5 (2025年12月5日実施)}

\subsection{I. 理想気体のエントロピー}

\subsubsection{この問題で学ぶこと}

エントロピー $S$ が状態量であり、経路に依らないこと。経路Aと経路Bで計算した結果が一致することで、$S$ が $(T,V)$ の一価関数であることを確認する。断熱自由膨張では $T$ は不変だが $S$ は増大する(不可逆過程)。

\subsubsection{問題}

$N$ mol の理想気体について、基準状態 $(T^*, V^*)$ のエントロピーを $S^*$ とする。
\begin{enumerate}
    \item 経路A(等温→断熱)で状態 $(T, V)$ に至るエントロピーを求めよ。
    \item 経路B(断熱→等温)で同様に求め、一致することを確かめよ。
    \item $dS = dU/T + (p/T)dV$ を積分して $S(T,V)$ を求めよ。
    \item 断熱自由膨張で体積 $V_{\mathrm{final}}$ になった後の温度を求めよ。
    \item 問4でエントロピーが増大することを示せ。
\end{enumerate}

\subsubsection{解答}

\paragraph{前提知識:エントロピーとは}

エントロピー $S$ は、熱力学第2法則に登場する状態量である。「乱雑さ」や不可逆性の度合いを表す。可逆過程では $dS = dQ_{\mathrm{rev}}/T$ で定義され、経路によらず一意的に定まる(状態量であることの数学的表現)。

\textbf{なぜ経路に依存しないか}:$dS$ が $dQ/T$ の形で、可逆過程では $dQ$ が $T$ と $dV$ などで表せるため、$dS$ は完全微分になる。積分が経路に依らない。

\textbf{断熱自由膨張}:壁を瞬間的に取り除いて膨張させる。$Q = 0$、$W = 0$ の不可逆過程。理想気体では $U$ が $T$ のみに依存するので温度は不変。しかし体積が増え、同じ温度・より大きい体積の状態は、より多くの「微視的状態」に対応する。エントロピーは増大する(クラウジウスの不等式 $dS \geq dQ/T$ で、$dQ=0$ のとき $dS > 0$)。

\paragraph{導出の戦略}

問1・2:経路Aと経路Bで $S(T,V)$ を計算し、一致することを確認する。経路Aは等温→断熱、経路Bは断熱→定積加熱。各区間で $dS = dQ/T$ を積分する。問3:$dS = dU/T + (p/T)dV$ を直接積分する。問4・5:断熱自由膨張では $\Delta U = 0$ だから $T$ 不変。エントロピーは $S(T,V)$ で $V$ が増えるから増大。

\paragraph{問1: 経路A}

経路A:$(T^*, V^*) \xrightarrow{\mathrm{等温}} (T^*, V') \xrightarrow{\mathrm{断熱}} (T, V)$。

第1区間(等温):等温では $dU=0$ なので $dQ = p\,dV$。$dS = dQ/T = p\,dV/T$。理想気体で $p/T = NR/V$ だから $dS = NR\,dV/V$。積分して $\Delta S_1 = NR \ln(V'/V^*)$。

第2区間(断熱):$dQ = 0$ なので $dS = 0$、$\Delta S_2 = 0$。

$V'$ は、$(T^*, V')$ から $(T, V)$ への断熱線上の点である。断熱線 $T V^{\gamma-1} = \mathrm{const}$ より $T^* (V')^{\gamma-1} = T V^{\gamma-1}$、したがって $V' = V (T/T^*)^{1/(\gamma-1)}$。

よって
\begin{equation}
S - S^* = NR \ln\frac{V'}{V^*} = NR \ln\left[\frac{V}{V^*}\left(\frac{T}{T^*}\right)^{\frac{1}{\gamma-1}}\right]
\end{equation}
$\gamma - 1 = R/C_V = 1/c$($c = C_V/(NR)$)を用いると
\begin{equation}
S - S^* = NR \ln\frac{V}{V^*} + NR \cdot c \ln\frac{T}{T^*} = NR\left[\ln\frac{V}{V^*} + c \ln\frac{T}{T^*}\right]
\end{equation}

\paragraph{問2: 経路B}

経路B:$(T^*, V^*) \xrightarrow{\mathrm{断熱}} (T', V) \xrightarrow{\text{定積加熱}} (T, V)$。

第1区間(断熱):$dS = 0$、$\Delta S_1 = 0$。

第2区間:体積 $V$ を一定にしたまま、温度を $T'$ から $T$ まで変化させる。\textbf{注意}:この区間は「等温過程」ではなく、定積過程である。$dQ = C_V dT$、$dS = C_V dT/T$。積分して $\Delta S_2 = C_V \ln(T/T') = NRc \ln(T/T')$。

断熱線 $T V^{\gamma-1} = \mathrm{const}$ より $T^*(V^*)^{\gamma-1} = T' V^{\gamma-1}$、したがって $T' = T^* (V^*/V)^{\gamma-1}$。

$\ln(T/T') = \ln(T/T^*) - (\gamma-1)\ln(V^*/V) = \ln(T/T^*) + (\gamma-1)\ln(V/V^*)$ である。

$\Delta S_2 = C_V \ln(T/T') = C_V \ln(T/T^*) + C_V(\gamma-1)\ln(V/V^*)$。ここで $C_V = cNR$、$C_V(\gamma-1) = C_V \cdot (C_p-C_V)/C_V = C_p - C_V = R$(1 mol あたり)だから、$N$ mol では $C_V(\gamma-1) = NR$。したがって
\begin{equation}
\Delta S_2 = NRc \ln\frac{T}{T^*} + NR \ln\frac{V}{V^*}
\end{equation}
ゆえに $S - S^* = NR\left[c \ln\frac{T}{T^*} + \ln\frac{V}{V^*}\right]$。経路Aの結果と一致する。

\paragraph{問3: 直接積分}

$dS = \frac{dU}{T} + \frac{p}{T}dV$。理想気体では $dU = C_V dT$、$p/T = NR/V$ だから
\begin{equation}
dS = \frac{C_V}{T}dT + \frac{NR}{V}dV
\end{equation}
積分して($C_V = cNR$ を用い)
\begin{equation}
S - S^* = cNR \ln\frac{T}{T^*} + NR \ln\frac{V}{V^*}
\end{equation}

\paragraph{問4: 断熱自由膨張後の温度}

断熱自由膨張では $Q = 0$、$W = 0$ なので $\Delta U = 0$。理想気体では $U$ が $T$ のみの関数だから $T$ は不変。膨張後の温度は $T$ のままである。

\paragraph{問5: エントロピー増大}

膨張前 $(T, V)$、膨張後 $(T, V_{\mathrm{f}})$($V_{\mathrm{f}} > V$)。問3の結果より
\begin{equation}
\Delta S = S(T, V_{\mathrm{f}}) - S(T, V) = NR \ln\frac{V_{\mathrm{f}}}{V} > 0
\end{equation}
不可逆過程ではエントロピーが増大する(クラウジウスの不等式)。

\textbf{なぜ経路が違っても $S$ が同じなのか}:エントロピー $S$ は状態 $(T, V)$ の関数であり、どの可逆経路で計算しても同じ値になる。これは $dS = dQ/T$ が完全微分であることの帰結である。経路A(等温→断熱)と経路B(断熱→定積加熱)は異なる経路だが、出発点と到着点が同じなので、積分 $\int dS$ は等しい。図\ref{fig:ex5_paths}の2つの経路で計算した $S - S^*$ が一致することは、$S$ が状態量であることの確認である。

\textbf{なぜ断熱自由膨張でエントロピーが増えるのか}:$Q = 0$ なので可逆的な $dS = dQ/T$ では $dS = 0$ だが、自由膨張は不可逆過程である。クラウジウスの不等式 $dS > dQ/T$(不可逆では不等号)より、$dQ = 0$ のときでも $dS > 0$ となり得る。微視的には、気体が広い領域に広がることで「とり得る配置の数」が増え、エントロピー(乱雑さの尺度)が増大する。図\ref{fig:ex5_free_expansion}を参照。

\begin{figure}[H]
    \centering
    \includegraphics[width=0.75\textwidth]{figures/ex5_entropy_paths.png}
    \caption{エントロピー計算の2つの経路(T-V図)。経路Aは赤(等温→断熱)、経路Bは緑(断熱→定積加熱)。どちらで計算しても $S(T,V)$ は同じ。}
    \label{fig:ex5_paths}
\end{figure}

\begin{figure}[H]
    \centering
    \includegraphics[width=0.85\textwidth]{figures/ex5_free_expansion.png}
    \caption{断熱自由膨張の概念図。壁を瞬間的に取り除くと、気体は真空に膨張する。$Q=0$、$W=0$ なので $\Delta U=0$。理想気体では $T$ 不変だが、体積が増えるため $S$ は増大する。}
    \label{fig:ex5_free_expansion}
\end{figure}

%--------------------------------------
\subsection{II. 自由断熱膨張とエントロピー}

\subsubsection{この問題で学ぶこと}

一般の系(理想気体でなくても)で、断熱自由膨張が不可逆であることを、エネルギーとエントロピーから示す。自由膨張後、同じ体積に準静的断熱で戻すと、内部エネルギーが増加する。エントロピーが温度の増加関数であることから、$\Delta S > 0$ が導かれる。

\subsubsection{問題}

一般の系で断熱自由膨張 $(T,V) \to (T', V')$ の後、準静的断熱過程で $(T'', V)$ に戻す。$U(T'', V) > U(T', V')$、したがって $T'' > T'$ を示し、エントロピーが温度の増加関数であることから $\Delta S > 0$ を導け。

\subsubsection{解答}

\paragraph{導出の戦略}

自由膨張では $Q=0$, $W=0$ だから $\Delta U = 0$。準静的断熱圧縮では、外界が系に仕事をするので系の $U$ は増加。$U$ が $T$ の増加関数なら $T'' > T'$。可逆断熱では $dS=0$、自由膨張は不可逆なので $S(T',V') > S(T,V)$。$S$ が $T$ の増加関数なら $S(T'',V) > S(T',V')$ と合わせて、自由膨張で $\Delta S > 0$。

\paragraph{計算の詳細}

断熱自由膨張では $\Delta U = 0$ なので $U(T', V') = U(T, V)$。準静的断熱圧縮 $(T', V') \to (T'', V)$ では、外界が系に仕事をするので $W > 0$(系が受け取る仕事)。断熱なので $Q = 0$、第1法則より $\Delta U = W > 0$。したがって $U(T'', V) > U(T', V')$。内部エネルギーが $T$ の増加関数なら $T'' > T'$。エントロピーも $T$ の増加関数なので $S(T'', V) > S(T', V')$。自由膨張 $(T,V) \to (T', V')$ は不可逆過程であり、可逆な断熱過程ではエントロピーは不変なので、$S(T', V') > S(T, V)$ が成り立つ。すなわち $\Delta S > 0$。

%--------------------------------------
\subsection{III--IV. 2つの系の熱接触とエントロピー}

\subsubsection{この問題で学ぶこと}

温度の異なる2つの系が熱接触して平衡に達するとき、エントロピーは増大する(不可逆過程)。相加・相乗平均の関係から $\Delta S \geq 0$ を示す。また、系が熱源と接触して温度変化するとき、クラウジウスの不等式 $\Delta S - Q/T \geq 0$ が成り立つ。

\subsubsection{問題}

2つの理想気体(各1mol、体積$V$、温度 $T_1$、$T_2$)が断熱壁で隔てられている。透熱壁に置き換えた後の平衡温度 $T_3$、全エントロピーの増加を求めよ。また熱源との接触でのエントロピー増大則 $\Delta S_A - Q/T > 0$ を確認せよ。

\subsubsection{解答}

\paragraph{導出の戦略}

平衡温度はエネルギー保存から $T_3 = (T_1 + T_2)/2$。エントロピー変化は、理想気体の $S(T,V)$ を用いて、終状態と初期状態の差を計算する。$T_3^2 \geq T_1 T_2$(相加・相乗平均)から $\Delta S \geq 0$ を示す。熱源との接触では、系の $\Delta S_A$ と熱源のエントロピー変化 $-Q/T$ の和が正であることを確認する。

\paragraph{平衡温度}

エネルギー保存より $T_3 = (T_1 + T_2)/2$。

\paragraph{エントロピー増加}

エントロピー $S = NR\ln(T^c V/N) + NS_0$ を用いる。初期:$S_{\mathrm{init}} = S(T_1, V) + S(T_2, V)$。終期:$S_{\mathrm{fin}} = 2S(T_3, V)$。$\Delta S = 2S(T_3) - S(T_1) - S(T_2) = 2NRc \ln T_3 - NRc(\ln T_1 + \ln T_2) = NRc \ln(T_3^2/(T_1 T_2))$。$T_3 = (T_1 + T_2)/2$ のとき、相加・相乗平均の関係から $T_3^2 \geq T_1 T_2$(等号は $T_1 = T_2$ のみ)。よって $\Delta S \geq 0$。

\paragraph{熱源との接触}

系が温度 $T$ の熱源と接触して $T_A$ から $T$ へ変化するとき、系のエントロピー変化は $\Delta S_A = NRc \ln(T/T_A)$、吸収する熱は $Q = C_V(T - T_A) = cNR(T - T_A)$。クラウジウスの不等式は $\Delta S_{\mathrm{tot}} = \Delta S_A - Q/T > 0$(熱源のエントロピー変化が $-Q/T$)。計算すると $\Delta S_A - Q/T = NRc\left(\ln\frac{T}{T_A} - \frac{T-T_A}{T}\right)$。$x = T/T_A > 1$ とおくと $\ln x - (x-1)/x > 0$ が成り立つ($f(x) = \ln x - (x-1)/x$ は $f(1) = 0$、$f'(x) > 0$ for $x > 1$)。

%--------------------------------------
\subsection{V. ファンデルワールス気体}

\subsubsection{問題}

ファンデルワールス気体のエントロピーを求め、断熱自由膨張後の温度 $T'$ および $\Delta S$ を計算せよ。

\subsubsection{解答}

\textbf{方針}:$dS = dU/T + (p/T)dV$ に $U = cNRT - aN^2/V$、$p = NRT/(V-bN) - aN^2/V^2$ を代入する。$dU = cNR\,dT + (aN^2/V^2)dV$、$p/T = NR/(V-bN) - aN^2/(TV^2)$。これらを代入すると $dS$ が $dT$ と $dV$ の線形結合になり、全微分となる(積分可能)。積分してあるいは断熱線 $T^c(V-bN) = \mathrm{const}$ に沿って $dS = 0$ を用い、等温線に沿う寄与を積分する。結果は
\begin{equation}
S = NRc \ln T + NR \ln(V - bN) + \mathrm{const}
\end{equation}

断熱自由膨張で $U$ は不変。$U = cNRT - aN^2/V$ より $cNRT - aN^2/V = cNRT' - aN^2/V'$。$V' > V$ のとき $aN^2(1/V - 1/V') > 0$ なので $T' > T$(分子間引力のポテンシャルエネルギーが増えるため、運動エネルギー=温度が上がる)。

$\Delta S = S(T', V') - S(T, V) = NRc \ln(T'/T) + NR \ln\frac{V'-bN}{V-bN}$。$T' > T$ かつ $V' > V$ なので $\Delta S > 0$。

\textbf{なぜファンデルワールス気体では自由膨張で温度が上がるのか}:理想気体では $U$ が $T$ のみの関数なので $\Delta U = 0$ から $T$ 不変。ファンデルワールス気体では $U = cNRT - aN^2/V$ であり、$V$ が増えると $-aN^2/V$ の項(分子間引力のポテンシャルエネルギー)が増える。$U$ が不変なので、運動エネルギー($\propto T$)が増えて温度が上昇する。これはジュール・トムソン効果の一例である。

% 演習6 (2025/12/19実施)
\section{演習6 (2025年12月19日実施)}

\subsection{I. 熱力学の関係式}

\subsubsection{この問題で学ぶこと}

熱力学の基本関係式($T\,dS = dU + p\,dV$、$dF = -S\,dT - p\,dV$)と偏微分の技巧から、内部エネルギーや熱容量に関する重要な関係式を導く。マクスウェル関係式は2階偏微分の対称性から得られる。

\subsubsection{問題}

$T\,dS = dU + p\,dV$、$dF = -S\,dT - p\,dV$ を用いて次を導け。
\begin{enumerate}
    \item $\left(\frac{\partial U}{\partial V}\right)_T = T^2 \frac{\partial}{\partial T}\left(\frac{p}{T}\right)_V$
    \item $C_p = C_V + T\left(\frac{\partial V}{\partial T}\right)_p \left(\frac{\partial p}{\partial T}\right)_V$
\end{enumerate}

\subsubsection{解答}

\paragraph{前提知識:熱力学の基本関係式}

\textbf{$T\,dS = dU + p\,dV$}:第1法則 $dU = dQ - p\,dV$ と、可逆過程での $dQ_{\mathrm{rev}} = T\,dS$ を組み合わせると、$dU = T\,dS - p\,dV$ すなわち $T\,dS = dU + p\,dV$ を得る。エントロピー $S$ を $U$, $V$ の関数として扱う表現である。

\textbf{$dF = -S\,dT - p\,dV$}:ヘルムホルツの自由エネルギー $F = U - TS$ の全微分は $dF = dU - T\,dS - S\,dT$。$dU = T\,dS - p\,dV$ を代入すると $dF = -S\,dT - p\,dV$。$F$ を $T$, $V$ の関数として扱う表現である。

\textbf{なぜ重要か}:偏微分の関係式(マクスウェル関係式)は、これらの完全微分の性質(2階偏微分の対称性)から導かれる。熱力学の様々な関係式を導くための基本ツールである。

\paragraph{導出の戦略}

マクスウェル関係式は、$F$ の2階偏微分 $\frac{\partial^2 F}{\partial T \partial V} = \frac{\partial^2 F}{\partial V \partial T}$ から $(\partial S/\partial V)_T = (\partial p/\partial T)_V$ を得る。問1は $dU = T\,dS - p\,dV$ を $V$ で偏微分し、マクスウェル関係式を代入。問2は $S(T,V)$ の全微分から $(\partial S/\partial T)_p$ を連鎖律で求める。

\paragraph{マクスウェル関係式の導出}

$dF = -S\,dT - p\,dV$ より、$F$ の2階偏微分の順序交換 $\frac{\partial^2 F}{\partial T \partial V} = \frac{\partial^2 F}{\partial V \partial T}$(滑らかな関数では偏微分の順序は交換可能)から
\begin{equation}
\left(\frac{\partial S}{\partial V}\right)_T = \left(\frac{\partial p}{\partial T}\right)_V
\end{equation}
が得られる(マクスウェル関係式の1つ)。

\paragraph{問1}

$dU = T\,dS - p\,dV$ より、$T$ を一定にして $V$ で偏微分すると
\begin{equation}
\left(\frac{\partial U}{\partial V}\right)_T = T\left(\frac{\partial S}{\partial V}\right)_T - p
\end{equation}
マクスウェル関係式 $(\partial S/\partial V)_T = (\partial p/\partial T)_V$ を代入して
\begin{equation}
\left(\frac{\partial U}{\partial V}\right)_T = T\left(\frac{\partial p}{\partial T}\right)_V - p
\end{equation}
一方、$T^2 \frac{\partial}{\partial T}(p/T)_V = T^2 \cdot \frac{T(\partial p/\partial T)_V - p}{T^2} = T(\partial p/\partial T)_V - p$ なので、与式と一致する。

\paragraph{問2}

$C_p = T(\partial S/\partial T)_p$、$C_V = T(\partial S/\partial T)_V$ である。$S$ を $T, V$ の関数とみて
\begin{equation}
dS = \left(\frac{\partial S}{\partial T}\right)_V dT + \left(\frac{\partial S}{\partial V}\right)_T dV
\end{equation}
$p$ を一定にしたときの $dS/dT$ は、$dV/dT = (\partial V/\partial T)_p$ を用いて
\begin{equation}
\left(\frac{\partial S}{\partial T}\right)_p = \left(\frac{\partial S}{\partial T}\right)_V + \left(\frac{\partial S}{\partial V}\right)_T \left(\frac{\partial V}{\partial T}\right)_p
\end{equation}
したがって
\begin{equation}
C_p - C_V = T\left(\frac{\partial S}{\partial V}\right)_T \left(\frac{\partial V}{\partial T}\right)_p = T\left(\frac{\partial p}{\partial T}\right)_V \left(\frac{\partial V}{\partial T}\right)_p
\end{equation}
(マクスウェル関係式を用いた)。

\textbf{なぜ $C_p > C_V$ なのか(原理的説明)}:定積では体積が変わらないので、加えた熱はすべて内部エネルギーの増加に使われる。定圧では体積が膨張するため、加えた熱の一部が膨張の仕事($p\Delta V$)に使われる。同じ温度上昇を得るには、定圧の方がより多くの熱が必要であり、$C_p > C_V$ となる。Mayerの関係式 $C_p - C_V = R$(1モルあたり)は、この差が気体定数 $R$ で与えられることを示す。

%--------------------------------------
\subsection{II. 輪ゴムの熱力学}

\subsubsection{この問題で学ぶこと}

熱力学は気体だけに限らない。輪ゴムのような弾性体にも適用できる。張力 $X$ と変位 $x$ が、気体の $p$ と $V$ に対応する。$k(T)$ が温度に依存すると、内部エネルギーと自由エネルギーの温度依存性が異なる。

\subsubsection{問題}

バネ定数 $k(T) = k_0 + k_1 T$ のバネについて、自由エネルギー $F(T,x)$ と内部エネルギー $U(T,x)$ を求めよ。

\subsubsection{解答}

\paragraph{前提知識:気体との対応}

気体の $p$-$V$ と弾性体の $X$-$x$ の対応:体積 $V \leftrightarrow$ 変位 $x$、圧力 $p \leftrightarrow$ 張力 $-X$。$(\partial F/\partial V)_T = -p$ に対応して $(\partial F/\partial x)_T = -X$ である。

バネの張力 $X = -k(T)x$(フックの法則)。気体の $p$-$V$ に対応して、バネでは $X$-$x$ が「力」と「変位」の組である。$k(T)$ が温度に比例する部分 $k_1 T$ を持つと、温度を上げると張力が増す。輪ゴムを素早く伸ばすと温度が上がる現象は、この温度依存性と整合する。

\paragraph{自由エネルギー}

$\left(\frac{\partial F}{\partial x}\right)_T = -X = k(T)x$ より、$x$ で積分して $F(T,x) = \frac{1}{2}k(T)x^2 + f(T)$。$f(T)$ は $x$ に依らない部分。比熱 $C$ が定数のとき、$F$ の $T$ 依存から $f(T) = -CT(\ln(T/T_0)-1)$ など。$F(T=0, x=0) = 0$ の条件で未定定数を決める。

\paragraph{内部エネルギー}

問Iの式(バネ版)$\left(\frac{\partial U}{\partial x}\right)_T = T^2 \frac{\partial}{\partial T}\left(\frac{X}{T}\right)_x$ を用いる。$X/T = -k(T)x/T = -(k_0/T + k_1)x$。$\frac{\partial}{\partial T}(X/T) = k_0 x/T^2$。よって $\left(\frac{\partial U}{\partial x}\right)_T = k_0 x$。積分して $U = \frac{1}{2}k_0 x^2 + g(T)$。$U(0,0)=0$ などで $g(T)$ を決める。$k_1 T$ の項は $F$ にはあるが $U$ には直接現れず、エントロピーに起因する。

\textbf{なぜ輪ゴムを急に伸ばすと熱くなるのか}:断熱的に(急に)伸ばすと $dQ = 0$。$dU = dQ + dW = dW$(外界がする仕事)。$k(T) = k_0 + k_1 T$ で $k_1 > 0$ なら、伸ばすと張力が増し、エントロピーが減る(秩序が増す)。$dU = T dS + X dx$ で、$dS < 0$、$X dx > 0$(伸ばすとき外界が仕事をする)のとき、$dU > 0$ となり温度が上昇する。輪ゴムを唇に当てて急に伸ばすと温かく感じる、という日常体験と整合する。

%--------------------------------------
\subsection{III. 理想気体の熱力学関数}

\subsubsection{問題}

$S = NR\ln(T^c V/N) + NS_0$、$U = cNRT$ から化学ポテンシャル $\mu$、ヘルムホルツの自由エネルギー $F$ を求めよ。オイラーの関係式を確認せよ。

\subsubsection{解答}

\textbf{化学ポテンシャルとは}:$\mu = (\partial F/\partial N)_{T,V}$ は、粒子を1つ追加したときの $F$ の増分。相平衡では両相の $\mu$ が等しい。

$F = U - TS = cNRT - T[NR\ln(T^c V/N) + NS_0]$。$\mu = (\partial F/\partial N)_{T,V}$ を計算する。$\ln(T^c V/N)$ の $N$ による微分は $\frac{\partial}{\partial N}\ln(T^c V) - \frac{\partial}{\partial N}\ln N = -1/N$ である。よって
\begin{equation}
\mu = cRT - RT\ln(T^c V/N) - RT - TS_0
\end{equation}
オイラーの関係式:$S$ が $U,V,N$ の1次同次関数(示量変数)なら $S = (\partial S/\partial U) U + (\partial S/\partial V) V + (\partial S/\partial N) N$。$\partial S/\partial U = 1/T$($U$ を $S,V,N$ の関数とみたときの逆関係)、$\partial S/\partial V = p/T = NR/V$、$\partial S/\partial N = -\mu/T$。よって $S = U/T + pV/T - \mu N/T$ となり、オイラー関係が成り立つ。化学ポテンシャルの単位はエネルギー/mol(または J/mol)である。

%--------------------------------------
\subsection{IV--V. 2成分混合気体とポアソン分布}

\subsubsection{この問題で学ぶこと}

異なる種類の気体を混合すると、エントロピーが増加する(混合のエントロピー)。統計力学では、大量の分子がランダムに分布するとき、小さな領域に入る分子数はポアソン分布に従う。二項分布の極限としてポアソン分布を導く。

\subsubsection{問題}

体積 $V$ の容器に $N_1$ mol、$N_2$ mol の2成分理想気体がある。仕切りを外して混合したときのエントロピー変化を求めよ。体積 $v$ に $n$ 個の分子が入る確率がポアソン分布になることを示せ。

\subsubsection{解答}

\paragraph{導出の戦略}

混合のエントロピー:各成分が $V_1$, $V_2$ から $V$ に広がる。理想気体の $S(T,V,N)$ で、$\Delta S = S_{\mathrm{fin}} - S_{\mathrm{init}}$ を計算。ポアソン分布:二項分布 $P(n) = \binom{N}{n} p^n (1-p)^{N-n}$ で $N \to \infty$、$p \to 0$、$Np = a$ 一定の極限をとる。母関数を用いて導出する。

\paragraph{混合のエントロピー}

\textbf{設定}:全体積 $V$ の容器に、動く仕切りで区切られた2つの部分がある。成分1は体積 $V_1$、成分2は体積 $V_2$ を占め、$V_1 + V_2 = V$。温度・圧力は平衡で等しい。仕切りを外すと、両成分が全体積 $V$ に広がる。

初期状態:$S_{\mathrm{init}} = S_1(T, V_1, N_1) + S_2(T, V_2, N_2)$。平衡条件から $V_1/N_1 = V_2/N_2$、また $V_1 + V_2 = V$、$N_1 + N_2 = N$ より $V_1/V = N_1/N$、$V_2/V = N_2/N$。

終状態:各成分が体積 $V$ に広がった理想混合気体。$S_{\mathrm{fin}} = S_1(T, V, N_1) + S_2(T, V, N_2)$(理想気体の混合では、各成分の partial pressure で計算)。エントロピー変化は
\begin{equation}
\Delta S_{\mathrm{mix}} = N_1 R \ln\frac{V}{V_1} + N_2 R \ln\frac{V}{V_2} = -N_1 R \ln\frac{N_1}{N} - N_2 R \ln\frac{N_2}{N}
\end{equation}
$> 0$(混合は不可逆過程)。

\textbf{なぜ混合でエントロピーが増えるのか}:仕切りを外す前は、成分1は $V_1$ に、成分2は $V_2$ に閉じ込められている。仕切りを外すと、各成分が全体積 $V$ に広がり、「とり得る配置の数」が増える。これは不可逆過程であり、エントロピーが増大する。図\ref{fig:ex6_mixing}を参照。

\begin{figure}[H]
    \centering
    \includegraphics[width=0.9\textwidth]{figures/ex6_mixing.png}
    \caption{2成分気体の混合。左:仕切りで分離された状態。右:仕切りを外した後の混合状態。混合は不可逆過程で $\Delta S > 0$。}
    \label{fig:ex6_mixing}
\end{figure}

\paragraph{ポアソン分布}

体積 $V$ に $N$ 個の分子が一様に分布する。部分体積 $v$ に1分子が入る確率は $p = v/V$。$n$ 個入る確率は二項分布 $P(n) = \binom{N}{n} p^n (1-p)^{N-n}$。

$N \to \infty$、$p \to 0$ で $Np = a$(一定)の極限をとる。母関数 $F(x) = \sum_n x^n P(n) = [(1-p)+px]^N = [1 + (x-1)p]^N$。$p = a/N$ として
\begin{equation}
F(x) = \left[1 + \frac{a(x-1)}{N}\right]^N \to e^{a(x-1)} \quad (N \to \infty)
\end{equation}
母関数の定義より、$x^n$ の係数が $P(n)$ である。$e^{a(x-1)} = e^{-a} e^{ax} = e^{-a} \sum_{n=0}^{\infty} \frac{(ax)^n}{n!} = e^{-a} \sum_{n=0}^{\infty} \frac{a^n}{n!} x^n$($e^x$ のテイラー展開)。したがって
\begin{equation}
P(n) = \frac{a^n}{n!} e^{-a}, \quad a = \langle n \rangle = Np
\end{equation}
これがポアソン分布である。ポアソン分布の性質:平均 $\langle n \rangle = a$、分散も $a$ である。統計力学では、体積 $v$ に $N$ 個の分子がランダムに分布するとき、$v \ll V$ なら $n$ はポアソン分布に従う(希薄な ideal gas の粒子数ゆらぎ)。

\begin{figure}[H]
    \centering
    \includegraphics[width=0.9\textwidth]{figures/ex6_binomial_poisson.png}
    \caption{二項分布とポアソン分布。$N$ が大きく $p$ が小さいとき、二項分布はポアソン分布に近づく。}
    \label{fig:ex6_poisson}
\end{figure}

% 演習7 (2026/01/09実施)
\section{演習7 (2026年1月9日実施)}

\subsection{I. スターリングの公式}

\subsubsection{この問題で学ぶこと}

$\Gamma$ 関数の被積分関数が鋭いピークを持つとき、その近傍だけで積分するラプラス近似の考え方。$N!$ の大きい $N$ での振る舞いを、$\sqrt{2\pi N} N^N e^{-N}$ で近似できる。統計力学で $\ln N!$ が頻出するため、この近似は必須である。

\subsubsection{問題}

$N! \approx \sqrt{2\pi N} N^N e^{-N}$ を証明せよ。$\Gamma(x+1)$ の被積分関数 $t^x e^{-t}$ の概形を描き、$x \gg 1$ でラプラス近似を用いて導け。

\subsubsection{解答}

\paragraph{前提知識:ラプラス近似とは}

大きな $N$ に対して $N!$ を計算するのは困難である。しかし統計力学では $\ln N!$ が頻出し、$N \gg 1$ のとき $\ln N! \approx N \ln N - N$ という近似(スターリング近似)が使われる。この近似を、$\Gamma$ 関数の積分表示にラプラス近似を適用して導く。

\textbf{ラプラス近似の直感的な考え方}:積分 $I = \int e^{-f(t)} dt$ で、$f(t)$ が $t = t^*$ で鋭い最小を持つとき、被積分関数 $e^{-f(t)}$ は $t^*$ 付近で大きな値を持つ。$t^*$ から離れると急激に小さくなるので、積分への主要な寄与は $t^*$ の近傍から来る。$f(t)$ を $t^*$ の周りで2次まで展開すると、被積分関数はガウス型になり、積分が実行できる。結果として $I \approx \sqrt{2\pi/f''(t^*)} \, e^{-f(t^*)}$ となる。

\paragraph{$\Gamma(x+1)$ への適用}

$\Gamma(x+1) = \int_0^{\infty} t^x e^{-t} dt = \int_0^{\infty} e^{-(x\ln t - t)} dt$。$f(t) = t - x\ln t$ とおくと被積分関数は $e^{-f(t)}$。$f'(t) = 1 - x/t = 0$ より $t^* = x$。$f''(t) = x/t^2$、$f''(t^*) = 1/x$。$f(t^*) = x - x\ln x$。ラプラス近似より
\begin{equation}
\Gamma(x+1) \approx \sqrt{\frac{2\pi}{f''(t^*)}} e^{-f(t^*)} = \sqrt{2\pi x} \, e^{-(x - x\ln x)} = \sqrt{2\pi x} \, x^x e^{-x}
\end{equation}
$x = N$(自然数)のとき $\Gamma(N+1) = N!$(演習1-Vで示した)。よって $N! \approx \sqrt{2\pi N} N^N e^{-N}$(スターリングの公式)。より精密には $N! \approx \sqrt{2\pi N} N^N e^{-N} (1 + 1/(12N) + \cdots)$ のように次の項もあるが、$N \gg 1$ では第1項で十分である。

\textbf{なぜピークは $t = x$ にできるのか}:$f(t) = t - x\ln t$ を $t$ で微分すると $f'(t) = 1 - x/t$。$f'(t) = 0$ より $t^* = x$。$t < x$ では $\ln t$ が小さく $f$ は大きく、$t > x$ では $t$ の増大より $-x\ln t$ の減少が効き $f$ は大きくなる。$t = x$ で $f$ が最小となり、$e^{-f(t)}$ が最大になる。図\ref{fig:ex7_gamma}のように、$x$ が大きいほどピークは鋭くなる。

\textbf{物理的考察}:$\ln N! \approx N\ln N - N$ は、$N$ 個のものを並べる方法の数が $e^{N\ln N - N} = (N/e)^N$ のオーダーであることを示す。統計力学では、$N$ 個の同種粒子の分配関数に $1/N!$ が現れる(不可弁別性)。$\ln Z$ に $\ln N!$ が含まれるため、スターリング近似は必須である。

\begin{figure}[H]
    \centering
    \includegraphics[width=0.75\textwidth]{figures/ex7_gamma_integrand.png}
    \caption{$\Gamma(x+1)$ の被積分関数 $t^x e^{-t}$。$x$ が大きいほど $t = x$ 付近に鋭いピークができる。}
    \label{fig:ex7_gamma}
\end{figure}

\begin{figure}[H]
    \centering
    \includegraphics[width=0.75\textwidth]{figures/ex7_stirling_error.png}
    \caption{スターリング近似の相対誤差。$N$ が大きいほど精度が上がる。}
    \label{fig:ex7_stirling}
\end{figure}

%--------------------------------------
\subsection{II. D次元球の体積}

\subsubsection{問題}

$V_D(R) = \frac{\pi^{D/2} R^D}{\Gamma(D/2 + 1)}$ をガウス積分から導け。

\subsubsection{解答}

\paragraph{ガウス積分}

$I_D = \int_{-\infty}^{\infty} \cdots \int_{-\infty}^{\infty} e^{-(x_1^2 + \cdots + x_D^2)} dx_1 \cdots dx_D$。各変数が独立なので $I_D = \left(\int_{-\infty}^{\infty} e^{-y^2} dy\right)^D = (\sqrt{\pi})^D = \pi^{D/2}$。

\paragraph{球座標への変換}

被積分関数は $r = \sqrt{x_1^2 + \cdots + x_D^2}$ のみの関数(動径方向にのみ依存)。したがって、半径 $r$ から $r+dr$ の球殻(厚さ $dr$)の体積を $C_D r^{D-1} dr$ と書くと、$C_D$ は半径1の $D$ 次元球の表面積である。$D=2$ のとき周長 $2\pi$、$D=3$ のとき表面積 $4\pi$ に対応する。よって
\begin{equation}
I_D = C_D \int_0^{\infty} r^{D-1} e^{-r^2} dr
\end{equation}
$u = r^2$ とおくと $du = 2r\,dr$、$r^{D-1} dr = \frac{1}{2} u^{(D/2)-1} du$。$\int_0^{\infty} u^{(D/2)-1} e^{-u} du = \Gamma(D/2)$ なので
\begin{equation}
I_D = C_D \cdot \frac{1}{2} \Gamma(D/2)
\end{equation}
$I_D = \pi^{D/2}$ より $C_D = \frac{2\pi^{D/2}}{\Gamma(D/2)}$。半径 $R$ の球の表面積は $S_D(R) = C_D R^{D-1}$、体積は
\begin{equation}
V_D(R) = \int_0^R S_D(r) dr = C_D \int_0^R r^{D-1} dr = \frac{C_D R^D}{D} = \frac{2\pi^{D/2} R^D}{D \, \Gamma(D/2)}
\end{equation}
$\Gamma(D/2 + 1) = (D/2) \Gamma(D/2)$ より $\frac{1}{D\,\Gamma(D/2)} = \frac{1}{2\Gamma(D/2+1)}$。したがって
\begin{equation}
V_D(R) = \frac{\pi^{D/2} R^D}{\Gamma(D/2 + 1)}
\end{equation}

\textbf{具体例の確認}:$D=2$ のとき $\Gamma(2)=1$ より $V_2(R) = \pi R^2$(円の面積)。$D=3$ のとき $\Gamma(5/2) = (3/2)(1/2)\sqrt{\pi} = (3/4)\sqrt{\pi}$ より $V_3(R) = \pi^{3/2} R^3 / [(3/4)\sqrt{\pi}] = (4/3)\pi R^3$(球の体積)。公式が正しいことが確認できる。

\textbf{なぜ統計力学で重要か}:$N$ 個の粒子の運動量空間は $3N$ 次元である。エネルギー $E$ 以下の状態数は、$3N$ 次元球の体積に比例する。本問の公式で $D = 3N$ とおくと、状態数の体積依存性が求まる。

%--------------------------------------
\subsection{III. 調和振動子の分配関数}

\subsubsection{この問題で学ぶこと}

統計力学の核心的概念である分配関数 $Z$。系が状態 $i$ にある確率は $e^{-\beta E_i}/Z$(ボルツマン分布)で与えられる。調和振動子(量子力学的)の分配関数を計算し、平均エネルギーを求める。固体の比熱(アインシュタイン模型)の基礎となる。

\subsubsection{問題}

$N$ 個の独立な調和振動子のエネルギー準位 $E = \hbar\omega(M + N/2)$($M = 0,1,2,\ldots$)について、分配関数 $Z_N$ と平均エネルギー $\langle E \rangle$ を求めよ。

\subsubsection{解答}

\paragraph{前提知識:分配関数とは}

熱平衡状態で、系が状態 $i$(エネルギー $E_i$)にある確率は、ボルツマン分布 $P(i) = e^{-\beta E_i}/Z$ で与えられる。ここで $\beta = 1/(k_B T)$ であり、$Z = \sum_i e^{-\beta E_i}$ は確率の規格化定数で、分配関数と呼ばれる。熱力学におけるヘルムホルツの自由エネルギーは $F = -k_B T \ln Z$ で与えられる。

\paragraph{1個の振動子}

$Z_1 = \sum_{m=0}^{\infty} e^{-\beta\hbar\omega(m+1/2)} = e^{-\beta\hbar\omega/2} \sum_{m=0}^{\infty} (e^{-\beta\hbar\omega})^m$。等比級数 $\sum_{m=0}^{\infty} x^m = 1/(1-x)$($|x|<1$)を用いる。$|e^{-\beta\hbar\omega}| < 1$ は $T>0$ で常に成り立つ(収束条件)。よって $Z_1 = \frac{e^{-\beta\hbar\omega/2}}{1 - e^{-\beta\hbar\omega}}$。独立な $N$ 個では $Z_N = Z_1^N$。

\paragraph{平均エネルギー}

$\langle E \rangle = \frac{1}{Z} \sum_i E_i e^{-\beta E_i} = -\frac{\partial}{\partial\beta} \ln Z$。$Z_N = Z_1^N$ より
\begin{equation}
\langle E \rangle = -\frac{\partial}{\partial\beta}(N \ln Z_1) = N \hbar\omega\left(\frac{1}{2} + \frac{1}{e^{\beta\hbar\omega}-1}\right)
\end{equation}

%--------------------------------------
\subsection{IV. 古典理想気体}

\subsubsection{この問題で学ぶこと}

古典統計力学における理想気体の分配関数。位相空間の積分から $Z_N$ を計算し、ヘルムホルツの自由エネルギー $F = -k_B T \ln Z_N$ が熱力学と一致することを確認する。また、状態数 $W_N(E)$ とエントロピー $S = k_B \ln W_N(E)$ の関係(ボルツマンの原理)を理解する。

\subsubsection{問題}

古典理想気体の分配関数を計算し、$F = -k_B T \ln Z_N$ が熱力学の式と一致することを確かめよ。また $W_N(E)$ を求め、$S = k_B \ln W_N(E)$ がエントロピーと一致することを示せ。

\subsubsection{解答}

\paragraph{導出の戦略}

分配関数 $Z_N = (1/N!h^{3N})\int e^{-\beta H} d\Gamma$。$H = \sum p^2/(2m)$ なので、運動量積分はガウス積分、位置積分は $V^N$。$W_N(E)$ は位相空間で $H \leq E$ の体積に比例し、$E^{3N/2}$ のオーダー。$E = (3/2)Nk_B T$ を代入して $S$ が熱力学と一致することを示す。

\paragraph{分配関数の計算}

$Z_N = \frac{1}{N! h^{3N}} \int d\Gamma \, e^{-\beta H}$。$H = \sum_{i,\alpha} \frac{p_{i\alpha}^2}{2m}$。運動量積分は各自由度で $\int_{-\infty}^{\infty} e^{-\beta p^2/(2m)} dp = \sqrt{2\pi m k_B T}$。$3N$ 自由度で $(\sqrt{2\pi m k_B T})^{3N}$。位置積分は $V^N$。したがって
\begin{equation}
Z_N = \frac{V^N}{N!} \left(\frac{2\pi m k_B T}{h^2}\right)^{3N/2} = \frac{1}{N!}\left(\frac{V}{\lambda^3}\right)^N
\end{equation}
$\lambda = h/\sqrt{2\pi m k_B T}$ は熱的ド・ブロイ波長。$F = -k_B T \ln Z_N$ にスターリング近似 $\ln N! \approx N\ln N - N$ を適用すると、熱力学の式 $F = Nk_B T[\ln(n\lambda^3) - 1]$($n = N/V$)と一致する。

\paragraph{状態数 $W_N(E)$}

エネルギー $E$ 以下の状態数は、位相空間で $H \leq E$ の体積に比例する。$H = p^2/(2m)$($p^2 = \sum p_{i\alpha}^2$)なので、$p^2 \leq 2mE$ の $3N$ 次元球の体積。$W_N(E) \propto V^N \cdot (2mE)^{3N/2}$。したがって $W_N(E) \propto E^{3N/2}$(微分として $dW/dE \propto E^{3N/2 - 1}$)。$S = k_B \ln W_N(E)$ に $E = \frac{3}{2}Nk_B T$ を代入すると、熱力学のエントロピーと一致する。

\textbf{なぜ $1/N!$ が必要か}:同種粒子は区別できない。$N$ 個の粒子の入れ替えは同じ状態なので、位相空間の体積を $N!$ で割る必要がある(ギブスのパラドックスを避ける)。

\textbf{なぜ $h^{3N}$ で割るか}:位相空間の「1状態」が占有する体積は $h$(プランク定数)のオーダーである。これは不確定性関係 $\Delta q \Delta p \sim h$ に起因する。量子統計との対応を正しくするために必要である。

\textbf{熱的ド・ブロイ波長 $\lambda$ の意味}:$\lambda = h/\sqrt{2\pi m k_B T}$ は、温度 $T$ での典型的な運動量 $\sim \sqrt{m k_B T}$ に対応するド・ブロイ波長である。$\lambda \ll$(粒子間距離)のとき古典近似が成り立つ。$n\lambda^3 \ll 1$ のとき理想気体の近似が有効である。

\textbf{図\ref{fig:ex7_W}の物理的意味}:$W_N(E) e^{-\beta E}$ のピーク位置 $E^*$ が、熱力学的な内部エネルギーに対応する。このピークが鋭いこと($N \gg 1$)が、熱力学がよく定義される理由である。微正準集団と正準集団の等価性は、このピークの鋭さに基づく。

\begin{figure}[H]
    \centering
    \includegraphics[width=0.75\textwidth]{figures/ex7_W_exp_betaE.png}
    \caption{被積分関数 $W_N(E) e^{-\beta E}$ は鋭いピークを持つ。ピーク位置 $E^*$ で $S$ と $E$ の関係が熱力学的な対応を与える。}
    \label{fig:ex7_W}
\end{figure}


\end{document}
