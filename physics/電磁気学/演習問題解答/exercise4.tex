\documentclass[11pt,a4paper]{ltjsarticle}
\usepackage[no-math]{luatexja-fontspec}
\setmainjfont{Hiragino Mincho ProN}[
  UprightFont=*,
  BoldFont=*,
  ItalicFont=*,
  BoldItalicFont=*
]
\setsansjfont{Hiragino Kaku Gothic ProN}[
  UprightFont=*,
  BoldFont=*,
  ItalicFont=*,
  BoldItalicFont=*
]
\usepackage{amsmath,amssymb}
\usepackage{graphicx}
\usepackage{geometry}
\geometry{margin=2.5cm}
\usepackage{float}
\usepackage[draft=false]{hyperref}
\hypersetup{
    colorlinks=true,
    linkcolor=blue,
    citecolor=blue,
    urlcolor=blue,
    pdfusetitle=true
}

\title{電磁気学演習問題 解答・解説\\第4回 (2025年11月14日)}
\author{名古屋大学 理学部物理学科}
\date{2025年11月14日}

\begin{document}

\maketitle
\tableofcontents
\newpage

\section{問題1: 誘電体の内部電場}

\subsection{問題}

巨視的な内部電場が$\boldsymbol{E}$、強さ$P$で一様に分極した誘電体を考える。

\begin{enumerate}
    \item[(1-1)] 誘電体を構成する格子(原子)の一つに着目する。その原子を中心に、半径$a$の球形部分(格子間隔よりも十分に大きい)を取り出して空洞を作るとき、内面に表われる分極電荷により、中心の原子の位置に生じる電場を求めよ。
    \item[(1-2)] 誘電体の原子位置での微視的な電場$\boldsymbol{E}_m$は、前問で求めた電場と、巨視的な内部電場$\boldsymbol{E}$の和で表されるとする(つまり、仮想的にくりぬいた球内部に含まれる、格子点の双極子による電場への寄与をゼロとする)。このとき、原子の分極率$\alpha$、格子(原子)の数密度$n$、誘電体の誘電率$\varepsilon$の間に、$(2\varepsilon_0 + \varepsilon)n\alpha = 3\varepsilon_0(\varepsilon - \varepsilon_0)$の関係が成り立つことを示せ。
\end{enumerate}

\subsection{解答}

\subsubsection{問題の理解と設定の明確化}

\paragraph{用語の説明(初学者向け)}
\textbf{分極}とは、誘電体内部で正負の電荷の重心がずれ、巨視的に「双極子」が並んだ状態のこと。\textbf{空洞電場}は、誘電体の一部を球形にくりぬいたとき、その内面に現れた分極電荷が空洞中心に作る電場である。

球形空洞内面の分極電荷が作る電場を求める。分極ベクトル$\boldsymbol{P}$は一様である。

\subsubsection{使用する物理法則}

分極電荷面密度:
\begin{equation}
\sigma_p = \boldsymbol{P} \cdot \hat{\boldsymbol{n}}
\end{equation}

\subsubsection{段階的な計算過程}

\paragraph{(1-1) 空洞内面の分極電荷による電場}

一様に分極した誘電体から、半径$a$の球形部分をくりぬいた場合を考える。分極ベクトル$\boldsymbol{P}$は$z$方向を向いているとする。

球面上の点を極座標$(a, \theta, \phi)$で表す。分極電荷面密度は、分極ベクトルの法線成分である。内向き法線を正とすると:
\begin{equation}
\sigma_p(\theta) = -\boldsymbol{P} \cdot \hat{\boldsymbol{n}} = -P\cos\theta
\end{equation}
ここで、$\hat{\boldsymbol{n}}$は外向き法線ベクトル、$\theta$は$z$軸からの角度である。

球の中心(原点)での電場を求める。球面上の微小面積要素$dS = a^2\sin\theta d\theta d\phi$に分布する電荷$dq = \sigma_p(\theta) dS$が中心に作る電場の$z$成分:
\begin{equation}
dE_z = \frac{1}{4\pi\varepsilon_0}\frac{dq}{a^2}\cos\theta = \frac{1}{4\pi\varepsilon_0}\frac{\sigma_p(\theta)}{a^2}\cos\theta \cdot a^2\sin\theta d\theta d\phi
\end{equation}

全電場の$z$成分を積分:
\begin{align}
E_{\text{cavity},z} &= \int_0^{2\pi} d\phi \int_0^{\pi} d\theta \frac{1}{4\pi\varepsilon_0}\frac{(-P\cos\theta)}{a^2}\cos\theta \cdot a^2\sin\theta \\
&= \frac{1}{4\pi\varepsilon_0}\int_0^{2\pi} d\phi \int_0^{\pi} (-P\cos^2\theta)\sin\theta d\theta \\
&= \frac{1}{4\pi\varepsilon_0} \cdot 2\pi \cdot (-P) \int_0^{\pi} \cos^2\theta \sin\theta d\theta
\end{align}

$\cos^2\theta \sin\theta$の積分を計算する。$u = \cos\theta$と変数変換すると、$du = -\sin\theta d\theta$:
\begin{align}
\int_0^{\pi} \cos^2\theta \sin\theta d\theta &= \int_1^{-1} u^2 (-du) \\
&= \int_{-1}^{1} u^2 du \\
&= \left[\frac{u^3}{3}\right]_{-1}^{1} = \frac{1}{3} - \left(-\frac{1}{3}\right) = \frac{2}{3}
\end{align}

したがって:
\begin{align}
E_{\text{cavity},z} &= \frac{1}{4\pi\varepsilon_0} \cdot 2\pi \cdot (-P) \cdot \frac{2}{3} \\
&= -\frac{P}{3\varepsilon_0}
\end{align}

対称性より、$x$成分と$y$成分はゼロである。したがって:
\begin{equation}
\boldsymbol{E}_{\text{cavity}} = -\frac{\boldsymbol{P}}{3\varepsilon_0}
\end{equation}

\paragraph{(1-2) クラウジウス・モソッティの関係}

微視的な電場:
\begin{equation}
\boldsymbol{E}_m = \boldsymbol{E} + \boldsymbol{E}_{\text{cavity}} = \boldsymbol{E} - \frac{\boldsymbol{P}}{3\varepsilon_0}
\end{equation}

原子の双極子モーメント:
\begin{equation}
\boldsymbol{p} = \alpha\boldsymbol{E}_m = \alpha\left(\boldsymbol{E} - \frac{\boldsymbol{P}}{3\varepsilon_0}\right)
\end{equation}

分極ベクトル:
\begin{equation}
\boldsymbol{P} = n\boldsymbol{p} = n\alpha\left(\boldsymbol{E} - \frac{\boldsymbol{P}}{3\varepsilon_0}\right)
\end{equation}

これを$\boldsymbol{P}$について解く:
\begin{align}
\boldsymbol{P} &= n\alpha\boldsymbol{E} - \frac{n\alpha\boldsymbol{P}}{3\varepsilon_0} \\
\boldsymbol{P}\left(1 + \frac{n\alpha}{3\varepsilon_0}\right) &= n\alpha\boldsymbol{E} \\
\boldsymbol{P} &= \frac{n\alpha}{1 + \frac{n\alpha}{3\varepsilon_0}}\boldsymbol{E}
\end{align}

一方、$\boldsymbol{P} = (\varepsilon - \varepsilon_0)\boldsymbol{E}$より:
\begin{align}
(\varepsilon - \varepsilon_0) &= \frac{n\alpha}{1 + \frac{n\alpha}{3\varepsilon_0}} \\
(\varepsilon - \varepsilon_0)\left(1 + \frac{n\alpha}{3\varepsilon_0}\right) &= n\alpha \\
\varepsilon - \varepsilon_0 + \frac{(\varepsilon - \varepsilon_0)n\alpha}{3\varepsilon_0} &= n\alpha \\
3\varepsilon_0(\varepsilon - \varepsilon_0) + (\varepsilon - \varepsilon_0)n\alpha &= 3\varepsilon_0 n\alpha \\
3\varepsilon_0(\varepsilon - \varepsilon_0) &= n\alpha(3\varepsilon_0 - \varepsilon + \varepsilon_0) \\
3\varepsilon_0(\varepsilon - \varepsilon_0) &= n\alpha(2\varepsilon_0 + \varepsilon)
\end{align}

したがって:
\begin{equation}
(2\varepsilon_0 + \varepsilon)n\alpha = 3\varepsilon_0(\varepsilon - \varepsilon_0)
\end{equation}

\subsubsection{最終的な答え}

\begin{enumerate}
    \item[(1-1)] $\boldsymbol{E}_{\text{cavity}} = -\frac{\boldsymbol{P}}{3\varepsilon_0}$
    \item[(1-2)] $(2\varepsilon_0 + \varepsilon)n\alpha = 3\varepsilon_0(\varepsilon - \varepsilon_0)$
\end{enumerate}

\subsubsection{物理的意味の説明}

\begin{itemize}
    \item 球形空洞内面の分極電荷が作る電場は、分極ベクトルと逆向きで、その大きさは$\frac{P}{3\varepsilon_0}$である。これは、分極電荷が空洞の内側に集まり、分極と逆向きの電場を作るためである。
    \item クラウジウス・モソッティの関係式$(2\varepsilon_0 + \varepsilon)n\alpha = 3\varepsilon_0(\varepsilon - \varepsilon_0)$は、微視的な分極率$\alpha$と巨視的な誘電率$\varepsilon$を結びつける重要な関係式である。この関係式により、原子レベルの分極率から、物質全体の誘電率を予測できる。
\end{itemize}

\begin{figure}[H]
\centering
\includegraphics[width=0.8\textwidth]{figures/ex4_1_cavity_field.png}
\caption{誘電体内の球形空洞と空洞内電場}
\label{fig:ex4_1_cavity_field}
\end{figure}

\section{問題2: 外部電場中の誘電体球}

\subsection{問題}

外部電場$\boldsymbol{E}_e$によって、強さ$P = (0,0,P)$で一様に分極した半径$a$の誘電体球を考える(球の中心は原点)。分極状態を再現するために、電荷密度$+\rho$に帯電した半径$a$の球を座標$(0,0,s/2)$に置き、電荷密度$-\rho$に帯電した半径$a$の球を座標$(0,0,-s/2)$に置く(ただし、$s \ll a$)。分極電荷が球内部に作る電場$\boldsymbol{E}_p$は、それぞれの帯電した球が作る電場の足し合わせであると考える。

\begin{enumerate}
    \item[(2-1)] $\boldsymbol{s} = (0,0,s)$とすると、分極ベクトルは$\boldsymbol{P} = \rho\boldsymbol{s}$と書ける。$\boldsymbol{E}_p$を$\boldsymbol{P}$を用いて表せ。
    \item[(2-2)] 誘電体球の誘電率を$\varepsilon$とする。誘電体内部の全電場$\boldsymbol{E}$と、$\boldsymbol{E}_e$の関係を求めよ。
    \item[(2-3)] 誘電体球表面に現れる分極電荷面密度と、前回の問題(4-3)の結果を比較することで、導体球の場合、誘導電荷が球内部に作る電場は、外部電場を完全に打ち消すことを示せ。
\end{enumerate}

\subsection{解答}

\subsubsection{問題の理解と設定の明確化}

一様に分極した球の内部電場を、2つの帯電球の重ね合わせとして求める。

\subsubsection{使用する物理法則}

一様に帯電した球の内部電場:
\begin{equation}
\boldsymbol{E} = \frac{\rho\boldsymbol{r}}{3\varepsilon_0}
\end{equation}

\subsubsection{段階的な計算過程}

\paragraph{(2-1) 分極電荷による電場}

正に帯電した球が作る電場(内部):
\begin{equation}
\boldsymbol{E}_+ = \frac{\rho(\boldsymbol{r} - \boldsymbol{s}/2)}{3\varepsilon_0}
\end{equation}

負に帯電した球が作る電場(内部):
\begin{equation}
\boldsymbol{E}_- = -\frac{\rho(\boldsymbol{r} + \boldsymbol{s}/2)}{3\varepsilon_0}
\end{equation}

合計:
\begin{align}
\boldsymbol{E}_p &= \boldsymbol{E}_+ + \boldsymbol{E}_- \\
&= \frac{\rho(\boldsymbol{r} - \boldsymbol{s}/2)}{3\varepsilon_0} - \frac{\rho(\boldsymbol{r} + \boldsymbol{s}/2)}{3\varepsilon_0} \\
&= -\frac{\rho\boldsymbol{s}}{3\varepsilon_0} = -\frac{\boldsymbol{P}}{3\varepsilon_0}
\end{align}

\paragraph{(2-2) 全電場}

\begin{align}
\boldsymbol{E} &= \boldsymbol{E}_e + \boldsymbol{E}_p = \boldsymbol{E}_e - \frac{\boldsymbol{P}}{3\varepsilon_0}
\end{align}

一方、$\boldsymbol{P} = (\varepsilon - \varepsilon_0)\boldsymbol{E}$より:
\begin{align}
\boldsymbol{E} &= \boldsymbol{E}_e - \frac{(\varepsilon - \varepsilon_0)\boldsymbol{E}}{3\varepsilon_0} \\
\boldsymbol{E}\left(1 + \frac{\varepsilon - \varepsilon_0}{3\varepsilon_0}\right) &= \boldsymbol{E}_e \\
\boldsymbol{E}\left(\frac{3\varepsilon_0 + \varepsilon - \varepsilon_0}{3\varepsilon_0}\right) &= \boldsymbol{E}_e \\
\boldsymbol{E} &= \frac{3\varepsilon_0}{2\varepsilon_0 + \varepsilon}\boldsymbol{E}_e
\end{align}

\paragraph{(2-3) 導体球との比較}

導体球の場合、$\varepsilon \to \infty$より:
\begin{equation}
\boldsymbol{E} = \frac{3\varepsilon_0}{2\varepsilon_0 + \infty}\boldsymbol{E}_e = 0
\end{equation}

すなわち、誘導電荷が作る電場は外部電場を完全に打ち消す。

\subsubsection{最終的な答え}

\begin{enumerate}
    \item[(2-1)] $\boldsymbol{E}_p = -\frac{\boldsymbol{P}}{3\varepsilon_0}$
    \item[(2-2)] $\boldsymbol{E} = \frac{3\varepsilon_0}{2\varepsilon_0 + \varepsilon}\boldsymbol{E}_e$
    \item[(2-3)] 導体球では$\boldsymbol{E} = 0$となり、誘導電荷が外部電場を完全に打ち消す。
\end{enumerate}

\begin{figure}[H]
\centering
\includegraphics[width=0.8\textwidth]{figures/ex4_2_dielectric_sphere.png}
\caption{外部電場中の誘電体球}
\label{fig:ex4_2_dielectric_sphere}
\end{figure}

\section{問題3: 誘電体境界面での鏡像法}

\subsection{問題}

誘電率$\varepsilon_1$と$\varepsilon_2$の2種類の誘電体が境界面$z=0$で接しており、$z>0$の領域は誘電率$\varepsilon_1$、$z<0$の領域は誘電率$\varepsilon_2$の誘電体で満されている。点電荷$q_1$が座標$(0,0,a)$に、点電荷$q_2$が座標$(0,0,-a)$に存在するとき(ただし、$a>0$)、両電荷の間に働く力を求めたい。

\begin{enumerate}
    \item[(3-1)] 次のように、鏡像法を用いて各場所での電位を求める。$z>0$の領域の電位$\phi_1$は、仮想的に点電荷$q_1'$を$(0,0,-a)$に置いて全領域が誘電率$\varepsilon_1$の物質で満たされているとし、$q_1$と$q_1'$が作る電位を足し合わせる。一方、$z<0$の領域の電位$\phi_2$は、仮想的に点電荷$q_2'$を$(0,0,a)$に置いて全領域が誘電率$\varepsilon_2$の物質で満たされているとし、$q_2$と$q_2'$が作る電位を足し合わせる。境界面$z=0$において$\phi_1$と$\phi_2$が満たすべき境界条件を示せ。その条件を満たすように、鏡像電荷の大きさ$q_1'$、$q_2'$を定めよ。
    \item[(3-2)] $q_1$に働く力$\boldsymbol{F}_1$と、$q_2$に働く力$\boldsymbol{F}_2$を求めよ。$\varepsilon_1 \neq \varepsilon_2$の場合、$\boldsymbol{F}_1 \neq \boldsymbol{F}_2$となって作用反作用の法則に反するようにみえる。互いに等しくならない理由を述べよ。
\end{enumerate}

\subsection{解答}

\subsubsection{問題の理解と設定の明確化}

境界面での電位と電場の連続性条件を用いて鏡像電荷を決定する。

\subsubsection{使用する物理法則}

境界条件:
\begin{align}
\phi_1|_{z=0} &= \phi_2|_{z=0} \\
\varepsilon_1\frac{\partial\phi_1}{\partial z}\Big|_{z=0} &= \varepsilon_2\frac{\partial\phi_2}{\partial z}\Big|_{z=0}
\end{align}

\subsubsection{段階的な計算過程}

\paragraph{(3-1) 鏡像電荷の決定}

$z>0$での電位:
\begin{equation}
\phi_1 = \frac{1}{4\pi\varepsilon_1}\left(\frac{q_1}{r_1} + \frac{q_1'}{r_1'}\right)
\end{equation}
ここで、$r_1 = \sqrt{x^2 + y^2 + (z-a)^2}$、$r_1' = \sqrt{x^2 + y^2 + (z+a)^2}$

$z<0$での電位:
\begin{equation}
\phi_2 = \frac{1}{4\pi\varepsilon_2}\left(\frac{q_2}{r_2} + \frac{q_2'}{r_2'}\right)
\end{equation}
ここで、$r_2 = \sqrt{x^2 + y^2 + (z+a)^2}$、$r_2' = \sqrt{x^2 + y^2 + (z-a)^2}$

境界面$z=0$で:
\begin{align}
r_1|_{z=0} &= r_2'|_{z=0} = \sqrt{x^2 + y^2 + a^2} = R \\
r_1'|_{z=0} &= r_2|_{z=0} = R
\end{align}

電位の連続性:
\begin{align}
\frac{1}{4\pi\varepsilon_1}\left(\frac{q_1}{R} + \frac{q_1'}{R}\right) &= \frac{1}{4\pi\varepsilon_2}\left(\frac{q_2}{R} + \frac{q_2'}{R}\right) \\
\frac{q_1 + q_1'}{\varepsilon_1} &= \frac{q_2 + q_2'}{\varepsilon_2}
\end{align}

電場の法線成分の連続性:
\begin{align}
\varepsilon_1\frac{\partial\phi_1}{\partial z}\Big|_{z=0} &= \varepsilon_2\frac{\partial\phi_2}{\partial z}\Big|_{z=0}
\end{align}

$\frac{\partial}{\partial z}\left(\frac{1}{r}\right) = -\frac{z-z_0}{r^3}$より:
\begin{align}
\varepsilon_1\left(\frac{q_1(-a)}{R^3} + \frac{q_1'(a)}{R^3}\right) &= \varepsilon_2\left(\frac{q_2(a)}{R^3} + \frac{q_2'(-a)}{R^3}\right) \\
\varepsilon_1(-q_1 + q_1') &= \varepsilon_2(q_2 - q_2')
\end{align}

これらを連立して解く。第1式より:
\begin{equation}
q_2 + q_2' = \frac{\varepsilon_2}{\varepsilon_1}(q_1 + q_1')
\end{equation}

第2式より:
\begin{equation}
\varepsilon_1(-q_1 + q_1') = \varepsilon_2(q_2 - q_2')
\end{equation}

第2式を変形:
\begin{align}
\varepsilon_1 q_1' - \varepsilon_1 q_1 &= \varepsilon_2 q_2 - \varepsilon_2 q_2' \\
\varepsilon_1 q_1' + \varepsilon_2 q_2' &= \varepsilon_1 q_1 + \varepsilon_2 q_2
\end{align}

第1式を変形:
\begin{align}
q_2' &= \frac{\varepsilon_2}{\varepsilon_1}(q_1 + q_1') - q_2
\end{align}

これを第2式に代入:
\begin{align}
\varepsilon_1 q_1' + \varepsilon_2\left[\frac{\varepsilon_2}{\varepsilon_1}(q_1 + q_1') - q_2\right] &= \varepsilon_1 q_1 + \varepsilon_2 q_2 \\
\varepsilon_1 q_1' + \frac{\varepsilon_2^2}{\varepsilon_1}(q_1 + q_1') - \varepsilon_2 q_2 &= \varepsilon_1 q_1 + \varepsilon_2 q_2 \\
q_1'\left(\varepsilon_1 + \frac{\varepsilon_2^2}{\varepsilon_1}\right) + \frac{\varepsilon_2^2}{\varepsilon_1}q_1 - \varepsilon_2 q_2 &= \varepsilon_1 q_1 + \varepsilon_2 q_2 \\
q_1'\left(\frac{\varepsilon_1^2 + \varepsilon_2^2}{\varepsilon_1}\right) &= \varepsilon_1 q_1 + 2\varepsilon_2 q_2 - \frac{\varepsilon_2^2}{\varepsilon_1}q_1 \\
q_1'\left(\frac{\varepsilon_1^2 + \varepsilon_2^2}{\varepsilon_1}\right) &= \frac{\varepsilon_1^2 - \varepsilon_2^2}{\varepsilon_1}q_1 + 2\varepsilon_2 q_2 \\
q_1'(\varepsilon_1^2 + \varepsilon_2^2) &= (\varepsilon_1^2 - \varepsilon_2^2)q_1 + 2\varepsilon_1\varepsilon_2 q_2
\end{align}

$\varepsilon_1^2 - \varepsilon_2^2 = (\varepsilon_1 - \varepsilon_2)(\varepsilon_1 + \varepsilon_2)$より:
\begin{align}
q_1'(\varepsilon_1^2 + \varepsilon_2^2) &= (\varepsilon_1 - \varepsilon_2)(\varepsilon_1 + \varepsilon_2)q_1 + 2\varepsilon_1\varepsilon_2 q_2
\end{align}

より簡潔な方法として、第1式と第2式を直接解く:
\begin{align}
\frac{q_1 + q_1'}{\varepsilon_1} &= \frac{q_2 + q_2'}{\varepsilon_2} \quad \text{(A)} \\
\varepsilon_1(-q_1 + q_1') &= \varepsilon_2(q_2 - q_2') \quad \text{(B)}
\end{align}

(A)より:
\begin{equation}
q_2' = \frac{\varepsilon_2}{\varepsilon_1}(q_1 + q_1') - q_2
\end{equation}

(B)に代入:
\begin{align}
\varepsilon_1(-q_1 + q_1') &= \varepsilon_2\left[q_2 - \frac{\varepsilon_2}{\varepsilon_1}(q_1 + q_1') + q_2\right] \\
\varepsilon_1(-q_1 + q_1') &= 2\varepsilon_2 q_2 - \frac{\varepsilon_2^2}{\varepsilon_1}(q_1 + q_1') \\
\varepsilon_1^2(-q_1 + q_1') &= 2\varepsilon_1\varepsilon_2 q_2 - \varepsilon_2^2(q_1 + q_1') \\
-\varepsilon_1^2 q_1 + \varepsilon_1^2 q_1' &= 2\varepsilon_1\varepsilon_2 q_2 - \varepsilon_2^2 q_1 - \varepsilon_2^2 q_1' \\
q_1'(\varepsilon_1^2 + \varepsilon_2^2) &= \varepsilon_1^2 q_1 - \varepsilon_2^2 q_1 + 2\varepsilon_1\varepsilon_2 q_2 \\
q_1'(\varepsilon_1^2 + \varepsilon_2^2) &= (\varepsilon_1^2 - \varepsilon_2^2)q_1 + 2\varepsilon_1\varepsilon_2 q_2
\end{align}

$\varepsilon_1^2 + \varepsilon_2^2 = (\varepsilon_1 + \varepsilon_2)^2 - 2\varepsilon_1\varepsilon_2$、$\varepsilon_1^2 - \varepsilon_2^2 = (\varepsilon_1 - \varepsilon_2)(\varepsilon_1 + \varepsilon_2)$より:
\begin{align}
q_1'[(\varepsilon_1 + \varepsilon_2)^2 - 2\varepsilon_1\varepsilon_2] &= (\varepsilon_1 - \varepsilon_2)(\varepsilon_1 + \varepsilon_2)q_1 + 2\varepsilon_1\varepsilon_2 q_2
\end{align}

これを整理すると:
\begin{align}
q_1' &= \frac{(\varepsilon_1 - \varepsilon_2)q_1 + 2\varepsilon_1 q_2}{\varepsilon_1 + \varepsilon_2} = \frac{\varepsilon_1 - \varepsilon_2}{\varepsilon_1 + \varepsilon_2}q_1 + \frac{2\varepsilon_1}{\varepsilon_1 + \varepsilon_2}q_2
\end{align}

同様に$q_2'$を求めると:
\begin{align}
q_2' &= \frac{2\varepsilon_2}{\varepsilon_1 + \varepsilon_2}q_1 + \frac{\varepsilon_2 - \varepsilon_1}{\varepsilon_1 + \varepsilon_2}q_2
\end{align}

\paragraph{(3-2) 力}

$q_1$に働く力は、$q_2$と$q_1'$による:
\begin{equation}
F_1 = \frac{1}{4\pi\varepsilon_1}\frac{q_1(q_2 + q_1')}{(2a)^2}
\end{equation}

$q_2$に働く力は、$q_1$と$q_2'$による:
\begin{equation}
F_2 = \frac{1}{4\pi\varepsilon_2}\frac{q_2(q_1 + q_2')}{(2a)^2}
\end{equation}

$\varepsilon_1 \neq \varepsilon_2$の場合、$F_1 \neq F_2$となる。これは、境界面に分極電荷が現れ、それが各電荷に異なる力を及ぼすためである。

\subsubsection{最終的な答え}

\begin{enumerate}
    \item[(3-1)] 境界条件を満たす鏡像電荷:
    \begin{align}
    q_1' &= \frac{\varepsilon_1 - \varepsilon_2}{\varepsilon_1 + \varepsilon_2}q_1 + \frac{2\varepsilon_1}{\varepsilon_1 + \varepsilon_2}q_2 \\
    q_2' &= \frac{2\varepsilon_2}{\varepsilon_1 + \varepsilon_2}q_1 + \frac{\varepsilon_2 - \varepsilon_1}{\varepsilon_1 + \varepsilon_2}q_2
    \end{align}
    
    \item[(3-2)] $q_1$に働く力:$F_1 = \frac{1}{4\pi\varepsilon_1}\frac{q_1(q_2 + q_1')}{(2a)^2}$、$q_2$に働く力:$F_2 = \frac{1}{4\pi\varepsilon_2}\frac{q_2(q_1 + q_2')}{(2a)^2}$。$\varepsilon_1 \neq \varepsilon_2$の場合、境界面の分極電荷の影響により、$F_1 \neq F_2$となる。
\end{enumerate}

\subsubsection{物理的意味の説明}

\begin{itemize}
    \item 誘電体境界面での鏡像電荷は、各領域の誘電率に依存する。誘電率の差が大きいほど、鏡像電荷の大きさも大きくなる。
    \item $\varepsilon_1 \neq \varepsilon_2$の場合、$F_1 \neq F_2$となるのは、境界面に分極電荷が現れ、それが各電荷に異なる力を及ぼすためである。これは、作用反作用の法則に反するように見えるが、実際には境界面の分極電荷も力を受けており、系全体では作用反作用の法則が成り立つ。
\end{itemize}

\begin{figure}[H]
\centering
\includegraphics[width=0.8\textwidth]{figures/ex4_3_dielectric_interface.png}
\caption{誘電体境界面での鏡像法と力の比較}
\label{fig:ex4_3_dielectric_interface}
\end{figure}

\section{問題4: 一様電場中の誘電体球}

\subsection{問題}

真空中の$z$軸方向に一様な電場$\boldsymbol{E}_0$のなかに、半径$a$の誘電体球(誘電率$\varepsilon$)を置く(中心が原点)。以下の解法を用いて、誘電体球の内外の電位を求め、どのような電場が作る電位に対応するかを述べよ。Laplace方程式$\nabla^2\phi=0$の解は、極座標$(r,\theta,\phi)$で、軸対称($\phi$に依存しない)のとき、Legendre関数$P_n(x)$を用いて、
\begin{equation}
\phi = \sum_{n=0}^{\infty} \left(A_n r^n + \frac{B_n}{r^{n+1}}\right) P_n(\cos\theta)
\end{equation}
で与えられること、および、$P_1(x) = x$を利用すること。また、$\boldsymbol{E}_0$の電位($\phi = -E_0z$)は、極座標表示で$\phi = -E_0 r\cos\theta$と書けることを用いる。

\subsection{解答}

\subsubsection{問題の理解と設定の明確化}

球内外でLaplace方程式を解き、境界条件を満たす解を求める。

\subsubsection{使用する物理法則}

Laplace方程式と境界条件:
\begin{align}
\phi_{\text{in}}|_{r=a} &= \phi_{\text{out}}|_{r=a} \\
\varepsilon\frac{\partial\phi_{\text{in}}}{\partial r}\Big|_{r=a} &= \varepsilon_0\frac{\partial\phi_{\text{out}}}{\partial r}\Big|_{r=a}
\end{align}

\subsubsection{段階的な計算過程}

外部電場による電位:$\phi_0 = -E_0 r\cos\theta$

球内の電位:
\begin{equation}
\phi_{\text{in}} = \sum_{n=0}^{\infty} A_n r^n P_n(\cos\theta)
\end{equation}
($r=0$で有限のため、$B_n$項は不要)

球外の電位:
\begin{equation}
\phi_{\text{out}} = -E_0 r\cos\theta + \sum_{n=0}^{\infty} \frac{B_n}{r^{n+1}} P_n(\cos\theta)
\end{equation}

境界条件を適用する。$r = a$で:

第1の境界条件(電位の連続性):
\begin{align}
\phi_{\text{in}}|_{r=a} &= \phi_{\text{out}}|_{r=a} \\
A_1 a P_1(\cos\theta) &= -E_0 a P_1(\cos\theta) + \frac{B_1}{a^2} P_1(\cos\theta)
\end{align}

$P_1(\cos\theta) = \cos\theta \neq 0$(一般に)より:
\begin{equation}
A_1 a = -E_0 a + \frac{B_1}{a^2} \quad \text{(1)}
\end{equation}

第2の境界条件(電束密度の法線成分の連続性):
\begin{align}
\varepsilon\frac{\partial\phi_{\text{in}}}{\partial r}\Big|_{r=a} &= \varepsilon_0\frac{\partial\phi_{\text{out}}}{\partial r}\Big|_{r=a} \\
\varepsilon A_1 P_1(\cos\theta) &= \varepsilon_0\left(-E_0 P_1(\cos\theta) - \frac{2B_1}{a^3}P_1(\cos\theta)\right)
\end{align}

したがって:
\begin{equation}
\varepsilon A_1 = \varepsilon_0\left(-E_0 - \frac{2B_1}{a^3}\right) \quad \text{(2)}
\end{equation}

(1)より$B_1 = a^3(A_1 + E_0)$。これを(2)に代入:
\begin{align}
\varepsilon A_1 &= \varepsilon_0\left(-E_0 - \frac{2a^3(A_1 + E_0)}{a^3}\right) \\
&= \varepsilon_0(-E_0 - 2A_1 - 2E_0) \\
&= \varepsilon_0(-2A_1 - 3E_0) \\
\varepsilon A_1 &= -2\varepsilon_0 A_1 - 3\varepsilon_0 E_0 \\
A_1(\varepsilon + 2\varepsilon_0) &= -3\varepsilon_0 E_0 \\
A_1 &= -\frac{3\varepsilon_0}{\varepsilon + 2\varepsilon_0}E_0
\end{align}

$B_1$を求める:
\begin{align}
B_1 &= a^3(A_1 + E_0) \\
&= a^3\left(-\frac{3\varepsilon_0}{\varepsilon + 2\varepsilon_0}E_0 + E_0\right) \\
&= a^3 E_0\left(1 - \frac{3\varepsilon_0}{\varepsilon + 2\varepsilon_0}\right) \\
&= a^3 E_0 \cdot \frac{\varepsilon + 2\varepsilon_0 - 3\varepsilon_0}{\varepsilon + 2\varepsilon_0} \\
&= a^3 E_0 \cdot \frac{\varepsilon - \varepsilon_0}{\varepsilon + 2\varepsilon_0}
\end{align}

したがって:
\begin{align}
\phi_{\text{in}} &= -\frac{3\varepsilon_0}{\varepsilon + 2\varepsilon_0}E_0 r\cos\theta \\
\phi_{\text{out}} &= -E_0 r\cos\theta + \frac{\varepsilon - \varepsilon_0}{\varepsilon + 2\varepsilon_0}\frac{E_0 a^3\cos\theta}{r^2}
\end{align}

球外の第2項は、双極子が作る電位に対応する。

\subsubsection{最終的な答え}

\begin{align}
\phi_{\text{in}} &= -\frac{3\varepsilon_0}{\varepsilon + 2\varepsilon_0}E_0 r\cos\theta \\
\phi_{\text{out}} &= -E_0 r\cos\theta + \frac{\varepsilon - \varepsilon_0}{\varepsilon + 2\varepsilon_0}\frac{E_0 a^3\cos\theta}{r^2}
\end{align}

球外の第2項は、双極子モーメント$\boldsymbol{p} = 4\pi\varepsilon_0\frac{\varepsilon - \varepsilon_0}{\varepsilon + 2\varepsilon_0}a^3\boldsymbol{E}_0$が作る電位に対応する。

\subsubsection{物理的意味の説明}

\begin{itemize}
    \item 誘電体球は外部電場によって分極し、双極子モーメントを持つ。この双極子モーメントは、外部電場に比例し、誘電率の差$(\varepsilon - \varepsilon_0)$に比例する。
    \item 球内部の電場は、外部電場より弱くなり、その比は$\frac{3\varepsilon_0}{\varepsilon + 2\varepsilon_0}$である。誘電率が大きいほど、内部電場は弱くなる。
    \item 球外部では、外部電場に加えて、誘電体球の分極による双極子場が重ね合わされる。この双極子場は、距離の2乗に反比例する。
\end{itemize}

\begin{figure}[H]
\centering
\includegraphics[width=0.8\textwidth]{figures/ex4_4_uniform_field.png}
\caption{一様電場中の誘電体球}
\label{fig:ex4_4_uniform_field}
\end{figure}

\end{document}
