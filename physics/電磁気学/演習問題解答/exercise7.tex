\documentclass[11pt,a4paper]{ltjsarticle}
\usepackage[no-math]{luatexja-fontspec}
\setmainjfont{Hiragino Mincho ProN}[
  UprightFont=*,
  BoldFont=*,
  ItalicFont=*,
  BoldItalicFont=*
]
\setsansjfont{Hiragino Kaku Gothic ProN}[
  UprightFont=*,
  BoldFont=*,
  ItalicFont=*,
  BoldItalicFont=*
]
\usepackage{amsmath,amssymb}
\usepackage{graphicx}
\usepackage{geometry}
\geometry{margin=2.5cm}
\usepackage{float}
\usepackage[draft=false]{hyperref}
\hypersetup{
    colorlinks=true,
    linkcolor=blue,
    citecolor=blue,
    urlcolor=blue,
    pdfusetitle=true
}

\title{電磁気学演習問題 解答・解説\\第7回 (2025年12月26日)}
\author{名古屋大学 理学部物理学科}
\date{2025年12月26日}

\begin{document}

\maketitle
\tableofcontents
\newpage

\section{問題1: 液体中の有極性分子の分極と電気感受率}

\subsection{問題}

時刻$t=0$で一様な電場$\boldsymbol{E}$をかけたときの分極ベクトルの大きさ$P(t)$は、時定数$\tau$を用いて$P(t) = \varepsilon_0\chi_0 E(1 - \exp(-t/\tau))$と表される。真空の誘電率を$\varepsilon_0$とする。有極性分子の分極が液体内の全電場$\boldsymbol{E}$に与える変化は無視する。

\begin{enumerate}
    \item[(1-1)] 上記の$P(t)$が微分方程式$\tau dP/dt + P = \varepsilon_0\chi_0 E$を満たすことを利用する。単一角振動数$\omega$で振動する電場$\boldsymbol{E}(\omega) = \boldsymbol{E}_0\exp(-i\omega t)$に対する電気感受率$\chi_e(\omega)$を、$\chi_0$、$\tau$、$\omega$を用いて求める。また、この液体の誘電率$\varepsilon(\omega)$の実部と虚部が、$\omega$に対してどのように変化するかを図示する。
    \item[(1-2)] 上述の液体による電場エネルギーの吸収率を求める。単位体積あたりの誘電体の微視的な変形エネルギーの時間変化$dw/dt = \text{Re}\{\boldsymbol{E}\} \cdot \text{Re}\{d\boldsymbol{P}/dt\}$を、前問の単一振動電場に対して計算し、一周期平均の$\langle dw/dt \rangle$を、$\varepsilon_0$、$\chi_0$、$\tau$、$\omega$、$E_0$を用いて求める。
\end{enumerate}

\subsection{解答}

\subsubsection{問題の理解と設定の明確化}

分極の時間発展方程式から、周波数依存の電気感受率を求める。

\subsubsection{使用する物理法則}

分極の時間発展方程式:
\begin{equation}
\tau\frac{dP}{dt} + P = \varepsilon_0\chi_0 E
\end{equation}

\subsubsection{段階的な計算過程}

\paragraph{(1-1) 電気感受率}

$P(\omega) = P_0\exp(-i\omega t)$と仮定して代入:
\begin{align}
\tau(-i\omega)P_0\exp(-i\omega t) + P_0\exp(-i\omega t) &= \varepsilon_0\chi_0 E_0\exp(-i\omega t) \\
P_0(1 - i\omega\tau) &= \varepsilon_0\chi_0 E_0
\end{align}

したがって:
\begin{equation}
\chi_e(\omega) = \frac{P_0}{\varepsilon_0 E_0} = \frac{\chi_0}{1 - i\omega\tau} = \frac{\chi_0(1 + i\omega\tau)}{1 + \omega^2\tau^2}
\end{equation}

誘電率:
\begin{equation}
\varepsilon(\omega) = \varepsilon_0[1 + \chi_e(\omega)] = \varepsilon_0\left[1 + \frac{\chi_0(1 + i\omega\tau)}{1 + \omega^2\tau^2}\right]
\end{equation}

実部と虚部:
\begin{align}
\text{Re}[\varepsilon(\omega)] &= \varepsilon_0\left[1 + \frac{\chi_0}{1 + \omega^2\tau^2}\right] \\
\text{Im}[\varepsilon(\omega)] &= \varepsilon_0\frac{\chi_0\omega\tau}{1 + \omega^2\tau^2}
\end{align}

\paragraph{(1-2) エネルギー吸収率}

単位体積あたりの誘電体の微視的な変形エネルギーの時間変化は:
\begin{equation}
\frac{dw}{dt} = \text{Re}\{\boldsymbol{E}\} \cdot \text{Re}\left\{\frac{d\boldsymbol{P}}{dt}\right\}
\end{equation}

電場と分極を複素表示で:
\begin{align}
\boldsymbol{E}(t) &= \boldsymbol{E}_0\exp(-i\omega t) = E_0\exp(-i\omega t)\hat{\boldsymbol{z}} \\
\boldsymbol{P}(t) &= \boldsymbol{P}_0\exp(-i\omega t) = P_0\exp(-i\omega t)\hat{\boldsymbol{z}}
\end{align}

前問より:
\begin{equation}
P_0 = \varepsilon_0\chi_e(\omega)E_0 = \varepsilon_0\frac{\chi_0(1 + i\omega\tau)}{1 + \omega^2\tau^2}E_0
\end{equation}

実部を取ると:
\begin{align}
\text{Re}\{E(t)\} &= E_0\cos(\omega t) \\
\text{Re}\{P(t)\} &= \text{Re}\{P_0\}\cos(\omega t) - \text{Im}\{P_0\}\sin(\omega t) \\
&= \varepsilon_0\frac{\chi_0}{1 + \omega^2\tau^2}E_0\cos(\omega t) + \varepsilon_0\frac{\chi_0\omega\tau}{1 + \omega^2\tau^2}E_0\sin(\omega t)
\end{align}

分極の時間微分:
\begin{align}
\frac{dP}{dt} &= -i\omega P_0\exp(-i\omega t) \\
\text{Re}\left\{\frac{dP}{dt}\right\} &= \omega\text{Im}\{P_0\}\cos(\omega t) + \omega\text{Re}\{P_0\}\sin(\omega t) \\
&= \omega\varepsilon_0\frac{\chi_0\omega\tau}{1 + \omega^2\tau^2}E_0\cos(\omega t) - \omega\varepsilon_0\frac{\chi_0}{1 + \omega^2\tau^2}E_0\sin(\omega t)
\end{align}

したがって:
\begin{align}
\frac{dw}{dt} &= E_0\cos(\omega t) \cdot \left[\omega\varepsilon_0\frac{\chi_0\omega\tau}{1 + \omega^2\tau^2}E_0\cos(\omega t) - \omega\varepsilon_0\frac{\chi_0}{1 + \omega^2\tau^2}E_0\sin(\omega t)\right] \\
&= \omega\varepsilon_0\frac{\chi_0\omega\tau}{1 + \omega^2\tau^2}E_0^2\cos^2(\omega t) - \omega\varepsilon_0\frac{\chi_0}{1 + \omega^2\tau^2}E_0^2\cos(\omega t)\sin(\omega t) \\
&= \omega\varepsilon_0\frac{\chi_0\omega\tau}{1 + \omega^2\tau^2}E_0^2\cos^2(\omega t) - \omega\varepsilon_0\frac{\chi_0}{2(1 + \omega^2\tau^2)}E_0^2\sin(2\omega t)
\end{align}

一周期平均を計算する:
\begin{align}
\left\langle\frac{dw}{dt}\right\rangle &= \frac{\omega}{2\pi}\int_0^{2\pi/\omega} \frac{dw}{dt} dt \\
&= \frac{\omega}{2\pi}\left[\omega\varepsilon_0\frac{\chi_0\omega\tau}{1 + \omega^2\tau^2}E_0^2\int_0^{2\pi/\omega}\cos^2(\omega t) dt - \omega\varepsilon_0\frac{\chi_0}{2(1 + \omega^2\tau^2)}E_0^2\int_0^{2\pi/\omega}\sin(2\omega t) dt\right]
\end{align}

$\int_0^{2\pi/\omega}\cos^2(\omega t) dt = \frac{\pi}{\omega}$、$\int_0^{2\pi/\omega}\sin(2\omega t) dt = 0$より:
\begin{align}
\left\langle\frac{dw}{dt}\right\rangle &= \frac{\omega}{2\pi} \cdot \omega\varepsilon_0\frac{\chi_0\omega\tau}{1 + \omega^2\tau^2}E_0^2 \cdot \frac{\pi}{\omega} \\
&= \frac{\omega\varepsilon_0\chi_0 E_0^2}{2}\frac{\omega\tau}{1 + \omega^2\tau^2}
\end{align}

\subsubsection{最終的な答え}

\begin{enumerate}
    \item[(1-1)] $\chi_e(\omega) = \frac{\chi_0}{1 - i\omega\tau} = \frac{\chi_0(1 + i\omega\tau)}{1 + \omega^2\tau^2}$、実部:$\text{Re}[\chi_e] = \frac{\chi_0}{1 + \omega^2\tau^2}$、虚部:$\text{Im}[\chi_e] = \frac{\chi_0\omega\tau}{1 + \omega^2\tau^2}$
    \item[(1-2)] $\left\langle\frac{dw}{dt}\right\rangle = \frac{\omega\varepsilon_0\chi_0 E_0^2}{2}\frac{\omega\tau}{1 + \omega^2\tau^2}$
\end{enumerate}

\subsubsection{物理的意味の説明}

\begin{itemize}
    \item 電気感受率$\chi_e(\omega)$は周波数依存性を持ち、$\omega\tau \ll 1$(低周波)では$\chi_e \approx \chi_0$、$\omega\tau \gg 1$(高周波)では$\chi_e \approx 0$となる。これは、分極が電場の変化に追従できなくなるためである。
    \item 誘電率の実部は、低周波では$\varepsilon_0(1 + \chi_0)$、高周波では$\varepsilon_0$に近づく。虚部は$\omega\tau = 1$付近で最大値を取り、エネルギー吸収が最大となる。
    \item エネルギー吸収率は、$\omega\tau = 1$付近で最大となり、この周波数で分極の緩和によるエネルギー散逸が最も大きい。
\end{itemize}

\begin{figure}[H]
\centering
\includegraphics[width=0.8\textwidth]{figures/ex7_1_dielectric_dispersion.png}
\caption{誘電率の周波数依存性(実部と虚部)}
\label{fig:ex7_1_dielectric_dispersion}
\end{figure}

\section{問題2: 遅延ポテンシャル}

\subsection{問題}

真空中の電荷密度$\rho(\boldsymbol{r})$と電流密度$\boldsymbol{i}(\boldsymbol{r})$が時間変化する場合を考える。スカラーポテンシャル$\phi$とベクトルポテンシャル$\boldsymbol{A}$は遅延ポテンシャルの式で与えられる。電場は$\boldsymbol{E} = -\nabla\phi - \partial\boldsymbol{A}/\partial t$と表せる。

\begin{enumerate}
    \item[(2-1)] 電場$\boldsymbol{E}(\boldsymbol{r},t)$が以下の式で与えられることを示す:
    \begin{equation}
    \boldsymbol{E}(\boldsymbol{r},t) = \frac{1}{4\pi\varepsilon_0}\left[\int \frac{\rho(\boldsymbol{r}',t')(\boldsymbol{r} - \boldsymbol{r}')}{|\boldsymbol{r} - \boldsymbol{r}'|^3} dV' + \int \frac{\partial\rho(\boldsymbol{r}',t')/\partial t'(\boldsymbol{r} - \boldsymbol{r}')}{c|\boldsymbol{r} - \boldsymbol{r}'|^2} dV' - \int \frac{\partial\boldsymbol{i}(\boldsymbol{r}',t')/\partial t'}{c^2|\boldsymbol{r} - \boldsymbol{r}'|} dV'\right]
    \end{equation}
    ただし、$t' = t - |\boldsymbol{r} - \boldsymbol{r}'|/c$である。
    \item[(2-2)] 同様に、磁場$\boldsymbol{B}(\boldsymbol{r},t)$の式を求める。
\end{enumerate}

\subsection{解答}

\subsubsection{問題の理解と設定の明確化}

遅延ポテンシャルから電場と磁場を計算する。

\subsubsection{使用する物理法則}

遅延ポテンシャル:
\begin{align}
\phi(\boldsymbol{r},t) &= \frac{1}{4\pi\varepsilon_0}\int \frac{\rho(\boldsymbol{r}',t')}{|\boldsymbol{r} - \boldsymbol{r}'|} dV' \\
\boldsymbol{A}(\boldsymbol{r},t) &= \frac{\mu_0}{4\pi}\int \frac{\boldsymbol{i}(\boldsymbol{r}',t')}{|\boldsymbol{r} - \boldsymbol{r}'|} dV'
\end{align}
ここで、$t' = t - |\boldsymbol{r} - \boldsymbol{r}'|/c$。

\subsubsection{段階的な計算過程}

\paragraph{(2-1) 電場の計算}

電場は:
\begin{equation}
\boldsymbol{E} = -\nabla\phi - \frac{\partial\boldsymbol{A}}{\partial t}
\end{equation}

まず、$\nabla\phi$を計算する。$\phi$は遅延ポテンシャル:
\begin{equation}
\phi(\boldsymbol{r},t) = \frac{1}{4\pi\varepsilon_0}\int \frac{\rho(\boldsymbol{r}',t')}{|\boldsymbol{r} - \boldsymbol{r}'|} dV'
\end{equation}
ここで、$t' = t - |\boldsymbol{r} - \boldsymbol{r}'|/c$である。

$\nabla\phi$を計算する際、$t'$が$\boldsymbol{r}$に依存することを考慮する必要がある:
\begin{align}
\nabla\phi &= \frac{1}{4\pi\varepsilon_0}\nabla\int \frac{\rho(\boldsymbol{r}',t')}{|\boldsymbol{r} - \boldsymbol{r}'|} dV' \\
&= \frac{1}{4\pi\varepsilon_0}\int\left[\frac{1}{|\boldsymbol{r} - \boldsymbol{r}'|}\nabla\rho(\boldsymbol{r}',t') + \rho(\boldsymbol{r}',t')\nabla\frac{1}{|\boldsymbol{r} - \boldsymbol{r}'|}\right] dV'
\end{align}

第1項について、$\rho(\boldsymbol{r}',t')$は$\boldsymbol{r}'$と$t'$の関数であり、$t' = t - |\boldsymbol{r} - \boldsymbol{r}'|/c$より:
\begin{align}
\nabla\rho(\boldsymbol{r}',t') &= \frac{\partial\rho}{\partial t'}\nabla t' \\
&= \frac{\partial\rho}{\partial t'}\nabla\left(t - \frac{|\boldsymbol{r} - \boldsymbol{r}'|}{c}\right) \\
&= -\frac{1}{c}\frac{\partial\rho}{\partial t'}\nabla|\boldsymbol{r} - \boldsymbol{r}'| \\
&= -\frac{1}{c}\frac{\partial\rho}{\partial t'}\frac{\boldsymbol{r} - \boldsymbol{r}'}{|\boldsymbol{r} - \boldsymbol{r}'|}
\end{align}

第2項について:
\begin{equation}
\nabla\frac{1}{|\boldsymbol{r} - \boldsymbol{r}'|} = -\frac{\boldsymbol{r} - \boldsymbol{r}'}{|\boldsymbol{r} - \boldsymbol{r}'|^3}
\end{equation}

したがって:
\begin{align}
\nabla\phi &= \frac{1}{4\pi\varepsilon_0}\int\left[-\frac{1}{c}\frac{\partial\rho}{\partial t'}\frac{\boldsymbol{r} - \boldsymbol{r}'}{|\boldsymbol{r} - \boldsymbol{r}'|^2} - \rho(\boldsymbol{r}',t')\frac{\boldsymbol{r} - \boldsymbol{r}'}{|\boldsymbol{r} - \boldsymbol{r}'|^3}\right] dV' \\
&= -\frac{1}{4\pi\varepsilon_0}\int\left[\frac{\rho(\boldsymbol{r}',t')(\boldsymbol{r} - \boldsymbol{r}')}{|\boldsymbol{r} - \boldsymbol{r}'|^3} + \frac{1}{c}\frac{\partial\rho(\boldsymbol{r}',t')}{\partial t'}\frac{\boldsymbol{r} - \boldsymbol{r}'}{|\boldsymbol{r} - \boldsymbol{r}'|^2}\right] dV'
\end{align}

次に、$\frac{\partial\boldsymbol{A}}{\partial t}$を計算する:
\begin{align}
\frac{\partial\boldsymbol{A}}{\partial t} &= \frac{\mu_0}{4\pi}\frac{\partial}{\partial t}\int \frac{\boldsymbol{i}(\boldsymbol{r}',t')}{|\boldsymbol{r} - \boldsymbol{r}'|} dV' \\
&= \frac{\mu_0}{4\pi}\int\frac{1}{|\boldsymbol{r} - \boldsymbol{r}'|}\frac{\partial\boldsymbol{i}(\boldsymbol{r}',t')}{\partial t'} \cdot \frac{\partial t'}{\partial t} dV'
\end{align}

$t' = t - |\boldsymbol{r} - \boldsymbol{r}'|/c$より、$\frac{\partial t'}{\partial t} = 1$である。したがって:
\begin{equation}
\frac{\partial\boldsymbol{A}}{\partial t} = \frac{\mu_0}{4\pi}\int\frac{1}{|\boldsymbol{r} - \boldsymbol{r}'|}\frac{\partial\boldsymbol{i}(\boldsymbol{r}',t')}{\partial t'} dV'
\end{equation}

電場は:
\begin{align}
\boldsymbol{E} &= -\nabla\phi - \frac{\partial\boldsymbol{A}}{\partial t} \\
&= \frac{1}{4\pi\varepsilon_0}\int\left[\frac{\rho(\boldsymbol{r}',t')(\boldsymbol{r} - \boldsymbol{r}')}{|\boldsymbol{r} - \boldsymbol{r}'|^3} + \frac{1}{c}\frac{\partial\rho(\boldsymbol{r}',t')}{\partial t'}\frac{\boldsymbol{r} - \boldsymbol{r}'}{|\boldsymbol{r} - \boldsymbol{r}'|^2}\right] dV' \\
&\quad - \frac{\mu_0}{4\pi}\int\frac{1}{|\boldsymbol{r} - \boldsymbol{r}'|}\frac{\partial\boldsymbol{i}(\boldsymbol{r}',t')}{\partial t'} dV'
\end{align}

$\mu_0 = 1/(\varepsilon_0 c^2)$より:
\begin{equation}
\boldsymbol{E}(\boldsymbol{r},t) = \frac{1}{4\pi\varepsilon_0}\left[\int \frac{\rho(\boldsymbol{r}',t')(\boldsymbol{r} - \boldsymbol{r}')}{|\boldsymbol{r} - \boldsymbol{r}'|^3} dV' + \int \frac{\partial\rho(\boldsymbol{r}',t')/\partial t'(\boldsymbol{r} - \boldsymbol{r}')}{c|\boldsymbol{r} - \boldsymbol{r}'|^2} dV' - \int \frac{\partial\boldsymbol{i}(\boldsymbol{r}',t')/\partial t'}{c^2|\boldsymbol{r} - \boldsymbol{r}'|} dV'\right]
\end{equation}

\paragraph{(2-2) 磁場の計算}

磁場は:
\begin{equation}
\boldsymbol{B} = \nabla \times \boldsymbol{A}
\end{equation}

ベクトルポテンシャル:
\begin{equation}
\boldsymbol{A}(\boldsymbol{r},t) = \frac{\mu_0}{4\pi}\int \frac{\boldsymbol{i}(\boldsymbol{r}',t')}{|\boldsymbol{r} - \boldsymbol{r}'|} dV'
\end{equation}

$\nabla \times \boldsymbol{A}$を計算する。ベクトル恒等式:
\begin{equation}
\nabla \times (f\boldsymbol{v}) = f\nabla \times \boldsymbol{v} + \nabla f \times \boldsymbol{v}
\end{equation}

ここで、$f = 1/|\boldsymbol{r} - \boldsymbol{r}'|$、$\boldsymbol{v} = \boldsymbol{i}(\boldsymbol{r}',t')$である。$\boldsymbol{i}(\boldsymbol{r}',t')$は$\boldsymbol{r}$に依存しないため、$\nabla \times \boldsymbol{i} = \boldsymbol{0}$である。したがって:
\begin{align}
\nabla \times \boldsymbol{A} &= \frac{\mu_0}{4\pi}\int \nabla\frac{1}{|\boldsymbol{r} - \boldsymbol{r}'|} \times \boldsymbol{i}(\boldsymbol{r}',t') dV' \\
&= \frac{\mu_0}{4\pi}\int \left(-\frac{\boldsymbol{r} - \boldsymbol{r}'}{|\boldsymbol{r} - \boldsymbol{r}'|^3}\right) \times \boldsymbol{i}(\boldsymbol{r}',t') dV' \\
&= \frac{\mu_0}{4\pi}\int \frac{\boldsymbol{i}(\boldsymbol{r}',t') \times (\boldsymbol{r} - \boldsymbol{r}')}{|\boldsymbol{r} - \boldsymbol{r}'|^3} dV'
\end{align}

ただし、$t'$が$\boldsymbol{r}$に依存するため、より正確には:
\begin{align}
\nabla \times \boldsymbol{A} &= \frac{\mu_0}{4\pi}\int\left[\frac{1}{|\boldsymbol{r} - \boldsymbol{r}'|}\nabla \times \boldsymbol{i}(\boldsymbol{r}',t') + \nabla\frac{1}{|\boldsymbol{r} - \boldsymbol{r}'|} \times \boldsymbol{i}(\boldsymbol{r}',t')\right] dV'
\end{align}

第1項について、$\nabla \times \boldsymbol{i}(\boldsymbol{r}',t') = \frac{\partial\boldsymbol{i}}{\partial t'}\nabla t' \times \boldsymbol{i}$の寄与を考慮すると:
\begin{align}
\nabla \times \boldsymbol{A} &= \frac{\mu_0}{4\pi}\int\left[\frac{1}{|\boldsymbol{r} - \boldsymbol{r}'|}\frac{\partial\boldsymbol{i}}{\partial t'}\nabla t' \times \boldsymbol{i} + \left(-\frac{\boldsymbol{r} - \boldsymbol{r}'}{|\boldsymbol{r} - \boldsymbol{r}'|^3}\right) \times \boldsymbol{i}(\boldsymbol{r}',t')\right] dV' \\
&= \frac{\mu_0}{4\pi}\int\left[-\frac{1}{c|\boldsymbol{r} - \boldsymbol{r}'|}\frac{\partial\boldsymbol{i}}{\partial t'}\frac{\boldsymbol{r} - \boldsymbol{r}'}{|\boldsymbol{r} - \boldsymbol{r}'|} \times \boldsymbol{i} + \frac{\boldsymbol{i}(\boldsymbol{r}',t') \times (\boldsymbol{r} - \boldsymbol{r}')}{|\boldsymbol{r} - \boldsymbol{r}'|^3}\right] dV' \\
&= \frac{\mu_0}{4\pi}\left[\int \frac{\boldsymbol{i}(\boldsymbol{r}',t') \times (\boldsymbol{r} - \boldsymbol{r}')}{|\boldsymbol{r} - \boldsymbol{r}'|^3} dV' + \int \frac{\partial\boldsymbol{i}(\boldsymbol{r}',t')/\partial t' \times (\boldsymbol{r} - \boldsymbol{r}')}{c|\boldsymbol{r} - \boldsymbol{r}'|^2} dV'\right]
\end{align}

したがって:
\begin{equation}
\boldsymbol{B}(\boldsymbol{r},t) = \frac{\mu_0}{4\pi}\left[\int \frac{\boldsymbol{i}(\boldsymbol{r}',t') \times (\boldsymbol{r} - \boldsymbol{r}')}{|\boldsymbol{r} - \boldsymbol{r}'|^3} dV' + \int \frac{\partial\boldsymbol{i}(\boldsymbol{r}',t')/\partial t' \times (\boldsymbol{r} - \boldsymbol{r}')}{c|\boldsymbol{r} - \boldsymbol{r}'|^2} dV'\right]
\end{equation}

\subsubsection{最終的な答え}

\begin{enumerate}
    \item[(2-1)] $\boldsymbol{E}(\boldsymbol{r},t) = \frac{1}{4\pi\varepsilon_0}\left[\int \frac{\rho(\boldsymbol{r}',t')(\boldsymbol{r} - \boldsymbol{r}')}{|\boldsymbol{r} - \boldsymbol{r}'|^3} dV' + \int \frac{\partial\rho(\boldsymbol{r}',t')/\partial t'(\boldsymbol{r} - \boldsymbol{r}')}{c|\boldsymbol{r} - \boldsymbol{r}'|^2} dV' - \int \frac{\partial\boldsymbol{i}(\boldsymbol{r}',t')/\partial t'}{c^2|\boldsymbol{r} - \boldsymbol{r}'|} dV'\right]$
    \item[(2-2)] $\boldsymbol{B}(\boldsymbol{r},t) = \frac{\mu_0}{4\pi}\left[\int \frac{\boldsymbol{i}(\boldsymbol{r}',t') \times (\boldsymbol{r} - \boldsymbol{r}')}{|\boldsymbol{r} - \boldsymbol{r}'|^3} dV' + \int \frac{\partial\boldsymbol{i}(\boldsymbol{r}',t')/\partial t' \times (\boldsymbol{r} - \boldsymbol{r}')}{c|\boldsymbol{r} - \boldsymbol{r}'|^2} dV'\right]$
\end{enumerate}

\subsubsection{物理的意味の説明}

\begin{itemize}
    \item 遅延ポテンシャルから導出される電場・磁場は、静電場・静磁場の項に加えて、時間変化による補正項を含む。
    \item 第1項は静電場・静磁場に対応し、第2項は時間変化による補正項である。これらの項により、電磁波の放射が記述される。
    \item $t' = t - |\boldsymbol{r} - \boldsymbol{r}'|/c$は、情報が光速$c$で伝播することを示している(遅延時間)。
\end{itemize}

\begin{figure}[H]
\centering
\includegraphics[width=0.8\textwidth]{figures/ex7_2_retarded_potential.png}
\caption{遅延ポテンシャルと電場の時間変化}
\label{fig:ex7_2_retarded_potential}
\end{figure}

\section{問題3: 微小誘電体球による光の吸収と散乱}

\subsection{問題}

真空(誘電率$\varepsilon_0$)に置かれた微小誘電体球(体積$V$、複素誘電率$\varepsilon = \varepsilon' + i\varepsilon''$)による光の吸収、散乱を考える。入射光は単色で、単一角振動数$\omega$で振動する外部電場$\boldsymbol{E}_i = \boldsymbol{E}_0\exp(-i\omega t)$とする。真空中の光のエネルギーの流れ(ポインティングベクトルの大きさ)$\boldsymbol{S}$は、光速$c$として、振動の一周期平均$\langle S \rangle = c\varepsilon_0|\boldsymbol{E}_0|^2/2$と書ける。誘電体球の大きさは光の波長に比べて十分小さいと考え、誘電体の分極は同位相で振動する双極子$\boldsymbol{p}$として近似できるとする。

\begin{enumerate}
    \item[(3-1)] 光の吸収断面積$C_a$を、$c$、$V$、$\varepsilon_0$、$\varepsilon'$、$\varepsilon''$、$\omega$を用いて求める。
    \item[(3-2)] 光の散乱断面積$C_s$を、$c$、$V$、$\varepsilon_0$、$\varepsilon'$、$\omega$を用いて求める。
\end{enumerate}

\subsection{解答}

\subsubsection{問題の理解と設定の明確化}

微小誘電体球の分極による光の吸収と散乱を、双極子近似で求める。

\subsubsection{使用する物理法則}

双極子の分極:
\begin{equation}
\boldsymbol{p} = \varepsilon_0\frac{\varepsilon - \varepsilon_0}{\varepsilon + 2\varepsilon_0}V\boldsymbol{E}_0
\end{equation}

\subsubsection{段階的な計算過程}

\paragraph{(3-1) 吸収断面積}

問題1-2の結果を用いて、単位体積あたりの吸収エネルギー率を求める。複素誘電率$\varepsilon = \varepsilon' + i\varepsilon''$の場合、誘電率の虚部$\varepsilon''$がエネルギー吸収に関与する。

問題1-2より、単位体積あたりの吸収エネルギー率の一周期平均は:
\begin{equation}
\left\langle\frac{dw_a}{dt}\right\rangle = \frac{\omega\varepsilon_0\chi_0 E_0^2}{2}\frac{\omega\tau}{1 + \omega^2\tau^2}
\end{equation}

複素誘電率の場合、$\varepsilon = \varepsilon_0(1 + \chi_e)$より、$\chi_e$の虚部は$\varepsilon''/\varepsilon_0$に対応する。より直接的に、問題1-2の結果を拡張すると:
\begin{equation}
\left\langle\frac{dw_a}{dt}\right\rangle = \frac{\omega\varepsilon'' E_0^2}{2}
\end{equation}

誘電体球全体での吸収エネルギー率:
\begin{align}
\left\langle\frac{dW_a}{dt}\right\rangle &= V \cdot \left\langle\frac{dw_a}{dt}\right\rangle \\
&= V \cdot \frac{\omega\varepsilon'' E_0^2}{2}
\end{align}

入射光のエネルギーフラックス(ポインティングベクトルの大きさの一周期平均):
\begin{equation}
\langle S \rangle = \frac{c\varepsilon_0|\boldsymbol{E}_0|^2}{2}
\end{equation}

吸収断面積の定義:
\begin{equation}
C_a = \frac{\langle dW_a/dt \rangle}{\langle S \rangle}
\end{equation}

したがって:
\begin{align}
C_a &= \frac{V \cdot \frac{\omega\varepsilon'' E_0^2}{2}}{\frac{c\varepsilon_0 E_0^2}{2}} \\
&= \frac{V\omega\varepsilon''}{c\varepsilon_0}
\end{align}

\paragraph{(3-2) 散乱断面積}

振動双極子が放射する電磁波のパワーを求める。振動双極子$\boldsymbol{p} = \boldsymbol{p}_0\exp(-i\omega t)$が作る電磁場から、放射パワーを計算する。

双極子放射の公式より、単位時間あたりに放射されるエネルギーは:
\begin{equation}
P = \frac{\mu_0}{6\pi c}\left|\frac{d^2\boldsymbol{p}}{dt^2}\right|^2
\end{equation}

$\boldsymbol{p} = \boldsymbol{p}_0\exp(-i\omega t)$より:
\begin{align}
\frac{d^2\boldsymbol{p}}{dt^2} &= (-i\omega)^2\boldsymbol{p}_0\exp(-i\omega t) = -\omega^2\boldsymbol{p}
\end{align}

したがって:
\begin{align}
\left|\frac{d^2\boldsymbol{p}}{dt^2}\right|^2 &= \omega^4|\boldsymbol{p}|^2
\end{align}

一周期平均:
\begin{align}
\left\langle\frac{dW_s}{dt}\right\rangle &= \frac{\mu_0\omega^4}{6\pi c}|\boldsymbol{p}_0|^2
\end{align}

双極子モーメントは、問題4(演習問題4)の結果より:
\begin{equation}
\boldsymbol{p} = \varepsilon_0\frac{\varepsilon - \varepsilon_0}{\varepsilon + 2\varepsilon_0}V\boldsymbol{E}_0
\end{equation}

散乱には実部のみが寄与するため、$\varepsilon'$を用いる:
\begin{equation}
\boldsymbol{p}_0 = \varepsilon_0\frac{\varepsilon' - \varepsilon_0}{\varepsilon' + 2\varepsilon_0}V\boldsymbol{E}_0
\end{equation}

したがって:
\begin{align}
\left\langle\frac{dW_s}{dt}\right\rangle &= \frac{\mu_0\omega^4}{6\pi c}\left|\varepsilon_0\frac{\varepsilon' - \varepsilon_0}{\varepsilon' + 2\varepsilon_0}V\boldsymbol{E}_0\right|^2 \\
&= \frac{\mu_0\omega^4\varepsilon_0^2 V^2 E_0^2}{6\pi c}\left(\frac{\varepsilon' - \varepsilon_0}{\varepsilon' + 2\varepsilon_0}\right)^2
\end{align}

散乱断面積:
\begin{align}
C_s &= \frac{\langle dW_s/dt \rangle}{\langle S \rangle} \\
&= \frac{\frac{\mu_0\omega^4\varepsilon_0^2 V^2 E_0^2}{6\pi c}\left(\frac{\varepsilon' - \varepsilon_0}{\varepsilon' + 2\varepsilon_0}\right)^2}{\frac{c\varepsilon_0 E_0^2}{2}} \\
&= \frac{\mu_0\omega^4\varepsilon_0 V^2}{3\pi c^2}\left(\frac{\varepsilon' - \varepsilon_0}{\varepsilon' + 2\varepsilon_0}\right)^2
\end{align}

$\mu_0 = 1/(\varepsilon_0 c^2)$より:
\begin{align}
C_s &= \frac{\omega^4 V^2}{3\pi c^4}\left(\frac{\varepsilon' - \varepsilon_0}{\varepsilon' + 2\varepsilon_0}\right)^2 \\
&= \frac{8\pi}{3}\frac{\omega^4 V^2}{c^4}\left(\frac{\varepsilon' - \varepsilon_0}{\varepsilon' + 2\varepsilon_0}\right)^2
\end{align}

\subsubsection{最終的な答え}

\begin{enumerate}
    \item[(3-1)] $C_a = \frac{V\omega\varepsilon''}{c\varepsilon_0}$
    \item[(3-2)] $C_s = \frac{8\pi}{3}\frac{\omega^4 V^2}{c^4}\left(\frac{\varepsilon' - \varepsilon_0}{\varepsilon' + 2\varepsilon_0}\right)^2$
\end{enumerate}

\subsubsection{物理的意味の説明}

\begin{itemize}
    \item 吸収断面積$C_a$は、誘電率の虚部$\varepsilon''$に比例する。$\varepsilon''$が大きいほど、エネルギー吸収が大きい。
    \item 散乱断面積$C_s$は、周波数の4乗に比例する(レイリー散乱)。これは、波長が短いほど散乱が強くなることを示している。
    \item 散乱断面積は、誘電体球の体積の2乗に比例する。これは、双極子モーメントが体積に比例し、放射パワーが双極子モーメントの2乗に比例するためである。
    \item 吸収と散乱の比は、$\frac{C_a}{C_s} \propto \frac{\varepsilon''}{\omega^3 V}$となり、周波数が高いほど散乱が優勢になる。
\end{itemize}

\begin{figure}[H]
\centering
\includegraphics[width=0.8\textwidth]{figures/ex7_3_scattering.png}
\caption{微小誘電体球による光の吸収と散乱}
\label{fig:ex7_3_scattering}
\end{figure}

\section{問題4: 媒質境界面での電磁波の反射}

\subsection{問題}

$z$軸方向に進行する、角振動数$\omega$、波数$k$の平面電磁波($\boldsymbol{E} = \boldsymbol{E}_0\exp[i(kz - \omega t)]$、$\boldsymbol{H} = \boldsymbol{H}_0\exp[i(kz - \omega t)]$)が、真空(誘電率$\varepsilon_0$、透磁率$\mu_0$)から、誘電率$\varepsilon$、透磁率$\mu$の物質($\varepsilon$と$\mu$は実数)の境界面に垂直に入射する。境界面を$z=0$とする。物質は$z \geq 0$の領域を占める。

\begin{enumerate}
    \item[(4-1)] 反射率を、$\varepsilon_0$、$\mu_0$、$\varepsilon$、$\mu$を用いて求める。
    \item[(4-2)] 入射波と反射波の電場成分の位相が、反射面で$\pi$だけ変化するための条件を示す。
    \item[(4-3)] 真空中からガラス面(屈折率$n = \sqrt{\varepsilon\mu/(\varepsilon_0\mu_0)} = 1.5$)に光が垂直入射するときの反射率を求める。ただし、ここでは$\mu = \mu_0$と考える。
    \item[(4-4)] もし$\varepsilon$が負であった場合、反射率はどうなるかを述べる。
\end{enumerate}

\subsection{解答}

\subsubsection{問題の理解と設定の明確化}

境界面での電場と磁場の連続性条件を用いて反射率を求める。

\subsubsection{使用する物理法則}

境界条件:
\begin{align}
E_{\parallel,1} &= E_{\parallel,2} \\
H_{\parallel,1} &= H_{\parallel,2}
\end{align}

\subsubsection{段階的な計算過程}

\paragraph{(4-1) 反射率}

入射波、反射波、透過波を以下のように表す:

入射波(真空側、$z < 0$):
\begin{align}
E_i(z,t) &= E_0\exp[i(k_0 z - \omega t)] \\
H_i(z,t) &= \frac{E_0}{Z_0}\exp[i(k_0 z - \omega t)]
\end{align}
ここで、$k_0 = \omega\sqrt{\varepsilon_0\mu_0}$、$Z_0 = \sqrt{\mu_0/\varepsilon_0}$は真空の特性インピーダンスである。

反射波(真空側、$z < 0$):
\begin{align}
E_r(z,t) &= rE_0\exp[i(-k_0 z - \omega t)] \\
H_r(z,t) &= -\frac{rE_0}{Z_0}\exp[i(-k_0 z - \omega t)]
\end{align}
反射波の磁場の符号が負なのは、進行方向が逆向きのためである。

透過波(物質側、$z > 0$):
\begin{align}
E_t(z,t) &= tE_0\exp[i(k z - \omega t)] \\
H_t(z,t) &= \frac{tE_0}{Z}\exp[i(k z - \omega t)]
\end{align}
ここで、$k = \omega\sqrt{\varepsilon\mu}$、$Z = \sqrt{\mu/\varepsilon}$は物質の特性インピーダンスである。

境界面$z = 0$での境界条件を適用する。

第1の境界条件(電場の接線成分の連続性):
\begin{align}
E_i(0,t) + E_r(0,t) &= E_t(0,t) \\
E_0\exp(-i\omega t) + rE_0\exp(-i\omega t) &= tE_0\exp(-i\omega t) \\
1 + r &= t \quad \text{(1)}
\end{align}

第2の境界条件(磁場の接線成分の連続性):
\begin{align}
H_i(0,t) + H_r(0,t) &= H_t(0,t) \\
\frac{E_0}{Z_0}\exp(-i\omega t) - \frac{rE_0}{Z_0}\exp(-i\omega t) &= \frac{tE_0}{Z}\exp(-i\omega t) \\
\frac{1 - r}{Z_0} &= \frac{t}{Z} \quad \text{(2)}
\end{align}

(1)より$t = 1 + r$を(2)に代入:
\begin{align}
\frac{1 - r}{Z_0} &= \frac{1 + r}{Z} \\
Z(1 - r) &= Z_0(1 + r) \\
Z - Zr &= Z_0 + Z_0 r \\
Z - Z_0 &= r(Z + Z_0) \\
r &= \frac{Z - Z_0}{Z + Z_0}
\end{align}

反射率は:
\begin{align}
R &= |r|^2 = \left|\frac{Z - Z_0}{Z + Z_0}\right|^2 = \left(\frac{Z - Z_0}{Z + Z_0}\right)^2
\end{align}

$Z = \sqrt{\mu/\varepsilon}$、$Z_0 = \sqrt{\mu_0/\varepsilon_0}$より:
\begin{equation}
R = \left(\frac{\sqrt{\mu/\varepsilon} - \sqrt{\mu_0/\varepsilon_0}}{\sqrt{\mu/\varepsilon} + \sqrt{\mu_0/\varepsilon_0}}\right)^2
\end{equation}

\paragraph{(4-2) 位相変化}

反射係数$r$の符号を調べる:
\begin{equation}
r = \frac{Z - Z_0}{Z + Z_0}
\end{equation}

$Z < Z_0$の場合、$Z - Z_0 < 0$、$Z + Z_0 > 0$より、$r < 0$となる。

$r < 0$の場合、反射波の電場は:
\begin{equation}
E_r = rE_0\exp[i(-k_0 z - \omega t)] = |r|E_0\exp[i(-k_0 z - \omega t + \pi)]
\end{equation}

これは、入射波の電場$E_i = E_0\exp[i(k_0 z - \omega t)]$と比較すると、位相が$\pi$だけずれていることを示す。

したがって、$Z < Z_0$(すなわち、物質の特性インピーダンスが真空より小さい)場合、反射面で位相が$\pi$変化する。

\paragraph{(4-3) ガラスの場合}

$\mu = \mu_0$の場合、特性インピーダンスは:
\begin{align}
Z &= \sqrt{\frac{\mu_0}{\varepsilon}} = \sqrt{\frac{\mu_0}{\varepsilon_0}} \cdot \sqrt{\frac{\varepsilon_0}{\varepsilon}} = Z_0 \cdot \frac{1}{n}
\end{align}

ここで、屈折率$n = \sqrt{\varepsilon/\varepsilon_0}$である。

反射率:
\begin{align}
R &= \left(\frac{Z - Z_0}{Z + Z_0}\right)^2 \\
&= \left(\frac{Z_0/n - Z_0}{Z_0/n + Z_0}\right)^2 \\
&= \left(\frac{1/n - 1}{1/n + 1}\right)^2 \\
&= \left(\frac{1 - n}{1 + n}\right)^2 \\
&= \left(\frac{n - 1}{n + 1}\right)^2
\end{align}

$n = 1.5$の場合:
\begin{align}
R &= \left(\frac{1.5 - 1}{1.5 + 1}\right)^2 \\
&= \left(\frac{0.5}{2.5}\right)^2 \\
&= \left(\frac{1}{5}\right)^2 = \frac{1}{25} = 0.04 = 4\%
\end{align}

\paragraph{(4-4) $\varepsilon < 0$の場合}

$\varepsilon < 0$の場合、特性インピーダンス:
\begin{equation}
Z = \sqrt{\frac{\mu}{\varepsilon}}
\end{equation}

$\varepsilon < 0$より、$Z$は純虚数となる。実際には:
\begin{equation}
Z = i\sqrt{\frac{\mu}{|\varepsilon|}} = i|Z|
\end{equation}

反射係数:
\begin{align}
r &= \frac{Z - Z_0}{Z + Z_0} = \frac{i|Z| - Z_0}{i|Z| + Z_0}
\end{align}

この場合、$|r|^2$を計算すると:
\begin{align}
|r|^2 &= \left|\frac{i|Z| - Z_0}{i|Z| + Z_0}\right|^2 \\
&= \frac{|i|Z| - Z_0|^2}{|i|Z| + Z_0|^2} \\
&= \frac{|Z|^2 + Z_0^2}{|Z|^2 + Z_0^2} = 1
\end{align}

したがって、反射率$R = 1$となり、全反射が起こる。これは、$\varepsilon < 0$の場合、物質内で電磁波が伝播できず、すべて反射されるためである。

\subsubsection{最終的な答え}

\begin{enumerate}
    \item[(4-1)] $R = \left(\frac{\sqrt{\mu/\varepsilon} - \sqrt{\mu_0/\varepsilon_0}}{\sqrt{\mu/\varepsilon} + \sqrt{\mu_0/\varepsilon_0}}\right)^2$
    \item[(4-2)] $Z < Z_0$(物質の特性インピーダンスが真空より小さい)のとき、反射面で位相が$\pi$変化する。
    \item[(4-3)] $R = 4\%$
    \item[(4-4)] $\varepsilon < 0$のとき、$R = 1$となり、全反射が起こる。
\end{enumerate}

\subsubsection{物理的意味の説明}

\begin{itemize}
    \item 反射率は、特性インピーダンスの差によって決まる。インピーダンスの差が大きいほど、反射率は大きくなる。
    \item 位相変化は、特性インピーダンスの大小関係によって決まる。物質のインピーダンスが真空より小さい場合、反射面で位相が$\pi$変化する。
    \item ガラスの場合、反射率は約4\%であり、大部分の光は透過する。これは、ガラスと真空のインピーダンスの差が小さいためである。
    \item $\varepsilon < 0$の場合(メタマテリアルなど)、全反射が起こる。これは、物質内で電磁波が伝播できないためである。
\end{itemize}

\begin{figure}[H]
\centering
\includegraphics[width=0.8\textwidth]{figures/ex7_4_reflection.png}
\caption{媒質境界面での電磁波の反射}
\label{fig:ex7_4_reflection}
\end{figure}

\end{document}
