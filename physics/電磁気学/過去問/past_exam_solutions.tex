\documentclass[11pt,a4paper]{ltjsarticle}
\usepackage[no-math]{luatexja-fontspec}
\setmainjfont{Hiragino Mincho ProN}[
  UprightFont=*,
  BoldFont=*,
  ItalicFont=*,
  BoldItalicFont=*
]
\setsansjfont{Hiragino Kaku Gothic ProN}[
  UprightFont=*,
  BoldFont=*,
  ItalicFont=*,
  BoldItalicFont=*
]
\usepackage{amsmath,amssymb}
\usepackage{graphicx}
\usepackage{geometry}
\geometry{margin=2.5cm}
\usepackage{float}
\usepackage[draft=false]{hyperref}
\hypersetup{
    colorlinks=true,
    linkcolor=blue,
    citecolor=blue,
    urlcolor=blue,
    pdfusetitle=true
}

\title{電磁気学 過去問 解答・解説}
\author{名古屋大学 理学部}
\date{2023年度・2024年度・2024年度再試験}

\setcounter{tocdepth}{1}

\begin{document}

\begin{center}
\vspace*{1.5cm}
{\LARGE 電磁気学 過去問 解答・解説}\\[1.2cm]
{\large 名古屋大学 理学部}\\[0.8cm]
2023年度・2024年度・2024年度再試験\\[1.5cm]
\end{center}
\clearpage

\tableofcontents
\clearpage

\section*{この解答解説について(初学者の方へ)}
\addcontentsline{toc}{section}{この解答解説について(初学者の方へ)}

各問題の解答では、次のような構成を心がけています。

\begin{itemize}
\item \textbf{この問題のポイント}:その問題で何を理解することが大切か、どのような流れで解くかを短くまとめています。
\item \textbf{解き方の流れ}:各小問をどの順に、どの公式や法則を使って解くかを箇条書きにしています。
\item \textbf{用語の説明}:分極、磁化、電束密度、誘電率、透磁率など、問題で使う用語を初出付近で説明しています。
\item \textbf{使用する物理法則}:Maxwell方程式、境界条件、オームの法則など、どの法則に基づいて式を立てているかを明示しています。
\item \textbf{計算のステップ}:式を立てるときに「ステップ1」「ステップ2」と区切り、途中を省略せずに記述しています。
\item \textbf{なぜこのように求まるか}:答えの式が成り立つ理由や、物理的な意味を補足しています。
\item \textbf{図}:設定の概念図、電場・磁場の向き、グラフなどを入れています。
\end{itemize}

\noindent
数式の変形では「明らかに」「同様に」に頼らず、途中のステップを書いています。ベクトル計算では成分を明示し、積分では変数変換や部分積分の過程を示しています。

\paragraph{読むときのコツ(初学者の方へ)}
\begin{itemize}
\item まず各問題の「この問題のポイント」と「解き方の流れ」を読むと、何を求めるか・どの順で式を立てるかが把握しやすくなります。
\item 用語(分極、磁化、電束密度、境界条件など)は初出付近の「用語の説明」で定義しています。わからなくなったらそこに戻って確認してください。
\item 「なぜ~か」「原理的な説明」「物理的考察」の段落は、式の意味や物理的なイメージを理解するためのものです。答えの式だけではなく、なぜその式になるかも読むと理解が深まります。
\item 図は設定のイメージや式の対応関係を示しています。図のキャプションにも式や条件を書いているので、図と本文を対応させて読むとよいです。
\item 2024年度・再試験では「2023年度問題1と同一」のように他年度を参照することがあります。その場合は該当年度の該当問題を開き、「解き方の流れ」と「なぜ~か」の段落をあわせて読むと理解しやすくなります。
\end{itemize}

\section*{類題一覧}
\addcontentsline{toc}{section}{類題一覧}

\begin{itemize}
\item \textbf{問題1(物質中のMaxwell方程式)}:2023・2024・2024再のいずれにも出題。静電磁場の4式(1-1)、$D$, $H$ を用いた書き換え(1-2)、時間変動を含む2式(1-3)の流れで共通。
\item \textbf{問題2}:2023は誘電体板・導体板の電場、2024・2024再は強誘電体球の内部電場と $E_p$。2024再は(2-1)(2-2)のみ。
\item \textbf{問題3}:2023は金属表面での平面電磁波の反射・透過、2024は極性分子の分極と電気感受率 $\chi_e(\omega)$、2024再も極性分子(3-1, 3-2)に加え、問題4で金属内電子の伝導率 $\sigma(\omega)$。
\end{itemize}

\newpage

%======================================================================
% 2023年度 統計物理1 本試験(2023年2月1日 10:45--11:45)
%======================================================================
\part{2023年度 本試験}
\setcounter{section}{0}

\section{問題I:理想気体の断熱自由膨張とエントロピー変化(類題:2024年度 問題I;演習5-II)}

\subsection{問題}

$N$ モルの理想気体のエントロピー $S(T,V)$ と内部エネルギー $U(T,V)$ は以下で与えられる($T$:温度、$V$:体積、$R$:気体定数、$c$, $S_0$:定数):
\begin{align}
S(T,V) &= NR \ln\left(\frac{T^c V}{N}\right) + N S_0, \\
U(T,V) &= c N R T.
\end{align}

\begin{enumerate}
\item 断熱自由膨張により、系が初期状態 $(T,V)$ から終状態 $(T',V')$ へ変化したとする($V < V'$)。$T'$ を $T$, $V$, $V'$, $c$, $R$, $N$ のいずれかの記号を用いて表せ。
\item 問1の操作後、系を準静的断熱過程により $(T',V')$ から $(T'',V)$ ともとの体積まで圧縮する。この一連の操作における、初期状態 $(T,V)$ と終状態 $(T'',V)$ のエントロピー変化を、$T$, $V$, $V'$, $c$, $R$, $N$ のいずれかの記号を用いて表せ。
\end{enumerate}

\subsection{解答}

\paragraph{この問題のポイント(初学者向け)}

断熱自由膨張では外界と熱・仕事のやりとりがないため内部エネルギー $U$ は不変。理想気体では $U$ は $T$ のみに依存するので $T' = T$。問2では準静的断熱過程で $S$ 一定を用い、全体のエントロピー変化を求める。

\paragraph{解き方の流れ}
\begin{enumerate}
\item 問1:断熱自由膨張では $Q=0$, $W=0$ なので $\Delta U = 0$。理想気体では $U = c N R T$ で $U$ は $V$ に依存しないので $T' = T$。
\item 問2:準静的断熱過程では $S$ 一定なので、$\Delta S$ は初期 $(T,V)$ と終状態 $(T'',V)$ のエントロピー差。断熱過程の関係式 $T^c V = \mathrm{const}$ で $T''$ を $T,V,V'$ で表し、$\Delta S$ に代入する。
\end{enumerate}

\paragraph{用語の説明}
\begin{itemize}
\item \textbf{断熱自由膨張}:外界と熱のやりとりがなく(断熱)、外から仕事もされない条件下で気体が膨張する過程。非準静的で不可逆。熱力学第一法則より $Q=0$, $W=0$ なら $\Delta U=0$。
\item \textbf{準静的断熱過程}:断熱しながら無限にゆっくり変化させる過程。可逆とみなせ、この間エントロピー $S$ は一定。
\item \textbf{理想気体の断熱過程}:$T^c V = \mathrm{const}$ が成り立つ($c$ は $U = c N R T$ の係数)。$dU = -p\,dV$ と $p = NRT/V$ から $c N R\,dT = -(NRT/V)\,dV$ を積分して得られる。
\end{itemize}

\paragraph{問1:$T'$}

断熱自由膨張では $Q=0$, $W=0$ より $\Delta U = 0$。理想気体では $U = c N R T$ で $U$ は $V$ に依存しないので
\begin{equation}
\boxed{T' = T}.
\end{equation}

\paragraph{なぜ温度が変わらないか(原理的な説明)}

断熱自由膨張では外界と熱・仕事のやりとりがないため $\Delta U = 0$。理想気体では $U$ は温度 $T$ のみの関数なので、体積が $V \to V'$ に変わっても $U$ が一定なら $T$ も不変である。

\paragraph{問2:エントロピー変化}

準静的断熱過程 $(T',V') \to (T'',V)$ では $S$ 一定。理想気体の断熱過程では $dU = -p\,dV$ と $p = NRT/V$ より $c N R\,dT/T = -N R\,dV/V$ を積分して $T^c V = \mathrm{const}$ が得られる。したがって $(T')^c V' = (T'')^c V$。$T'=T$ より
\begin{equation}
T'' = T \left(\frac{V'}{V}\right)^{1/c}.
\end{equation}
初期エントロピー $S_{\mathrm{i}} = NR\ln(T^c V/N) + N S_0$、終状態 $S_{\mathrm{f}} = NR\ln((T'')^c V/N) + N S_0$ より
\begin{equation}
\Delta S = S_{\mathrm{f}} - S_{\mathrm{i}} = N R \ln\frac{(T'')^c V}{T^c V}
= N R c \ln\frac{T''}{T}.
\end{equation}
ここに問1の $T'=T$ と断熱過程の関係 $(T')^c V' = (T'')^c V$ より得た $T''/T = (V'/V)^{1/c}$ を代入する。$\ln(T''/T) = \ln(V'/V)^{1/c} = (1/c)\ln(V'/V)$ なので
\begin{equation}
\Delta S = N R c \cdot \frac{1}{c}\ln\frac{V'}{V}
= \boxed{N R \ln\frac{V'}{V}}.
\end{equation}
($V' > V$ なので $\Delta S > 0$。断熱自由膨張は不可逆過程のため、エントロピーが増大する。)

\paragraph{なぜエントロピーが増大するか(物理的考察)}

断熱自由膨張 $(T,V) \to (T',V')$ は不可逆過程である。気体が急に膨張するため、同じ体積変化を準静的に行う経路と比べて「無秩序に」広がり、熱力学第二法則よりエントロピーは増大する。準静的断熱圧縮 $(T',V') \to (T'',V)$ では $S$ 一定なので、全体の $\Delta S$ は断熱自由膨張の段階で生じた増加 $N R \ln(V'/V)$ に等しい。

\begin{figure}[H]
\centering
\includegraphics[width=0.85\textwidth]{figures/past2024_ex1_why_entropy.png}
\caption{問題I:断熱自由膨張で気体が $V \to V'$ に広がるとエントロピーが増大する。準静的断熱圧縮では $S$ 一定。}
\label{fig:past2023main_ex1_why}
\end{figure}

\begin{figure}[H]
\centering
\includegraphics[width=0.85\textwidth]{figures/past2024_ex1_path.png}
\caption{問題I:$(T,V) \to (T',V')$(断熱自由膨張)、$(T',V') \to (T'',V)$(準静的断熱圧縮)。}
\label{fig:past2023main_ex1_path}
\end{figure}

%----------------------------------------------------------------------
\section{問題II:混合に関するエントロピー変化(類題:2023年度再試験 問題I、2025年度 問題II;演習3-II, 演習5-III--IV)}
%----------------------------------------------------------------------

\subsection{問題}

断熱壁で左右に仕切られた容器がある。左側に温度 $T_A$ の理想気体が 1 モル・体積 $V_A$、右側に温度 $T_B$ の理想気体が 1 モル・体積 $V_B$ 入っている。$T_A > T_B$ とする。容器全体は外界から孤立している。理想気体のエントロピーは $S(T,V) = NR\ln(T^c V/N) + N S_0$ を用いる。

\begin{enumerate}
\item 左右を隔てる断熱壁を(気体の移動がないように)透熱壁に置き換えた。十分時間が経った後の、左右の気体の等しくなった温度 $T$ を求めよ。
\item 問1の終状態における容器全体のエントロピーが、初期状態に比べてどれだけ変化したか。$c$, $R$, $T_A$, $T_B$, $V_A$, $V_B$, $S_0$ の記号を用いて示せ。
\item 問2で計算した全エントロピー変化が正であることを証明せよ。
\item 同じ初期状態 $(T_A,V_A)$, $(T_B,V_B)$ から出発し、最終的に温度が等しくなる状態 $\{(T,V_A),(T,V_B)\}$ を達成する操作の例を、$\{(T_A,V_A),(T_B,V_B)\} \to \{(T,V_A),(T,V_B)\}$ の形式で述べよ。
\item 問4と同様の操作のうち、仕切り壁を左右に動かしたり、断熱壁と透熱壁に交換したりできるとする。終状態の体積は $V_A$, $V_B$ のまま、気体の移動はない。この操作で達成できる最低温度 $T_{\mathrm{min}}$ を求めよ。また、どのような操作で $T_{\mathrm{min}}$ が達成できるか、操作方法を具体的に述べよ。
\item 達成できる最高温度 $T_{\mathrm{max}}$ を求めよ。また、どのような操作で $T_{\mathrm{max}}$ が達成できるか、操作方法を具体的に述べよ。
\end{enumerate}

\subsection{解答}

\paragraph{この問題のポイント(初学者向け)}

孤立系では全内部エネルギーが保存するので、問1の終温度 $T$ はエネルギー保存から決まる。問2はエントロピー変化、問3はその正性の証明。問5・問6は「仕切り壁の操作で達成できる終温度の範囲」を求める発展である。

\paragraph{解き方の流れ}
\begin{enumerate}
\item 問1:孤立系なので全内部エネルギー $U_{\mathrm{total}}$ は一定。左室・右室とも 1 モルなので $c R T_A + c R T_B = 2 c R T$ より $T$ を求める。
\item 問2:初期状態(左 $T_A,V_A$、右 $T_B,V_B$)と終状態(両室とも $T$、体積 $V_A,V_B$ のまま)のエントロピーを $S(T,V)$ で計算し、差をとる。
\item 問3:相加・相乗平均の不等式 $(T_A+T_B)^2 \ge 4 T_A T_B$(等号は $T_A=T_B$ のときのみ)を用いて $\Delta S > 0$ を示す。
\item 問5:可逆操作では $S_{\mathrm{f}}(T) \ge S_{\mathrm{i}}$ が成り立つ。$S_{\mathrm{f}}(T)$ は $T$ の増加関数なので、等号 $S_{\mathrm{f}}(T)=S_{\mathrm{i}}$ を満たす最小の $T$ が $T_{\mathrm{min}}$。
\item 問6:内部エネルギー保存より、両室が同じ温度 $T$ のとき $T$ は $(T_A+T_B)/2$ を超えられない。その最大値が $T_{\mathrm{max}}$。
\end{enumerate}

\paragraph{記号・用語}
左室・右室とも 1 モルなので $N_A = N_B = 1$。\textbf{透熱壁}:熱を通し気体は通さない。十分時間が経つと両側の温度が等しくなる(熱平衡)。\textbf{孤立系}:外界と仕事・熱のやりとりがない系。熱力学第一法則より全内部エネルギーは保存する。

\paragraph{問1:終状態の温度 $T$}

内部エネルギー保存より $c R T_A + c R T_B = c R T + c R T$(各室 1 モル)。よって
\begin{equation}
\boxed{T = \frac{T_A + T_B}{2}}.
\end{equation}
(算術平均。高温側から低温側へ熱が流れ、両室の温度がこの $T$ で一致する。)

\paragraph{問2:エントロピー変化}

初期状態のエントロピーは $S_{\mathrm{initial}} = R \ln(T_A^c V_A) + S_0 + R \ln(T_B^c V_B) + S_0$(各室 1 モル)。終状態では両室とも温度 $T$、体積は $V_A$, $V_B$ のままなので $S_{\mathrm{final}} = R \ln(T^c V_A) + S_0 + R \ln(T^c V_B) + S_0$。したがって
\begin{equation}
\Delta S = S_{\mathrm{final}} - S_{\mathrm{initial}}
= R \ln\frac{T^c}{T_A^c} + R \ln\frac{T^c}{T_B^c}
= R \ln\frac{T^2}{T_A T_B}.
\end{equation}
問1の $T = (T_A+T_B)/2$ を代入して
\begin{equation}
\boxed{\Delta S = R \ln\frac{(T_A+T_B)^2}{4 T_A T_B}}.
\end{equation}

\paragraph{問3:$\Delta S > 0$ の証明}

相加・相乗平均の不等式より、$T_A \neq T_B$ のとき
\begin{equation}
\frac{T_A + T_B}{2} > \sqrt{T_A T_B}
\quad \Rightarrow \quad
(T_A+T_B)^2 > 4 T_A T_B.
\end{equation}
(等号は $T_A = T_B$ のときのみ。これは $(T_A+T_B)^2 - 4 T_A T_B = T_A^2 - 2 T_A T_B + T_B^2 = (T_A - T_B)^2 \ge 0$ からも確かめられる。)したがって $\Delta S = R \ln\bigl((T_A+T_B)^2/(4 T_A T_B)\bigr) > 0$。物理的には、熱が高温から低温へ流れる過程は不可逆であり、熱力学第二法則から孤立系のエントロピーは増大するので $\Delta S > 0$ である。

\paragraph{問4:操作の例}

断熱壁を透熱壁に置き換え、十分時間を待つ。これで $\{(T_A,V_A),(T_B,V_B)\} \to \{(T,V_A),(T,V_B)\}$($T = (T_A+T_B)/2$)が達成できる。

\paragraph{問5:最低温度 $T_{\mathrm{min}}$ とその操作}

可逆操作では熱力学第二法則より $S_{\mathrm{f}}(T) \ge S_{\mathrm{i}}$ でなければならない。終状態のエントロピー $S_{\mathrm{f}}(T) = R \ln(T^c V_A) + R \ln(T^c V_B) + \mathrm{const}$ は $T$ の増加関数なので、\textbf{等号 $S_{\mathrm{f}}(T)=S_{\mathrm{i}}$ を満たす最小の $T$} が達成可能な最低温度 $T_{\mathrm{min}}$ である。$S_{\mathrm{f}} - S_{\mathrm{i}} = R \ln(T^2/(T_A T_B)) \ge 0$ より $\ln(T^2/(T_A T_B)) \ge 0$、すなわち $T^2/(T_A T_B) \ge 1$。$T$, $T_A$, $T_B > 0$ なので $T^2 \ge T_A T_B$、したがって $T \ge \sqrt{T_A T_B}$。等号は可逆過程で達成されるので
\begin{equation}
\boxed{T_{\mathrm{min}} = \sqrt{T_A T_B}}.
\end{equation}
($T_A$ と $T_B$ の幾何平均。算術平均 $(T_A+T_B)/2$ より小さく、可逆操作を駆使すれば問1より低い終温度を実現できる。)

操作の例:左室を断熱壁で囲い準静的断熱膨張させて冷却、右室を断熱圧縮して加熱するなど、両室のエントロピーを変えずに可逆的に温度を揃える。一方を断熱膨張・他方を断熱圧縮する組み合わせで、$S_{\mathrm{f}} = S_{\mathrm{i}}$ を満たす終温度 $T_{\mathrm{min}}$ を実現する。

\paragraph{問6:最高温度 $T_{\mathrm{max}}$ とその操作}

内部エネルギー保存より、両室が同じ温度 $T$ になるとき $2 c R T = c R (T_A + T_B)$ なので $T$ は $(T_A+T_B)/2$ を超えられない。したがって
\begin{equation}
\boxed{T_{\mathrm{max}} = \frac{T_A + T_B}{2}}.
\end{equation}
操作:問1・問4と同様、断熱壁を透熱壁に置き換えて熱平衡に達させる。これが $T_{\mathrm{max}}$ を達成する。

\begin{figure}[H]
\centering
\includegraphics[width=0.75\textwidth]{figures/past2023main_ex2_setup.png}
\caption{問題IIの設定(本試験):左右に仕切られた容器。左は 1 モル・$T_A,V_A$、右は 1 モル・$T_B,V_B$。}
\label{fig:past2023main_ex2_setup}
\end{figure}

\begin{figure}[H]
\centering
\includegraphics[width=0.75\textwidth]{figures/past2023main_ex2_Tmin_Tmax.png}
\caption{問題II:達成できる終温度 $T$ の範囲。$T_{\mathrm{min}} = \sqrt{T_A T_B}$(幾何平均)$\le T \le$ $T_{\mathrm{max}} = (T_A+T_B)/2$(算術平均)。}
\label{fig:past2023main_ex2_Tmin_Tmax}
\end{figure}

%----------------------------------------------------------------------
\section{問題III:N個の独立な調和振動子(類題:2023年度再試験 問題III、2024・2025年度 問題IV;演習7-III)}
%----------------------------------------------------------------------

\subsection{問題}

$N$ 個の独立な調和振動子を量子的に扱う。$l$ 番目の振動子のエネルギーは $E_l = \hbar\omega m_l$($m_l = 0,1,2,\ldots$)、$\omega$ は振動子の角振動数である。系が温度 $T$ の大きな熱源と接しているとき、状態 $i$ にある確率は $P(i) = e^{-\beta E_i}/Z_N$($\beta = 1/(k_B T)$)。状態 $i$ は $(m_1,\ldots,m_N)$ で指定され、$E_i = \hbar\omega(m_1+\cdots+m_N) = \hbar\omega M$ とする。

\begin{enumerate}
\item $N=1$ のとき $Z_1$ を計算せよ。
\item $N=1$ のときエネルギーの平均値 $\langle E \rangle$ を $\beta$, $\hbar\omega$ などで表せ。
\item 一般の $N$ のとき $Z_N$ を計算せよ。
\item 一般の $N$ のとき $\langle E \rangle$ を $\beta$, $\hbar\omega$, $N$ などで表せ。
\item $N$ が十分大きいとき、系のエネルギーが $E$ である確率 $P(E)$ は $E^*$ で最大となり鋭いピークを持つ。$E^*$ を $\beta$, $\hbar\omega$, $N$ などで表せ。
\item エネルギー $E$ のゆらぎ(分散)$\sigma^2 = \langle (E - \langle E \rangle)^2 \rangle$ を $\beta$, $\hbar\omega$, $N$ などで表せ。
\end{enumerate}

\subsection{解答}

\paragraph{問題III の全体像(詳解)}

本問は\textbf{同じ設定}($N$ 個の独立な量子調和振動子が温度 $T$ の熱源と接触)を、三つの観点から扱います。

\begin{enumerate}
\item \textbf{問1〜問4:分配関数と平均エネルギー}\\
個々の量子状態 $i$(各振動子の量子数 $(m_1,\ldots,m_N)$ で指定)を取る確率は $P(i) = e^{-\beta E_i}/Z_N$ である。ここから $Z_N$ と平均エネルギー $\langle E \rangle = -\partial\ln Z_N/\partial\beta$ を求める。これは「熱力学的な代表値」であり、温度 $T$ が決まると $\langle E \rangle$ が一意に決まる。
\item \textbf{問5:最確エネルギー $E^*$}\\
「系の全エネルギーが $E$ である」という事象は、エネルギー $E$ を持つ\textbf{たくさんの状態}の和である。$E = \hbar\omega M$ のとき、そのような状態の数は $W_N(M)$ 個ある。したがって「エネルギーが $E$ である確率」は $P(E) \propto W_N(M)\,e^{-\beta E}$ となり、これが最大になる $E$ が\textbf{最確エネルギー} $E^*$ である。$N \gg 1$ のとき $P(E)$ は $E^*$ のまわりに鋭いピークを持ち、$E^* = \langle E \rangle$ となる。
\item \textbf{問6:エネルギーのゆらぎ}\\
カノニカル分布では系のエネルギーは熱源とのやりとりで揺らぐ。その揺らぎの大きさ(分散 $\sigma^2 = \langle (E-\langle E \rangle)^2 \rangle$)は、統計力学の一般公式 $\sigma^2 = k_B T^2 C_V$ で与えられる。$C_V$ は問4の $\langle E \rangle$ を $T$ で微分して求める。
\end{enumerate}

\noindent
つまり、\textbf{問1〜4で「平均」を、問5で「最も起こりやすい値」を、問6で「揺らぎの幅」を求めている}という流れです。問5と問6は、カノニカル分布でエネルギーが一定でないこと(揺らぎ)を explicitly に扱う部分です。

\paragraph{この問題のポイント(初学者向け)}

$N$ 個の独立な量子調和振動子が温度 $T$ の熱源と接触している場合を扱う。\textbf{分配関数} $Z$ は「すべての状態についてボルツマン因子 $e^{-\beta E}$ を足し合わせたもの」であり、確率の規格化定数になる。平均エネルギーは $\langle E \rangle = -\partial\ln Z/\partial\beta$ で求まる。問5の $W_N(M)$ は「全エネルギーが $E = \hbar\omega M$ であるような状態の数」であり、重複組合せで与えられる。問6では、熱源と接触した系のエネルギー揺らぎが $\langle (\delta E)^2 \rangle = k_B T^2 C_V$ で与えられる公式を用いる。

\paragraph{用語の説明(統計力学)}
本問では「系が温度 $T$ の大きな熱源と接している」とあるので、\textbf{カノニカル分布}の取り扱いになる。用語集(本冊の「用語集(統計力学)」)も参照されたい。
\begin{itemize}
\item \textbf{カノニカル分布}:系が熱源と接触して熱平衡にあるとき、系の量子状態 $i$(エネルギー $E_i$)を取る確率は $P(i) = e^{-\beta E_i}/Z$ で与えられる。$Z$ は分配関数、$\beta = 1/(k_B T)$ である。熱源とエネルギーをやりとりするため、系のエネルギーは一定ではなく揺らぐ。
\item \textbf{分配関数} $Z$:$Z = \sum_i e^{-\beta E_i}$。すべての状態についてボルツマン因子を足し合わせたもので、$P(i)$ の規格化を保証する($\sum_i P(i)=1$)。
\item \textbf{ボルツマン因子}:$e^{-\beta E}$。エネルギーが低い状態ほど確率が高くなる重みである。
\item \textbf{熱源}:温度 $T$ を一定に保つ外界。系が熱源と接すると、熱平衡で系の温度も $T$ になる。
\end{itemize}

\paragraph{問1〜問4:分配関数と平均エネルギー}

\textbf{問1($Z_1$)の計算の流れ}:1個の振動子のエネルギーは $E_m = \hbar\omega m$($m = 0,1,2,\ldots$)なので、分配関数の定義 $Z = \sum_{\text{状態}} e^{-\beta E}$ より
\[
Z_1 = \sum_{m=0}^\infty e^{-\beta\hbar\omega m}
= e^0 + e^{-\beta\hbar\omega} + e^{-2\beta\hbar\omega} + \cdots.
\]
$x = e^{-\beta\hbar\omega}$ とおくと $Z_1 = 1 + x + x^2 + \cdots = \sum_{m=0}^\infty x^m$ は等比級数である。$|x|<1$($\beta\hbar\omega>0$、つまり $T>0$ のとき成立)のとき、等比級数の和の公式 $\sum_{m=0}^\infty x^m = 1/(1-x)$ より
\begin{equation}
Z_1 = \frac{1}{1 - e^{-\beta\hbar\omega}}.
\end{equation}

\textbf{問2($\langle E \rangle\big|_{N=1}$)の計算の流れ}:カノニカル分布では平均エネルギーは $\langle E \rangle = \sum_i E_i P(i) = (1/Z)\sum_i E_i e^{-\beta E_i}$ で定義される。一方、$Z = \sum_i e^{-\beta E_i}$ を $\beta$ で微分すると $\frac{\partial Z}{\partial\beta} = -\sum_i E_i e^{-\beta E_i}$ なので $\frac{\partial\ln Z}{\partial\beta} = \frac{1}{Z}\frac{\partial Z}{\partial\beta} = -\langle E \rangle$ となる。したがって \textbf{$\langle E \rangle = -\partial\ln Z/\partial\beta$} が成り立つ。$N=1$ のとき $\ln Z_1 = -\ln(1-e^{-\beta\hbar\omega})$ を $\beta$ で微分すると $\frac{\partial\ln Z_1}{\partial\beta} = \frac{\hbar\omega\,e^{-\beta\hbar\omega}}{1-e^{-\beta\hbar\omega}}$ なので、$\langle E \rangle\big|_{N=1} = -\frac{\partial\ln Z_1}{\partial\beta} = \frac{\hbar\omega}{e^{\beta\hbar\omega}-1}$ である。

\textbf{なぜ $Z_N = (Z_1)^N$ か}:各振動子は互いに独立に量子数 $m_i$ を取る。全状態の和は「1番目の振動子のすべての状態」「2番目の振動子のすべての状態」…の組み合わせについて $e^{-\beta(E_1+E_2+\cdots+E_N)} = e^{-\beta E_1}e^{-\beta E_2}\cdots e^{-\beta E_N}$ を足すことになる。変数分離でき、$\sum_{m_1}\sum_{m_2}\cdots\sum_{m_N} (\cdots) = \bigl(\sum_{m_1} e^{-\beta\hbar\omega m_1}\bigr)\bigl(\sum_{m_2} e^{-\beta\hbar\omega m_2}\bigr)\cdots = (Z_1)^N$ となる。以上より
\begin{equation}
Z_1 = \frac{1}{1-e^{-\beta\hbar\omega}}, \quad
\langle E \rangle \big|_{N=1} = \frac{\hbar\omega}{e^{\beta\hbar\omega}-1}, \quad
Z_N = \left(\frac{1}{1-e^{-\beta\hbar\omega}}\right)^N, \quad
\boxed{\langle E \rangle = N\,\frac{\hbar\omega}{e^{\beta\hbar\omega}-1}}.
\end{equation}
($N$ 個のときは $\ln Z_N = N\ln Z_1$ なので $\langle E \rangle = -\partial\ln Z_N/\partial\beta = -N\,\partial\ln Z_1/\partial\beta = N\times$(1個の平均エネルギー)である。)

\paragraph{最確エネルギーとは(詳しい説明)}

\textbf{最確エネルギー $E^*$}とは、「系の全エネルギーを1回測定したとき、最も出やすい値」のことです。つまり、\textbf{確率 $P(E)$ が最大になるエネルギー $E$} です。

カノニカル分布では、個々の量子状態 $i$ を取る確率は $P(i) = e^{-\beta E_i}/Z_N$ です。しかし「系のエネルギーが $E$ である」という事象は、\textbf{エネルギーが $E$ であるようなすべての状態}が含まれる事象です。全エネルギーが $E = \hbar\omega M$ になるのは、$m_1+\cdots+m_N = M$ を満たす状態 $(m_1,\ldots,m_N)$ のときで、そのような状態の数が $W_N(M)$ 個あります。それぞれの状態 $i$ を取る確率は $e^{-\beta E_i}/Z_N = e^{-\beta E}/Z_N$($E_i=E$ なので)なので、「エネルギーが $E$ である確率」は
\[
P(E) = \frac{\text{(エネルギー $E$ の状態の数)}\times \text{(その1つを取る確率)}}{\text{(規格化)}}
\propto W_N(M) \cdot e^{-\beta E}.
\]
となります。$W_N(M)$ は $M$ が大きいほど増加します(エネルギーを多くの振動子に分け与えるやり方が増える)。一方 $e^{-\beta E}$ は $E$ が大きいほど減少します(高温でない限り、高エネルギーは起こりにくい)。この二つの効果のバランスで、$P(E)$ はある $E^*$ で最大になります。その $E^*$ が最確エネルギーです。

$N \gg 1$ のとき、$P(E)$ は $E^*$ のまわりに\textbf{鋭いピーク}を持ち、ほとんどすべての確率が $E \approx E^*$ の近くに集中します。そのため、\textbf{最確値 $E^*$ と平均値 $\langle E \rangle$ は一致し}、どちらも「平衡状態の代表的なエネルギー」を表します。問5では $E^*$ を、問4では $\langle E \rangle$ を別の方法で求めており、結果が同じ $N\hbar\omega/(e^{\beta\hbar\omega}-1)$ になることで、この一致が確かめられます。

\paragraph{問5:最確エネルギー $E^*$ の計算}

\textbf{なぜ $P(E) \propto W_N(M)\,e^{-\beta E}$ か}:上で述べたとおり、エネルギー $E = \hbar\omega M$ を取る状態が $W_N(M)$ 個あり、それぞれの確率が $e^{-\beta E}/Z_N$ に比例するので、$P(E) \propto W_N(M)\,e^{-\beta E}$ である。状態数 $W_N(M)$ は $M$ の増加とともに増え、ボルツマン因子 $e^{-\beta E}$ は $E$ の増加とともに減る。その積が最大になる $M$ が最確エネルギーに対応する。

$W_N(M)$ は $M = m_1+\cdots+m_N$ を満たす非負整数の組 $(m_1,\ldots,m_N)$ の個数(重複組合せ)で、$W_N(M) = (M+N-1)!/((N-1)!\,M!)$ である。$P(E)$ が最大になる $M$ を求めるには、$\ln P$ の $M$ による極大を考えればよい($\ln$ は単調増加なので $P$ と $\ln P$ の最大点は同じ)。よって
\[
\frac{\partial}{\partial M}\bigl( \ln W_N(M) - \beta\hbar\omega M \bigr) = 0
\]
を解く。$M \gg 1$, $N \gg 1$ のときスターリングの公式 $\ln n! \approx n\ln n - n$ を用いると、$\ln W_N(M) \approx (M+N-1)\ln(M+N-1) - (N-1)\ln(N-1) - M\ln M - (\text{定数})$ を $M$ で微分して $\partial\ln W_N(M)/\partial M \approx \ln(M+N) - \ln M = \ln((M+N)/M)$ となる(再試験 問題III 問6に同じ式の詳しい導出あり)。したがって
\[
\ln\frac{M+N}{M} - \beta\hbar\omega = 0
\quad \Rightarrow \quad
\frac{M+N}{M} = e^{\beta\hbar\omega}
\quad \Rightarrow \quad
1 + \frac{N}{M} = e^{\beta\hbar\omega}
\quad \Rightarrow \quad
M = \frac{N}{e^{\beta\hbar\omega}-1}.
\]
$E^* = \hbar\omega M^*$ なので
\begin{equation}
\boxed{E^* = \frac{N\hbar\omega}{e^{\beta\hbar\omega}-1} = \langle E \rangle}.
\end{equation}
(最確エネルギー $E^*$ と平均 $\langle E \rangle$ が一致する。$N \gg 1$ のとき $P(E)$ は $E^*$ のまわりに鋭いピークを持ち、測定するとほぼ必ず $E \approx E^*$ 付近の値が得られる。)

\paragraph{問6:ゆらぎ $\sigma^2$}

\textbf{なぜ $\langle (\delta E)^2 \rangle = k_B T^2 C_V$ が使えるか}:本問では系が「温度 $T$ の熱源と接している」ので、\textbf{カノニカル分布}が適用される。カノニカル分布では、系のエネルギー $E$ は熱源とのやりとりで揺らぎ、その分散は統計力学の一般公式 $\langle (\delta E)^2 \rangle = k_B T^2 C_V$ で与えられる($C_V$ は定積比熱)。導出の概略:$\langle E \rangle = -\partial\ln Z/\partial\beta$ と $\langle E^2 \rangle$ の関係から $\langle (\delta E)^2 \rangle = \langle E^2 \rangle - \langle E \rangle^2 = \partial^2\ln Z/\partial\beta^2$ となり、$\partial/\partial\beta = -k_B T^2 \partial/\partial T$ と $C_V = (\partial\langle E\rangle/\partial T)_V$ を用いると上式が得られる。教科書の「カノニカル分布のゆらぎ」を参照されたい。

\textbf{問6の計算の流れ}:公式 $\sigma^2 = k_B T^2 C_V$ を使うには、定積比熱 $C_V = (\partial\langle E \rangle/\partial T)_V$ が必要である。問4で $\langle E \rangle = N\hbar\omega/(e^{\beta\hbar\omega}-1)$ を既に求めているので、これを $T$ で微分して $C_V$ を得てから $\sigma^2$ に代入する。$T$ で直接微分する代わりに、$\beta = 1/(k_B T)$ なので $d\beta/dT = -1/(k_B T^2) = -\beta^2/k_B$ を用いて $\partial\langle E \rangle/\partial T = (\partial\langle E \rangle/\partial\beta)(d\beta/dT)$ と計算する。$\langle E \rangle$ を $\beta$ で微分すると
\begin{equation}
\frac{\partial \langle E \rangle}{\partial \beta}
= N\hbar\omega \cdot \frac{(-1)\cdot(-\hbar\omega)\,e^{\beta\hbar\omega}}{(e^{\beta\hbar\omega}-1)^2}
= -N\,\frac{(\hbar\omega)^2 e^{\beta\hbar\omega}}{(e^{\beta\hbar\omega}-1)^2}.
\end{equation}
したがって
\begin{equation}
C_V = \frac{\partial \langle E \rangle}{\partial T}
= \frac{\partial \langle E \rangle}{\partial \beta}\,\frac{d\beta}{dT}
= \left(-N\,\frac{(\hbar\omega)^2 e^{\beta\hbar\omega}}{(e^{\beta\hbar\omega}-1)^2}\right)
\cdot \left(-\frac{\beta^2}{k_B}\right)
= N k_B\,\frac{(\beta\hbar\omega)^2 e^{\beta\hbar\omega}}{(e^{\beta\hbar\omega}-1)^2}.
\end{equation}
$\sigma^2 = k_B T^2 C_V$ に代入し、$k_B T^2 = 1/(k_B \beta^2)$ を用いると
\begin{equation}
\boxed{\sigma^2 = \langle (\delta E)^2 \rangle = k_B T^2 C_V
= N (\hbar\omega)^2 \frac{e^{\beta\hbar\omega}}{(e^{\beta\hbar\omega}-1)^2}}.
\end{equation}
($N$ が大きいとき相対ゆらぎ $\sigma/\langle E \rangle$ は $1/\sqrt{N}$ のオーダーで小さくなる。)

\paragraph{問題III のまとめ(問1〜問6のつながり)}

問1〜4では、カノニカル分布の基本量である\textbf{分配関数 $Z_N$}と\textbf{平均エネルギー $\langle E \rangle$}を求めた。これらは「温度 $T$ が与えられたときの平衡状態の代表値」を表す。問5では、同じ平衡状態において「系のエネルギーを1回測定したとき最も出やすい値」である\textbf{最確エネルギー $E^*$}を求め、$E^* = \langle E \rangle$ となることを確認した。$N \gg 1$ のとき $P(E)$ が $E^*$ 付近に鋭く集中するため、最確値と平均値が一致する。問6では、エネルギーが熱源とのやりとりで揺らぐ\textbf{幅}(分散 $\sigma^2$)を、公式 $\sigma^2 = k_B T^2 C_V$ と問4で得た $\langle E \rangle$ から $C_V$ を求めることで計算した。$N$ が大きいとき $\sigma/\langle E \rangle \propto 1/\sqrt{N}$ なので、粒子数が増えるほど相対的な揺らぎは小さくなり、「平衡のエネルギー」はより明確に決まる。

\begin{figure}[H]
\centering
\includegraphics[width=0.85\textwidth]{figures/past2023_ex3_W_P.png}
\caption{問題III:$P(E)$ は $E^*$ で鋭いピークを持ち、$N$ が大きいとき $\sigma/\langle E \rangle \propto 1/\sqrt{N}$ で小さくなる。}
\label{fig:past2023main_ex3}
\end{figure}

%======================================================================
% 2023年度 統計物理1 再試験(2023年2月8日 10:45--11:45)
%======================================================================
\part{2023年度 再試験}
\setcounter{section}{0}

\section{問題I:理想気体の混合(類題:2023年度本試験 問題II、2025年度 問題II;演習3-II, 演習5-III--IV)}

\subsection{問題}

体積 $V$ の容器に入った $N$ モルの理想気体のエントロピー $S(T,V)$ と内部エネルギー $U(T,V)$ は以下で与えられる($T$ は温度、$V$ は体積、$R$ は気体定数、$c$ と $S_0$ は定数):
\begin{align}
S(T,V) &= NR \ln\left(\frac{T^c V}{N}\right) + N S_0, \\
U(T,V) &= c N R T.
\end{align}

容器が壁で左右に仕切られている。左側には温度 $T_A$ の理想気体が 1 モル入っており体積は $V_A$、右側には温度 $T_B$ の理想気体が 2 モル入っており体積は $V_B$ である。仕切り壁は断熱壁であり、$T_A > T_B$ とする。容器全体は外界から断熱壁で覆われ、孤立している。

\begin{enumerate}
\item 左右を隔てる断熱壁を(気体の移動がないように)透熱壁に置き換えた。十分に時間が経過し、左右の気体の温度が等しくなったときの終状態の温度 $T$ を求めよ。
\item 問1の終状態における容器全体のエントロピーが、初期状態に比べてどれだけ変化したか。$c$, $R$, $T_A$, $T_B$, $V_A$, $V_B$, $S_0$ の記号を用いて示せ。
\item 問2で計算した全エントロピー変化が正であることを証明せよ。
\item 同じ初期状態から出発し、最終的に温度が等しくなる状態 $\{(T,V_A),(T,V_B)\}$ を達成する操作のうち、仕切り壁を動かしたり透熱壁に交換したりできるとする。終状態の左右の体積は $V_A$, $V_B$ のまま、左右の部屋の間で気体の移動はないとする。この操作で達成できる最低温度 $T_{\mathrm{min}}$ を求めよ。
\end{enumerate}

\subsection{解答}

\paragraph{この問題のポイント(初学者向け)}

この問題では、\textbf{孤立系で内部エネルギーが保存すること}と、\textbf{エントロピーが不可逆過程で増大すること}を理解することが大切です。問1は「熱が流れたあと、両方の部屋の温度がどうなるか」を内部エネルギー保存から求めます。問2はそのときのエントロピー変化、問3はその変化が正であることの証明、問4は「うまく操作すればどこまで温度を下げられるか」という発展です。

\paragraph{解き方の流れ}

\begin{enumerate}
\item 問1:孤立系なので全内部エネルギー $U_{\mathrm{total}}$ は一定。初期と終状態で $U_{\mathrm{total}}$ を書き、等号で結んで $T$ を求める。
\item 問2:与えられた $S(T,V)$ の式で、初期(左室 $T_A,V_A$、右室 $T_B,V_B$)と終状態(両室とも $T$、体積は $V_A,V_B$ のまま)のエントロピーを計算し、差をとる。
\item 問3:$\Delta S > 0$ を、対数の不等式 $\ln t \le t-1$ または熱力学第二法則から示す。
\item 問4:可逆操作では $S_{\mathrm{f}} \ge S_{\mathrm{i}}$ の等号が成り立つ。それを満たす最小の $T$ が $T_{\mathrm{min}}$。
\end{enumerate}

\paragraph{用語の説明}
\begin{itemize}
\item \textbf{透熱壁}:熱のみを通し、気体は通さない壁。十分時間が経つと両側の温度が等しくなる(\textbf{熱平衡})。
\item \textbf{断熱壁}:熱も気体も通さない壁。両側の温度は独立に保たれる。
\item \textbf{孤立系}:外界と仕事・熱のやりとりがない系。熱力学第一法則より、全体の内部エネルギーは保存する。
\item \textbf{可逆・不可逆}:可逆過程では外界を含めた全エントロピーが変わらない。不可逆過程(例:熱が高温から低温へ一方向に流れる)では全エントロピーは増大する。
\item \textbf{加重平均}:重みをつけた平均。問1の $T = (N_A T_A + N_B T_B)/(N_A+N_B)$ は、モル数 $N_A$, $N_B$ を重みとした温度の平均である。
\end{itemize}

\paragraph{記号の整理}
左室:モル数 $N_A = 1$、初期温度 $T_A$、体積 $V_A$。右室:モル数 $N_B = 2$、初期温度 $T_B$、体積 $V_B$。透熱壁に替えた後の共通温度を $T$ とする。

\paragraph{問1:終状態の温度 $T$}

\paragraph{使用する物理法則}
孤立系では外界と仕事・熱のやりとりがないため、\textbf{熱力学第一法則}より全内部エネルギー $U_{\mathrm{total}}$ は一定である。理想気体では $U = c N R T$ であり、$U$ は温度 $T$ のみに依存し体積には依存しない。

\paragraph{計算のステップ}
内部エネルギー保存則を用いる。気体の移動はないので左室のモル数は $N_A$、右室は $N_B$ のまま。左室の内部エネルギーは $U_A = c N_A R T$、右室は $U_B = c N_B R T$ であり、理想気体では $U$ は $T$ のみに依存する。

\textbf{ステップ1:}初期状態の全内部エネルギーを書く。
\begin{equation}
U_{\mathrm{initial}} = c N_A R T_A + c N_B R T_B = cR(N_A T_A + N_B T_B).
\end{equation}

\textbf{ステップ2:}終状態では両室とも温度 $T$ なので
\begin{equation}
U_{\mathrm{final}} = c N_A R T + c N_B R T = c R (N_A + N_B) T.
\end{equation}

\textbf{ステップ3:}孤立系なので $U_{\mathrm{initial}} = U_{\mathrm{final}}$ より
\begin{equation}
c R (N_A + N_B) T = c R (N_A T_A + N_B T_B).
\end{equation}
$cR \neq 0$ で割り、
\begin{equation}
\boxed{T = \frac{N_A T_A + N_B T_B}{N_A + N_B} = \frac{T_A + 2T_B}{3}}.
\end{equation}

\paragraph{なぜこのように求まるか(原理的な説明)}

透熱壁に替えると、左室(高温 $T_A$)と右室(低温 $T_B$)の間で熱が流れる。熱力学第零法則により、十分時間が経つと両側の温度は等しくなる。その最終温度を決めるのが\textbf{内部エネルギー保存則}である。孤立系では外界との仕事・熱のやりとりがないため、全内部エネルギー $U_{\mathrm{total}}$ は一定である。理想気体では $U = c N R T$ で $U$ は $T$ のみに依存するので、「左室の $U$ の減少量」と「右室の $U$ の増加量」が等しくなるように熱が左から右に流れ、結果として $T$ が $T_A$ と $T_B$ のモル数で重みづけた平均(加重平均)になる。$T_A > T_B$ のとき、高温側から低温側へ熱が流れるため、必ず $T_B < T < T_A$ が成り立つ。

\begin{figure}[H]
\centering
\includegraphics[width=0.8\textwidth]{figures/past2023_ex1_energy_flow.png}
\caption{問1の物理:透熱壁を通して熱 $Q$ が高温側から低温側へ流れ、両室の温度が $T$ で一致する。内部エネルギー保存により $T$ が一意に決まる。}
\label{fig:past2023_ex1_energy_flow}
\end{figure}

\paragraph{問2:エントロピー変化}

初期状態のエントロピーは、左室が $S_A^{\mathrm{i}} = N_A R \ln(T_A^c V_A / N_A) + N_A S_0$、右室が $S_B^{\mathrm{i}} = N_B R \ln(T_B^c V_B / N_B) + N_B S_0$ なので
\begin{equation}
S_{\mathrm{initial}} = N_A R \ln\frac{T_A^c V_A}{N_A} + N_A S_0 + N_B R \ln\frac{T_B^c V_B}{N_B} + N_B S_0.
\end{equation}
終状態では両室とも温度 $T$、体積は $V_A$, $V_B$ のままなので
\begin{equation}
S_{\mathrm{final}} = N_A R \ln\frac{T^c V_A}{N_A} + N_A S_0 + N_B R \ln\frac{T^c V_B}{N_B} + N_B S_0.
\end{equation}
エントロピー変化は
\begin{align}
\Delta S &= S_{\mathrm{final}} - S_{\mathrm{initial}} \notag \\
&= N_A R \ln\frac{T^c}{T_A^c} + N_B R \ln\frac{T^c}{T_B^c}
= N_A R \ln\frac{T}{T_A} + N_B R \ln\frac{T}{T_B}.
\end{align}
したがって
\begin{equation}
\boxed{\Delta S = N_A R \ln\frac{T}{T_A} + N_B R \ln\frac{T}{T_B}
= R \ln\frac{T}{T_A} + 2R \ln\frac{T}{T_B}
= R \ln\frac{T^3}{T_A T_B^2}}.
\end{equation}

\paragraph{なぜこの式になるか(物理的考察)}

エントロピー $S = NR\ln(T^c V/N) + NS_0$ では、温度が $T_A \to T$ または $T_B \to T$ に変わることで $S$ が変化する。左室では $T_A > T$ なので $T/T_A < 1$、$\ln(T/T_A) < 0$ であり、高温だった左室のエントロピーは\textbf{減少}する(熱を失うため)。右室では $T_B < T$ なので $T/T_B > 1$、$\ln(T/T_B) > 0$ であり、低温だった右室のエントロピーは\textbf{増加}する(熱を得るため)。熱力学第二法則から、不可逆過程(熱が高温から低温へ一方向に流れる)では全エントロピーは増大するので、右室の増加量が左室の減少量を上回り、$\Delta S > 0$ となる。式の形 $\Delta S = N_A R \ln(T/T_A) + N_B R \ln(T/T_B)$ は、各室の「温度比の対数」にモル数と $R$ をかけた寄与の和であり、$T$ が問1で求めた平衡温度であることから一意に定まる。

\paragraph{問3:$\Delta S > 0$ の証明}

\paragraph{証明の流れ(初学者向け)}
問2で得た $\Delta S = N_A R \ln(T/T_A) + N_B R \ln(T/T_B)$ が正であることを示す。方法は二通りある。(1) 数学的な方法:対数について成り立つ不等式 $\ln t \le t-1$($t>0$、等号は $t=1$ のみ)を用いて $\Delta S/R$ を上からおさえ、等号が $T_A=T_B$ のときのみであることから $\Delta S > 0$ を導く。(2) 物理的な方法:熱が高温から低温へ流れる過程は不可逆であり、熱力学第二法則から孤立系のエントロピーは増大するので $\Delta S > 0$ である。ここでは(1)の計算も示す。

\textbf{用いる不等式}:実数 $t > 0$ に対して $\ln t \le t - 1$ が成り立ち、等号は $t = 1$ のときのみである($y = \ln t$ と $y = t-1$ は $t=1$ で接する)。この不等式は、対数関数の性質から導かれる。

$T$ は $T_A$ と $T_B$ の加重平均なので、$T_B < T < T_A$ である。$x = T/T_A$, $y = T/T_B$ とおくと $x < 1$, $y > 1$ である。$\Delta S / R = \ln(T/T_A) + 2\ln(T/T_B) = \ln x + 2\ln y$。ここで $T = (T_A + 2T_B)/3$ より
\begin{equation}
\frac{T}{T_A} = \frac{1}{3}\left(1 + 2\frac{T_B}{T_A}\right), \quad
\frac{T}{T_B} = \frac{1}{3}\left(\frac{T_A}{T_B} + 2\right).
\end{equation}
不等式 $\ln t \le t-1$($t>0$、等号は $t=1$ のみ)を用いる。$t = T_A/T$ とすると $T_A > T$ より $t > 1$ なので $\ln(T_A/T) \ge (T_A/T)-1$、すなわち $\ln(T/T_A) \le (T/T_A)-1$。同様に $t = T/T_B$ とすると $t > 1$ なので $\ln(T/T_B) \le (T/T_B)-1$。よって
\begin{align}
\frac{\Delta S}{R} &= \ln\frac{T}{T_A} + 2\ln\frac{T}{T_B}
\le \left(\frac{T}{T_A}-1\right) + 2\left(\frac{T}{T_B}-1\right) \notag \\
&= \frac{T}{T_A} + \frac{2T}{T_B} - 3.
\end{align}
$T = (N_A T_A + N_B T_B)/(N_A+N_B)$ を代入した右辺 $T/T_A + 2T/T_B - 3$ は、$T_A = T_B$ のとき $T = T_A$ なので $1 + 2 - 3 = 0$ となり、$T_A \neq T_B$ のときは正になる($T$ が $T_A$ と $T_B$ の加重平均であることから、右辺は $(T_A-T_B)^2$ に比例する正の量になる)。また、不等式 $\ln t \le t-1$ の等号は $t=1$ のときのみなので、$T_A \neq T_B$ では $\ln(T/T_A)$ と $\ln(T/T_B)$ の少なくとも一方で等号が成り立たず、$\Delta S/R$ は右辺より真に小さい。いずれにせよ、熱が高温から低温へ流れる過程は不可逆であり、熱力学第二法則から孤立系のエントロピーは増大するので $\Delta S > 0$ である。

\paragraph{物理的意味}

熱が高温 $T_A$ から低温 $T_B$ へ流れる過程は\textbf{不可逆}である。自然に逆方向(低温から高温へ熱が流れる)には戻らない。熱力学第二法則は「孤立系のエントロピーは減少しない」であり、この過程で $\Delta S > 0$ となることは、その数学的な反映である。不等式 $\ln t \le t-1$ を用いた証明は、対数関数の性質から $\Delta S$ の正性を代数的に導いている。

\paragraph{問4:達成できる最低温度 $T_{\mathrm{min}}$}

体積 $V_A$, $V_B$ は固定、気体の移動なし、終状態で両室の温度を $T$ に揃える。可逆操作のみ許すとき、全エントロピーは減少しない。最低温度を達成するのは、エントロピーを一定に保つ可逆過程で、終状態の温度をできるだけ低くした場合である。

初期エントロピーは
\begin{equation}
S_{\mathrm{i}} = N_A R \ln\frac{T_A^c V_A}{N_A} + N_A S_0 + N_B R \ln\frac{T_B^c V_B}{N_B} + N_B S_0.
\end{equation}
終状態の温度を $T$ としたときのエントロピーは
\begin{equation}
S_{\mathrm{f}}(T) = N_A R \ln\frac{T^c V_A}{N_A} + N_A S_0 + N_B R \ln\frac{T^c V_B}{N_B} + N_B S_0.
\end{equation}
$S_{\mathrm{f}}(T) \ge S_{\mathrm{i}}$ でなければならない。$S_{\mathrm{f}}(T) - S_{\mathrm{i}} = N_A R \ln(T/T_A) + N_B R \ln(T/T_B) \ge 0$ より
\begin{equation}
\left(\frac{T}{T_A}\right)^{N_A} \left(\frac{T}{T_B}\right)^{N_B} \ge 1
\quad \Rightarrow \quad
T^{N_A+N_B} \ge T_A^{N_A} T_B^{N_B}.
\end{equation}
したがって $T \ge (T_A^{N_A} T_B^{N_B})^{1/(N_A+N_B)}$。等号は可逆過程で達成されるので、
\begin{equation}
\boxed{T_{\mathrm{min}} = T_A^{N_A/(N_A+N_B)} T_B^{N_B/(N_A+N_B)}
= T_A^{1/3} T_B^{2/3}}.
\end{equation}

\paragraph{なぜ幾何平均が最低温度か(原理的な説明)}

問1の操作(透熱壁に替えるだけ)では、終状態の温度は\textbf{算術平均の重みづけ} $T = (N_A T_A + N_B T_B)/(N_A+N_B)$ で決まった。一方、仕切り壁を動かしたり透熱・断熱を切り替えたりする\textbf{可逆操作}を許すと、エントロピーを増やさない範囲で終状態の温度を変えられる。可逆過程では $S_{\mathrm{f}} = S_{\mathrm{i}}$ とできるので、$S_{\mathrm{f}}(T) \ge S_{\mathrm{i}}$ の等号が成り立つような最小の $T$ が $T_{\mathrm{min}}$ である。条件 $(T/T_A)^{N_A}(T/T_B)^{N_B} = 1$ を解くと $T$ は $T_A$ と $T_B$ の\textbf{重みづけ幾何平均}になる。幾何平均は算術平均より小さくなる($T_A \neq T_B$ のとき)ので、$T_{\mathrm{min}} < T_{\mathrm{問1}}$ であり、可逆操作を駆使すれば問1より低い終温度を実現できる。

$T_{\mathrm{min}}$ を達成する操作の例:左室を断熱壁で囲い準静的断熱膨張させて冷却し、右室を準静的断熱圧縮して加熱するなど、両室のエントロピーを変えずに可逆的に温度を揃える。一方を断熱膨張・他方を断熱圧縮する組み合わせで、$S_{\mathrm{f}} = S_{\mathrm{i}}$ を満たす終温度 $T_{\mathrm{min}}$ を実現する。逆に、可逆操作で最高温度を目指すと算術平均より高い温度も実現可能である。

\begin{figure}[H]
\centering
\includegraphics[width=0.75\textwidth]{figures/past2023_ex1_Tmin.png}
\caption{問4:終状態の温度 $T$ の取り得る範囲。$T_{\mathrm{min}}$(幾何平均)は問1の $T$(算術平均)より低い。}
\label{fig:past2023_ex1_Tmin}
\end{figure}

\begin{figure}[H]
\centering
\includegraphics[width=0.75\textwidth]{figures/past2023_ex1_setup.png}
\caption{問題Iの設定:左右に仕切られた容器。左は 1 モル・$T_A,V_A$、右は 2 モル・$T_B,V_B$。}
\label{fig:past2023_ex1_setup}
\end{figure}

%----------------------------------------------------------------------
\section{問題II:変な気体(類題:2024・2025年度 問題III;演習6-I, 演習4-III)}
%----------------------------------------------------------------------

\subsection{問題}

体積 $V$ の中に、ある気体が入っている。この系のエントロピーが
\begin{equation}
S(T,V) = \sigma T^3 V
\end{equation}
で与えられるとする($\sigma$ は正の定数)。以下の問に答えよ。

\begin{enumerate}
\item この系の内部エネルギー $E(T,V)$ を、$\sigma$, $T$, $V$ を用いて表わせ。ただし $E(T=0,V)=0$ とする。
\item この系の圧力 $p(T,V)$ を、$\sigma$, $T$, $V$ を用いて表わせ。
\item この気体が温度 $T$ の熱源と接して熱平衡にあり、体積 $V$ は一定とする。このときの揺らぎの分散 $\langle \delta E^2 \rangle$ を、ボルツマン定数 $k_B$、$\sigma$、$T$、$V$ を用いて表わせ。ここで $\delta E = E - \langle E \rangle$ である。
\end{enumerate}

ヒント:定積比熱は $C_V = (\partial E/\partial T)_V = T(\partial S/\partial T)_V$ とも書ける。$S$ を $T,V$ の関数とみなすと $T\,dS = C_V\,dT + \bigl((\partial E/\partial V)_T + p\bigr)\,dV$ と書ける。

\subsection{解答}

\paragraph{この問題のポイント(初学者向け)}

エントロピーが $S(T,V) = \sigma T^3 V$ という「変な」形で与えられた気体を扱う。問1は熱力学の関係 $C_V = T(\partial S/\partial T)_V$ から定積比熱を求め、$E(T=0,V)=0$ として $E$ を積分で求める。問2は $T\,dS = dE + p\,dV$ から圧力 $p$ を求める。問3は統計力学の結果「熱源と接触した系のエネルギー揺らぎは $\langle \delta E^2 \rangle = k_B T^2 C_V$」を用いる。

\paragraph{解き方の流れ}

\begin{enumerate}
\item 問1:$C_V = T(\partial S/\partial T)_V$ で $C_V$ を求め、$E = \int_0^T C_V\,dT'$ で $E(T,V)$ を求める。
\item 問2:$T(\partial S/\partial V)_T = (\partial E/\partial V)_T + p$ から $p$ を求める。
\item 問3:$\langle \delta E^2 \rangle = k_B T^2 C_V$ に $C_V$ を代入する。
\end{enumerate}

\paragraph{問1:内部エネルギー $E(T,V)$}

熱力学の基本関係式 $T\,dS = dE + p\,dV$ より、体積一定では $dS = (1/T)\,dE$、すなわち $(\partial S/\partial T)_V = (1/T)(\partial E/\partial T)_V$。したがって
\begin{equation}
C_V = \left(\frac{\partial E}{\partial T}\right)_V = T\left(\frac{\partial S}{\partial T}\right)_V.
\end{equation}
$S = \sigma T^3 V$ より $(\partial S/\partial T)_V = 3\sigma T^2 V$ なので
\begin{equation}
C_V = T \cdot 3\sigma T^2 V = 3\sigma T^3 V.
\end{equation}
$E(T=0,V)=0$ として、$E(T,V) = \int_0^T C_V(T',V)\,dT'$ を計算する:
\begin{equation}
E(T,V) = \int_0^T 3\sigma (T')^3 V\,dT' = 3\sigma V \left[\frac{(T')^4}{4}\right]_0^T = \frac{3}{4}\sigma T^4 V.
\end{equation}
よって
\begin{equation}
\boxed{E(T,V) = \frac{3}{4}\sigma T^4 V}.
\end{equation}

\paragraph{なぜこのように求まるか(原理的な説明)}

熱力学の基本関係式 $T\,dS = dE + p\,dV$ より、体積一定では $dE = T\,dS$ である。したがって $(\partial E/\partial T)_V = T(\partial S/\partial T)_V = C_V$ が成り立つ。つまり\textbf{定積比熱 $C_V$ は、温度 $T$ でエントロピーの温度微分 $(\partial S/\partial T)_V$ をかけたもの}である。$S = \sigma T^3 V$ を $T$ で微分すると $3\sigma T^2 V$ なので、$C_V = 3\sigma T^3 V$ となり、$E(T,V)$ は $T$ について $0$ から $T$ まで積分して $E = (3/4)\sigma T^4 V$ を得る。$E(T=0,V)=0$ としたのは、絶対零度では内部エネルギーを 0 に取る慣習に従うためである。この「変な気体」では $E \propto T^4 V$ であり、理想気体の $E \propto T$ とは異なる。放射場(黒体放射)のエネルギー密度が $T^4$ に比例するのと類似した振る舞いである。

\paragraph{問2:圧力 $p(T,V)$}

熱力学の基本関係式 $T\,dS = dE + p\,dV$ で、$T$ を一定にすると $dT=0$ なので $dS = (\partial S/\partial V)_T\,dV$、$dE = (\partial E/\partial V)_T\,dV$ である。よって $T(\partial S/\partial V)_T\,dV = (\partial E/\partial V)_T\,dV + p\,dV$ が任意の $dV$ で成り立つので、
\begin{equation}
T\left(\frac{\partial S}{\partial V}\right)_T = \left(\frac{\partial E}{\partial V}\right)_T + p.
\end{equation}
$S = \sigma T^3 V$ より $(\partial S/\partial V)_T = \sigma T^3$。$E = (3/4)\sigma T^4 V$ より $(\partial E/\partial V)_T = (3/4)\sigma T^4$。したがって
\begin{equation}
T \cdot \sigma T^3 = \frac{3}{4}\sigma T^4 + p
\quad \Rightarrow \quad
\sigma T^4 = \frac{3}{4}\sigma T^4 + p
\quad \Rightarrow \quad
p = \sigma T^4 - \frac{3}{4}\sigma T^4 = \frac{1}{4}\sigma T^4.
\end{equation}
よって
\begin{equation}
\boxed{p(T,V) = \frac{1}{4}\sigma T^4}.
\end{equation}
($p$ は $V$ に依存しない。この「変な気体」の状態方程式に相当する。)

\paragraph{なぜ圧力が $V$ に依存しないか(物理的考察)}

$T\,dS = dE + p\,dV$ で $T$ 一定とすると $T(\partial S/\partial V)_T = (\partial E/\partial V)_T + p$ である。$S = \sigma T^3 V$ より $(\partial S/\partial V)_T = \sigma T^3$、$E = (3/4)\sigma T^4 V$ より $(\partial E/\partial V)_T = (3/4)\sigma T^4$。代入すると $p = \sigma T^4 - (3/4)\sigma T^4 = (1/4)\sigma T^4$ となり、$p$ は $T$ のみの関数で $V$ に依存しない。理想気体では $p \propto T/V$ だったが、この「変な気体」では $p \propto T^4$ であり、体積を変えても圧力は変わらない。黒体放射の圧力 $p = u/3$($u$ はエネルギー密度)と同様の $T^4$ 依存性である。

\begin{figure}[H]
\centering
\includegraphics[width=0.8\textwidth]{figures/past2023_ex2_E_p.png}
\caption{問題II:変な気体の $E(T,V)$ と $p(T,V)$。$E \propto T^4 V$、$p \propto T^4$($V$ に依存しない)。}
\label{fig:past2023_ex2_E_p}
\end{figure}

\paragraph{問3:揺らぎの分散 $\langle \delta E^2 \rangle$}

問題文の「この気体が温度 $T$ の熱源と接して熱平衡にあり」という条件から、\textbf{カノニカル分布}が適用される。カノニカル分布では、系のエネルギー $E$ は熱源との熱のやりとりで揺らぎ、その分散は統計力学の一般公式 $\langle \delta E^2 \rangle = k_B T^2 C_V$ で与えられる($C_V$ は定積比熱。用語集・本試験問題III 問6の「なぜこの公式が使えるか」を参照)。ヒントの $C_V = T(\partial S/\partial T)_V$ は問1で用いた。問1より $C_V = 3\sigma T^3 V$ なので、これを代入して
\begin{equation}
\boxed{\langle \delta E^2 \rangle = k_B T^2 \cdot 3\sigma T^3 V = 3\sigma k_B T^5 V}.
\end{equation}

(導出の補足:ボルツマンの原理 $S = k_B \ln W(E)$ と $W(E) \propto e^{S/k_B}$ から、熱浴と合わせた孤立系のエネルギー保存と状態数の最大条件により、$P(E) \propto e^{S(E)/k_B} e^{-\beta E}$ の形になり、$\langle (\Delta E)^2 \rangle = k_B T^2 C_V$ が得られる。)

\paragraph{なぜ揺らぎが $k_B T^2 C_V$ か(物理的考察)}

熱源と接触した系では、ミクロには内部エネルギー $E$ は一定ではなく、熱の出入りによって揺らいでいる。カノニカル分布では、エネルギー $E$ を取る確率は $P(E) \propto W(E)\,e^{-\beta E}$ で与えられ、その分散は $\langle \delta E^2 \rangle = k_B T^2 C_V$ となる。$C_V$ が大きいほど「熱容量が大きい」=温度を少し変えるのに多くの熱が必要であり、その分だけエネルギーの揺らぎも大きくなる。$T^2$ が効くのは、高温になるほど熱のやりとりが活発になり揺らぎが増すためである。この「変な気体」では $C_V = 3\sigma T^3 V$ なので、$\langle \delta E^2 \rangle = 3\sigma k_B T^5 V$ となり、$T^5$ に比例する。

%----------------------------------------------------------------------
\section{問題III:N個の独立な調和振動子(類題:2024・2025年度 問題IV;演習7-III)}
%----------------------------------------------------------------------

\subsection{問題}

$N$ 個の独立な調和振動子を量子的に扱う。$i$ 番目の振動子のエネルギーは $E_i = \hbar\omega m_i$($m_i = 0,1,2,\ldots$)とし、$\omega$ は振動子の角振動数である。この系が温度 $T$ の大きな熱源と接しているとき、系が量子状態 $i$ にある確率は分配関数 $Z_N$ を用いて $P(i) = e^{-\beta E_i}/Z_N$ で与えられる。ここで $\beta = 1/(k_B T)$ である。状態 $i$ は量子数の組 $(m_1,\ldots,m_N)$ で指定され、全エネルギーは $E_i = \hbar\omega(m_1+\cdots+m_N) \equiv \hbar\omega M$、$M = m_1+\cdots+m_N$ とする。

\begin{enumerate}
\item $N=1$ のとき $Z_1$ を計算せよ。
\item $N=1$ のときエネルギーの平均値 $\langle E \rangle$ を計算せよ。
\item $Z_N$ を計算せよ。
\item $Z_N$ を用いてエネルギーの平均値 $\langle E \rangle$ を計算せよ。
\item $Z_N = \sum_{M=0}^\infty W_N(M)\,e^{-\beta\hbar\omega M}$ とする。$W_N(M)$ を $N$ と $M$ で表せ。
\item 系がエネルギー $E$ を取る確率は $P(E) \propto W_N(M)\,e^{-\beta E}$ であり、$E^*$ で鋭いピークを持つ。$M \gg 1$ かつ $N \gg 1$ として、$E^*$ を $\beta$, $\hbar\omega$, $N$ などで表せ。必要ならスターリングの公式 $\ln N! \approx N\ln N - N$ を用いてよい。
\end{enumerate}

\subsection{解答}

\paragraph{この問題のポイント(初学者向け)}

$N$ 個の独立な量子調和振動子が温度 $T$ の熱源と接触している場合を扱う。\textbf{分配関数} $Z$ は「すべての状態についてボルツマン因子 $e^{-\beta E}$ を足し合わせたもの」であり、確率の規格化定数になる。平均エネルギーは $\langle E \rangle = -\partial\ln Z/\partial\beta$ で求まる。問5の $W_N(M)$ は「全エネルギーが $E = \hbar\omega M$ であるような状態の数」であり、重複組合せで与えられる。問6では、$P(E) \propto W_N(M)\,e^{-\beta E}$ が最大になるエネルギー $E^*$ を求め、それが $\langle E \rangle$ と一致することを確認する。

\paragraph{用語の説明(統計力学)}
「系が温度 $T$ の大きな熱源と接している」ので\textbf{カノニカル分布}である。状態 $i$ を取る確率は $P(i)=e^{-\beta E_i}/Z_N$($\beta=1/(k_B T)$)。\textbf{分配関数} $Z_N$ は全状態について $e^{-\beta E_i}$ を足したもので、\textbf{ボルツマン因子} $e^{-\beta E}$ がエネルギーが低い状態ほど重みを大きくする。用語集(本冊)も参照。

\paragraph{解き方の流れ}

\begin{enumerate}
\item 問1:1個の振動子の $Z_1 = \sum_{m=0}^\infty e^{-\beta\hbar\omega m}$ を等比級数で求める。
\item 問2:$\langle E \rangle = -\partial\ln Z_1/\partial\beta$ で $\langle E \rangle$ を求める($Z$ が求まれば微分するだけで平均エネルギーが得られる)。
\item 問3:独立な $N$ 個なので $Z_N = (Z_1)^N$(状態の和が積に分離するため)。
\item 問4:$\langle E \rangle = -\partial\ln Z_N/\partial\beta = N \times$(1個の平均エネルギー)。
\item 問5:$M = m_1+\cdots+m_N$ を満たす非負整数の組の数は重複組合せ $\binom{M+N-1}{N-1}$。
\item 問6:$\ln P(E)$ を $M$ で微分して 0 とおき、$M \gg 1$, $N \gg 1$ でスターリングの公式を使って $E^*$ を求める。
\end{enumerate}

\paragraph{問1:$Z_1$}

1 個の振動子では $m_1 = 0,1,2,\ldots$ に対し $E = \hbar\omega m_1$ なので
\begin{equation}
Z_1 = \sum_{m_1=0}^\infty e^{-\beta\hbar\omega m_1}
= \sum_{m=0}^\infty (e^{-\beta\hbar\omega})^m.
\end{equation}
$|e^{-\beta\hbar\omega}|<1$ のとき等比級数として
\begin{equation}
\boxed{Z_1 = \frac{1}{1 - e^{-\beta\hbar\omega}}}.
\end{equation}

\paragraph{なぜこの形になるか(原理的な説明)}

量子調和振動子のエネルギーは $E_m = \hbar\omega m$($m = 0,1,2,\ldots$)と離散的である。カノニカル分布では、状態 $m$ を取る確率は $P(m) \propto e^{-\beta E_m} = e^{-\beta\hbar\omega m}$ である。分配関数 $Z_1 = \sum_{m=0}^\infty e^{-\beta\hbar\omega m}$ は、$x = e^{-\beta\hbar\omega}$ とおくと $1 + x + x^2 + \cdots$ という等比級数になり、$|x|<1$($\beta\hbar\omega>0$)のとき和は $1/(1-x) = 1/(1-e^{-\beta\hbar\omega})$ である。つまり、\textbf{すべてのエネルギー固有状態についてボルツマン因子 $e^{-\beta E}$ を足し合わせたもの}が分配関数であり、確率の規格化定数になっている。

\paragraph{問2:$N=1$ のときの $\langle E \rangle$}

$\langle E \rangle = -\frac{\partial}{\partial\beta}\ln Z_1$ を用いる。
\begin{equation}
\ln Z_1 = -\ln(1 - e^{-\beta\hbar\omega})
\quad \Rightarrow \quad
\frac{\partial \ln Z_1}{\partial \beta} = -\frac{(-\hbar\omega)\,e^{-\beta\hbar\omega}}{1-e^{-\beta\hbar\omega}}
= \frac{\hbar\omega\,e^{-\beta\hbar\omega}}{1-e^{-\beta\hbar\omega}}.
\end{equation}
したがって
\begin{equation}
\langle E \rangle = -\frac{\partial \ln Z_1}{\partial \beta}
= \frac{\hbar\omega}{e^{\beta\hbar\omega}-1}.
\end{equation}
よって
\begin{equation}
\boxed{\langle E \rangle = \frac{\hbar\omega}{e^{\beta\hbar\omega}-1}}.
\end{equation}

\paragraph{なぜ $\langle E \rangle = -\partial\ln Z/\partial\beta$ か(原理的な説明)}

カノニカル分布では、エネルギーの平均は $\langle E \rangle = \sum_i E_i P(i) = (1/Z)\sum_i E_i e^{-\beta E_i}$ で与えられる。一方、$\ln Z = \ln\sum_i e^{-\beta E_i}$ を $\beta$ で微分すると $\frac{\partial \ln Z}{\partial \beta} = -\frac{1}{Z}\sum_i E_i e^{-\beta E_i} = -\langle E \rangle$ となる。したがって \textbf{$\langle E \rangle = -\frac{\partial}{\partial\beta}\ln Z$} が成り立つ。これは分配関数 $Z$ さえ求まれば、微分するだけで平均エネルギーが得られることを意味する。$T \to 0$($\beta \to \infty$)のとき $\langle E \rangle \to 0$(基底状態)、$T \to \infty$($\beta \to 0$)のとき $\langle E \rangle \sim k_B T$(等分配則に近づく)となる。

\paragraph{問3:$Z_N$}

各振動子が独立なので、分配関数は積になる:
\begin{equation}
Z_N = (Z_1)^N = \boxed{\left(\frac{1}{1-e^{-\beta\hbar\omega}}\right)^N}.
\end{equation}

\paragraph{問4:$Z_N$ を用いた $\langle E \rangle$}

$\langle E \rangle = -\frac{\partial}{\partial\beta}\ln Z_N = -N \frac{\partial}{\partial\beta}\ln Z_1$ より、問2の結果を $N$ 倍して
\begin{equation}
\boxed{\langle E \rangle = N\,\frac{\hbar\omega}{e^{\beta\hbar\omega}-1}}.
\end{equation}

\paragraph{問5:状態数 $W_N(M)$}

$M = m_1 + m_2 + \cdots + m_N$ を満たす非負整数の組 $(m_1,\ldots,m_N)$ の個数を数える。$M$ 個のボールを $N$ 個の箱に分ける重複組合せに等しい(各 $m_i$ が箱 $i$ のボール数)。その数は
\begin{equation}
\binom{M+N-1}{N-1} = \frac{(M+N-1)!}{(N-1)!\,M!}.
\end{equation}
よって
\begin{equation}
\boxed{W_N(M) = \frac{(M+N-1)!}{(N-1)!\,M!}}.
\end{equation}

\paragraph{なぜこの組み合わせになるか(物理的考察)}

$M = m_1 + m_2 + \cdots + m_N$ を満たす非負整数の組 $(m_1,\ldots,m_N)$ の個数は、「$M$ 個の区別できないボールを $N$ 個の区別できる箱に分ける方法の数」に等しい。これは\textbf{重複組合せ}であり、$N-1$ 個の仕切り「$|$」と $M$ 個のボール「$\circ$」を一列に並べる並べ方の数 $\binom{M+N-1}{N-1}$ で与えられる。例えば $\circ\circ|\circ||\circ$ は $m_1=2$, $m_2=1$, $m_3=0$, $m_4=1$ に対応する(左から箱1,2,...,N に入るボール数)。各振動子が独立に $m_i = 0,1,2,\ldots$ を取れるので、全エネルギー $E = \hbar\omega M$ に対する状態数が $W_N(M)$ であり、$M$ が大きいほど多くの組み合わせがある(エントロピー $S = k_B \ln W_N(M)$ が $M$ について増加する)。

\begin{figure}[H]
\centering
\includegraphics[width=0.85\textwidth]{figures/past2023_ex3_balls_bins.png}
\caption{問題III 問5:$W_N(M)$ の重複組合せのイメージ。$M$ 個のボールと $N-1$ 個の仕切りを一列に並べ、左から箱1,2,\ldots,N に入るボール数が $(m_1,\ldots,m_N)$ になる。}
\label{fig:past2023_ex3_balls_bins}
\end{figure}

\paragraph{問6:最確エネルギー $E^*$}

$P(E) \propto W_N(M)\,e^{-\beta E}$($E = \hbar\omega M$)が最大になる $M$ を求める。$f(M) = \ln W_N(M) - \beta\hbar\omega M$ の極大で $\partial f/\partial M = 0$ とおく。スターリングの公式 $\ln n! \approx n\ln n - n$ を用いて
\begin{align}
\ln W_N(M) &\approx (M+N-1)\ln(M+N-1) - (M+N-1) - (N-1)\ln(N-1) + (N-1) - M\ln M + M \notag \\
&\approx (M+N)\ln(M+N) - (N-1)\ln(N-1) - M\ln M - N + 1 \quad (M,N\gg1).
\end{align}
$M$ で微分する。$\frac{\partial}{\partial M}(M+N-1)\ln(M+N-1) = \ln(M+N-1) + 1$、$\frac{\partial}{\partial M} M\ln M = \ln M + 1$ なので、$M,N\gg1$ のとき $(M+N-1)\approx M+N$ として
\begin{equation}
\frac{\partial}{\partial M}\ln W_N(M) \approx \ln(M+N) + 1 - (\ln M + 1) = \ln\frac{M+N}{M}.
\end{equation}
したがって $\frac{\partial f}{\partial M} = \ln\frac{M+N}{M} - \beta\hbar\omega = 0$ より
\begin{equation}
\frac{M+N}{M} = e^{\beta\hbar\omega}
\quad \Rightarrow \quad
1 + \frac{N}{M} = e^{\beta\hbar\omega}
\quad \Rightarrow \quad
M = \frac{N}{e^{\beta\hbar\omega}-1}.
\end{equation}
$E^* = \hbar\omega M^*$ なので
\begin{equation}
\boxed{E^* = \frac{N\hbar\omega}{e^{\beta\hbar\omega}-1}}.
\end{equation}
これは問4の $\langle E \rangle$ と一致する(熱力学極限でピークが平均に一致する)。

\paragraph{なぜ $E^*$ が平均 $\langle E \rangle$ と一致するか(物理的考察)}

系がエネルギー $E = \hbar\omega M$ を取る確率は $P(E) \propto W_N(M)\,e^{-\beta E}$ である。$W_N(M)$ は $M$ の増加とともに増え(状態数が増える)、$e^{-\beta E}$ は $E$ の増加とともに減る(ボルツマン因子)。その積が最大になる $E^*$ が最確エネルギーである。$N \gg 1$、$M \gg 1$ の熱力学極限では、$P(E)$ は $E^*$ のまわりに鋭いピークを持ち、相対的な揺らぎ $\sqrt{\langle \delta E^2 \rangle}/\langle E \rangle$ は $1/\sqrt{N}$ のオーダーで小さくなる。そのため、\textbf{最確値 $E^*$ と平均値 $\langle E \rangle$ が一致し}、マクロな観測ではどちらも同じ「平衡のエネルギー」を表す。

\begin{figure}[H]
\centering
\includegraphics[width=0.85\textwidth]{figures/past2023_ex3_W_P.png}
\caption{問題III:$P(E) \propto W_N(M)\,e^{-\beta E}$ は $E^*$ で鋭いピークを持つ。$N$ が大きいほどピークは鋭くなり、$E^* = \langle E \rangle$ に一致する。}
\label{fig:past2023_ex3_W_P}
\end{figure}

\begin{figure}[H]
\centering
\includegraphics[width=0.8\textwidth]{figures/past2023_oscillator_Z.png}
\caption{調和振動子の分配関数と平均エネルギー(概念図)。}
\label{fig:past2023_oscillator}
\end{figure}

%======================================================================
% 2024年度 統計物理1 本試験(2024年1月24日 10:30--12:00)
%======================================================================
\part{2024年度 本試験}
\setcounter{section}{0}

\section{問題I:断熱自由膨張とエントロピー変化(類題:演習5-II)}

\subsection{問題}

$N$ モルの理想気体のエントロピー $S(T,V)$ と内部エネルギー $U(T,V)$ は以下で与えられる($T$:温度、$V$:体積、$R$:気体定数、$c$, $S_0$:定数):
\begin{align}
S(T,V) &= NR \ln\left(\frac{T^c V}{N}\right) + N S_0, \\
U(T,V) &= c N R T.
\end{align}

\begin{enumerate}
\item 系が断熱自由膨張により状態 $(T,V)$ から $(T',V')$ へ変化したとする($V < V'$)。$T'$ を $T$, $V$, $V'$, $c$, $R$, $N$ のいずれかの記号を用いて表せ。
\item 問1の後、系を準静的断熱過程により $(T',V')$ から $(T'',V)$ ともとの体積 $V$ まで圧縮する。一連の操作
\[
(T,V) \xrightarrow{\text{断熱自由膨張}} (T',V')
\xrightarrow{\text{準静的断熱過程}} (T'',V)
\]
における、初期状態 $(T,V)$ と終状態 $(T'',V)$ のエントロピー変化を、$T$, $V$, $V'$, $c$, $R$, $N$ のいずれかの記号を用いて表せ。
\end{enumerate}

\subsection{解答}

\paragraph{この問題のポイント(初学者向け)}

断熱自由膨張では\textbf{外界と熱のやりとりも仕事のやりとりもない}ため、内部エネルギー $U$ は変わらない。理想気体では $U$ は $T$ のみに依存するので、$T' = T$ となる。一方、エントロピーは不可逆過程で増大するので、一連の操作の後には $\Delta S > 0$ となる。問2では「準静的断熱過程では $S$ 一定」を使い、$\Delta S$ を断熱自由膨張の部分だけで表す。

\paragraph{解き方の流れ}

\begin{enumerate}
\item 問1:断熱自由膨張では $Q=0$, $W=0$ なので $\Delta U = 0$。理想気体では $U = c N R T$ で $U$ は $V$ に依存しないので、$T' = T$。
\item 問2:準静的断熱過程では $S$ 一定なので、$\Delta S$ は初期 $(T,V)$ と終状態 $(T'',V)$ のエントロピー差。断熱過程の関係式で $T''$ を $T,V,V'$ で表し、$\Delta S$ に代入する。
\end{enumerate}

\paragraph{用語の説明}
\begin{itemize}
\item \textbf{断熱自由膨張}:外界と熱のやりとりがなく(断熱)、外から仕事もされない条件下で気体が膨張する過程。非準静的で不可逆。熱力学第一法則 $dU = \delta Q - \delta W$ より $Q=0$, $W=0$ なら $\Delta U=0$。
\item \textbf{準静的断熱過程}:断熱しながら無限にゆっくり変化させる過程。可逆とみなせる。可逆断熱過程ではエントロピーは一定($dS = \delta Q_{\mathrm{rev}}/T = 0$)。
\item \textbf{理想気体の断熱過程}:$T^c V = \mathrm{const}$ が成り立つ($c$ は $U = c N R T$ の係数)。これは $dU = -p\,dV$ と $p = NRT/V$ から導かれる。
\end{itemize}

\paragraph{問1:断熱自由膨張後の温度 $T'$}

断熱自由膨張では $Q=0$, $W=0$ なので $\Delta U = 0$。理想気体では $U = c N R T$ であり、$U$ は $T$ のみに依存するので、$U$ が変わらなければ $T$ も変わらない。したがって
\begin{equation}
\boxed{T' = T}.
\end{equation}

\paragraph{なぜ温度が変わらないか(原理的な説明)}

断熱自由膨張では、外界と熱のやりとりがなく($Q=0$)、また気体が外に仕事をしない($W=0$)。熱力学第一法則 $dU = \delta Q - \delta W$ より $\Delta U = 0$ である。理想気体では内部エネルギー $U$ は温度 $T$ のみの関数($U = c N R T$)であり、体積 $V$ には依存しない。したがって、体積が $V$ から $V'$ に増えても $U$ が一定なら $T$ も不変である。\textbf{理想気体の断熱自由膨張では温度は変化しない}。これは理想気体の仮定(分子間力なし)に基づく結果であり、実在気体ではジュールの実験でわかるようにわずかに温度が変化する場合がある。

\paragraph{問2:一連の操作のエントロピー変化}

初期状態のエントロピーは $S_{\mathrm{i}} = NR\ln(T^c V/N) + N S_0$。終状態 $(T'',V)$ のエントロピーは $S_{\mathrm{f}} = NR\ln((T'')^c V/N) + N S_0$。したがって
\begin{equation}
\Delta S = S_{\mathrm{f}} - S_{\mathrm{i}}
= NR \ln\frac{(T'')^c}{T^c}
= N R c \ln\frac{T''}{T}.
\end{equation}
問1より $T' = T$。準静的断熱過程 $(T',V') \to (T'',V)$ ではエントロピーが一定なので、\textbf{理想気体の断熱過程の関係式} $T^c V = \mathrm{const}$ を用いる。導出:断熱では $dU = -p\,dV$。$U = c N R T$ より $dU = c N R\,dT$、$p = NRT/V$ より $c N R\,dT = -(NRT/V)\,dV$。$T$, $V$ で分離して積分すると $c \ln T = -\ln V + \mathrm{const}$、すなわち $T^c V = \mathrm{const}$。したがって $(T')^c V' = (T'')^c V$。$T'=T$ なので $T^c V' = (T'')^c V$、よって
\begin{equation}
T'' = T \left(\frac{V'}{V}\right)^{1/c}.
\end{equation}
($c$ は $U = c N R T$ の係数であり、断熱指数 $\gamma = (c+1)/c$ を使うと $T V^{\gamma-1} = T V^{1/c}$ なので $T^c V = \mathrm{const}$ と $T V^{\gamma-1} = \mathrm{const}$ は同じ関係である。)

これを $\Delta S$ に代入する:
\begin{equation}
\Delta S = N R c \ln\frac{T''}{T}
= N R c \ln\left(\frac{V'}{V}\right)^{1/c}
= N R \ln\frac{V'}{V}.
\end{equation}
よって
\begin{equation}
\boxed{\Delta S = N R \ln\frac{V'}{V}}.
\end{equation}
($V' > V$ なので $\Delta S > 0$。断熱自由膨張は不可逆過程のため、エントロピーが増大する。)

\paragraph{なぜエントロピーが増大するか(物理的考察)}

一連の操作のうち、\textbf{断熱自由膨張} $(T,V) \to (T',V')$ は不可逆過程である。気体が急に膨張するため、途中の状態は平衡ではなく、同じ体積変化を準静的に行う経路と比べて「無秩序に」広がる。熱力学第二法則により、孤立系の不可逆過程ではエントロピーは増大する。準静的断熱過程 $(T',V') \to (T'',V)$ ではエントロピーは一定なので、\textbf{全体の $\Delta S$ は断熱自由膨張の段階で生じたエントロピー増加} $N R \ln(V'/V)$ に等しい。$V' > V$ なので $\ln(V'/V) > 0$ であり、気体がより広い体積に広がった分だけ「配置の無秩序さ」が増し、エントロピーが増えたと解釈できる。

\begin{figure}[H]
\centering
\includegraphics[width=0.85\textwidth]{figures/past2024_ex1_why_entropy.png}
\caption{問題I:断熱自由膨張で気体が $V \to V'$ に広がるとエントロピーが増大する。準静的断熱圧縮では $S$ 一定。}
\label{fig:past2024_ex1_why_entropy}
\end{figure}

\begin{figure}[H]
\centering
\includegraphics[width=0.85\textwidth]{figures/past2024_ex1_path.png}
\caption{問題Iの過程:$(T,V) \to (T',V')$(断熱自由膨張)、$(T',V') \to (T'',V)$(準静的断熱圧縮)。}
\label{fig:past2024_ex1_path}
\end{figure}

%----------------------------------------------------------------------
\section{問題II:左右に仕切られた容器内の理想気体(類題:演習6-IV--V)}
%----------------------------------------------------------------------

\subsection{問題}

体積 $V$ の断熱壁の容器が、中央で左右に仕切られた透熱壁で分かれている。透熱壁は左右に自由に動ける。系の温度を $T$ とする。左に $N_A$ モル、右に $N_B$ モルの単原子分子理想気体が入っており、左右の圧力は釣り合い、熱平衡にある。

$N_A$ モルと $N_B$ モルの2種類の(単原子)理想気体からなる混合系のエントロピーは
\begin{equation}
S(T,V,N_A,N_B) = N_A R \ln\frac{T^c V}{N_A} + N_A S_{0,A}
+ N_B R \ln\frac{T^c V}{N_B} + N_B S_{0,B}
\end{equation}
で与えられる($c$, $S_{0,A}$, $S_{0,B}$ は定数)。

\begin{enumerate}
\item 左右が同一種の気体の場合、左右を隔てる壁を静かに取り外した。十分時間が経った後の全エントロピー変化 $\Delta S$ を、$T$, $R$, $c$, $N_A$, $N_B$ で表せ。
\item 左右が異なる種類の気体の場合、壁にかかる圧力 $p$ を、$T$, $R$, $c$, $V$, $N_A$, $N_B$ で表せ。
\item 問2の状況で壁を静かに取り外した。十分時間が経った後のエントロピー変化 $\Delta S$ を、$V$ を用いずに表せ。
\item 問3で求めた全エントロピー変化が非負であることを示せ。
\item 問3の終状態における気体Aの化学ポテンシャル $\mu_A$ を導け。
\end{enumerate}

\subsection{解答}

\paragraph{この問題のポイント(初学者向け)}

問1は\textbf{同一種}の気体で壁を取り外す場合。左右とも同じ気体で、もともと温度・圧力が釣り合っているので、壁を取ってもマクロには同じ状態であり $\Delta S = 0$。問2〜5は\textbf{異なる種類}の気体(例:ネオンとアルゴン)。問2は壁にかかる圧力、問3は壁を取り外したときのエントロピー増加(混合のエントロピー)、問4はその非負性の証明、問5は化学ポテンシャルである。

\paragraph{解き方の流れ}

\begin{enumerate}
\item 問1:初期と終状態のエントロピーを $S(T,V,N)$ の式で書き、$V_A/V = N_A/(N_A+N_B)$ などを使って $\Delta S$ を計算すると 0 になる。
\item 問2:$p V = (N_A+N_B) R T$ より $p$ を求める。
\item 問3:各気体が体積 $V_A$, $V_B$ から $V$ に広がるので、$\Delta S = N_A R \ln(V/V_A) + N_B R \ln(V/V_B)$。$V/V_A$, $V/V_B$ を $N_A$, $N_B$ で表す。
\item 問4:$x_A \ln x_A + x_B \ln x_B \le 0$($x_A+x_B=1$)を用いる。
\item 問5:理想気体の化学ポテンシャル $\mu_A = \mu_A^0(T) + R T \ln(p_A/p^0)$ で、$p_A = x_A p$ とする。
\end{enumerate}

\paragraph{記号と設定}

左室:体積 $V_A$、モル数 $N_A$、圧力 $p$。右室:体積 $V_B$、モル数 $N_B$、圧力 $p$。透熱壁で温度 $T$ が共通。圧力釣り合い:$p V_A = N_A R T$、$p V_B = N_B R T$。全体の体積は $V = V_A + V_B$。

\paragraph{問1:同一種で壁を取り外したときの $\Delta S$}

初期:左室 $S_A^{\mathrm{i}} = N_A R \ln(T^c V_A/N_A) + N_A S_0$、右室 $S_B^{\mathrm{i}} = N_B R \ln(T^c V_B/N_B) + N_B S_0$(同一種なので $S_{0,A}=S_{0,B}=S_0$)。終状態:全体で体積 $V$、モル数 $N_A+N_B$、温度 $T$。混合後のエントロピーは、単一の理想気体として
\begin{equation}
S_{\mathrm{f}} = (N_A+N_B) R \ln\frac{T^c V}{N_A+N_B} + (N_A+N_B) S_0.
\end{equation}
初期の全エントロピーは
\begin{equation}
S_{\mathrm{i}} = N_A R \ln\frac{T^c V_A}{N_A} + N_A S_0
+ N_B R \ln\frac{T^c V_B}{N_B} + N_B S_0.
\end{equation}
変化は
\begin{align}
\Delta S &= (N_A+N_B) R \ln\frac{T^c V}{N_A+N_B}
- N_A R \ln\frac{T^c V_A}{N_A}
- N_B R \ln\frac{T^c V_B}{N_B} \notag \\
&= N_A R \ln\frac{V}{V_A}\frac{N_A}{N_A+N_B}
+ N_B R \ln\frac{V}{V_B}\frac{N_B}{N_A+N_B}.
\end{align}
同一種で温度 $T$・圧力 $p$ が共通なので、状態方程式 $p V_A = N_A R T$、$p V_B = N_B R T$、$V_A + V_B = V$ が成り立つ。$p V = (N_A+N_B) R T$ なので $V_A/V = (N_A R T/p)/((N_A+N_B) R T/p) = N_A/(N_A+N_B)$、すなわち $V_A = V N_A/(N_A+N_B)$。同様に $V_B = V N_B/(N_A+N_B)$。したがって
\begin{equation}
\frac{V}{V_A} = \frac{N_A+N_B}{N_A}, \quad
\frac{V}{V_B} = \frac{N_A+N_B}{N_B}.
\end{equation}
これらを $\Delta S$ の式に代入すると、
$\frac{V}{V_A}\frac{N_A}{N_A+N_B} = \frac{N_A+N_B}{N_A}\cdot\frac{N_A}{N_A+N_B} = 1$、同様に $\frac{V}{V_B}\frac{N_B}{N_A+N_B} = 1$。したがって
\begin{equation}
\Delta S = N_A R \ln 1 + N_B R \ln 1 = 0.
\end{equation}
同一種で圧力・温度が釣り合っているとき、壁を取り外してもマクロな状態は変わらず可逆である。よって
\begin{equation}
\boxed{\Delta S = 0}.
\end{equation}

\paragraph{なぜ同一種では $\Delta S = 0$ か(物理的考察)}

左右が同一種の気体で、透熱壁で温度 $T$ が共通、圧力も釣り合っているとき、左室の体積 $V_A$ と右室の体積 $V_B$ の比は $V_A : V_B = N_A : N_B$ である。壁を取り外すと、両方の気体が全体積 $V = V_A + V_B$ に広がるが、\textbf{同一種なので区別がつかず}、マクロには「$N_A+N_B$ モルの同一気体が体積 $V$ にある」という1つの平衡状態になる。初期状態も、圧力・温度が同じなので、実質同じマクロ状態を別の仕切り方で表現しているに過ぎない。したがって可逆的に壁を元に戻せ、$\Delta S = 0$ である。

\paragraph{問2:異なる種類のとき壁にかかる圧力 $p$}

左室の状態方程式:$p V_A = N_A R T$(単原子理想気体)。右室:$p V_B = N_B R T$。壁にかかる圧力は左右で等しく、釣り合いのとき左から $p$、右から $p$ なので、壁に働く正味の力は 0。壁にかかる圧力の「大きさ」は $p$ である。左室の体積を $V_A$ とすると $p V_A = N_A R T$、$V_A + V_B = V$、$p V_B = N_B R T$ より $p(V_A+V_B) = (N_A+N_B) R T$、よって $p = (N_A+N_B) R T / V$。したがって
\begin{equation}
\boxed{p = \frac{(N_A+N_B) R T}{V}}.
\end{equation}

\paragraph{問3:異なる種類で壁を取り外したときの $\Delta S$($V$ を用いずに)}

初期:左室 $N_A$ モル・体積 $V_A$・温度 $T$、右室 $N_B$ モル・体積 $V_B$・温度 $T$。$V_A + V_B = V$。終状態:混合気体、体積 $V$、温度 $T$、$N_A$ モルの気体Aと $N_B$ モルの気体Bが同じ体積 $V$ を占める。$\Delta S$ は $N_A R \ln(V/V_A) + N_B R \ln(V/V_B)$ となるが、問題では $V$ を用いずに表すので、$V_A$, $V_B$, $V$ を $p V_A = N_A R T$、$p V = (N_A+N_B) R T$ などで $N_A$, $N_B$, $T$, $R$ で表して $V$ を消去する。

初期エントロピー:
\begin{equation}
S_{\mathrm{i}} = N_A R \ln\frac{T^c V_A}{N_A} + N_A S_{0,A}
+ N_B R \ln\frac{T^c V_B}{N_B} + N_B S_{0,B}.
\end{equation}
終状態:気体Aは体積 $V_A$ から $V$ に、気体Bは $V_B$ から $V$ に広がる。混合系のエントロピーは(各成分が体積 $V$ を持つとして)
\begin{equation}
S_{\mathrm{f}} = N_A R \ln\frac{T^c V}{N_A} + N_A S_{0,A}
+ N_B R \ln\frac{T^c V}{N_B} + N_B S_{0,B}.
\end{equation}
よって
\begin{equation}
\Delta S = N_A R \ln\frac{V}{V_A} + N_B R \ln\frac{V}{V_B}.
\end{equation}
$p V_A = N_A R T$、$p V = (N_A+N_B) R T$ より $V/V_A = p V/(p V_A) = (N_A+N_B) R T / (N_A R T) = (N_A+N_B)/N_A$。同様に $V/V_B = (N_A+N_B)/N_B$。よって
\begin{equation}
\boxed{\Delta S = N_A R \ln\frac{N_A+N_B}{N_A} + N_B R \ln\frac{N_A+N_B}{N_B}}.
\end{equation}

\paragraph{問4:$\Delta S \ge 0$ の証明}

問3の結果 $\Delta S = N_A R \ln\frac{N_A+N_B}{N_A} + N_B R \ln\frac{N_A+N_B}{N_B}$ を、モル分率 $x_A = N_A/(N_A+N_B)$、$x_B = N_B/(N_A+N_B)$($x_A+x_B=1$)で書き直す。$\ln\frac{N_A+N_B}{N_A} = \ln(1/x_A) = -\ln x_A$、$\ln\frac{N_A+N_B}{N_B} = -\ln x_B$ なので
\begin{equation}
\Delta S = N_A R (-\ln x_A) + N_B R (-\ln x_B)
= -R (N_A \ln x_A + N_B \ln x_B).
\end{equation}
$N_A = (N_A+N_B) x_A$、$N_B = (N_A+N_B) x_B$ を代入すると $N_A \ln x_A + N_B \ln x_B = (N_A+N_B)(x_A \ln x_A + x_B \ln x_B)$ である。したがって
\begin{equation}
\frac{\Delta S}{R} = -(N_A+N_B)(x_A \ln x_A + x_B \ln x_B).
\end{equation}
$0 < x < 1$ のとき $\ln x < 0$ なので $x \ln x < 0$($x=1$ のときは $\ln 1 = 0$ なので $x\ln x = 0$)。したがって $x_A \ln x_A \le 0$、$x_B \ln x_B \le 0$(等号は $x_A=1$ または $x_B=1$ のときのみ)であり、$\Delta S \ge 0$ となる。等号は $x_A=1$ または $x_B=1$、すなわち一方の気体だけが存在するときである。

\paragraph{なぜ異種混合で $\Delta S > 0$ か(物理的考察)}

異なる種類の気体(例:ネオンとアルゴン)が壁を取り外して混合すると、\textbf{各気体が相手の領域にも広がる}。気体Aは体積 $V_A$ から $V$ に、気体Bは $V_B$ から $V$ に広がり、両方が同じ空間を占める。異種なので「どちらがどちらか」は区別でき、この「混合」は不可逆である(自然には分離しない)。配置の取り方が増えた分だけエントロピーが増大し、$\Delta S = N_A R \ln[(N_A+N_B)/N_A] + N_B R \ln[(N_A+N_B)/N_B] > 0$ となる。熱力学第二法則(孤立系のエントロピーは減少しない)の反映である。

\begin{figure}[H]
\centering
\includegraphics[width=0.8\textwidth]{figures/past2024_ex2_entropy_same_diff.png}
\caption{問題II:同一種で壁を取り外すと $\Delta S = 0$(可逆)。異種で混合すると $\Delta S > 0$(不可逆)。}
\label{fig:past2024_ex2_entropy}
\end{figure}

\paragraph{問5:終状態の気体Aの化学ポテンシャル $\mu_A$}

化学ポテンシャルは $\mu_A = (\partial G/\partial N_A)_{T,p,N_B}$ で定義される。理想気体では、ギブス自由エネルギー $G$ から $\mu_A = \mu_A^0(T) + R T \ln(p_A/p^0)$ が導かれる($\mu_A^0(T)$ は標準圧力 $p^0$ における化学ポテンシャル)。終状態では気体Aの分圧は $p_A = p \cdot N_A/(N_A+N_B) = x_A p$(ダルトンの分圧の法則)。$p = (N_A+N_B) R T / V$ なので $p_A = N_A R T / V$ でもある。標準状態を $p^0$ とすると
\begin{equation}
\mu_A = \mu_A^0(T) + R T \ln\frac{p_A}{p^0}
= \mu_A^0(T) + R T \ln\frac{N_A R T}{V p^0}.
\end{equation}
あるいは、モル分率 $x_A = N_A/(N_A+N_B)$ と全圧 $p$ を用いて $p_A = x_A p$ なので
\begin{equation}
\boxed{\mu_A = \mu_A^0(T) + R T \ln\frac{x_A p}{p^0}}.
\end{equation}

\paragraph{化学ポテンシャルの物理的意味}

化学ポテンシャル $\mu_A$ は、成分Aの粒子を1モル追加したときのギブス自由エネルギー $G$ の増分である。理想気体では $\mu_A = \mu_A^0(T) + R T \ln(p_A/p^0)$ となり、\textbf{分圧 $p_A$ が高いほど $\mu_A$ は大きい}。混合後の終状態では、気体Aは全圧 $p$ のうち分圧 $p_A = x_A p$ で存在する。$x_A < 1$ なので $p_A < p$ であり、純粋なAだけのときより $\mu_A$ は低くなる。これは「混合によってAの化学ポテンシャルが下がり、拡散が進む」という直感と一致する。

\begin{figure}[H]
\centering
\includegraphics[width=0.75\textwidth]{figures/past2024_ex2_setup.png}
\caption{問題IIの設定:透熱壁で仕切られた容器。左 $N_A$ モル、右 $N_B$ モル。}
\label{fig:past2024_ex2_setup}
\end{figure}

%----------------------------------------------------------------------
\section{問題III:変な気体(類題:2023年度 問題II;演習6-I, 演習4-III)}
%----------------------------------------------------------------------

\subsection{問題}

体積 $V$ の中に、ある気体が入っている。この系のエントロピーが $S(T,V) = \sigma T^3 V$ で与えられるとする($\sigma$ は正の定数)。以下の問に答えよ。

\begin{enumerate}
\item この系の内部エネルギー $E(T,V)$ を、$\sigma$, $T$, $V$ を用いて表わせ。ただし $E(T=0,V)=0$ とする。
\item この系の圧力 $p(T,V)$ を、$\sigma$, $T$, $V$ を用いて表わせ。
\item この気体が温度 $T$ の熱源と接して熱平衡にあり、体積 $V$ は一定とする。このときの揺らぎの分散 $\langle \delta E^2 \rangle$ を、ボルツマン定数 $k_B$、$\sigma$、$T$、$V$ を用いて表わせ。ここで $\delta E = E - \langle E \rangle$ である。
\end{enumerate}

ヒント:定積比熱は $C_V = (\partial E/\partial T)_V = T(\partial S/\partial T)_V$ とも書ける。$S$ を $T,V$ の関数とみなすと $T\,dS = C_V\,dT + \bigl((\partial E/\partial V)_T + p\bigr)\,dV$ と書ける。

\subsection{解答}

【類題】2023年度 問題II と同一内容です。演習問題解説では演習6-I(熱力学の関係式)、演習4-III(内部エネルギーの方程式)が関連します。

\paragraph{なぜこのように解くか(問1・問2)}
問1ではエントロピー $S(T,V)$ が与えられているので、熱力学の関係 $C_V = T(\partial S/\partial T)_V$(体積一定での定積比熱)から $C_V$ を求め、$E(T=0,V)=0$ のもとで $E = \int_0^T C_V\,dT'$ で内部エネルギーを求める。問2では $T\,dS = dE + p\,dV$ を $T$ 一定で $dV$ の係数比較し、$(\partial S/\partial V)_T$ と $(\partial E/\partial V)_T$ から $p$ を導く。

\paragraph{問1:$E(T,V)$}

ヒントの $C_V = T(\partial S/\partial T)_V$ を用いる。$S = \sigma T^3 V$ より $(\partial S/\partial T)_V = 3\sigma T^2 V$ なので $C_V = T \cdot 3\sigma T^2 V = 3\sigma T^3 V$。体積一定で $E(T=0,V)=0$ として $E(T,V) = \int_0^T C_V(T',V)\,dT'$ を計算すると
\begin{equation}
\boxed{E(T,V) = \frac{3}{4}\sigma T^4 V}.
\end{equation}

\paragraph{問2:$p(T,V)$}

熱力学の基本関係式 $T\,dS = dE + p\,dV$ で $T$ 一定とすると $T(\partial S/\partial V)_T = (\partial E/\partial V)_T + p$。$(\partial S/\partial V)_T = \sigma T^3$、$(\partial E/\partial V)_T = (3/4)\sigma T^4$ を代入して $T\cdot\sigma T^3 = (3/4)\sigma T^4 + p$、したがって
\begin{equation}
\boxed{p(T,V) = \frac{1}{4}\sigma T^4}.
\end{equation}
($p$ は $V$ に依存しない。)

\paragraph{問3:$\langle \delta E^2 \rangle$}

問題文の「温度 $T$ の熱源と接して熱平衡にあり」から\textbf{カノニカル分布}が適用される(用語集参照)。カノニカル分布では、系のエネルギーの分散は $\langle \delta E^2 \rangle = k_B T^2 C_V$ で与えられる($C_V$ は定積比熱。なぜこの公式でよいかは2023年度 問題II 問3・本試験問題III 問6を参照)。問1で $C_V = 3\sigma T^3 V$ を求めているので、これを代入して
\begin{equation}
\boxed{\langle \delta E^2 \rangle = 3\sigma k_B T^5 V}.
\end{equation}

詳細な導出・物理的考察は2023年度 問題II(本冊)を参照してください。図は図\ref{fig:past2023_ex2_E_p}(2023年度 問題II)を参照。

%----------------------------------------------------------------------
\section{問題IV:N個の独立な調和振動子(類題:2023年度 問題III;演習7-III)}
%----------------------------------------------------------------------

\subsection{問題}

$N$ 個の独立な調和振動子を量子的に扱う。$i$ 番目の振動子のエネルギーは $E_i = \hbar\omega m_i$($m_i = 0,1,2,\ldots$)とし、$\omega$ は振動子の角振動数である。この系が温度 $T$ の大きな熱源と接しているとき、系が量子状態 $i$ にある確率は分配関数 $Z_N$ を用いて $P(i) = e^{-\beta E_i}/Z_N$ で与えられる。ここで $\beta = 1/(k_B T)$ である。状態 $i$ は量子数の組 $(m_1,\ldots,m_N)$ で指定され、全エネルギーは $E_i = \hbar\omega(m_1+\cdots+m_N) \equiv \hbar\omega M$、$M = m_1+\cdots+m_N$ とする。

\begin{enumerate}
\item $N=1$ のとき $Z_1$ を計算せよ。
\item $N=1$ のときエネルギーの平均値 $\langle E \rangle$ を計算せよ。
\item $Z_N$ を計算せよ。
\item $Z_N$ を用いてエネルギーの平均値 $\langle E \rangle$ を計算せよ。
\item $Z_N = \sum_{M=0}^\infty W_N(M)\,e^{-\beta\hbar\omega M}$ とする。$W_N(M)$ を $N$ と $M$ で表せ。
\item 系がエネルギー $E$ を取る確率は $P(E) \propto W_N(M)\,e^{-\beta E}$ であり、$E^*$ で鋭いピークを持つ。$M \gg 1$ かつ $N \gg 1$ として、$E^*$ を $\beta$, $\hbar\omega$, $N$ などで表せ。必要ならスターリングの公式 $\ln N! \approx N\ln N - N$ を用いてよい。
\end{enumerate}

\subsection{解答}

【類題】2023年度 問題III と同一内容です。演習問題解説では演習7-III(調和振動子の分配関数)が類題です。

\paragraph{問1:$Z_1$}

1個の振動子のエネルギーは $E_m = \hbar\omega m$($m = 0,1,2,\ldots$)。分配関数は $Z_1 = \sum_{m=0}^\infty e^{-\beta\hbar\omega m}$ で、$|e^{-\beta\hbar\omega}|<1$ のとき等比級数の和として
\begin{equation}
\boxed{Z_1 = \frac{1}{1 - e^{-\beta\hbar\omega}}}.
\end{equation}

\paragraph{問2:$N=1$ のときの $\langle E \rangle$}

カノニカル分布では平均エネルギーは $\langle E \rangle = -\partial\ln Z/\partial\beta$ で与えられる($\ln Z$ を $\beta$ で微分すると $-\langle E \rangle$ になるため。2023年度 問題III 問2の「なぜ」を参照)。$\ln Z_1 = -\ln(1-e^{-\beta\hbar\omega})$ を $\beta$ で微分して
\begin{equation}
\boxed{\langle E \rangle = \frac{\hbar\omega}{e^{\beta\hbar\omega}-1}}.
\end{equation}

\paragraph{問3:$Z_N$}

各振動子が独立なので、全状態の和は各振動子の分配関数の積になる:$Z_N = (Z_1)^N$。よって
\begin{equation}
\boxed{Z_N = \left(\frac{1}{1-e^{-\beta\hbar\omega}}\right)^N}.
\end{equation}

\paragraph{問4:$Z_N$ を用いた $\langle E \rangle$}

$\langle E \rangle = -\partial\ln Z_N/\partial\beta = -N\,\partial\ln Z_1/\partial\beta$ より、問2の結果を $N$ 倍して
\begin{equation}
\boxed{\langle E \rangle = N\,\frac{\hbar\omega}{e^{\beta\hbar\omega}-1}}.
\end{equation}

\paragraph{問5:$W_N(M)$}

$W_N(M)$ は「全エネルギーが $E = \hbar\omega M$ であるような状態の数」である。$M = m_1+\cdots+m_N$ を満たす非負整数の組 $(m_1,\ldots,m_N)$ の個数は、$M$ 個のボールを $N$ 個の箱に分ける重複組合せに等しく、
\begin{equation}
\boxed{W_N(M) = \frac{(M+N-1)!}{(N-1)!\,M!}}.
\end{equation}

\paragraph{問6:最確エネルギー $E^*$}

系がエネルギー $E = \hbar\omega M$ を取る確率は $P(E) \propto W_N(M)\,e^{-\beta E}$(状態数 $W_N(M)$ とボルツマン因子の積)なので、これが最大になる $M$ を求める。$P$ の最大は $\ln P$ の最大と同じだから、$\ln P$ を $M$ で微分して 0 とおく:$\partial[\ln W_N(M) - \beta\hbar\omega M]/\partial M = 0$。$M \gg 1$, $N \gg 1$ でスターリングの公式 $\ln n! \approx n\ln n - n$ を用いると $\partial\ln W_N(M)/\partial M \approx \ln((M+N)/M)$ となる(2023年度 問題III 問6の導出参照)。したがって $\ln\frac{M+N}{M} = \beta\hbar\omega$ より $1+N/M = e^{\beta\hbar\omega}$、$M = N/(e^{\beta\hbar\omega}-1)$。よって
\begin{equation}
\boxed{E^* = \frac{N\hbar\omega}{e^{\beta\hbar\omega}-1}}.
\end{equation}
(問4の $\langle E \rangle$ と一致する。$N \gg 1$ のとき $P(E)$ は $E^*$ で鋭いピークを持つ。)

詳細な導出・物理的考察は2023年度 問題III(本冊)を参照してください。図は図\ref{fig:past2023_oscillator}、図\ref{fig:past2023_ex3_W_P}(2023年度 問題III)を参照。

%======================================================================
% 2024年度 電磁気学 期末 再試験(2024年2月6日)
%======================================================================
\part{2024年度 再試験}
\setcounter{section}{0}
\renewcommand{\theHsection}{2024r.\arabic{section}}
\renewcommand{\theHsubsection}{2024r.\arabic{section}.\arabic{subsection}}

%----------------------------------------------------------------------
\section{問題1:物質中のMaxwell方程式(類題:2023年度 問題1、2024年度 問題1;第2回 問題1)}\label{sec:2024r-1}
%----------------------------------------------------------------------

\subsection{問題}\label{sec:2024r-1-prob}

静電磁場における物質中のMaxwell方程式を考え、その後、時間変動する電磁場に拡張する。真空中の誘電率を $\varepsilon_0$、透磁率を $\mu_0$ とする。

\begin{enumerate}
\item 物質内部に真電荷(電荷密度 $\rho$)、真電流(電流密度 $\mathbf{i}$)、分極電荷(電荷密度 $\rho_p$)、磁化電流(電流密度 $\mathbf{i}_M$)が存在する場合に、電場 $\mathbf{E}$、磁束密度 $\mathbf{B}$、$\varepsilon_0$、$\mu_0$、$\rho$、$\mathbf{i}$、$\rho_p$、$\mathbf{i}_M$ だけを用いて、4つのMaxwell方程式を示せ。
\item 分極ベクトル $\mathbf{P}$ と磁化ベクトル $\mathbf{M}$ を用いて $\rho_p = -\nabla\cdot\mathbf{P}$、$\mathbf{i}_M = \nabla\times\mathbf{M}/\mu_0$ と表せる。物質の誘電率 $\varepsilon$ は $\varepsilon\mathbf{E} = \varepsilon_0\mathbf{E}+\mathbf{P}$ で定義され、透磁率 $\mu$ は $\mathbf{B}/\mu = (\mathbf{B}-\mathbf{M})/\mu_0$ で定義される。前問で求めたMaxwell方程式のうち、分極電荷と磁化電流を含む2つの式を、$\mathbf{E}$、$\mathbf{B}$、$\varepsilon$、$\mu$、$\rho$、$\mathbf{i}$ だけを用いて書き直せ。ただし $\varepsilon$、$\mu$ は場所の関数とする。
\item 時間変動する電磁場を考える。物質中のMaxwell方程式のうち、時間変動する項を含む2つの式を、$\mathbf{E}$、$\mathbf{B}$、$\varepsilon_0$、$\mu_0$、$\mathbf{i}$、$\mathbf{P}$、$\mathbf{M}$ を用いて示せ。
\end{enumerate}

\subsection{解答}\label{sec:2024r-1-ans}

\paragraph{この問題のポイント}

2023・2024年度問題1と同一である。1-1で4式、1-2で $\nabla\cdot(\varepsilon\mathbf{E})=\rho$ と $\nabla\times(\mathbf{B}/\mu)=\mathbf{i}$、1-3で時間変動を含む2式を示す。

\paragraph{解き方の流れ}
\begin{enumerate}
\item 問1:静電磁場の4式を書く。問2:$\mathbf{D}$、$\mathbf{B}/\mu$ の形に書き直す。問3:ファラデーとアンペールの時間変動を含む式を書く(詳細は2023年度問題1参照)。
\end{enumerate}

\paragraph{1-1:4つのMaxwell方程式}

静電磁場では、求める4式は
\解答
\begin{equation}
\boxed{\nabla\cdot\mathbf{E} = \frac{\rho + \rho_p}{\varepsilon_0}}, \quad
\boxed{\nabla\cdot\mathbf{B} = 0}, \quad
\boxed{\nabla\times\mathbf{E} = \mathbf{0}}, \quad
\boxed{\nabla\times\mathbf{B} = \mu_0(\mathbf{i} + \mathbf{i}_M)}.
\end{equation}
($\mathbf{i}_M = \nabla\times\mathbf{M}/\mu_0$ の定義に合わせるなら、右辺は $\mu_0\mathbf{i} + \nabla\times\mathbf{M}$。)

\paragraph{1-2:$\mathbf{E},\mathbf{B},\varepsilon,\mu,\rho,\mathbf{i}$ だけを用いた2式}

\textbf{分極電荷を含む式}:1-1のガウスの法則 $\nabla\cdot\mathbf{E}=(\rho+\rho_p)/\varepsilon_0$ に $\rho_p=-\nabla\cdot\mathbf{P}$ を代入し、移項すると $\nabla\cdot(\varepsilon_0\mathbf{E}+\mathbf{P})=\rho$。$\mathbf{D}=\varepsilon\mathbf{E}=\varepsilon_0\mathbf{E}+\mathbf{P}$ なので、
\begin{equation}
\nabla\cdot(\varepsilon\mathbf{E}) = \rho.
\end{equation}
\textbf{磁化電流を含む式}:1-1のアンペールの法則 $\nabla\times\mathbf{B}=\mu_0\mathbf{i}+\nabla\times\mathbf{M}$ と、問題文の透磁率の定義 $\mathbf{B}/\mu=(\mathbf{B}-\mathbf{M})/\mu_0$ から、$\nabla\times(\mathbf{B}/\mu)=\mathbf{i}$ が得られる(詳細は2023年度問題1参照)。よって
\解答
\begin{equation}
\boxed{\nabla\cdot(\varepsilon\mathbf{E}) = \rho}, \qquad \boxed{\nabla\times\left(\frac{\mathbf{B}}{\mu}\right) = \mathbf{i}}.
\end{equation}

\paragraph{1-3:時間変動する項を含む2式}

\解答
\begin{equation}
\boxed{\nabla\times\mathbf{E} = -\frac{\partial\mathbf{B}}{\partial t}}
\end{equation}
\begin{equation}
\boxed{\nabla\times\mathbf{B} = \mu_0\mathbf{i} + \mu_0\varepsilon_0\frac{\partial\mathbf{E}}{\partial t} + \mu_0\frac{\partial\mathbf{P}}{\partial t} + \nabla\times\mathbf{M}}.
\end{equation}

\paragraph{物理的意味(初学者向け)}

物質中では分極電荷・磁化電流が電磁場の源になる。1-1のガウスの法則の右辺 $\rho+\rho_p$ は「真電荷と分極電荷の和」、アンペールの法則の右辺は「真電流と磁化電流の和」である。1-2で $\mathbf{D}=\varepsilon\mathbf{E}$、$\mathbf{B}/\mu$ を導入すると、式の上では真電荷・真電流だけが右辺に現れる。詳細な導出・図は2023年度問題1を参照。

\begin{figure}[H]
\centering
\includegraphics[width=0.9\textwidth]{figures/em_maxwell_concept.png}
\caption{問題1:物質中のMaxwell方程式の物理的意味。電場の源は真電荷と分極電荷の和、磁場の回転の源は真電流と磁化電流の和。}
\label{fig:em2024retake_maxwell}
\end{figure}

%----------------------------------------------------------------------
\section{問題2:強誘電体球の電場(類題:2024年度 問題2;第4回 問題1・2、第1回 問題1)}\label{sec:2024r-2}
%----------------------------------------------------------------------

\subsection{問題}\label{sec:2024r-2-prob}

半径 $a$ の強誘電体球が $\mathbf{P} = (0,0,P)$ で一様に自発分極している。球の中心は原点、外部電場はゼロ、真電荷はない。分極状態を再現するため、電荷密度 $+\rho$ で一様に帯電した半径 $a$ の球を $(0,0,s/2)$ に、電荷密度 $-\rho$ で一様に帯電した半径 $a$ の球を $(0,0,-s/2)$ に置く($s \ll a$)。分極電荷が球内部に作る電場 $\mathbf{E}_p$ は、これらの帯電球が作る電場の足し合わせとする。

\begin{enumerate}
\item 中心が原点にあり、電荷密度 $+\rho$ で一様に帯電した半径 $a$ の球のみを考える。原点から位置ベクトル $\mathbf{r}$ の場所における球内部の電場 $\mathbf{E}$ を $\rho$、$\varepsilon_0$、$\mathbf{r}$ を用いて求めよ。
\item 問題文の通り、中心が原点から $z$ 軸上に微小距離 $\pm s/2$ だけずれた場所に置かれた $\pm\rho$ の2つの帯電球が作る電場の足し合わせを考える。$\mathbf{s} = (0,0,s)$ とすると $\mathbf{P} = \rho\mathbf{s}$ と書けることを用いて、$\mathbf{E}_p$ を $\mathbf{P}$ と $\varepsilon_0$ を用いて表せ。
\end{enumerate}

\subsection{解答}\label{sec:2024r-2-ans}

\paragraph{この問題のポイント}

2024年度問題2の(2-1)(2-2)と同一である。一様帯電球のガウスの法則から内部電場を求め、2つの帯電球の和から $\mathbf{E}_p=-\mathbf{P}/(3\varepsilon_0)$ を導く。

\paragraph{用語の説明}
\begin{itemize}
\item \textbf{強誘電体}:外部電場がなくても自発分極 $\mathbf{P}$ を持つ物質。本問では一様に $\mathbf{P}=(0,0,P)$ で分極している球を考える。
\end{itemize}

\paragraph{解き方の流れ}
\begin{enumerate}
\item 問(2-1):球対称なガウスの法則で $\mathbf{E}=(\rho/(3\varepsilon_0))\mathbf{r}$ を求める。
\item 問(2-2):中心が $(0,0,\pm s/2)$ の2球の電場の和を計算し、$\mathbf{P}=\rho\mathbf{s}$ で $\mathbf{E}_p=-\mathbf{P}/(3\varepsilon_0)$ を得る。
\end{enumerate}

\paragraph{2-1:一様帯電球内部の電場}

中心が原点、半径 $a$、電荷密度 $+\rho$ の球を考える。ガウスの法則:閉曲面 $S$ について、\textbf{電束の総和} $\int_S \mathbf{E}\cdot\mathbf{n}\,dS$ は、$S$ の内部の全電荷 $Q_{\mathrm{in}}$ を $\varepsilon_0$ で割ったものに等しい。球対称なので、原点から距離 $r$($r \le a$)の点では $\mathbf{E}$ は動径方向で大きさは $r$ のみに依存する。半径 $r$ の球面でガウスの法則を適用すると、左辺は $E(r)\times 4\pi r^2$、右辺は $Q_{\mathrm{in}}/\varepsilon_0 = (4\pi r^3/3)\rho/\varepsilon_0$ となる。よって $4\pi r^2 E(r) = (4\pi r^3/3)\rho/\varepsilon_0$ から $E(r) = \rho r/(3\varepsilon_0)$。位置 $\mathbf{r}$ では動径外向きに $\mathbf{E} = (\rho/(3\varepsilon_0))\mathbf{r}$。したがって
\解答
\begin{equation}
\boxed{\mathbf{E} = \frac{\rho}{3\varepsilon_0}\mathbf{r}}.
\end{equation}

\paragraph{なぜ一様帯電球内部の電場が $\mathbf{E}=(\rho/(3\varepsilon_0))\mathbf{r}$ になるか(原理的な説明)}

球対称な電荷分布では、ガウスの法則から、半径 $r$ の球面内の全電荷 $Q_{\mathrm{in}} = (4\pi r^3/3)\rho$ が作る電場は、球面上一様で動径外向きに $E(r)=Q_{\mathrm{in}}/(4\pi\varepsilon_0 r^2) = \rho r/(3\varepsilon_0)$ となる。つまり内部の点では、その点より内側の電荷だけが寄与し、外側の電荷の寄与は球対称性からゼロである。

\paragraph{2-2:分極電場 $\mathbf{E}_p$}

$+\rho$ の球の中心が $(0,0,s/2)$、$-\rho$ の球の中心が $(0,0,-s/2)$ にあり、$s \ll a$ とする。\textbf{原点付近の点} $\mathbf{r}$ は、$s$ が十分小さいため、\textbf{両方の球の内部}にあるとみなせる。したがって、$+\rho$ の球が作る電場は問(2-1)の公式が使え、その球の中心 $(0,0,s/2)$ からの位置ベクトル $\mathbf{r}_+ = \mathbf{r} - (0,0,s/2)$ を用いて $\mathbf{E}_+ = (\rho/(3\varepsilon_0))\mathbf{r}_+$。同様に $-\rho$ の球が作る電場は $\mathbf{r}_- = \mathbf{r} - (0,0,-s/2)$ として $\mathbf{E}_- = (-\rho/(3\varepsilon_0))\mathbf{r}_-$。したがって
\begin{equation}
\mathbf{E}_p = \mathbf{E}_+ + \mathbf{E}_- = \frac{\rho}{3\varepsilon_0}(\mathbf{r}_+ - \mathbf{r}_-) = \frac{\rho}{3\varepsilon_0}\bigl((\mathbf{r} - (0,0,s/2)) - (\mathbf{r} - (0,0,-s/2))\bigr) = \frac{\rho}{3\varepsilon_0}(0,0,-s) = -\frac{\rho\mathbf{s}}{3\varepsilon_0}.
\end{equation}
$\mathbf{P} = \rho\mathbf{s}$ なので
\解答
\begin{equation}
\boxed{\mathbf{E}_p = -\frac{\mathbf{P}}{3\varepsilon_0}}.
\end{equation}

\paragraph{なぜそう求まるか(原理的な説明)}

一様分極を「$z$ 軸上に微小にずれた $+\rho$ と $-\rho$ の2つの一様帯電球」で等価に置き換えると、各球内部の電場はガウスの法則から中心からの位置ベクトルに比例する。2つの電場を足し合わせると、$\mathbf{r}$ に比例する項は打ち消し、ずれベクトル $\mathbf{s}$ に比例する項だけが残り $\mathbf{E}_p = -\rho\mathbf{s}/(3\varepsilon_0) = -\mathbf{P}/(3\varepsilon_0)$ となる。この\textbf{脱分極電場}は分極と逆向きに働く。

\begin{figure}[H]
\centering
\includegraphics[width=0.85\textwidth]{figures/em_depolarization_sphere.png}
\caption{問題2:一様分極を2つの帯電球で再現するモデル。内部では2球の電場の和が $\mathbf{E}_p=-\mathbf{P}/(3\varepsilon_0)$ という脱分極電場になる。}
\label{fig:em2024retake_depolarization}
\end{figure}

%----------------------------------------------------------------------
\section{問題3:液体中の極性分子の分極と電気感受率(類題:2024年度 問題3;第7回 問題1)}\label{sec:2024r-3}
%----------------------------------------------------------------------

\subsection{問題}\label{sec:2024r-3-prob}

時刻 $t=0$ で一様な電場 $\mathbf{E}$ がかけられたとき、水分子が回転して向きが揃い、分極ベクトルの大きさ $P(t)$ が時定数 $\tau$ で $P(t) = \varepsilon_0\chi_{e0}E(1-\exp(-t/\tau))$ のように増加する($\varepsilon_0$ は真空の誘電率)。水分子の分極が液体内の全電場に与える変化は無視する。

\begin{enumerate}
\item 上記の $P(t)$ が微分方程式 $\tau\,dP/dt + P = \varepsilon_0\chi_{e0}E$ を満たすことを利用し、単一角振動数 $\omega$ で振動する電場 $E(\omega) = E_0\exp(-\mathrm{i}\omega t)$ に対する電気感受率 $\chi_e(\omega)$ を、$\chi_{e0}$、$\tau$、$\omega$ を用いて求めよ。ヒント:$P(\omega) = P_0\exp(-\mathrm{i}\omega t)$ とし、$\chi_e(\omega) = P(\omega)/(\varepsilon_0 E(\omega))$ を計算する。
\item 前問で求めた電気感受率 $\chi_e(\omega)$ の実部 $\chi_e'$ が $\omega$ に対してどのように変化するか図示し、$\omega = \tau^{-1} = \omega_0$ における値を図に記入せよ。
\end{enumerate}

\subsection{解答}\label{sec:2024r-3-ans}

\paragraph{この問題のポイント}

2024年度問題3の(3-1)(3-2)と同一である。記号 $\chi_0$ が $\chi_{e0}$ になっているだけ。緩和型の微分方程式から周波数依存の電気感受率を求める。

\begin{figure}[H]
\centering
\includegraphics[width=0.85\textwidth]{figures/em2024_polar_molecule.png}
\caption{問題3の設定:液体中の極性分子(水分子)と電場。電場をかけると分子の双極子が電場方向に向きを変えようとするが、液体中では分子の回転に粘性抵抗があり、時定数 $\tau$ で遅れて追従する。}
\label{fig:em2024retake_polar_molecule}
\end{figure}

\paragraph{解き方の流れ}
\begin{enumerate}
\item 問(3-1):$\tau dP/dt+P=\varepsilon_0\chi_{e0}E$ に $E=E_0 e^{-\mathrm{i}\omega t}$、$P=P_0 e^{-\mathrm{i}\omega t}$ を代入し、$\chi_e(\omega)=\chi_{e0}/(1-\mathrm{i}\omega\tau)$ を導く。
\item 問(3-2):$\mathrm{Re}[\chi_e(\omega)]=\chi_{e0}/(1+\omega^2\tau^2)$ を図示し、$\omega_0=1/\tau$ で $\chi_{e0}/2$ を記入する。
\end{enumerate}

\paragraph{3-1:電気感受率 $\chi_e(\omega)$}

\subparagraph{ステップ1:なぜ $P = P_0\exp(-\mathrm{i}\omega t)$ とおけるか}

微分方程式 $\tau\frac{dP}{dt} + P = \varepsilon_0\chi_{e0} E$ は、$P$ について\textbf{線形}である。線形微分方程式では、\textbf{入力 $E$ が単一角振動数 $\omega$ で振動するとき、定常状態の解 $P$ も同じ角振動数 $\omega$ で振動する}。したがって、$E = E_0\exp(-\mathrm{i}\omega t)$ に対して $P = P_0\exp(-\mathrm{i}\omega t)$ とおける($P_0$ は複素数になりうる)。

\subparagraph{ステップ2:微分方程式へ代入して $P_0$ を求める}

$E = E_0\exp(-\mathrm{i}\omega t)$、$P = P_0\exp(-\mathrm{i}\omega t)$ を代入する。$dP/dt = -\mathrm{i}\omega P_0\exp(-\mathrm{i}\omega t)$ より、左辺は $\tau(-\mathrm{i}\omega)P_0\exp(-\mathrm{i}\omega t) + P_0\exp(-\mathrm{i}\omega t) = \exp(-\mathrm{i}\omega t)[\tau(-\mathrm{i}\omega)P_0 + P_0]$、右辺は $\varepsilon_0\chi_{e0} E_0\exp(-\mathrm{i}\omega t)$。$\exp(-\mathrm{i}\omega t)$ を消去すると
\[
\tau(-\mathrm{i}\omega)P_0 + P_0 = \varepsilon_0\chi_{e0} E_0.
\]
左辺を $P_0$ でくくると $P_0(1 - \mathrm{i}\omega\tau) = \varepsilon_0\chi_{e0} E_0$。よって $P_0 = \varepsilon_0\chi_{e0} E_0/(1 - \mathrm{i}\omega\tau)$。

\subparagraph{ステップ3:電気感受率の定義から $\chi_e(\omega)$ を得る}

電気感受率は $\chi_e(\omega) = P(\omega)/(\varepsilon_0 E(\omega)) = P_0/(\varepsilon_0 E_0)$ である。したがって
\[
\chi_e(\omega) = \frac{P_0}{\varepsilon_0 E_0} = \frac{\varepsilon_0\chi_{e0} E_0/(1-\mathrm{i}\omega\tau)}{\varepsilon_0 E_0} = \frac{\chi_{e0}}{1 - \mathrm{i}\omega\tau}.
\]
分母を実数化するには分子・分母に $1+\mathrm{i}\omega\tau$ を掛けると、$\chi_e(\omega) = \chi_{e0}(1+\mathrm{i}\omega\tau)/(1+\omega^2\tau^2)$ となる。
\解答
\begin{equation}
\boxed{\chi_e(\omega) = \frac{\chi_{e0}}{1 - \mathrm{i}\omega\tau} = \frac{\chi_{e0}(1+\mathrm{i}\omega\tau)}{1+\omega^2\tau^2}}.
\end{equation}

\subparagraph{複素感受率の意味(実部と虚部)}

\textbf{実部 $\mathrm{Re}[\chi_e] = \chi_{e0}/(1+\omega^2\tau^2)$}:電場と同位相の分極成分(分極能)。\textbf{虚部 $\mathrm{Im}[\chi_e] = \chi_{e0}\omega\tau/(1+\omega^2\tau^2)$}:電場と90度位相ずれた成分で、エネルギー吸収に対応し、$\omega\sim 1/\tau$ 付近でピークを持つ。図\ref{fig:em2024retake_chi_complex} 参照。

\begin{figure}[H]
\centering
\includegraphics[width=0.9\textwidth]{figures/em2024_chi_complex.png}
\caption{問題3(3-1):複素電気感受率の意味。実部は電場と同位相の分極(分極能)、虚部は90度位相ずれ(エネルギー吸収)に対応する。}
\label{fig:em2024retake_chi_complex}
\end{figure}

\begin{figure}[H]
\centering
\includegraphics[width=0.9\textwidth]{figures/em_debye_relaxation.png}
\caption{問題3:分極の時間応答 $P(t)$(左)と周波数応答 $\mathrm{Re}[\chi_e(\omega)]$(右)。時定数 $\tau$ で遅れる応答のため、高周波では感受率が低下する。}
\label{fig:em2024retake_debye}
\end{figure}

\paragraph{3-2:実部 $\chi_e'$ の図示}

\subparagraph{実部の式の導出}

$\chi_e(\omega) = \chi_{e0}(1+\mathrm{i}\omega\tau)/(1+\omega^2\tau^2)$ の実部を取ると、
\[
\mathrm{Re}[\chi_e(\omega)] = \frac{\chi_{e0}}{1+\omega^2\tau^2}.
\]

\subparagraph{グラフの形と $\omega_0$ における値}

$\omega=0$ で $\chi_{e0}$、$\omega\to\infty$ で 0。$\omega = \omega_0 = 1/\tau$ では $\omega^2\tau^2 = 1$ より $\mathrm{Re}[\chi_e(\omega_0)] = \chi_{e0}/2$。横軸 $\omega$、縦軸 $\mathrm{Re}[\chi_e(\omega)]$ のグラフは、$\omega=0$ で $\chi_{e0}$ から始まり、なだらかに減少して 0 に近づく曲線となる。$\omega_0 = 1/\tau$ の位置に $\chi_{e0}/2$ を記入する。
\解答
$\mathrm{Re}[\chi_e(\omega)] = \chi_{e0}/(1+\omega^2\tau^2)$ の概略を描き、$\omega_0 = 1/\tau$ において $\boxed{\chi_{e0}/2}$ を記入する。

\begin{figure}[H]
\centering
\includegraphics[width=0.85\textwidth]{figures/em2024_chi_real.png}
\caption{問題3(3-2):電気感受率の実部 $\mathrm{Re}[\chi_e(\omega)]$。横軸は $\omega\tau$。$\omega_0=1/\tau$ で $\chi_{e0}/2$。(本図では $\chi_0$ と表記しているが、再試験では $\chi_{e0}$ に対応。)}
\label{fig:em2024retake_chi_real}
\end{figure}

\paragraph{なぜ高周波で感受率が低下するか(物理的考察)}

微分方程式 $\tau\dot{P}+P=\varepsilon_0\chi_{e0} E$ は、分極が電場の変化に「時定数 $\tau$ で遅れて」追従することを表す(緩和型)。直流($\omega=0$)では $P=\varepsilon_0\chi_{e0} E$ で感受率は $\chi_{e0}$。角振動数 $\omega$ が $1/\tau$ 程度になると、電場の向きが変わるのが速く、分極が追いつかなくなる。極性分子の配向分極は、分子が回転して電場の向きに揃う過程であり、隣接分子との衝突などで時定数 $\tau$ の遅れが生じる。そのため $\omega$ が大きいほど感受率の実部は小さくなり、このような周波数分散を\textbf{デバイ緩和}という。

%----------------------------------------------------------------------
\section{問題4:金属内の電子の運動と電気伝導率(類題:第2回 問題2・3)}\label{sec:2024r-4}
%----------------------------------------------------------------------

\subsection{問題}\label{sec:2024r-4-prob}

電場中に置かれた金属内を移動する電子には、電場によるクーロン力と、運動と逆向きに働く抵抗力(電子の運動速度 $\mathbf{v}$ に比例し、$\gamma m\mathbf{v}$ と書ける。$m$ は電子の質量)が働く。電子の電荷を $q$、数密度を $n$ とする。

\begin{enumerate}
\item 電流密度 $\mathbf{i}$ と運動速度 $\mathbf{v}$ の関係を示せ。
\item 電場が一定の場合、力が釣り合って定常運動となり、オームの法則 $\mathbf{i} = \sigma\mathbf{E}$ が成り立つ。このときの電気伝導率 $\sigma$ を求めよ。
\item 電場が $\mathbf{E} = \mathbf{E}_0\exp(-\mathrm{i}\omega t)$ によって時間変動するとき、複素電気伝導率 $\sigma(\omega)$ の実部 $\sigma'$ と虚部 $\sigma''$ を求めよ。
\end{enumerate}

\subsection{解答}\label{sec:2024r-4-ans}

\paragraph{この問題のポイント(初学者向け)}

電子の運動方程式 $m d\mathbf{v}/dt = q\mathbf{E} - \gamma m\mathbf{v}$ から、定常状態では $q\mathbf{E} = \gamma m\mathbf{v}$ となり、電流密度 $\mathbf{i} = nq\mathbf{v}$ と合わせてオームの法則 $\mathbf{i} = \sigma\mathbf{E}$ および $\sigma = nq^2/(\gamma m)$ を得る。時間変動電場では $\mathbf{v} \propto e^{-\mathrm{i}\omega t}$ とおき、複素伝導率 $\sigma(\omega)$ の実部・虚部を求める。

\paragraph{解き方の流れ}
\begin{enumerate}
\item 問(4-1):電流密度の定義から $\mathbf{i}=nq\mathbf{v}$ を示す。
\item 問(4-2):定常状態で $d\mathbf{v}/dt=\mathbf{0}$ より $\mathbf{v}=(q/(\gamma m))\mathbf{E}$。$\mathbf{i}=nq\mathbf{v}$ から $\sigma=nq^2/(\gamma m)$ を導く。
\item 問(4-3):$E=E_0 e^{-\mathrm{i}\omega t}$、$v=v_0 e^{-\mathrm{i}\omega t}$ を運動方程式に代入し、$\sigma(\omega)$ の実部 $\sigma'$ と虚部 $\sigma''$ を求める。
\end{enumerate}

\paragraph{使用する物理法則・用語}

運動方程式:$m\frac{d\mathbf{v}}{dt} = q\mathbf{E} - \gamma m\mathbf{v}$。電流密度の定義:$\mathbf{i} = nq\mathbf{v}$($n$ は電子数密度、$q$ は電子の電荷。電子なら $q<0$。)抵抗力の係数 $\gamma$ は\textbf{減衰係数}で、散乱(フォノンや不純物との衝突)の頻度を表し、単位は 1/秒の次元を持つ。$\gamma$ が大きいほど定常速度は小さくなり、伝導率 $\sigma = nq^2/(\gamma m)$ は小さくなる。

\paragraph{4-1:電流密度 $\mathbf{i}$ と速度 $\mathbf{v}$ の関係}

電流密度は、単位面積を単位時間に通過する電荷の流れである。電子が速度 $\mathbf{v}$ で運動しているとき、数密度 $n$、電荷 $q$ なら、$\mathbf{i} = nq\mathbf{v}$。したがって
\解答
\begin{equation}
\boxed{\mathbf{i} = nq\mathbf{v}}.
\end{equation}

\paragraph{なぜ電流密度が $\mathbf{i}=nq\mathbf{v}$ か(原理的な説明)}

電流の大きさ $I$ は、断面を単位時間に通過する電荷の総量である。断面積 $S$、長さ $v\,\Delta t$ の細い管を考え、電子が速度 $v$ で管に沿って動いているとすると、$\Delta t$ の間に通過する電子数は $n S v\,\Delta t$、電荷は $n q S v\,\Delta t$ なので、電流は $I = nqvS$、電流密度の大きさは $i = I/S = nqv$。ベクトルで書くと $\mathbf{i} = nq\mathbf{v}$ となる。

\paragraph{4-2:定常状態の電気伝導率 $\sigma$}

定常状態では $d\mathbf{v}/dt = \mathbf{0}$ なので、運動方程式は $q\mathbf{E} - \gamma m\mathbf{v} = \mathbf{0}$、すなわち $\mathbf{v} = (q/(\gamma m))\mathbf{E}$。したがって
\begin{equation}
\mathbf{i} = nq\mathbf{v} = \frac{nq^2}{\gamma m}\mathbf{E}.
\end{equation}
オームの法則 $\mathbf{i} = \sigma\mathbf{E}$ と比較して、
\解答
\begin{equation}
\boxed{\sigma = \frac{nq^2}{\gamma m}}.
\end{equation}

\paragraph{なぜオームの法則が成り立つか(物理的考察)}

定常状態では、電場による力 $q\mathbf{E}$ と抵抗力 $\gamma m\mathbf{v}$ が釣り合い、$\mathbf{v} = (q/(\gamma m))\mathbf{E}$ となる。つまり速度は電場に比例する。電流密度 $\mathbf{i}=nq\mathbf{v}$ も電場に比例し、$\mathbf{i}=\sigma\mathbf{E}$(オームの法則)が成り立つ。伝導率 $\sigma = nq^2/(\gamma m)$ は、キャリア数 $n$ が多く、質量 $m$ が小さく、抵抗力 $\gamma$ が小さいほど大きくなる。金属では自由電子の数密度が高く、$\gamma$ は散乱(フォノンや不純物との衝突)の頻度を表す。

\paragraph{4-3:時間変動電場における複素伝導率 $\sigma(\omega)$ の実部・虚部}

$\mathbf{E} = \mathbf{E}_0\exp(-\mathrm{i}\omega t)$ のとき、定常振動解として $\mathbf{v} = \mathbf{v}_0\exp(-\mathrm{i}\omega t)$ を仮定する。運動方程式 $m\frac{d\mathbf{v}}{dt} = q\mathbf{E} - \gamma m\mathbf{v}$ に代入すると、$m(-\mathrm{i}\omega)\mathbf{v}_0 = q\mathbf{E}_0 - \gamma m\mathbf{v}_0$。したがって $(-\mathrm{i}\omega m + \gamma m)\mathbf{v}_0 = q\mathbf{E}_0$、$\mathbf{v}_0 = \frac{q\mathbf{E}_0}{m(\gamma - \mathrm{i}\omega)}$。電流密度は $\mathbf{i} = nq\mathbf{v} = nq\mathbf{v}_0 e^{-\mathrm{i}\omega t} = \sigma(\omega)\mathbf{E}$ となるので、
\begin{equation}
\sigma(\omega) = \frac{nq^2}{m(\gamma - \mathrm{i}\omega)} = \frac{nq^2}{m}\,\frac{1}{\gamma - \mathrm{i}\omega}.
\end{equation}
分母を実数化:$\frac{1}{\gamma - \mathrm{i}\omega} = \frac{\gamma + \mathrm{i}\omega}{\gamma^2 + \omega^2}$ なので、
\begin{equation}
\sigma(\omega) = \frac{nq^2}{m}\,\frac{\gamma + \mathrm{i}\omega}{\gamma^2 + \omega^2}.
\end{equation}
実部と虚部は
\begin{equation}
\sigma' = \mathrm{Re}[\sigma(\omega)] = \frac{nq^2}{m}\,\frac{\gamma}{\gamma^2 + \omega^2}, \qquad
\sigma'' = \mathrm{Im}[\sigma(\omega)] = \frac{nq^2}{m}\,\frac{\omega}{\gamma^2 + \omega^2}.
\end{equation}
したがって
\解答
\begin{equation}
\boxed{\sigma' = \frac{nq^2\gamma}{m(\gamma^2 + \omega^2)}, \qquad \sigma'' = \frac{nq^2\omega}{m(\gamma^2 + \omega^2)}}.
\end{equation}

\paragraph{なぜ高周波で $\sigma'$ が減り $\sigma''$ が効くか(原理的な説明)}

電場が $E=E_0 e^{-\mathrm{i}\omega t}$ で振動するとき、電子の運動方程式 $m\dot{v} = qE - \gamma m v$ の定常振動解は $v = (q/(m(\gamma-\mathrm{i}\omega)))E$ となる。$\omega=0$(直流)では $v = (q/(\gamma m))E$ で、電流は電場と同相であり、$\sigma$ は実数でジュール損失を表す。$\omega$ が大きくなると、電子の\textbf{慣性}($m\dot{v}$ の項)が効き、速度は電場に対して位相遅れを持つ。その結果、電流の電場と同相の成分(実部 $\sigma'$)は減り、$90^\circ$ ずれた成分(虚部 $\sigma''$)が現れる。$\sigma'$ は $\omega^2$ が $\gamma^2$ より大きくなると減少し、$\sigma''$ は $\omega$ に比例して増加する領域がある。高周波では電子が電場の変化に追従しきれず、伝導率の実効値が低下する。

\begin{figure}[H]
\centering
\includegraphics[width=0.85\textwidth]{figures/em2024retake_sigma_omega.png}
\caption{問題4(4-3):複素電気伝導率の実部 $\sigma'$ と虚部 $\sigma''$ の周波数依存性。$\omega\sim\gamma$ で $\sigma'$ が減り、$\sigma''$ がピーク近くになる。}
\label{fig:em2024retake_sigma}
\end{figure}

\begin{figure}[H]
\centering
\includegraphics[width=0.85\textwidth]{figures/em_drude_physics.png}
\caption{問題4:高周波で $\sigma'$ が減る理由(電子の慣性・位相遅れ)の概念図。実部はジュール損失、虚部は慣性による位相ずれに対応する。}
\label{fig:em2024retake_drude}
\end{figure}


\end{document}
