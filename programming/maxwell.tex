\documentclass[a4paper]{article}
\usepackage[utf8]{inputenc}
\usepackage{xeCJK}
\usepackage{amsmath}
\usepackage{amsfonts}
\usepackage{amssymb}
\usepackage{bm}
\usepackage{xcolor}

% 日本語フォント設定
\setCJKmainfont{Hiragino Mincho Pro}

% 日付を2026年1月20日に設定(青文字)
\date{\textcolor{blue}{2026年1月20日}}

% タイトルと著者
\title{マックスウェル方程式}
\author{\textcolor{blue}{川嶋宥翔}}

\begin{document}

\maketitle

\begin{center}
\section*{概要}

真空中のマックスウェル方程式を $\mathrm{LATEX}$ で表示する。
\end{center}

\section{ベクトル表示}

電場を $E$、磁場を $B$、電荷密度を $\rho_q$、電流密度を $j$ とすると、真空中のマックスウェル方程式は以下のようになる。

\begin{equation}
\nabla \cdot \boldsymbol{E} = \frac{\rho_q}{\epsilon_0} \tag{1}
\end{equation}

\begin{equation}
\nabla \times \boldsymbol{E} = -\frac{\partial \boldsymbol{B}}{\partial t} \tag{2}
\end{equation}

\begin{equation}
\nabla \cdot \boldsymbol{B} = 0 \tag{3}
\end{equation}

\begin{equation}
\nabla \times \boldsymbol{B} = \mu_0 \boldsymbol{j} + c^{-2} \frac{\partial \boldsymbol{E}}{\partial t} \tag{4}
\end{equation}

ここで $\epsilon_0$ は誘電率、$\mu_0$ は透磁率、$c$ は光速である。

\section{成分表示}

式 (1) と (2) を $x, y, z$ のデカルト座標表示を用いて成分表示すると以下のようになる。

\subsection*{式 (1) の成分表示}

$\nabla = \left( \frac{\partial}{\partial x}, \frac{\partial}{\partial y}, \frac{\partial}{\partial z} \right)$、$\boldsymbol{E} = (E_x, E_y, E_z)$ とすると、

\begin{align}
\nabla \cdot \boldsymbol{E} &= \frac{\partial E_x}{\partial x} + \frac{\partial E_y}{\partial y} + \frac{\partial E_z}{\partial z} = \frac{\rho_q}{\epsilon_0} \tag{1'}
\end{align}

\subsection*{式 (2) の成分表示}

$\nabla \times \boldsymbol{E}$ の各成分は以下のようになる。

$x$ 成分:
\begin{equation}
\frac{\partial E_z}{\partial y} - \frac{\partial E_y}{\partial z} = -\frac{\partial B_x}{\partial t} \tag{2a}
\end{equation}

$y$ 成分:
\begin{equation}
\frac{\partial E_x}{\partial z} - \frac{\partial E_z}{\partial x} = -\frac{\partial B_y}{\partial t} \tag{2b}
\end{equation}

$z$ 成分:
\begin{equation}
\frac{\partial E_y}{\partial x} - \frac{\partial E_x}{\partial y} = -\frac{\partial B_z}{\partial t} \tag{2c}
\end{equation}

\end{document}
